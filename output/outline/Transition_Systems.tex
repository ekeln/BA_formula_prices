%
\begin{isabellebody}%
\setisabellecontext{Transition{\isacharunderscore}{\kern0pt}Systems}%
%
\isadelimtheory
%
\endisadelimtheory
%
\isatagtheory
%
\endisatagtheory
{\isafoldtheory}%
%
\isadelimtheory
%
\endisadelimtheory
%
\isadelimdocument
%
\endisadelimdocument
%
\isatagdocument
%
\isamarkupsection{Labelled Transition Systems%
}
\isamarkuptrue%
%
\endisatagdocument
{\isafolddocument}%
%
\isadelimdocument
%
\endisadelimdocument
%
\begin{isamarkuptext}%
\label{sec:LTS}%
\end{isamarkuptext}\isamarkuptrue%
%
\isadelimdocument
%
\endisadelimdocument
%
\isatagdocument
%
\isamarkupsubsubsection{Definition 1 (Labeled transition Systems)%
}
\isamarkuptrue%
%
\endisatagdocument
{\isafolddocument}%
%
\isadelimdocument
%
\endisadelimdocument
%
\begin{isamarkuptext}%
A Labelled Transition System (LTS) is a tuple $\mathcal{S} = (\Proc, \Act, \rightarrow)$ where $\Proc$ is the set of processes, 
$\Act$ is the set of actions and $\rightarrow$ $\subseteq \Proc \times \Act \times \Proc$ is a transition relation.\\

In concurrency theory, it is customary that the semantics of Reactive Systems are given in terms of labelled transition systems.
The processes represent the states a reactive system can be in. A transition of one state into another, 
caused by performing an action, can be understood as moving along the corresponding edge in the transition relation. \\

- Example?\\

- examples (to reuse later?)???%
\end{isamarkuptext}\isamarkuptrue%
%
\isadelimdocument
%
\endisadelimdocument
%
\isatagdocument
%
\isamarkupsubsubsection{Some more Definitions%
}
\isamarkuptrue%
%
\endisatagdocument
{\isafolddocument}%
%
\isadelimdocument
%
\endisadelimdocument
%
\begin{isamarkuptext}%
The $\alpha$-derivatives of a process are the processes that can be reached with one $\alpha$-transition:%
\end{isamarkuptext}\isamarkuptrue%
%
\begin{isamarkuptext}%
- image finite? nötig?\\
- image countable?
- initial actions
- deadlock
- relevant actions?
- step sequence!
-%
\end{isamarkuptext}\isamarkuptrue%
%
\isadelimdocument
%
\endisadelimdocument
%
\isatagdocument
%
\isamarkupsubsection{Isabelle%
}
\isamarkuptrue%
%
\endisatagdocument
{\isafolddocument}%
%
\isadelimdocument
%
\endisadelimdocument
%
\begin{isamarkuptext}%
Zustände: \isa{{\isacharprime}{\kern0pt}s} und Aktionen \isa{{\isacharprime}{\kern0pt}a}, Transitionsrelation ist locale trans. Ein LTS wird dann durch
seine Transitionsrelation definiert.%
\end{isamarkuptext}\isamarkuptrue%
\isacommand{locale}\isamarkupfalse%
\ lts\ {\isacharequal}{\kern0pt}\ \isanewline
\ \ \isakeyword{fixes}\ tran\ {\isacharcolon}{\kern0pt}{\isacharcolon}{\kern0pt}\ {\isacartoucheopen}{\isacharprime}{\kern0pt}s\ {\isasymRightarrow}\ {\isacharprime}{\kern0pt}a\ {\isasymRightarrow}\ {\isacharprime}{\kern0pt}s\ {\isasymRightarrow}\ bool{\isacartoucheclose}\isanewline
\ \ \ \ {\isacharparenleft}{\kern0pt}{\isachardoublequoteopen}{\isacharunderscore}{\kern0pt}\ {\isasymmapsto}{\isacharunderscore}{\kern0pt}\ {\isacharunderscore}{\kern0pt}{\isachardoublequoteclose}\ {\isacharbrackleft}{\kern0pt}{\isadigit{7}}{\isadigit{0}}{\isacharcomma}{\kern0pt}\ {\isadigit{7}}{\isadigit{0}}{\isacharcomma}{\kern0pt}\ {\isadigit{7}}{\isadigit{0}}{\isacharbrackright}{\kern0pt}\ {\isadigit{8}}{\isadigit{0}}{\isacharparenright}{\kern0pt}\isanewline
\isakeyword{begin}\isanewline
\isanewline
\isacommand{abbreviation}\isamarkupfalse%
\ derivatives\ {\isacharcolon}{\kern0pt}{\isacharcolon}{\kern0pt}\ {\isacartoucheopen}{\isacharprime}{\kern0pt}s\ {\isasymRightarrow}\ {\isacharprime}{\kern0pt}a\ {\isasymRightarrow}\ {\isacharprime}{\kern0pt}s\ set{\isacartoucheclose}\isanewline
\ \ \isakeyword{where}\isanewline
{\isacartoucheopen}derivatives\ p\ {\isasymalpha}\ {\isasymequiv}\ {\isacharbraceleft}{\kern0pt}p{\isacharprime}{\kern0pt}{\isachardot}{\kern0pt}\ p\ {\isasymmapsto}{\isasymalpha}\ p{\isacharprime}{\kern0pt}{\isacharbraceright}{\kern0pt}{\isacartoucheclose}%
\begin{isamarkuptext}%
Transition System is image-finite%
\end{isamarkuptext}\isamarkuptrue%
\isacommand{definition}\isamarkupfalse%
\ image{\isacharunderscore}{\kern0pt}finite\ \isakeyword{where}\isanewline
{\isacartoucheopen}image{\isacharunderscore}{\kern0pt}finite\ {\isasymequiv}\ {\isacharparenleft}{\kern0pt}{\isasymforall}p\ {\isasymalpha}{\isachardot}{\kern0pt}\ finite\ {\isacharparenleft}{\kern0pt}derivatives\ p\ {\isasymalpha}{\isacharparenright}{\kern0pt}{\isacharparenright}{\kern0pt}{\isacartoucheclose}\isanewline
\isanewline
\isacommand{definition}\isamarkupfalse%
\ image{\isacharunderscore}{\kern0pt}countable\ {\isacharcolon}{\kern0pt}{\isacharcolon}{\kern0pt}\ {\isacartoucheopen}bool{\isacartoucheclose}\isanewline
\ \ \isakeyword{where}\ {\isacartoucheopen}image{\isacharunderscore}{\kern0pt}countable\ {\isasymequiv}\ {\isacharparenleft}{\kern0pt}{\isasymforall}\ p\ {\isasymalpha}{\isachardot}{\kern0pt}\ countable\ {\isacharparenleft}{\kern0pt}derivatives\ p\ {\isasymalpha}{\isacharparenright}{\kern0pt}{\isacharparenright}{\kern0pt}{\isacartoucheclose}%
\begin{isamarkuptext}%
stimmt definition? definition benötigt nach umstieg auf sets?%
\end{isamarkuptext}\isamarkuptrue%
\isacommand{definition}\isamarkupfalse%
\ lts{\isacharunderscore}{\kern0pt}finite\ \isakeyword{where}\isanewline
{\isacartoucheopen}lts{\isacharunderscore}{\kern0pt}finite\ {\isasymequiv}\ {\isacharparenleft}{\kern0pt}finite\ {\isacharparenleft}{\kern0pt}UNIV\ {\isacharcolon}{\kern0pt}{\isacharcolon}{\kern0pt}\ {\isacharprime}{\kern0pt}s\ set{\isacharparenright}{\kern0pt}{\isacharparenright}{\kern0pt}{\isacartoucheclose}\isanewline
\isanewline
\isacommand{abbreviation}\isamarkupfalse%
\ initial{\isacharunderscore}{\kern0pt}actions{\isacharcolon}{\kern0pt}{\isacharcolon}{\kern0pt}\ {\isacartoucheopen}{\isacharprime}{\kern0pt}s\ {\isasymRightarrow}\ {\isacharprime}{\kern0pt}a\ set{\isacartoucheclose}\isanewline
\ \ \isakeyword{where}\isanewline
{\isacartoucheopen}initial{\isacharunderscore}{\kern0pt}actions\ p\ {\isasymequiv}\ {\isacharbraceleft}{\kern0pt}{\isasymalpha}{\isacharbar}{\kern0pt}{\isasymalpha}{\isachardot}{\kern0pt}\ {\isacharparenleft}{\kern0pt}{\isasymexists}p{\isacharprime}{\kern0pt}{\isachardot}{\kern0pt}\ p\ {\isasymmapsto}{\isasymalpha}\ p{\isacharprime}{\kern0pt}{\isacharparenright}{\kern0pt}{\isacharbraceright}{\kern0pt}{\isacartoucheclose}\isanewline
\isanewline
\isacommand{abbreviation}\isamarkupfalse%
\ deadlock\ {\isacharcolon}{\kern0pt}{\isacharcolon}{\kern0pt}\ {\isacartoucheopen}{\isacharprime}{\kern0pt}s\ {\isasymRightarrow}\ bool{\isacartoucheclose}\ \isakeyword{where}\isanewline
{\isacartoucheopen}deadlock\ p\ {\isasymequiv}\ {\isacharparenleft}{\kern0pt}{\isasymforall}a{\isachardot}{\kern0pt}\ derivatives\ p\ a\ {\isacharequal}{\kern0pt}\ {\isacharbraceleft}{\kern0pt}{\isacharbraceright}{\kern0pt}{\isacharparenright}{\kern0pt}{\isacartoucheclose}%
\begin{isamarkuptext}%
nötig?%
\end{isamarkuptext}\isamarkuptrue%
\isacommand{abbreviation}\isamarkupfalse%
\ relevant{\isacharunderscore}{\kern0pt}actions\ {\isacharcolon}{\kern0pt}{\isacharcolon}{\kern0pt}\ {\isacartoucheopen}{\isacharprime}{\kern0pt}a\ set{\isacartoucheclose}\isanewline
\ \ \isakeyword{where}\isanewline
{\isacartoucheopen}relevant{\isacharunderscore}{\kern0pt}actions\ {\isasymequiv}\ {\isacharbraceleft}{\kern0pt}a{\isachardot}{\kern0pt}\ {\isasymexists}p\ p{\isacharprime}{\kern0pt}{\isachardot}{\kern0pt}\ p\ {\isasymmapsto}a\ p{\isacharprime}{\kern0pt}{\isacharbraceright}{\kern0pt}{\isacartoucheclose}\isanewline
\isanewline
\isacommand{inductive}\isamarkupfalse%
\ step{\isacharunderscore}{\kern0pt}sequence\ {\isacharcolon}{\kern0pt}{\isacharcolon}{\kern0pt}\ {\isacartoucheopen}{\isacharprime}{\kern0pt}s\ {\isasymRightarrow}\ {\isacharprime}{\kern0pt}a\ list\ {\isasymRightarrow}\ {\isacharprime}{\kern0pt}s\ {\isasymRightarrow}\ bool{\isacartoucheclose}\ {\isacharparenleft}{\kern0pt}{\isacartoucheopen}{\isacharunderscore}{\kern0pt}\ {\isasymmapsto}{\isachardollar}{\kern0pt}\ {\isacharunderscore}{\kern0pt}\ {\isacharunderscore}{\kern0pt}{\isacartoucheclose}{\isacharbrackleft}{\kern0pt}{\isadigit{7}}{\isadigit{0}}{\isacharcomma}{\kern0pt}{\isadigit{7}}{\isadigit{0}}{\isacharcomma}{\kern0pt}{\isadigit{7}}{\isadigit{0}}{\isacharbrackright}{\kern0pt}\ {\isadigit{8}}{\isadigit{0}}{\isacharparenright}{\kern0pt}\ \isakeyword{where}\isanewline
{\isacartoucheopen}p\ {\isasymmapsto}{\isachardollar}{\kern0pt}\ {\isacharbrackleft}{\kern0pt}{\isacharbrackright}{\kern0pt}\ p{\isacartoucheclose}\ {\isacharbar}{\kern0pt}\isanewline
{\isacartoucheopen}p\ {\isasymmapsto}{\isachardollar}{\kern0pt}\ {\isacharparenleft}{\kern0pt}a{\isacharhash}{\kern0pt}rt{\isacharparenright}{\kern0pt}\ p{\isacharprime}{\kern0pt}{\isacharprime}{\kern0pt}{\isacartoucheclose}\ \isakeyword{if}\ {\isacartoucheopen}{\isasymexists}p{\isacharprime}{\kern0pt}{\isachardot}{\kern0pt}\ p\ {\isasymmapsto}\ a\ p{\isacharprime}{\kern0pt}\ {\isasymand}\ p{\isacharprime}{\kern0pt}\ {\isasymmapsto}{\isachardollar}{\kern0pt}\ rt\ p{\isacharprime}{\kern0pt}{\isacharprime}{\kern0pt}{\isacartoucheclose}\isanewline
\isanewline
\isacommand{end}\isamarkupfalse%
\isanewline
%
\isadelimtheory
%
\endisadelimtheory
%
\isatagtheory
%
\endisatagtheory
{\isafoldtheory}%
%
\isadelimtheory
%
\endisadelimtheory
%
\end{isabellebody}%
\endinput
%:%file=~/Documents/Isabelle_HOL/Transition_Systems.thy%:%
%:%24=8%:%
%:%36=9%:%
%:%45=11%:%
%:%57=13%:%
%:%58=14%:%
%:%59=15%:%
%:%60=16%:%
%:%61=17%:%
%:%62=18%:%
%:%63=19%:%
%:%64=20%:%
%:%65=21%:%
%:%66=22%:%
%:%75=25%:%
%:%87=27%:%
%:%91=31%:%
%:%92=32%:%
%:%93=33%:%
%:%94=34%:%
%:%95=35%:%
%:%96=36%:%
%:%97=37%:%
%:%106=39%:%
%:%118=41%:%
%:%119=42%:%
%:%121=43%:%
%:%122=43%:%
%:%123=44%:%
%:%124=45%:%
%:%125=46%:%
%:%126=47%:%
%:%127=48%:%
%:%128=48%:%
%:%129=49%:%
%:%130=50%:%
%:%132=52%:%
%:%134=54%:%
%:%135=54%:%
%:%136=55%:%
%:%137=56%:%
%:%138=57%:%
%:%139=57%:%
%:%140=58%:%
%:%142=60%:%
%:%144=61%:%
%:%145=61%:%
%:%146=62%:%
%:%147=63%:%
%:%148=64%:%
%:%149=64%:%
%:%150=65%:%
%:%151=66%:%
%:%152=67%:%
%:%153=68%:%
%:%154=68%:%
%:%155=69%:%
%:%157=71%:%
%:%159=72%:%
%:%160=72%:%
%:%161=73%:%
%:%162=74%:%
%:%163=75%:%
%:%164=76%:%
%:%165=76%:%
%:%166=77%:%
%:%167=78%:%
%:%168=79%:%
%:%169=80%:%
%:%170=80%:%
