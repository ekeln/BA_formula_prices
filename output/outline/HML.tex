%
\begin{isabellebody}%
\setisabellecontext{HML}%
%
\isadelimtheory
%
\endisadelimtheory
%
\isatagtheory
%
\endisatagtheory
{\isafoldtheory}%
%
\isadelimtheory
%
\endisadelimtheory
%
\isadelimdocument
%
\endisadelimdocument
%
\isatagdocument
%
\isamarkupsection{Hennessy--Milner logic%
}
\isamarkuptrue%
%
\endisatagdocument
{\isafolddocument}%
%
\isadelimdocument
%
\endisadelimdocument
%
\begin{isamarkuptext}%
For the purpose of this thesis, we focus on the modal-logical characterizations of equivalences, using Hennessy--Milner logic (HML). 
First introduced by Matthew Hennessy and Robin Milner (citation), HML is a modal logic for expressing properties of systems described by LTS.
Intuitively, HML describes observations on an LTS and two processes are considered equivalent under HML when there exists no observation that distinguishes between them.
(citation) defined the modal-logical language as consisting of (finite) conjunctions, negations and a (modal) possibility operator:
$$\varphi ::= t\!t \mid \varphi_1 \;\wedge\; \varphi_2 \mid \neg\varphi \mid \langle\alpha\rangle\varphi$$
(where $\alpha$ ranges over the set of actions. The paper also proves that this characterization of strong bisimilarity
corresponds to a relational definition that is effectively the same as in (...). Their result can be expressed as follows:
for image-finite LTSs, two processes are strongly bisimilar iff they satisfy the same set of HML formulas. We call this the modal characterisation of
strong bisimilarity. By allowing for conjunction of arbitrary width (infinitary HML), the modal characterization of strong bisimilarity can be proved for arbitrary LTS. This is done in (...)%
\end{isamarkuptext}\isamarkuptrue%
%
\begin{isamarkuptext}%
Mention: HML to capture equivalences (Spectrum, HM theorem)%
\end{isamarkuptext}\isamarkuptrue%
%
\isadelimdocument
%
\endisadelimdocument
%
\isatagdocument
%
\isamarkupsubsubsection{Hennessy--Milner logic%
}
\isamarkuptrue%
%
\endisatagdocument
{\isafolddocument}%
%
\isadelimdocument
%
\endisadelimdocument
%
\begin{isamarkuptext}%
The syntax of Hennessy--Milner logic over a set $\Sigma$ of actions, (HML) - richtige font!!!!![$\Sigma$], is defined by the grammar:
\begin{align*}
    \varphi &::= \langle a \rangle \varphi && \text{with } a \in \Sigma \\
            &\quad | \, \bigwedge_{i \in I} \psi_i \\
    \psi &::= \neg \varphi \, | \, \varphi.
\end{align*}%
\end{isamarkuptext}\isamarkuptrue%
%
\begin{isamarkuptext}%
The data type \isa{{\isacharparenleft}{\kern0pt}{\isacharprime}{\kern0pt}a{\isacharcomma}{\kern0pt}\ {\isacharprime}{\kern0pt}i{\isacharparenright}{\kern0pt}hml} formalizes the definition of HML formulas above. It is parameterized by the type of actions \isa{{\isacharprime}{\kern0pt}a} for $\Sigma$
and an index type \isa{{\isacharprime}{\kern0pt}i}. We use an index sets of arbitrary type \isa{I\ {\isacharcolon}{\kern0pt}{\isacharcolon}{\kern0pt}\ {\isacharprime}{\kern0pt}i\ set} and a mapping \isa{F\ {\isacharcolon}{\kern0pt}{\isacharcolon}{\kern0pt}\ {\isacharprime}{\kern0pt}i\ {\isasymRightarrow}\ {\isacharparenleft}{\kern0pt}{\isacharprime}{\kern0pt}a{\isacharcomma}{\kern0pt}\ {\isacharprime}{\kern0pt}i{\isacharparenright}{\kern0pt}\ hml} to formalize
conjunctions so that each element of \isa{I} is mapped to a formula%
\footnote{Note that the formalization via an arbitrary set (...) does not yield a valid type, since \isa{set} is not a bounded natural functor.}%.%
\end{isamarkuptext}\isamarkuptrue%
\isacommand{datatype}\isamarkupfalse%
\ {\isacharparenleft}{\kern0pt}{\isacharprime}{\kern0pt}a{\isacharcomma}{\kern0pt}\ {\isacharprime}{\kern0pt}i{\isacharparenright}{\kern0pt}hml\ {\isacharequal}{\kern0pt}\isanewline
TT\ {\isacharbar}{\kern0pt}\isanewline
hml{\isacharunderscore}{\kern0pt}pos\ {\isacartoucheopen}{\isacharprime}{\kern0pt}a{\isacartoucheclose}\ {\isacartoucheopen}{\isacharparenleft}{\kern0pt}{\isacharprime}{\kern0pt}a{\isacharcomma}{\kern0pt}\ {\isacharprime}{\kern0pt}i{\isacharparenright}{\kern0pt}hml{\isacartoucheclose}\ {\isacharbar}{\kern0pt}\isanewline
hml{\isacharunderscore}{\kern0pt}conj\ {\isachardoublequoteopen}{\isacharprime}{\kern0pt}i\ set{\isachardoublequoteclose}\ {\isachardoublequoteopen}{\isacharprime}{\kern0pt}i\ set{\isachardoublequoteclose}\ {\isachardoublequoteopen}{\isacharprime}{\kern0pt}i\ {\isasymRightarrow}\ {\isacharparenleft}{\kern0pt}{\isacharprime}{\kern0pt}a{\isacharcomma}{\kern0pt}\ {\isacharprime}{\kern0pt}i{\isacharparenright}{\kern0pt}\ hml{\isachardoublequoteclose}%
\begin{isamarkuptext}%
Note that in canonical definitions of HML \isa{TT} is not usually part of the syntax,
but is instead synonymous to \isa{{\isasymAnd}{\isacharbraceleft}{\kern0pt}{\isacharbraceright}{\kern0pt}}.
We include \isa{TT} in the definition to enable Isabelle to infer that the type \isa{hml} is not empty..
This formalization allows for conjunctions of arbitrary - even of infinite - width and has been
taken from \cite{Pohlmann2021ReducingReactive} (Appendix B).%
\end{isamarkuptext}\isamarkuptrue%
\isacommand{context}\isamarkupfalse%
\ lts\ \isakeyword{begin}\isanewline
\isanewline
\isacommand{primrec}\isamarkupfalse%
\ hml{\isacharunderscore}{\kern0pt}semantics\ {\isacharcolon}{\kern0pt}{\isacharcolon}{\kern0pt}\ {\isacartoucheopen}{\isacharprime}{\kern0pt}s\ {\isasymRightarrow}\ {\isacharparenleft}{\kern0pt}{\isacharprime}{\kern0pt}a{\isacharcomma}{\kern0pt}\ {\isacharprime}{\kern0pt}s{\isacharparenright}{\kern0pt}hml\ {\isasymRightarrow}\ bool{\isacartoucheclose}\isanewline
{\isacharparenleft}{\kern0pt}{\isacartoucheopen}{\isacharunderscore}{\kern0pt}\ {\isasymTurnstile}\ {\isacharunderscore}{\kern0pt}{\isacartoucheclose}\ {\isacharbrackleft}{\kern0pt}{\isadigit{5}}{\isadigit{0}}{\isacharcomma}{\kern0pt}\ {\isadigit{5}}{\isadigit{0}}{\isacharbrackright}{\kern0pt}\ {\isadigit{5}}{\isadigit{0}}{\isacharparenright}{\kern0pt}\isanewline
\isakeyword{where}\isanewline
hml{\isacharunderscore}{\kern0pt}sem{\isacharunderscore}{\kern0pt}tt{\isacharcolon}{\kern0pt}\ {\isacartoucheopen}{\isacharparenleft}{\kern0pt}{\isacharunderscore}{\kern0pt}\ {\isasymTurnstile}\ TT{\isacharparenright}{\kern0pt}\ {\isacharequal}{\kern0pt}\ True{\isacartoucheclose}\ {\isacharbar}{\kern0pt}\isanewline
hml{\isacharunderscore}{\kern0pt}sem{\isacharunderscore}{\kern0pt}pos{\isacharcolon}{\kern0pt}\ {\isacartoucheopen}{\isacharparenleft}{\kern0pt}p\ {\isasymTurnstile}\ {\isacharparenleft}{\kern0pt}hml{\isacharunderscore}{\kern0pt}pos\ {\isasymalpha}\ {\isasymphi}{\isacharparenright}{\kern0pt}{\isacharparenright}{\kern0pt}\ {\isacharequal}{\kern0pt}\ {\isacharparenleft}{\kern0pt}{\isasymexists}\ q{\isachardot}{\kern0pt}\ {\isacharparenleft}{\kern0pt}p\ {\isasymmapsto}{\isasymalpha}\ q{\isacharparenright}{\kern0pt}\ {\isasymand}\ q\ {\isasymTurnstile}\ {\isasymphi}{\isacharparenright}{\kern0pt}{\isacartoucheclose}\ {\isacharbar}{\kern0pt}\isanewline
hml{\isacharunderscore}{\kern0pt}sem{\isacharunderscore}{\kern0pt}conj{\isacharcolon}{\kern0pt}\ {\isacartoucheopen}{\isacharparenleft}{\kern0pt}p\ {\isasymTurnstile}\ {\isacharparenleft}{\kern0pt}hml{\isacharunderscore}{\kern0pt}conj\ I\ J\ {\isasympsi}s{\isacharparenright}{\kern0pt}{\isacharparenright}{\kern0pt}\ {\isacharequal}{\kern0pt}\ {\isacharparenleft}{\kern0pt}{\isacharparenleft}{\kern0pt}{\isasymforall}i\ {\isasymin}\ I{\isachardot}{\kern0pt}\ p\ {\isasymTurnstile}\ {\isacharparenleft}{\kern0pt}{\isasympsi}s\ i{\isacharparenright}{\kern0pt}{\isacharparenright}{\kern0pt}\ {\isasymand}\ {\isacharparenleft}{\kern0pt}{\isasymforall}j\ {\isasymin}\ J{\isachardot}{\kern0pt}\ {\isasymnot}{\isacharparenleft}{\kern0pt}p\ {\isasymTurnstile}\ {\isacharparenleft}{\kern0pt}{\isasympsi}s\ j{\isacharparenright}{\kern0pt}{\isacharparenright}{\kern0pt}{\isacharparenright}{\kern0pt}{\isacharparenright}{\kern0pt}{\isacartoucheclose}\isanewline
\isanewline
\isacommand{definition}\isamarkupfalse%
\ HML{\isacharunderscore}{\kern0pt}true\ \isakeyword{where}\isanewline
{\isachardoublequoteopen}HML{\isacharunderscore}{\kern0pt}true\ {\isasymphi}\ {\isasymequiv}\ {\isasymforall}s{\isachardot}{\kern0pt}\ s\ {\isasymTurnstile}\ {\isasymphi}{\isachardoublequoteclose}\isanewline
\isanewline
\isacommand{lemma}\isamarkupfalse%
\ \isanewline
\ \ \isakeyword{fixes}\ s{\isacharcolon}{\kern0pt}{\isacharcolon}{\kern0pt}{\isacharprime}{\kern0pt}s\isanewline
\ \ \isakeyword{assumes}\ {\isachardoublequoteopen}HML{\isacharunderscore}{\kern0pt}true\ {\isacharparenleft}{\kern0pt}hml{\isacharunderscore}{\kern0pt}conj\ I\ J\ {\isasymPhi}{\isacharparenright}{\kern0pt}{\isachardoublequoteclose}\isanewline
\ \ \isakeyword{shows}\ {\isachardoublequoteopen}{\isasymforall}{\isasymphi}\ {\isasymin}\ {\isasymPhi}\ {\isacharbackquote}{\kern0pt}\ I{\isachardot}{\kern0pt}\ HML{\isacharunderscore}{\kern0pt}true\ {\isasymphi}{\isachardoublequoteclose}\isanewline
%
\isadelimproof
\ \ %
\endisadelimproof
%
\isatagproof
\isacommand{using}\isamarkupfalse%
\ HML{\isacharunderscore}{\kern0pt}true{\isacharunderscore}{\kern0pt}def\ assms\ \isacommand{by}\isamarkupfalse%
\ auto%
\endisatagproof
{\isafoldproof}%
%
\isadelimproof
%
\endisadelimproof
%
\begin{isamarkuptext}%
Two states are HML equivalent if they satisfy the same formula.%
\end{isamarkuptext}\isamarkuptrue%
\isacommand{definition}\isamarkupfalse%
\ HML{\isacharunderscore}{\kern0pt}equivalent\ {\isacharcolon}{\kern0pt}{\isacharcolon}{\kern0pt}\ {\isacartoucheopen}{\isacharprime}{\kern0pt}s\ {\isasymRightarrow}\ {\isacharprime}{\kern0pt}s\ {\isasymRightarrow}\ bool{\isacartoucheclose}\ \isakeyword{where}\isanewline
\ \ {\isacartoucheopen}HML{\isacharunderscore}{\kern0pt}equivalent\ p\ q\ {\isasymequiv}\ {\isacharparenleft}{\kern0pt}{\isasymforall}\ {\isasymphi}{\isacharcolon}{\kern0pt}{\isacharcolon}{\kern0pt}{\isacharparenleft}{\kern0pt}{\isacharprime}{\kern0pt}a{\isacharcomma}{\kern0pt}\ {\isacharprime}{\kern0pt}s{\isacharparenright}{\kern0pt}\ hml{\isachardot}{\kern0pt}\ {\isacharparenleft}{\kern0pt}p\ {\isasymTurnstile}\ {\isasymphi}{\isacharparenright}{\kern0pt}\ {\isasymlongleftrightarrow}\ {\isacharparenleft}{\kern0pt}q\ {\isasymTurnstile}\ {\isasymphi}{\isacharparenright}{\kern0pt}{\isacharparenright}{\kern0pt}{\isacartoucheclose}%
\begin{isamarkuptext}%
An HML formula \isa{{\isasymphi}l} implies another (\isa{{\isasymphi}r}) if the fact that some process \isa{p} satisfies \isa{{\isasymphi}l}
implies that \isa{p} must also satisfy \isa{{\isasymphi}r}, no matter the process \isa{p}.%
\end{isamarkuptext}\isamarkuptrue%
\isacommand{definition}\isamarkupfalse%
\ hml{\isacharunderscore}{\kern0pt}impl\ {\isacharcolon}{\kern0pt}{\isacharcolon}{\kern0pt}\ {\isachardoublequoteopen}{\isacharparenleft}{\kern0pt}{\isacharprime}{\kern0pt}a{\isacharcomma}{\kern0pt}\ {\isacharprime}{\kern0pt}s{\isacharparenright}{\kern0pt}\ hml\ {\isasymRightarrow}\ {\isacharparenleft}{\kern0pt}{\isacharprime}{\kern0pt}a{\isacharcomma}{\kern0pt}\ {\isacharprime}{\kern0pt}s{\isacharparenright}{\kern0pt}\ hml\ {\isasymRightarrow}\ bool{\isachardoublequoteclose}\ {\isacharparenleft}{\kern0pt}\isakeyword{infix}\ {\isachardoublequoteopen}{\isasymRrightarrow}{\isachardoublequoteclose}\ {\isadigit{6}}{\isadigit{0}}{\isacharparenright}{\kern0pt}\ \ \isakeyword{where}\isanewline
\ \ {\isachardoublequoteopen}{\isasymphi}l\ {\isasymRrightarrow}\ {\isasymphi}r\ {\isasymequiv}\ {\isacharparenleft}{\kern0pt}{\isasymforall}p{\isachardot}{\kern0pt}\ {\isacharparenleft}{\kern0pt}p\ {\isasymTurnstile}\ {\isasymphi}l{\isacharparenright}{\kern0pt}\ {\isasymlongrightarrow}\ {\isacharparenleft}{\kern0pt}p\ {\isasymTurnstile}\ {\isasymphi}r{\isacharparenright}{\kern0pt}{\isacharparenright}{\kern0pt}{\isachardoublequoteclose}\isanewline
\isanewline
\isacommand{lemma}\isamarkupfalse%
\ hml{\isacharunderscore}{\kern0pt}impl{\isacharunderscore}{\kern0pt}iffI{\isacharcolon}{\kern0pt}\ {\isachardoublequoteopen}{\isasymphi}l\ {\isasymRrightarrow}\ {\isasymphi}r\ {\isacharequal}{\kern0pt}\ {\isacharparenleft}{\kern0pt}{\isasymforall}p{\isachardot}{\kern0pt}\ {\isacharparenleft}{\kern0pt}p\ {\isasymTurnstile}\ {\isasymphi}l{\isacharparenright}{\kern0pt}\ {\isasymlongrightarrow}\ {\isacharparenleft}{\kern0pt}p\ {\isasymTurnstile}\ {\isasymphi}r{\isacharparenright}{\kern0pt}{\isacharparenright}{\kern0pt}{\isachardoublequoteclose}\isanewline
%
\isadelimproof
\ \ %
\endisadelimproof
%
\isatagproof
\isacommand{using}\isamarkupfalse%
\ hml{\isacharunderscore}{\kern0pt}impl{\isacharunderscore}{\kern0pt}def\ \isacommand{by}\isamarkupfalse%
\ force%
\endisatagproof
{\isafoldproof}%
%
\isadelimproof
%
\endisadelimproof
%
\isadelimdocument
%
\endisadelimdocument
%
\isatagdocument
%
\isamarkupsubsection{Equivalence%
}
\isamarkuptrue%
%
\endisatagdocument
{\isafolddocument}%
%
\isadelimdocument
%
\endisadelimdocument
%
\begin{isamarkuptext}%
A HML formula \isa{{\isasymphi}l} is said to be equivalent to some other HML formula \isa{{\isasymphi}r} (written \isa{{\isasymphi}l\ {\isasymLleftarrow}{\isasymRrightarrow}\ {\isasymphi}r})
iff process \isa{p} satisfies \isa{{\isasymphi}l} iff it also satisfies \isa{{\isasymphi}r}, no matter the process \isa{p}.

We have chosen to define this equivalence by appealing to HML formula implication (c.f. pre-order).%
\end{isamarkuptext}\isamarkuptrue%
\isacommand{definition}\isamarkupfalse%
\ hml{\isacharunderscore}{\kern0pt}formula{\isacharunderscore}{\kern0pt}eq\ {\isacharcolon}{\kern0pt}{\isacharcolon}{\kern0pt}\ {\isachardoublequoteopen}{\isacharparenleft}{\kern0pt}{\isacharprime}{\kern0pt}a{\isacharcomma}{\kern0pt}\ {\isacharprime}{\kern0pt}s{\isacharparenright}{\kern0pt}\ hml\ {\isasymRightarrow}\ {\isacharparenleft}{\kern0pt}{\isacharprime}{\kern0pt}a{\isacharcomma}{\kern0pt}\ {\isacharprime}{\kern0pt}s{\isacharparenright}{\kern0pt}\ hml\ {\isasymRightarrow}\ bool{\isachardoublequoteclose}\ {\isacharparenleft}{\kern0pt}\isakeyword{infix}\ {\isachardoublequoteopen}{\isasymLleftarrow}{\isasymRrightarrow}{\isachardoublequoteclose}\ {\isadigit{6}}{\isadigit{0}}{\isacharparenright}{\kern0pt}\ \ \isakeyword{where}\isanewline
\ \ {\isachardoublequoteopen}{\isasymphi}l\ {\isasymLleftarrow}{\isasymRrightarrow}\ {\isasymphi}r\ {\isasymequiv}\ {\isasymphi}l\ {\isasymRrightarrow}\ {\isasymphi}r\ {\isasymand}\ {\isasymphi}r\ {\isasymRrightarrow}\ {\isasymphi}l{\isachardoublequoteclose}%
\begin{isamarkuptext}%
\isa{{\isasymLleftarrow}{\isasymRrightarrow}} is truly an equivalence relation.%
\end{isamarkuptext}\isamarkuptrue%
\isacommand{lemma}\isamarkupfalse%
\ hml{\isacharunderscore}{\kern0pt}eq{\isacharunderscore}{\kern0pt}equiv{\isacharcolon}{\kern0pt}\ {\isachardoublequoteopen}equivp\ {\isacharparenleft}{\kern0pt}{\isasymLleftarrow}{\isasymRrightarrow}{\isacharparenright}{\kern0pt}{\isachardoublequoteclose}\isanewline
%
\isadelimproof
\ \ %
\endisadelimproof
%
\isatagproof
\isacommand{by}\isamarkupfalse%
\ {\isacharparenleft}{\kern0pt}smt\ {\isacharparenleft}{\kern0pt}verit{\isacharcomma}{\kern0pt}\ del{\isacharunderscore}{\kern0pt}insts{\isacharparenright}{\kern0pt}\ equivpI\ hml{\isacharunderscore}{\kern0pt}formula{\isacharunderscore}{\kern0pt}eq{\isacharunderscore}{\kern0pt}def\ hml{\isacharunderscore}{\kern0pt}impl{\isacharunderscore}{\kern0pt}def\ reflpI\ sympI\ transpI{\isacharparenright}{\kern0pt}%
\endisatagproof
{\isafoldproof}%
%
\isadelimproof
\isanewline
%
\endisadelimproof
\isanewline
\isacommand{lemma}\isamarkupfalse%
\ equiv{\isacharunderscore}{\kern0pt}der{\isacharcolon}{\kern0pt}\isanewline
\ \ \isakeyword{assumes}\ {\isachardoublequoteopen}HML{\isacharunderscore}{\kern0pt}equivalent\ p\ q{\isachardoublequoteclose}\ {\isachardoublequoteopen}{\isasymexists}p{\isacharprime}{\kern0pt}{\isachardot}{\kern0pt}\ p\ {\isasymmapsto}{\isasymalpha}\ p{\isacharprime}{\kern0pt}{\isachardoublequoteclose}\isanewline
\ \ \isakeyword{shows}\ {\isachardoublequoteopen}{\isasymexists}p{\isacharprime}{\kern0pt}\ q{\isacharprime}{\kern0pt}{\isachardot}{\kern0pt}\ {\isacharparenleft}{\kern0pt}HML{\isacharunderscore}{\kern0pt}equivalent\ p{\isacharprime}{\kern0pt}\ q{\isacharprime}{\kern0pt}{\isacharparenright}{\kern0pt}\ {\isasymand}\ q\ {\isasymmapsto}{\isasymalpha}\ q{\isacharprime}{\kern0pt}{\isachardoublequoteclose}\isanewline
%
\isadelimproof
\ \ %
\endisadelimproof
%
\isatagproof
\isacommand{using}\isamarkupfalse%
\ assms\ hml{\isacharunderscore}{\kern0pt}semantics{\isachardot}{\kern0pt}simps\isanewline
\ \ \isacommand{unfolding}\isamarkupfalse%
\ HML{\isacharunderscore}{\kern0pt}equivalent{\isacharunderscore}{\kern0pt}def\ \isanewline
\ \ \isacommand{by}\isamarkupfalse%
\ metis%
\endisatagproof
{\isafoldproof}%
%
\isadelimproof
\isanewline
%
\endisadelimproof
\isanewline
\isanewline
\isacommand{lemma}\isamarkupfalse%
\ equiv{\isacharunderscore}{\kern0pt}trans{\isacharcolon}{\kern0pt}\ {\isachardoublequoteopen}transp\ HML{\isacharunderscore}{\kern0pt}equivalent{\isachardoublequoteclose}\isanewline
%
\isadelimproof
\ \ %
\endisadelimproof
%
\isatagproof
\isacommand{by}\isamarkupfalse%
\ {\isacharparenleft}{\kern0pt}simp\ add{\isacharcolon}{\kern0pt}\ HML{\isacharunderscore}{\kern0pt}equivalent{\isacharunderscore}{\kern0pt}def\ transp{\isacharunderscore}{\kern0pt}def{\isacharparenright}{\kern0pt}%
\endisatagproof
{\isafoldproof}%
%
\isadelimproof
%
\endisadelimproof
%
\begin{isamarkuptext}%
A formula distinguishes one state from another if its true for the
  first and false for the second.%
\end{isamarkuptext}\isamarkuptrue%
\isacommand{abbreviation}\isamarkupfalse%
\ distinguishes\ {\isacharcolon}{\kern0pt}{\isacharcolon}{\kern0pt}\ \ {\isacartoucheopen}{\isacharparenleft}{\kern0pt}{\isacharprime}{\kern0pt}a{\isacharcomma}{\kern0pt}\ {\isacharprime}{\kern0pt}s{\isacharparenright}{\kern0pt}\ hml\ {\isasymRightarrow}\ {\isacharprime}{\kern0pt}s\ {\isasymRightarrow}\ {\isacharprime}{\kern0pt}s\ {\isasymRightarrow}\ bool{\isacartoucheclose}\ \isakeyword{where}\isanewline
\ \ \ {\isacartoucheopen}distinguishes\ {\isasymphi}\ p\ q\ {\isasymequiv}\ p\ {\isasymTurnstile}\ {\isasymphi}\ {\isasymand}\ {\isasymnot}\ q\ {\isasymTurnstile}\ {\isasymphi}{\isacartoucheclose}\isanewline
\isanewline
\isacommand{lemma}\isamarkupfalse%
\ hml{\isacharunderscore}{\kern0pt}equiv{\isacharunderscore}{\kern0pt}sym{\isacharcolon}{\kern0pt}\isanewline
\ \ \isakeyword{shows}\ {\isacartoucheopen}symp\ HML{\isacharunderscore}{\kern0pt}equivalent{\isacartoucheclose}\isanewline
%
\isadelimproof
%
\endisadelimproof
%
\isatagproof
\isacommand{unfolding}\isamarkupfalse%
\ HML{\isacharunderscore}{\kern0pt}equivalent{\isacharunderscore}{\kern0pt}def\ symp{\isacharunderscore}{\kern0pt}def\ \isacommand{by}\isamarkupfalse%
\ simp%
\endisatagproof
{\isafoldproof}%
%
\isadelimproof
%
\endisadelimproof
%
\begin{isamarkuptext}%
If two states are not HML equivalent then there must be a
  distinguishing formula.%
\end{isamarkuptext}\isamarkuptrue%
\isacommand{lemma}\isamarkupfalse%
\ hml{\isacharunderscore}{\kern0pt}distinctions{\isacharcolon}{\kern0pt}\isanewline
\ \ \isakeyword{fixes}\ state{\isacharcolon}{\kern0pt}{\isacharcolon}{\kern0pt}{\isachardoublequoteopen}{\isacharprime}{\kern0pt}s{\isachardoublequoteclose}\isanewline
\ \ \isakeyword{assumes}\ {\isacartoucheopen}{\isasymnot}\ HML{\isacharunderscore}{\kern0pt}equivalent\ p\ q{\isacartoucheclose}\isanewline
\ \ \isakeyword{shows}\ {\isacartoucheopen}{\isasymexists}{\isasymphi}{\isachardot}{\kern0pt}\ distinguishes\ {\isasymphi}\ p\ q{\isacartoucheclose}\isanewline
%
\isadelimproof
%
\endisadelimproof
%
\isatagproof
\isacommand{proof}\isamarkupfalse%
{\isacharminus}{\kern0pt}\isanewline
\ \ \isacommand{from}\isamarkupfalse%
\ assms\ \isacommand{have}\isamarkupfalse%
\ {\isachardoublequoteopen}{\isasymnot}\ {\isacharparenleft}{\kern0pt}{\isasymforall}\ {\isasymphi}{\isacharcolon}{\kern0pt}{\isacharcolon}{\kern0pt}{\isacharparenleft}{\kern0pt}{\isacharprime}{\kern0pt}a{\isacharcomma}{\kern0pt}\ {\isacharprime}{\kern0pt}s{\isacharparenright}{\kern0pt}\ hml{\isachardot}{\kern0pt}\ {\isacharparenleft}{\kern0pt}p\ {\isasymTurnstile}\ {\isasymphi}{\isacharparenright}{\kern0pt}\ {\isasymlongleftrightarrow}\ {\isacharparenleft}{\kern0pt}q\ {\isasymTurnstile}\ {\isasymphi}{\isacharparenright}{\kern0pt}{\isacharparenright}{\kern0pt}{\isachardoublequoteclose}\ \isanewline
\ \ \ \ \isacommand{unfolding}\isamarkupfalse%
\ HML{\isacharunderscore}{\kern0pt}equivalent{\isacharunderscore}{\kern0pt}def\ \isacommand{by}\isamarkupfalse%
\ blast\isanewline
\ \ \isacommand{then}\isamarkupfalse%
\ \isacommand{obtain}\isamarkupfalse%
\ {\isasymphi}{\isacharcolon}{\kern0pt}{\isacharcolon}{\kern0pt}{\isachardoublequoteopen}{\isacharparenleft}{\kern0pt}{\isacharprime}{\kern0pt}a{\isacharcomma}{\kern0pt}\ {\isacharprime}{\kern0pt}s{\isacharparenright}{\kern0pt}\ hml{\isachardoublequoteclose}\ \isakeyword{where}\ {\isachardoublequoteopen}{\isacharparenleft}{\kern0pt}p\ {\isasymTurnstile}\ {\isasymphi}{\isacharparenright}{\kern0pt}\ {\isasymnoteq}\ {\isacharparenleft}{\kern0pt}q\ {\isasymTurnstile}\ {\isasymphi}{\isacharparenright}{\kern0pt}{\isachardoublequoteclose}\ \isacommand{by}\isamarkupfalse%
\ blast\isanewline
\ \ \isacommand{then}\isamarkupfalse%
\ \isacommand{have}\isamarkupfalse%
\ {\isachardoublequoteopen}{\isacharparenleft}{\kern0pt}{\isacharparenleft}{\kern0pt}p\ {\isasymTurnstile}\ {\isasymphi}{\isacharparenright}{\kern0pt}\ {\isasymand}\ {\isasymnot}{\isacharparenleft}{\kern0pt}q\ {\isasymTurnstile}\ {\isasymphi}{\isacharparenright}{\kern0pt}{\isacharparenright}{\kern0pt}\ {\isasymor}\ {\isacharparenleft}{\kern0pt}{\isasymnot}{\isacharparenleft}{\kern0pt}p\ {\isasymTurnstile}\ {\isasymphi}{\isacharparenright}{\kern0pt}\ {\isasymand}\ {\isacharparenleft}{\kern0pt}q\ {\isasymTurnstile}\ {\isasymphi}{\isacharparenright}{\kern0pt}{\isacharparenright}{\kern0pt}{\isachardoublequoteclose}\isanewline
\ \ \ \ \isacommand{by}\isamarkupfalse%
\ blast\isanewline
\ \ \isacommand{then}\isamarkupfalse%
\ \isacommand{show}\isamarkupfalse%
\ {\isacharquery}{\kern0pt}thesis\isanewline
\ \ \isacommand{proof}\isamarkupfalse%
\isanewline
\ \ \ \ \isacommand{show}\isamarkupfalse%
\ {\isachardoublequoteopen}distinguishes\ {\isasymphi}\ p\ q\ {\isasymLongrightarrow}\ {\isasymexists}{\isasymphi}{\isachardot}{\kern0pt}\ distinguishes\ {\isasymphi}\ p\ q{\isachardoublequoteclose}\ \isacommand{by}\isamarkupfalse%
\ blast\isanewline
\ \ \isacommand{next}\isamarkupfalse%
\isanewline
\ \ \ \ \isacommand{assume}\isamarkupfalse%
\ assm{\isacharcolon}{\kern0pt}\ {\isachardoublequoteopen}{\isasymnot}\ p\ {\isasymTurnstile}\ {\isasymphi}\ {\isasymand}\ q\ {\isasymTurnstile}\ {\isasymphi}{\isachardoublequoteclose}\isanewline
\ \ \ \ \isacommand{define}\isamarkupfalse%
\ n{\isasymphi}\ \isakeyword{where}\ {\isachardoublequoteopen}n{\isasymphi}\ {\isasymequiv}{\isacharparenleft}{\kern0pt}hml{\isacharunderscore}{\kern0pt}conj\ {\isacharparenleft}{\kern0pt}{\isacharbraceleft}{\kern0pt}{\isacharbraceright}{\kern0pt}{\isacharcolon}{\kern0pt}{\isacharcolon}{\kern0pt}{\isacharprime}{\kern0pt}s\ set{\isacharparenright}{\kern0pt}\ \isanewline
\ \ \ \ \ \ \ \ \ \ \ \ \ \ \ \ \ \ \ \ \ \ \ \ \ \ {\isacharbraceleft}{\kern0pt}state{\isacharbraceright}{\kern0pt}\ \isanewline
\ \ \ \ \ \ \ \ \ \ \ \ \ \ \ \ \ \ \ \ \ \ \ \ \ \ {\isacharparenleft}{\kern0pt}{\isasymlambda}j{\isachardot}{\kern0pt}\ if\ j\ {\isacharequal}{\kern0pt}\ state\ then\ {\isasymphi}\ else\ undefined{\isacharparenright}{\kern0pt}{\isacharparenright}{\kern0pt}{\isachardoublequoteclose}\isanewline
\ \ \ \ \isacommand{have}\isamarkupfalse%
\ {\isachardoublequoteopen}p\ {\isasymTurnstile}\ n{\isasymphi}\ {\isasymand}\ {\isasymnot}\ q\ {\isasymTurnstile}\ n{\isasymphi}{\isachardoublequoteclose}\ \isanewline
\ \ \ \ \ \ \isacommand{unfolding}\isamarkupfalse%
\ n{\isasymphi}{\isacharunderscore}{\kern0pt}def\isanewline
\ \ \ \ \ \ \isacommand{using}\isamarkupfalse%
\ hml{\isacharunderscore}{\kern0pt}semantics{\isachardot}{\kern0pt}simps\ assm\isanewline
\ \ \ \ \ \ \isacommand{by}\isamarkupfalse%
\ force\isanewline
\ \ \ \ \isacommand{then}\isamarkupfalse%
\ \isacommand{show}\isamarkupfalse%
\ {\isacharquery}{\kern0pt}thesis\isanewline
\ \ \ \ \ \ \isacommand{by}\isamarkupfalse%
\ blast\isanewline
\ \ \isacommand{qed}\isamarkupfalse%
\isanewline
\isacommand{qed}\isamarkupfalse%
%
\endisatagproof
{\isafoldproof}%
%
\isadelimproof
\isanewline
%
\endisadelimproof
\isanewline
\isacommand{end}\isamarkupfalse%
\ \isanewline
%
\isadelimtheory
%
\endisadelimtheory
%
\isatagtheory
\isacommand{end}\isamarkupfalse%
%
\endisatagtheory
{\isafoldtheory}%
%
\isadelimtheory
%
\endisadelimtheory
%
\end{isabellebody}%
\endinput
%:%file=~/Documents/Isabelle_HOL/HML.thy%:%
%:%24=7%:%
%:%36=8%:%
%:%37=9%:%
%:%38=10%:%
%:%39=11%:%
%:%40=12%:%
%:%41=13%:%
%:%42=14%:%
%:%43=15%:%
%:%44=16%:%
%:%48=19%:%
%:%57=20%:%
%:%69=21%:%
%:%70=22%:%
%:%71=23%:%
%:%72=24%:%
%:%73=25%:%
%:%74=26%:%
%:%78=28%:%
%:%79=29%:%
%:%80=30%:%
%:%81=31%:%
%:%83=34%:%
%:%84=34%:%
%:%85=35%:%
%:%86=36%:%
%:%87=37%:%
%:%89=39%:%
%:%90=40%:%
%:%91=41%:%
%:%92=42%:%
%:%93=43%:%
%:%95=45%:%
%:%96=45%:%
%:%97=46%:%
%:%98=47%:%
%:%99=47%:%
%:%100=48%:%
%:%101=49%:%
%:%102=50%:%
%:%103=51%:%
%:%104=52%:%
%:%105=53%:%
%:%106=54%:%
%:%107=54%:%
%:%108=55%:%
%:%109=56%:%
%:%110=57%:%
%:%111=57%:%
%:%112=58%:%
%:%113=59%:%
%:%114=60%:%
%:%117=61%:%
%:%121=61%:%
%:%122=61%:%
%:%123=61%:%
%:%132=63%:%
%:%134=64%:%
%:%135=64%:%
%:%136=65%:%
%:%138=67%:%
%:%139=68%:%
%:%141=70%:%
%:%142=70%:%
%:%143=71%:%
%:%144=72%:%
%:%145=73%:%
%:%146=73%:%
%:%149=74%:%
%:%153=74%:%
%:%154=74%:%
%:%155=74%:%
%:%169=76%:%
%:%181=79%:%
%:%182=80%:%
%:%183=81%:%
%:%184=82%:%
%:%186=85%:%
%:%187=85%:%
%:%188=86%:%
%:%190=88%:%
%:%192=89%:%
%:%193=89%:%
%:%196=90%:%
%:%200=90%:%
%:%201=90%:%
%:%206=90%:%
%:%209=91%:%
%:%210=92%:%
%:%211=92%:%
%:%212=93%:%
%:%213=94%:%
%:%216=95%:%
%:%220=95%:%
%:%221=95%:%
%:%222=96%:%
%:%223=96%:%
%:%224=97%:%
%:%225=97%:%
%:%230=97%:%
%:%233=98%:%
%:%234=99%:%
%:%235=100%:%
%:%236=100%:%
%:%239=101%:%
%:%243=101%:%
%:%244=101%:%
%:%253=104%:%
%:%254=105%:%
%:%256=107%:%
%:%257=107%:%
%:%258=108%:%
%:%259=109%:%
%:%260=110%:%
%:%261=110%:%
%:%262=111%:%
%:%269=112%:%
%:%270=112%:%
%:%271=112%:%
%:%280=115%:%
%:%281=116%:%
%:%283=119%:%
%:%284=119%:%
%:%285=120%:%
%:%286=121%:%
%:%287=122%:%
%:%294=123%:%
%:%295=123%:%
%:%296=124%:%
%:%297=124%:%
%:%298=124%:%
%:%299=125%:%
%:%300=125%:%
%:%301=125%:%
%:%302=126%:%
%:%303=126%:%
%:%304=126%:%
%:%305=126%:%
%:%306=127%:%
%:%307=127%:%
%:%308=127%:%
%:%309=128%:%
%:%310=128%:%
%:%311=129%:%
%:%312=129%:%
%:%313=129%:%
%:%314=130%:%
%:%315=130%:%
%:%316=131%:%
%:%317=131%:%
%:%318=131%:%
%:%319=132%:%
%:%320=132%:%
%:%321=133%:%
%:%322=133%:%
%:%323=134%:%
%:%324=134%:%
%:%326=136%:%
%:%327=137%:%
%:%328=137%:%
%:%329=138%:%
%:%330=138%:%
%:%331=139%:%
%:%332=139%:%
%:%333=140%:%
%:%334=140%:%
%:%335=141%:%
%:%336=141%:%
%:%337=141%:%
%:%338=142%:%
%:%339=142%:%
%:%340=143%:%
%:%341=143%:%
%:%342=144%:%
%:%348=144%:%
%:%351=145%:%
%:%352=146%:%
%:%353=146%:%
%:%360=147%:%
