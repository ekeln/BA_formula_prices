%
\begin{isabellebody}%
\setisabellecontext{Introduction}%
%
\isadelimtheory
%
\endisadelimtheory
%
\isatagtheory
%
\endisatagtheory
{\isafoldtheory}%
%
\isadelimtheory
%
\endisadelimtheory
%
\isadelimdocument
%
\endisadelimdocument
%
\isatagdocument
%
\isamarkupchapter{Introduction%
}
\isamarkuptrue%
%
\endisatagdocument
{\isafolddocument}%
%
\isadelimdocument
%
\endisadelimdocument
%
\begin{isamarkuptext}%
In this thesis, I show the corrspondence between various equivalences popular in the reactive
systems community and and coordinates of a price function, as introduced by Benjamin Bisping (citation).
I formalised the concepts and proofs discussed in this thesis in the interactive proof assistant Isabelle (citation).

- Reactive Systmes \\
  - what are they \\
  - modelling (via lts etc) \\
  - Semantics of resysts \\
    - Verification \\
    - different notions of equivalence (because of nondeteminism?) -> van glabbeek \\
    - Different definitions of semantics -> HML/relational/... \\
 --> linear-time--branching-time spectrum understood through properties of HML \\
  --> capture expressiveness capabilities of HML formulas via a function \\
--> Contribution o Paper: The in (citation) introduced expressiveness function 
and its coordinates captures the linear time branching time spectrum.. \\
- Isabelle:\\
  - formalization of concepts, proofs \\
  - what is isabelle \\
  - difference between mathematical concepts and their implementation? \\%
\end{isamarkuptext}\isamarkuptrue%
%
\isadelimtheory
%
\endisadelimtheory
%
\isatagtheory
%
\endisatagtheory
{\isafoldtheory}%
%
\isadelimtheory
%
\endisadelimtheory
%
\end{isabellebody}%
\endinput
%:%file=~/Documents/Isabelle_HOL/Introduction.thy%:%
%:%24=8%:%
%:%36=9%:%
%:%37=10%:%
%:%38=11%:%
%:%39=12%:%
%:%40=13%:%
%:%41=14%:%
%:%42=15%:%
%:%43=16%:%
%:%44=17%:%
%:%45=18%:%
%:%46=19%:%
%:%47=20%:%
%:%48=21%:%
%:%49=22%:%
%:%50=23%:%
%:%51=24%:%
%:%52=25%:%
%:%53=26%:%
%:%54=27%:%
