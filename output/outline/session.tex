%
\begin{isabellebody}%
\setisabellecontext{Introduction}%
%
\isadelimtheory
%
\endisadelimtheory
%
\isatagtheory
%
\endisatagtheory
{\isafoldtheory}%
%
\isadelimtheory
%
\endisadelimtheory
%
\isadelimdocument
%
\endisadelimdocument
%
\isatagdocument
%
\isamarkupchapter{Introduction%
}
\isamarkuptrue%
%
\endisatagdocument
{\isafolddocument}%
%
\isadelimdocument
%
\endisadelimdocument
%
\begin{isamarkuptext}%
In this thesis, I show the correspondence between various equivalences popular in the reactive
systems community and coordinates of a formula price function, as introduced by Bisping (citation).
I formalized the concepts and proofs discussed in this thesis in the interactive proof assistant Isabelle (citation).

\textit{Reactive systems} are computing systems that continuously interact with their environment, reacting to external stimuli and producing outputs accordingly(Harel).
At a high level of abstraction, they can be seen as a collection of interacting processes. Modeling and verification of these processes is often referred to as \textit{Process Theory}. \\\\

Modeling is the activity of abstracting real-world systems by capturing essential features while omitting unnecessary details, often by mathematical structures.
Verification of these systems involves proving statements regarding the behavior of a system model. Often, verification tasks aim to show that a system's observed behavior aligns with its intended behavior.
That requires a criterion of similar behavior, or \textit{semantics of equality}. Depending on the requirements of a particular user, many different such criterions have been defined.
For a subset of processes, namely the class of sequential processes lacking internal behavior, (Glaabbeck) classified many such semantics. 
The processes in this subset can only perform one action at a time. Furthermore, this class is restricted to \textit{concrete} processes; processes in which no internal actions occur.
This classification involved partially ordering them by the relation 'makes strictly more identifications on processes than' (Glabbeeck). The resulting complete lattice is
referred to as the (infinitary) linear-time--branching-time spectrum. \footnote{On Infinity?} 
\footnote{Linear time describes identification via the order of events, while branching time captures the branching possibilities in system executions.}
\\\\
\textit{Reactive systems} are computing systems that continually interact with their environment, responding to external stimuli and generating outputs accordingly (Harel). 
At a high level of abstraction, they can be viewed as a collection of interacting processes. The modeling and verification of these processes are often referred to as \textit{Process Theory}.

Modeling is the process of abstracting real-world systems by capturing essential features while omitting unnecessary details, 
often through mathematical structures. 

The verification of these systems involves proving statements regarding the behavior of a system model.
Verification tasks typically aim to demonstrate that a system's observed behavior aligns with its intended behavior. 
This requires a criterion for similar behavior, or for the \textit{semantics of equality}. Depending on the requirements of a particular user, various such criteria have been defined.
For a subset of processes, specifically the class of sequential processes lacking internal behavior, Glabbeck classified many such semantics. 
These processes can only execute one action at a time. Moreover, this class is confined to \textit{concrete} processes, 
where internal actions are absent. Glabbeck's classification involved partially ordering these processes by the relation 'makes strictly more identifications on processes than' (Glabbeeck). 
The resulting complete lattice is known as the (infinitary) linear-time--branching-time spectrum.



For a subset of processes, namely the class of sequential processes lacking internal behavior, Glabbeck classified many such semantics. These processes can only perform one action at a time and are restricted to \textit{concrete} processes, where no internal actions occur. Glabbeck's classification involved partially ordering them by the relation 'makes strictly more identifications on processes than' (Glabbeeck). The resulting complete lattice is referred to as the (infinitary) linear-time--branching-time spectrum \footnote{On Infinity?} \footnote{Linear time describes identification via the order of events, while branching time captures the branching possibilities in system executions}.

more on LT BT spectrum?

Systems with this kind of processes can be modeled using labeled transition systems (Kel). An LTS is a triple of a set of processes, or states of the system,
a set of possible actions and a transition relation between a process, an action and another process. The outgoing transitions of each process correspond to the actions the system can perform in that state, 
yielding a subsequent state. In accordance with our restriction to concrete processes, we do not distinguish between different kinds of actions. 
\footnote{A popular notion of identification is internal behavior, LTS capable of modeling internal behavior use a fixed action to express internal behavior. This extension allows for additional semantics that have been investigated, for instance, in (Glabeeck).}
\\\\
In the context of this spectrum, demonstrating that a system model's observed behavior aligns with the behavior of a model of the specification involves 
finding the finest notions of behavioral equivalence that equate them. Special bisimulation games and algorithms capable of answering equivalence questions 
by performing a 'spectroscopy' of the differences between two processes have been developed (Deciding all at once)((accounting for silent steps), evtl hier weglassen oder mention: anderes spektrum)(process equiv as energy games)(A game for lt bt spectr).
These approaches rechart the linear-time--branching-time spectrum using \textit{formula prices} that capture the expressive capabilities of Hennessy-Milner Logic (HML).

This thesis provides a machine-checkable proof that certain price bounds correspond to the modal-logical characterizations of named equivalences. 
More precisely, a formula \isa{{\isasymphi}} is in an observation language $\mathcal{O_X}$ iff its price is within the given price bound.
Concretely, for every expressiveness price bound $e_X$, i derive the sublanguage of Hennessy--Miler logic $\mathcal{O_X}$ and show that a formula \isa{{\isasymphi}} is in $\mathcal{O_X}$ precisely if its price \isa{expr{\isacharparenleft}{\kern0pt}{\isasymphi}{\isacharparenright}{\kern0pt}} is less than or equal to $e_X$.
Then i show that $\mathcal{O_X}$ has exactly the same distinguishing power as the modal-logical characterization of that equivalence.

For the class of sequential processes, that can at most perform one action at a time, and that do not posses internal behavior.
(cite glabbeeck) classified many such semantics by partially ordering them by the relation 'makes strictly more identifications on processes than'. However,
The term \emph{reactive system} (citation) describes computing systems that continuously interact with their environment. Unlike sequential systems, 
the behavior or reactive systems is inherently event-driven and concurrent. They can be modeled by labeled directed graphs called \emph{labeled transition systems} (LTSs) (citation),
where the nodes of an LTS describe the states of a reactive system and the edges describe transitions between those states.


Strucutre:\\
Foundations: LTS, Bismilarity (weil besonders dadurch das es feinste äquivalenz ist,verbindung zu HML(HM Theorem), HML \\
Formula Pricing - capturing expressiveness using formula prices \\
Korrespondenz zwischen koordinaten und äquivalenzen beweise\\
diskussion?\\
appendix?\\

The semantics of reactive systems can be modeled as equivalences, that determine whether or not two systems behave similarly.
In the literature on concurrent systems many different notion of equivalence can be found, the maybe best known being \emph{(strong) bisimilarity}.
Rab van Glabbeek`s \emph{linear-time--branching-spectrum}(citation) ordered some of the most popular in a hierachy of equivalences.
-> New Paper characterizes them different... (HML beschreibung als erstes?!!)


- Reactive Systmes \\
  - modelling (via lts etc) \\
  - Semantics of resysts \\
    - Verification \\
    - different notions of equivalence (because of nondeteminism?) -> van glabbeek \\
    - Different definitions of semantics -> HML/relational/... \\
 --> linear-time--branching-time spectrum understood through properties of HML \\
  --> capture expressiveness capabilities of HML formulas via a function \\
--> Contribution o Paper: The in (citation) introduced expressiveness function 
and its coordinates captures the linear time branching time spectrum.. \\
- Isabelle:\\
  - formalization of concepts, proofs \\
  - what is isabelle \\
  - difference between mathematical concepts and their implementation? \\%
\end{isamarkuptext}\isamarkuptrue%
%
\isadelimtheory
%
\endisadelimtheory
%
\isatagtheory
%
\endisatagtheory
{\isafoldtheory}%
%
\isadelimtheory
%
\endisadelimtheory
%
\end{isabellebody}%
\endinput
%:%file=~/Documents/Isabelle_HOL/Introduction.thy%:%
%:%24=8%:%
%:%36=9%:%
%:%37=10%:%
%:%38=11%:%
%:%39=12%:%
%:%40=13%:%
%:%41=14%:%
%:%42=15%:%
%:%43=16%:%
%:%44=17%:%
%:%45=18%:%
%:%46=19%:%
%:%47=20%:%
%:%48=21%:%
%:%49=22%:%
%:%50=23%:%
%:%51=24%:%
%:%52=25%:%
%:%53=26%:%
%:%54=27%:%
%:%55=28%:%
%:%56=29%:%
%:%57=30%:%
%:%58=31%:%
%:%59=32%:%
%:%60=33%:%
%:%61=34%:%
%:%62=35%:%
%:%63=36%:%
%:%64=37%:%
%:%65=38%:%
%:%66=39%:%
%:%67=40%:%
%:%68=41%:%
%:%69=42%:%
%:%70=43%:%
%:%71=44%:%
%:%72=45%:%
%:%73=46%:%
%:%74=47%:%
%:%75=48%:%
%:%76=49%:%
%:%77=50%:%
%:%78=51%:%
%:%79=52%:%
%:%80=53%:%
%:%81=54%:%
%:%82=55%:%
%:%83=56%:%
%:%84=57%:%
%:%85=58%:%
%:%86=59%:%
%:%87=60%:%
%:%88=61%:%
%:%89=62%:%
%:%90=63%:%
%:%91=64%:%
%:%92=65%:%
%:%93=66%:%
%:%94=67%:%
%:%95=68%:%
%:%96=69%:%
%:%97=70%:%
%:%98=71%:%
%:%99=72%:%
%:%100=73%:%
%:%101=74%:%
%:%102=75%:%
%:%103=76%:%
%:%104=77%:%
%:%105=78%:%
%:%106=79%:%
%:%107=80%:%
%:%108=81%:%
%:%109=82%:%
%:%110=83%:%
%:%111=84%:%
%:%112=85%:%
%:%113=86%:%
%:%114=87%:%
%:%115=88%:%
%:%116=89%:%
%:%117=90%:%
%:%118=91%:%
%:%119=92%:%
%:%120=93%:%

%
\begin{isabellebody}%
\setisabellecontext{Foundations}%
%
\isadelimtheory
%
\endisadelimtheory
%
\isatagtheory
%
\endisatagtheory
{\isafoldtheory}%
%
\isadelimtheory
%
\endisadelimtheory
%
\isadelimdocument
%
\endisadelimdocument
%
\isatagdocument
%
\isamarkupchapter{Foundations%
}
\isamarkuptrue%
%
\endisatagdocument
{\isafolddocument}%
%
\isadelimdocument
%
\endisadelimdocument
%
\begin{isamarkuptext}%
In this chapter, relevant concepts will be introduced as well as formalised in Isabelle.%
\end{isamarkuptext}\isamarkuptrue%
%
\begin{isamarkuptext}%
- mention sources (Ben / Max Pohlmann?)%
\end{isamarkuptext}\isamarkuptrue%
%
\isadelimtheory
%
\endisadelimtheory
%
\isatagtheory
%
\endisatagtheory
{\isafoldtheory}%
%
\isadelimtheory
%
\endisadelimtheory
%
\end{isabellebody}%
\endinput
%:%file=~/Documents/Isabelle_HOL/Foundations.thy%:%
%:%24=8%:%
%:%36=10%:%
%:%40=13%:%

%
\begin{isabellebody}%
\setisabellecontext{Transition{\isacharunderscore}{\kern0pt}Systems}%
%
\isadelimtheory
%
\endisadelimtheory
%
\isatagtheory
%
\endisatagtheory
{\isafoldtheory}%
%
\isadelimtheory
%
\endisadelimtheory
%
\isadelimdocument
%
\endisadelimdocument
%
\isatagdocument
%
\isamarkupsection{Labelled Transition Systems%
}
\isamarkuptrue%
%
\endisatagdocument
{\isafolddocument}%
%
\isadelimdocument
%
\endisadelimdocument
%
\begin{isamarkuptext}%
\label{sec:LTS}%
\end{isamarkuptext}\isamarkuptrue%
%
\isadelimdocument
%
\endisadelimdocument
%
\isatagdocument
%
\isamarkupsubsubsection{Definition 1 (Labeled transition Systems)%
}
\isamarkuptrue%
%
\endisatagdocument
{\isafolddocument}%
%
\isadelimdocument
%
\endisadelimdocument
%
\begin{isamarkuptext}%
A Labelled Transition System (LTS) is a tuple $\mathcal{S} = (\Proc, \Act, \rightarrow)$ where $\Proc$ is the set of processes, 
$\Act$ is the set of actions and $\rightarrow$ $\subseteq \Proc \times \Act \times \Proc$ is a transition relation.\\

In concurrency theory, it is customary that the semantics of Reactive Systems are given in terms of labelled transition systems.
The processes represent the states a reactive system can be in. A transition of one state into another, 
caused by performing an action, can be understood as moving along the corresponding edge in the transition relation. \\

- Example?\\

- examples (to reuse later?)???%
\end{isamarkuptext}\isamarkuptrue%
%
\isadelimdocument
%
\endisadelimdocument
%
\isatagdocument
%
\isamarkupsubsubsection{Some more Definitions%
}
\isamarkuptrue%
%
\endisatagdocument
{\isafolddocument}%
%
\isadelimdocument
%
\endisadelimdocument
%
\begin{isamarkuptext}%
The $\alpha$-derivatives of a process are the processes that can be reached with one $\alpha$-transition:%
\end{isamarkuptext}\isamarkuptrue%
%
\begin{isamarkuptext}%
- image finite? nötig?\\
- image countable?
- initial actions
- deadlock
- relevant actions?
- step sequence!
-%
\end{isamarkuptext}\isamarkuptrue%
%
\isadelimdocument
%
\endisadelimdocument
%
\isatagdocument
%
\isamarkupsubsection{Isabelle%
}
\isamarkuptrue%
%
\endisatagdocument
{\isafolddocument}%
%
\isadelimdocument
%
\endisadelimdocument
%
\begin{isamarkuptext}%
Zustände: \isa{{\isacharprime}{\kern0pt}s} und Aktionen \isa{{\isacharprime}{\kern0pt}a}, Transitionsrelation ist locale trans. Ein LTS wird dann durch
seine Transitionsrelation definiert.%
\end{isamarkuptext}\isamarkuptrue%
\isacommand{locale}\isamarkupfalse%
\ lts\ {\isacharequal}{\kern0pt}\ \isanewline
\ \ \isakeyword{fixes}\ tran\ {\isacharcolon}{\kern0pt}{\isacharcolon}{\kern0pt}\ {\isacartoucheopen}{\isacharprime}{\kern0pt}s\ {\isasymRightarrow}\ {\isacharprime}{\kern0pt}a\ {\isasymRightarrow}\ {\isacharprime}{\kern0pt}s\ {\isasymRightarrow}\ bool{\isacartoucheclose}\isanewline
\ \ \ \ {\isacharparenleft}{\kern0pt}{\isachardoublequoteopen}{\isacharunderscore}{\kern0pt}\ {\isasymmapsto}{\isacharunderscore}{\kern0pt}\ {\isacharunderscore}{\kern0pt}{\isachardoublequoteclose}\ {\isacharbrackleft}{\kern0pt}{\isadigit{7}}{\isadigit{0}}{\isacharcomma}{\kern0pt}\ {\isadigit{7}}{\isadigit{0}}{\isacharcomma}{\kern0pt}\ {\isadigit{7}}{\isadigit{0}}{\isacharbrackright}{\kern0pt}\ {\isadigit{8}}{\isadigit{0}}{\isacharparenright}{\kern0pt}\isanewline
\isakeyword{begin}\isanewline
\isanewline
\isacommand{abbreviation}\isamarkupfalse%
\ derivatives\ {\isacharcolon}{\kern0pt}{\isacharcolon}{\kern0pt}\ {\isacartoucheopen}{\isacharprime}{\kern0pt}s\ {\isasymRightarrow}\ {\isacharprime}{\kern0pt}a\ {\isasymRightarrow}\ {\isacharprime}{\kern0pt}s\ set{\isacartoucheclose}\isanewline
\ \ \isakeyword{where}\isanewline
{\isacartoucheopen}derivatives\ p\ {\isasymalpha}\ {\isasymequiv}\ {\isacharbraceleft}{\kern0pt}p{\isacharprime}{\kern0pt}{\isachardot}{\kern0pt}\ p\ {\isasymmapsto}{\isasymalpha}\ p{\isacharprime}{\kern0pt}{\isacharbraceright}{\kern0pt}{\isacartoucheclose}%
\begin{isamarkuptext}%
Transition System is image-finite%
\end{isamarkuptext}\isamarkuptrue%
\isacommand{definition}\isamarkupfalse%
\ image{\isacharunderscore}{\kern0pt}finite\ \isakeyword{where}\isanewline
{\isacartoucheopen}image{\isacharunderscore}{\kern0pt}finite\ {\isasymequiv}\ {\isacharparenleft}{\kern0pt}{\isasymforall}p\ {\isasymalpha}{\isachardot}{\kern0pt}\ finite\ {\isacharparenleft}{\kern0pt}derivatives\ p\ {\isasymalpha}{\isacharparenright}{\kern0pt}{\isacharparenright}{\kern0pt}{\isacartoucheclose}\isanewline
\isanewline
\isacommand{definition}\isamarkupfalse%
\ image{\isacharunderscore}{\kern0pt}countable\ {\isacharcolon}{\kern0pt}{\isacharcolon}{\kern0pt}\ {\isacartoucheopen}bool{\isacartoucheclose}\isanewline
\ \ \isakeyword{where}\ {\isacartoucheopen}image{\isacharunderscore}{\kern0pt}countable\ {\isasymequiv}\ {\isacharparenleft}{\kern0pt}{\isasymforall}\ p\ {\isasymalpha}{\isachardot}{\kern0pt}\ countable\ {\isacharparenleft}{\kern0pt}derivatives\ p\ {\isasymalpha}{\isacharparenright}{\kern0pt}{\isacharparenright}{\kern0pt}{\isacartoucheclose}%
\begin{isamarkuptext}%
stimmt definition? definition benötigt nach umstieg auf sets?%
\end{isamarkuptext}\isamarkuptrue%
\isacommand{definition}\isamarkupfalse%
\ lts{\isacharunderscore}{\kern0pt}finite\ \isakeyword{where}\isanewline
{\isacartoucheopen}lts{\isacharunderscore}{\kern0pt}finite\ {\isasymequiv}\ {\isacharparenleft}{\kern0pt}finite\ {\isacharparenleft}{\kern0pt}UNIV\ {\isacharcolon}{\kern0pt}{\isacharcolon}{\kern0pt}\ {\isacharprime}{\kern0pt}s\ set{\isacharparenright}{\kern0pt}{\isacharparenright}{\kern0pt}{\isacartoucheclose}\isanewline
\isanewline
\isacommand{abbreviation}\isamarkupfalse%
\ initial{\isacharunderscore}{\kern0pt}actions{\isacharcolon}{\kern0pt}{\isacharcolon}{\kern0pt}\ {\isacartoucheopen}{\isacharprime}{\kern0pt}s\ {\isasymRightarrow}\ {\isacharprime}{\kern0pt}a\ set{\isacartoucheclose}\isanewline
\ \ \isakeyword{where}\isanewline
{\isacartoucheopen}initial{\isacharunderscore}{\kern0pt}actions\ p\ {\isasymequiv}\ {\isacharbraceleft}{\kern0pt}{\isasymalpha}{\isacharbar}{\kern0pt}{\isasymalpha}{\isachardot}{\kern0pt}\ {\isacharparenleft}{\kern0pt}{\isasymexists}p{\isacharprime}{\kern0pt}{\isachardot}{\kern0pt}\ p\ {\isasymmapsto}{\isasymalpha}\ p{\isacharprime}{\kern0pt}{\isacharparenright}{\kern0pt}{\isacharbraceright}{\kern0pt}{\isacartoucheclose}\isanewline
\isanewline
\isacommand{abbreviation}\isamarkupfalse%
\ deadlock\ {\isacharcolon}{\kern0pt}{\isacharcolon}{\kern0pt}\ {\isacartoucheopen}{\isacharprime}{\kern0pt}s\ {\isasymRightarrow}\ bool{\isacartoucheclose}\ \isakeyword{where}\isanewline
{\isacartoucheopen}deadlock\ p\ {\isasymequiv}\ {\isacharparenleft}{\kern0pt}{\isasymforall}a{\isachardot}{\kern0pt}\ derivatives\ p\ a\ {\isacharequal}{\kern0pt}\ {\isacharbraceleft}{\kern0pt}{\isacharbraceright}{\kern0pt}{\isacharparenright}{\kern0pt}{\isacartoucheclose}%
\begin{isamarkuptext}%
nötig?%
\end{isamarkuptext}\isamarkuptrue%
\isacommand{abbreviation}\isamarkupfalse%
\ relevant{\isacharunderscore}{\kern0pt}actions\ {\isacharcolon}{\kern0pt}{\isacharcolon}{\kern0pt}\ {\isacartoucheopen}{\isacharprime}{\kern0pt}a\ set{\isacartoucheclose}\isanewline
\ \ \isakeyword{where}\isanewline
{\isacartoucheopen}relevant{\isacharunderscore}{\kern0pt}actions\ {\isasymequiv}\ {\isacharbraceleft}{\kern0pt}a{\isachardot}{\kern0pt}\ {\isasymexists}p\ p{\isacharprime}{\kern0pt}{\isachardot}{\kern0pt}\ p\ {\isasymmapsto}a\ p{\isacharprime}{\kern0pt}{\isacharbraceright}{\kern0pt}{\isacartoucheclose}\isanewline
\isanewline
\isacommand{inductive}\isamarkupfalse%
\ step{\isacharunderscore}{\kern0pt}sequence\ {\isacharcolon}{\kern0pt}{\isacharcolon}{\kern0pt}\ {\isacartoucheopen}{\isacharprime}{\kern0pt}s\ {\isasymRightarrow}\ {\isacharprime}{\kern0pt}a\ list\ {\isasymRightarrow}\ {\isacharprime}{\kern0pt}s\ {\isasymRightarrow}\ bool{\isacartoucheclose}\ {\isacharparenleft}{\kern0pt}{\isacartoucheopen}{\isacharunderscore}{\kern0pt}\ {\isasymmapsto}{\isachardollar}{\kern0pt}\ {\isacharunderscore}{\kern0pt}\ {\isacharunderscore}{\kern0pt}{\isacartoucheclose}{\isacharbrackleft}{\kern0pt}{\isadigit{7}}{\isadigit{0}}{\isacharcomma}{\kern0pt}{\isadigit{7}}{\isadigit{0}}{\isacharcomma}{\kern0pt}{\isadigit{7}}{\isadigit{0}}{\isacharbrackright}{\kern0pt}\ {\isadigit{8}}{\isadigit{0}}{\isacharparenright}{\kern0pt}\ \isakeyword{where}\isanewline
{\isacartoucheopen}p\ {\isasymmapsto}{\isachardollar}{\kern0pt}\ {\isacharbrackleft}{\kern0pt}{\isacharbrackright}{\kern0pt}\ p{\isacartoucheclose}\ {\isacharbar}{\kern0pt}\isanewline
{\isacartoucheopen}p\ {\isasymmapsto}{\isachardollar}{\kern0pt}\ {\isacharparenleft}{\kern0pt}a{\isacharhash}{\kern0pt}rt{\isacharparenright}{\kern0pt}\ p{\isacharprime}{\kern0pt}{\isacharprime}{\kern0pt}{\isacartoucheclose}\ \isakeyword{if}\ {\isacartoucheopen}{\isasymexists}p{\isacharprime}{\kern0pt}{\isachardot}{\kern0pt}\ p\ {\isasymmapsto}\ a\ p{\isacharprime}{\kern0pt}\ {\isasymand}\ p{\isacharprime}{\kern0pt}\ {\isasymmapsto}{\isachardollar}{\kern0pt}\ rt\ p{\isacharprime}{\kern0pt}{\isacharprime}{\kern0pt}{\isacartoucheclose}\isanewline
\isanewline
\isacommand{end}\isamarkupfalse%
\isanewline
%
\isadelimtheory
%
\endisadelimtheory
%
\isatagtheory
%
\endisatagtheory
{\isafoldtheory}%
%
\isadelimtheory
%
\endisadelimtheory
%
\end{isabellebody}%
\endinput
%:%file=~/Documents/Isabelle_HOL/Transition_Systems.thy%:%
%:%24=8%:%
%:%36=9%:%
%:%45=11%:%
%:%57=13%:%
%:%58=14%:%
%:%59=15%:%
%:%60=16%:%
%:%61=17%:%
%:%62=18%:%
%:%63=19%:%
%:%64=20%:%
%:%65=21%:%
%:%66=22%:%
%:%75=25%:%
%:%87=27%:%
%:%91=31%:%
%:%92=32%:%
%:%93=33%:%
%:%94=34%:%
%:%95=35%:%
%:%96=36%:%
%:%97=37%:%
%:%106=39%:%
%:%118=41%:%
%:%119=42%:%
%:%121=43%:%
%:%122=43%:%
%:%123=44%:%
%:%124=45%:%
%:%125=46%:%
%:%126=47%:%
%:%127=48%:%
%:%128=48%:%
%:%129=49%:%
%:%130=50%:%
%:%132=52%:%
%:%134=54%:%
%:%135=54%:%
%:%136=55%:%
%:%137=56%:%
%:%138=57%:%
%:%139=57%:%
%:%140=58%:%
%:%142=60%:%
%:%144=61%:%
%:%145=61%:%
%:%146=62%:%
%:%147=63%:%
%:%148=64%:%
%:%149=64%:%
%:%150=65%:%
%:%151=66%:%
%:%152=67%:%
%:%153=68%:%
%:%154=68%:%
%:%155=69%:%
%:%157=71%:%
%:%159=72%:%
%:%160=72%:%
%:%161=73%:%
%:%162=74%:%
%:%163=75%:%
%:%164=76%:%
%:%165=76%:%
%:%166=77%:%
%:%167=78%:%
%:%168=79%:%
%:%169=80%:%
%:%170=80%:%

%
\begin{isabellebody}%
\setisabellecontext{Equivalences}%
%
\isadelimtheory
%
\endisadelimtheory
%
\isatagtheory
%
\endisatagtheory
{\isafoldtheory}%
%
\isadelimtheory
%
\endisadelimtheory
%
\isadelimdocument
%
\endisadelimdocument
%
\isatagdocument
%
\isamarkupsection{Behavioral Equivalence of Processes%
}
\isamarkuptrue%
%
\endisatagdocument
{\isafolddocument}%
%
\isadelimdocument
%
\endisadelimdocument
%
\begin{isamarkuptext}%
As discussed in the previous sections, LTSs model the behaviour of reactive systems. That behaviour
is observable by the environment in terms of transitions performed by the system. Depending on different criteria
on what constitutes equal behavior has led to a large number of equivalences for concurrent processes. Those equivalences are often
defined in term of relations on LTSs or sets of executions. The finest commonly used \textit{extensional behavioral equivalence}
is \textit{Bisimilarity}. In extensional equivalences, only observable behavior is taken into account, without considering the
identity of the processes. This sets bisimilarity apart from stronger graph equivalences like \textit{graph isomorphism}, 
here the (intensional) identity of processes is relevant. The coarsest commonly used equivalence is \textit{trace equivalence}.

- LT-BT spectrum (between them there is a lattice of equivalences ...)
- Wir behandeln bisimilarität hier gesondert wegen dessen bezihung zu HML (HM-Theorem) (s.h. Introduction, doppelung vermeiden).
- example bisimilarity

Informally, we call two processes bisimilar if...%
\end{isamarkuptext}\isamarkuptrue%
%
\isadelimdocument
%
\endisadelimdocument
%
\isatagdocument
%
\isamarkupsubsubsection{Bisimilarity%
}
\isamarkuptrue%
%
\endisatagdocument
{\isafolddocument}%
%
\isadelimdocument
%
\endisadelimdocument
%
\begin{isamarkuptext}%
The notion of strong bisimilarity can be formalised through \emph{strong bisimulation} (SB) relations, introduced originally in (citation Park). 
A binary relation $\mathcal{R}$ over the set of processes $\Proc$ is an SB iff for all $(p,q) \in \mathcal{R}$:
\begin{align*}
&\forall p' \in \Proc,\; \alpha \in \Act .\; p \xrightarrow{\alpha} p' \longrightarrow
\exists q' \in \Proc .\; q \xrightarrow{\alpha} q' \wedge (p',q') \in \mathcal{R}, \text{ and}\\
&\forall q' \in \Proc,\; \alpha \in \Act .\; q \xrightarrow{\alpha} q' \longrightarrow
\exists p' \in \Proc .\; p \xrightarrow{\alpha} p' \wedge (p',q') \in \mathcal{R}.
\end{align*}%
\end{isamarkuptext}\isamarkuptrue%
%
\isadelimtheory
%
\endisadelimtheory
%
\isatagtheory
\isacommand{end}\isamarkupfalse%
%
\endisatagtheory
{\isafoldtheory}%
%
\isadelimtheory
%
\endisadelimtheory
%
\end{isabellebody}%
\endinput
%:%file=~/Documents/Isabelle_HOL/Equivalences.thy%:%
%:%24=7%:%
%:%36=9%:%
%:%37=10%:%
%:%38=11%:%
%:%39=12%:%
%:%40=13%:%
%:%41=14%:%
%:%42=15%:%
%:%43=16%:%
%:%44=17%:%
%:%45=18%:%
%:%46=19%:%
%:%47=20%:%
%:%48=21%:%
%:%57=24%:%
%:%69=25%:%
%:%70=26%:%
%:%71=27%:%
%:%72=28%:%
%:%73=29%:%
%:%74=30%:%
%:%75=31%:%
%:%76=32%:%
%:%84=33%:%

%
\begin{isabellebody}%
\setisabellecontext{HML}%
%
\isadelimtheory
%
\endisadelimtheory
%
\isatagtheory
%
\endisatagtheory
{\isafoldtheory}%
%
\isadelimtheory
%
\endisadelimtheory
%
\isadelimdocument
%
\endisadelimdocument
%
\isatagdocument
%
\isamarkupsection{Hennessy--Milner logic%
}
\isamarkuptrue%
%
\endisatagdocument
{\isafolddocument}%
%
\isadelimdocument
%
\endisadelimdocument
%
\begin{isamarkuptext}%
For the purpose of this thesis, we focus on the modal-logical characterizations of equivalences, using Hennessy--Milner logic (HML). 
First introduced by Matthew Hennessy and Robin Milner (citation), HML is a modal logic for expressing properties of systems described by LTS.
Intuitively, HML describes observations on an LTS and two processes are considered equivalent under HML if there exists no observation that distinguishes between them.
(citation) defined the modal-logical language as consisting of (finite) conjunctions, negations and a (modal) possibility operator:
$$\varphi ::= t\!t \mid \varphi_1 \;\wedge\; \varphi_2 \mid \neg\varphi \mid \langle\alpha\rangle\varphi$$
(where $\alpha$ ranges over the set of actions.) The paper also proves that this language characterizes a relation that is effectively the same as bisimilarity. 
This theorem is called the Hennessy--Milner Theorem and can be expressed as follows: for image-finite LTSs, two processes are bisimilar iff they satisfy the same set of HML formulas. We call this the modal characterization of
bisimilarity. (Infinitary) Hennessy--Milner logic extends the original definition by allowing for conjunction of arbitrary width. 
This yields the modal characterization of bisimilarity for arbitrary LTS (cite). In (Section Bisimilarity) we provide an intuition of the proof along with the Isabelle proof.
In the following sections we mean the infinitary version when talking about HML.%
\end{isamarkuptext}\isamarkuptrue%
%
\isadelimdocument
%
\endisadelimdocument
%
\isatagdocument
%
\isamarkupsubsubsection{Definition 2.2.1 (Hennessy--Milner logic)%
}
\isamarkuptrue%
%
\endisatagdocument
{\isafolddocument}%
%
\isadelimdocument
%
\endisadelimdocument
%
\begin{isamarkuptext}%
\textbf{Syntax} \textit{The \textnormal{syntax of Hennessy--Milner logic} over a set $\Sigma$ of actions HML[$\Sigma$] is defined by the grammar:}
\begin{align*}
    \varphi &::= \langle a \rangle \varphi && \text{with } a \in \Sigma \\
            &\quad | \, \bigwedge_{i \in I} \psi_i \\
    \psi &::= \neg \varphi \, | \, \varphi.
\end{align*}
Where $I$ denotes an index set. The empty conjunction \textsf{T} $\coloneqq \bigwedge\varnothing$ is usually omitted in writing.%
\end{isamarkuptext}\isamarkuptrue%
%
\begin{isamarkuptext}%
The data type \isa{{\isacharparenleft}{\kern0pt}{\isacharprime}{\kern0pt}a{\isacharcomma}{\kern0pt}\ {\isacharprime}{\kern0pt}i{\isacharparenright}{\kern0pt}hml} formalizes the definition of HML formulas above. It is parameterized by the type of actions \isa{{\isacharprime}{\kern0pt}a} for $\Sigma$
and an index type \isa{{\isacharprime}{\kern0pt}i}. We use an index sets of arbitrary type \isa{I\ {\isacharcolon}{\kern0pt}{\isacharcolon}{\kern0pt}\ {\isacharprime}{\kern0pt}i\ set} and a mapping \isa{F\ {\isacharcolon}{\kern0pt}{\isacharcolon}{\kern0pt}\ {\isacharprime}{\kern0pt}i\ {\isasymRightarrow}\ {\isacharparenleft}{\kern0pt}{\isacharprime}{\kern0pt}a{\isacharcomma}{\kern0pt}\ {\isacharprime}{\kern0pt}i{\isacharparenright}{\kern0pt}\ hml} to formalize
conjunctions so that each element of \isa{I} is mapped to a formula%
\footnote{Note that the formalization via an arbitrary set, i.e. \isa{hml{\isacharunderscore}{\kern0pt}conj\ {\isasymopen}{\isacharparenleft}{\kern0pt}{\isacharprime}{\kern0pt}a{\isacharparenright}{\kern0pt}hml\ set{\isasymclose}} does not yield a valid type, since \isa{set} is not a bounded natural functor.}%.%
\end{isamarkuptext}\isamarkuptrue%
\isacommand{datatype}\isamarkupfalse%
\ {\isacharparenleft}{\kern0pt}{\isacharprime}{\kern0pt}a{\isacharcomma}{\kern0pt}\ {\isacharprime}{\kern0pt}i{\isacharparenright}{\kern0pt}hml\ {\isacharequal}{\kern0pt}\isanewline
TT\ {\isacharbar}{\kern0pt}\isanewline
hml{\isacharunderscore}{\kern0pt}pos\ {\isacartoucheopen}{\isacharprime}{\kern0pt}a{\isacartoucheclose}\ {\isacartoucheopen}{\isacharparenleft}{\kern0pt}{\isacharprime}{\kern0pt}a{\isacharcomma}{\kern0pt}\ {\isacharprime}{\kern0pt}i{\isacharparenright}{\kern0pt}hml{\isacartoucheclose}\ {\isacharbar}{\kern0pt}\isanewline
hml{\isacharunderscore}{\kern0pt}conj\ {\isacartoucheopen}{\isacharprime}{\kern0pt}i\ set{\isacartoucheclose}\ {\isacartoucheopen}{\isacharprime}{\kern0pt}i\ set{\isacartoucheclose}\ {\isacartoucheopen}{\isacharprime}{\kern0pt}i\ {\isasymRightarrow}\ {\isacharparenleft}{\kern0pt}{\isacharprime}{\kern0pt}a{\isacharcomma}{\kern0pt}\ {\isacharprime}{\kern0pt}i{\isacharparenright}{\kern0pt}\ hml{\isacartoucheclose}%
\begin{isamarkuptext}%
Note that in the Isabelle formalization differs from the mathematical definition by including a special formula \isa{TT} for \textsf{T} as part of the syntax. 
This is to to enable Isabelle to infer that the type \isa{hml} is not empty. Semantically, \isa{TT} is synonymous to \isa{{\isasymAnd}{\isacharbraceleft}{\kern0pt}{\isacharbraceright}{\kern0pt}}.
Corresponding to the mathematical definition, this formalization allows for conjunctions of arbitrary - even of infinite - width.%
\end{isamarkuptext}\isamarkuptrue%
%
\begin{isamarkuptext}%
$\langle a \rangle$ captures the observation of an $a$-transition by the system. 
Similar to propositional logic, conjunctions are used to describe multiple properties of a state that must hold simultaneously. Each conjunct represents a possible branching or execution path of the system. 
$\lnot\varphi$ indicate the absence of behavior represented by the subformula $\varphi$. 

\textbf{Semantics} \textit{The \textnormal{semantics of HML} parametrized by $\Sigma$ (on LTS processes) are given by the relation $\models$ : $(\Proc, \text{HML}[\Sigma])$:}
\begin{align*}
  p &\models \langle \alpha \rangle\varphi && \text{if there exists } q \text{ such that } q\in\mathit{Der}(p, \alpha) \text{ and } q \models\varphi \\
  p &\models \bigwedge_{i \in I} \psi_i && \text{if } p\models\psi_i \text{ for all } i\in I \\
  p &\models \lnot\varphi && \text{if } p\not\models\varphi
\end{align*}%
\end{isamarkuptext}\isamarkuptrue%
\isacommand{context}\isamarkupfalse%
\ lts\ \isakeyword{begin}\isanewline
\isanewline
\isacommand{primrec}\isamarkupfalse%
\ hml{\isacharunderscore}{\kern0pt}semantics\ {\isacharcolon}{\kern0pt}{\isacharcolon}{\kern0pt}\ {\isacartoucheopen}{\isacharprime}{\kern0pt}s\ {\isasymRightarrow}\ {\isacharparenleft}{\kern0pt}{\isacharprime}{\kern0pt}a{\isacharcomma}{\kern0pt}\ {\isacharprime}{\kern0pt}s{\isacharparenright}{\kern0pt}hml\ {\isasymRightarrow}\ bool{\isacartoucheclose}\isanewline
{\isacharparenleft}{\kern0pt}{\isacartoucheopen}{\isacharunderscore}{\kern0pt}\ {\isasymTurnstile}\ {\isacharunderscore}{\kern0pt}{\isacartoucheclose}\ {\isacharbrackleft}{\kern0pt}{\isadigit{5}}{\isadigit{0}}{\isacharcomma}{\kern0pt}\ {\isadigit{5}}{\isadigit{0}}{\isacharbrackright}{\kern0pt}\ {\isadigit{5}}{\isadigit{0}}{\isacharparenright}{\kern0pt}\isanewline
\isakeyword{where}\isanewline
hml{\isacharunderscore}{\kern0pt}sem{\isacharunderscore}{\kern0pt}tt{\isacharcolon}{\kern0pt}\ {\isacartoucheopen}{\isacharparenleft}{\kern0pt}{\isacharunderscore}{\kern0pt}\ {\isasymTurnstile}\ TT{\isacharparenright}{\kern0pt}\ {\isacharequal}{\kern0pt}\ True{\isacartoucheclose}\ {\isacharbar}{\kern0pt}\isanewline
hml{\isacharunderscore}{\kern0pt}sem{\isacharunderscore}{\kern0pt}pos{\isacharcolon}{\kern0pt}\ {\isacartoucheopen}{\isacharparenleft}{\kern0pt}p\ {\isasymTurnstile}\ {\isacharparenleft}{\kern0pt}hml{\isacharunderscore}{\kern0pt}pos\ {\isasymalpha}\ {\isasymphi}{\isacharparenright}{\kern0pt}{\isacharparenright}{\kern0pt}\ {\isacharequal}{\kern0pt}\ {\isacharparenleft}{\kern0pt}{\isasymexists}\ q{\isachardot}{\kern0pt}\ {\isacharparenleft}{\kern0pt}p\ {\isasymmapsto}{\isasymalpha}\ q{\isacharparenright}{\kern0pt}\ {\isasymand}\ q\ {\isasymTurnstile}\ {\isasymphi}{\isacharparenright}{\kern0pt}{\isacartoucheclose}\ {\isacharbar}{\kern0pt}\isanewline
hml{\isacharunderscore}{\kern0pt}sem{\isacharunderscore}{\kern0pt}conj{\isacharcolon}{\kern0pt}\ {\isacartoucheopen}{\isacharparenleft}{\kern0pt}p\ {\isasymTurnstile}\ {\isacharparenleft}{\kern0pt}hml{\isacharunderscore}{\kern0pt}conj\ I\ J\ {\isasympsi}s{\isacharparenright}{\kern0pt}{\isacharparenright}{\kern0pt}\ {\isacharequal}{\kern0pt}\ {\isacharparenleft}{\kern0pt}{\isacharparenleft}{\kern0pt}{\isasymforall}i\ {\isasymin}\ I{\isachardot}{\kern0pt}\ p\ {\isasymTurnstile}\ {\isacharparenleft}{\kern0pt}{\isasympsi}s\ i{\isacharparenright}{\kern0pt}{\isacharparenright}{\kern0pt}\ {\isasymand}\ {\isacharparenleft}{\kern0pt}{\isasymforall}j\ {\isasymin}\ J{\isachardot}{\kern0pt}\ {\isasymnot}{\isacharparenleft}{\kern0pt}p\ {\isasymTurnstile}\ {\isacharparenleft}{\kern0pt}{\isasympsi}s\ j{\isacharparenright}{\kern0pt}{\isacharparenright}{\kern0pt}{\isacharparenright}{\kern0pt}{\isacharparenright}{\kern0pt}{\isacartoucheclose}%
\begin{isamarkuptext}%
A formula that is true for all processes in a LTS can be considered a property that holds universally for the system, akin to a tautology in classical logic.%
\end{isamarkuptext}\isamarkuptrue%
\isacommand{definition}\isamarkupfalse%
\ HML{\isacharunderscore}{\kern0pt}true\ \isakeyword{where}\isanewline
{\isachardoublequoteopen}HML{\isacharunderscore}{\kern0pt}true\ {\isasymphi}\ {\isasymequiv}\ {\isasymforall}s{\isachardot}{\kern0pt}\ s\ {\isasymTurnstile}\ {\isasymphi}{\isachardoublequoteclose}\isanewline
%
\isadelimproof
%
\endisadelimproof
%
\isatagproof
%
\endisatagproof
{\isafoldproof}%
%
\isadelimproof
%
\endisadelimproof
%
\isadelimdocument
%
\endisadelimdocument
%
\isatagdocument
%
\isamarkupsubsubsection{Definition 2.2.2%
}
\isamarkuptrue%
%
\endisatagdocument
{\isafolddocument}%
%
\isadelimdocument
%
\endisadelimdocument
%
\begin{isamarkuptext}%
\textit{As discussed, equivalences in LTS can be defined in terms of \textsf{HML} subsets. Two processes are equivalent regarding a subset of HML if they satisfy the same formulas of that subset. 
A subset provides a \textnormal{modal-logical characterization $\mathcal{O}_X$} of an equivalence X if, according to that subset, the same processes are considered equivalent as they are under the colloquial definition of that equivalence.
We denote \textnormal{X-equivalence} of two processes $p$ $q$ by $p \sim_X q$. If they processes are equivalent for every formula in HML, they are bisimilar $p \sim_B q$. A formula \textnormal{distinguishes} one state from another if it is true for the former and false for the latter.}%
\end{isamarkuptext}\isamarkuptrue%
\isacommand{definition}\isamarkupfalse%
\ HML{\isacharunderscore}{\kern0pt}subset{\isacharunderscore}{\kern0pt}equivalent\ {\isacharcolon}{\kern0pt}{\isacharcolon}{\kern0pt}\ {\isacartoucheopen}{\isacharparenleft}{\kern0pt}{\isacharprime}{\kern0pt}a{\isacharcomma}{\kern0pt}\ {\isacharprime}{\kern0pt}s{\isacharparenright}{\kern0pt}hml\ set\ {\isasymRightarrow}\ {\isacharprime}{\kern0pt}s\ {\isasymRightarrow}\ {\isacharprime}{\kern0pt}s\ {\isasymRightarrow}\ bool{\isacartoucheclose}\ \isakeyword{where}\isanewline
\ \ {\isacartoucheopen}HML{\isacharunderscore}{\kern0pt}subset{\isacharunderscore}{\kern0pt}equivalent\ X\ p\ q\ {\isasymequiv}\ {\isacharparenleft}{\kern0pt}{\isasymforall}{\isasymphi}\ {\isasymin}\ X{\isachardot}{\kern0pt}\ {\isacharparenleft}{\kern0pt}p\ {\isasymTurnstile}\ {\isasymphi}{\isacharparenright}{\kern0pt}\ {\isasymlongleftrightarrow}\ {\isacharparenleft}{\kern0pt}q\ {\isasymTurnstile}\ {\isasymphi}{\isacharparenright}{\kern0pt}{\isacharparenright}{\kern0pt}{\isacartoucheclose}\isanewline
\isanewline
\isacommand{definition}\isamarkupfalse%
\ HML{\isacharunderscore}{\kern0pt}equivalent\ {\isacharcolon}{\kern0pt}{\isacharcolon}{\kern0pt}\ {\isachardoublequoteopen}{\isacharprime}{\kern0pt}s\ {\isasymRightarrow}\ {\isacharprime}{\kern0pt}s\ {\isasymRightarrow}\ bool{\isachardoublequoteclose}\ \isakeyword{where}\isanewline
\ \ {\isachardoublequoteopen}HML{\isacharunderscore}{\kern0pt}equivalent\ p\ q\ {\isasymequiv}\ HML{\isacharunderscore}{\kern0pt}subset{\isacharunderscore}{\kern0pt}equivalent\ {\isacharbraceleft}{\kern0pt}{\isasymphi}{\isachardot}{\kern0pt}\ True{\isacharbraceright}{\kern0pt}\ p\ q{\isachardoublequoteclose}\isanewline
\isanewline
\isacommand{abbreviation}\isamarkupfalse%
\ distinguishes\ {\isacharcolon}{\kern0pt}{\isacharcolon}{\kern0pt}\ \ {\isacartoucheopen}{\isacharparenleft}{\kern0pt}{\isacharprime}{\kern0pt}a{\isacharcomma}{\kern0pt}\ {\isacharprime}{\kern0pt}s{\isacharparenright}{\kern0pt}\ hml\ {\isasymRightarrow}\ {\isacharprime}{\kern0pt}s\ {\isasymRightarrow}\ {\isacharprime}{\kern0pt}s\ {\isasymRightarrow}\ bool{\isacartoucheclose}\ \isakeyword{where}\isanewline
\ \ \ {\isacartoucheopen}distinguishes\ {\isasymphi}\ p\ q\ {\isasymequiv}\ p\ {\isasymTurnstile}\ {\isasymphi}\ {\isasymand}\ {\isasymnot}\ q\ {\isasymTurnstile}\ {\isasymphi}{\isacartoucheclose}%
\begin{isamarkuptext}%
$\cdot \sim_X \cdot$ is an equivalence relation.%
\end{isamarkuptext}\isamarkuptrue%
\isacommand{lemma}\isamarkupfalse%
\ subs{\isacharunderscore}{\kern0pt}equiv{\isacharunderscore}{\kern0pt}refl{\isacharcolon}{\kern0pt}\ {\isachardoublequoteopen}reflp\ {\isacharparenleft}{\kern0pt}HML{\isacharunderscore}{\kern0pt}subset{\isacharunderscore}{\kern0pt}equivalent\ X{\isacharparenright}{\kern0pt}{\isachardoublequoteclose}\isanewline
%
\isadelimproof
\ \ %
\endisadelimproof
%
\isatagproof
\isacommand{using}\isamarkupfalse%
\ reflpI\ HML{\isacharunderscore}{\kern0pt}subset{\isacharunderscore}{\kern0pt}equivalent{\isacharunderscore}{\kern0pt}def\isanewline
\ \ \isacommand{by}\isamarkupfalse%
\ metis%
\endisatagproof
{\isafoldproof}%
%
\isadelimproof
\isanewline
%
\endisadelimproof
\isanewline
\isacommand{lemma}\isamarkupfalse%
\ subs{\isacharunderscore}{\kern0pt}equiv{\isacharunderscore}{\kern0pt}trans{\isacharcolon}{\kern0pt}\ {\isachardoublequoteopen}transp\ {\isacharparenleft}{\kern0pt}HML{\isacharunderscore}{\kern0pt}subset{\isacharunderscore}{\kern0pt}equivalent\ X{\isacharparenright}{\kern0pt}{\isachardoublequoteclose}\isanewline
%
\isadelimproof
\ \ %
\endisadelimproof
%
\isatagproof
\isacommand{using}\isamarkupfalse%
\ HML{\isacharunderscore}{\kern0pt}subset{\isacharunderscore}{\kern0pt}equivalent{\isacharunderscore}{\kern0pt}def\ transp{\isacharunderscore}{\kern0pt}def\isanewline
\ \ \isacommand{by}\isamarkupfalse%
\ force%
\endisatagproof
{\isafoldproof}%
%
\isadelimproof
\isanewline
%
\endisadelimproof
\isanewline
\isacommand{lemma}\isamarkupfalse%
\ subs{\isacharunderscore}{\kern0pt}equiv{\isacharunderscore}{\kern0pt}sym{\isacharcolon}{\kern0pt}\isanewline
\ \ \isakeyword{shows}\ {\isachardoublequoteopen}symp\ {\isacharparenleft}{\kern0pt}HML{\isacharunderscore}{\kern0pt}subset{\isacharunderscore}{\kern0pt}equivalent\ X{\isacharparenright}{\kern0pt}{\isachardoublequoteclose}\isanewline
%
\isadelimproof
\ \ %
\endisadelimproof
%
\isatagproof
\isacommand{unfolding}\isamarkupfalse%
\ HML{\isacharunderscore}{\kern0pt}subset{\isacharunderscore}{\kern0pt}equivalent{\isacharunderscore}{\kern0pt}def\ symp{\isacharunderscore}{\kern0pt}def\ \isanewline
\ \ \isacommand{by}\isamarkupfalse%
\ blast\isanewline
\isanewline
%
\endisatagproof
{\isafoldproof}%
%
\isadelimproof
%
\endisadelimproof
%
\isadelimvisible
%
\endisadelimvisible
%
\isatagvisible
%
\endisatagvisible
{\isafoldvisible}%
%
\isadelimvisible
%
\endisadelimvisible
%
\begin{isamarkuptext}%
If two states are not HML equivalent, there must be a distinguishing formula.%
\end{isamarkuptext}\isamarkuptrue%
\isacommand{lemma}\isamarkupfalse%
\ hml{\isacharunderscore}{\kern0pt}distinctions{\isacharcolon}{\kern0pt}\isanewline
\ \ \isakeyword{fixes}\ state{\isacharcolon}{\kern0pt}{\isacharcolon}{\kern0pt}\ {\isacharprime}{\kern0pt}s\isanewline
\ \ \isakeyword{assumes}\ {\isacartoucheopen}{\isasymnot}\ HML{\isacharunderscore}{\kern0pt}equivalent\ p\ q{\isacartoucheclose}\isanewline
\ \ \isakeyword{shows}\ {\isacartoucheopen}{\isasymexists}{\isasymphi}{\isachardot}{\kern0pt}\ distinguishes\ {\isasymphi}\ p\ q{\isacartoucheclose}\isanewline
%
\isadelimproof
%
\endisadelimproof
%
\isatagproof
\isacommand{proof}\isamarkupfalse%
{\isacharminus}{\kern0pt}\isanewline
\ \ \isacommand{from}\isamarkupfalse%
\ assms\ \isacommand{have}\isamarkupfalse%
\ {\isachardoublequoteopen}{\isasymnot}\ {\isacharparenleft}{\kern0pt}{\isasymforall}\ {\isasymphi}{\isacharcolon}{\kern0pt}{\isacharcolon}{\kern0pt}{\isacharparenleft}{\kern0pt}{\isacharprime}{\kern0pt}a{\isacharcomma}{\kern0pt}\ {\isacharprime}{\kern0pt}s{\isacharparenright}{\kern0pt}\ hml{\isachardot}{\kern0pt}\ {\isacharparenleft}{\kern0pt}p\ {\isasymTurnstile}\ {\isasymphi}{\isacharparenright}{\kern0pt}\ {\isasymlongleftrightarrow}\ {\isacharparenleft}{\kern0pt}q\ {\isasymTurnstile}\ {\isasymphi}{\isacharparenright}{\kern0pt}{\isacharparenright}{\kern0pt}{\isachardoublequoteclose}\ \isanewline
\ \ \ \ \isacommand{unfolding}\isamarkupfalse%
\ HML{\isacharunderscore}{\kern0pt}equivalent{\isacharunderscore}{\kern0pt}def\ HML{\isacharunderscore}{\kern0pt}subset{\isacharunderscore}{\kern0pt}equivalent{\isacharunderscore}{\kern0pt}def\ \isacommand{by}\isamarkupfalse%
\ blast\isanewline
\ \ \isacommand{then}\isamarkupfalse%
\ \isacommand{obtain}\isamarkupfalse%
\ {\isasymphi}{\isacharcolon}{\kern0pt}{\isacharcolon}{\kern0pt}{\isachardoublequoteopen}{\isacharparenleft}{\kern0pt}{\isacharprime}{\kern0pt}a{\isacharcomma}{\kern0pt}\ {\isacharprime}{\kern0pt}s{\isacharparenright}{\kern0pt}\ hml{\isachardoublequoteclose}\ \isakeyword{where}\ {\isachardoublequoteopen}{\isacharparenleft}{\kern0pt}p\ {\isasymTurnstile}\ {\isasymphi}{\isacharparenright}{\kern0pt}\ {\isasymnoteq}\ {\isacharparenleft}{\kern0pt}q\ {\isasymTurnstile}\ {\isasymphi}{\isacharparenright}{\kern0pt}{\isachardoublequoteclose}\ \isacommand{by}\isamarkupfalse%
\ blast\isanewline
\ \ \isacommand{then}\isamarkupfalse%
\ \isacommand{have}\isamarkupfalse%
\ {\isachardoublequoteopen}{\isacharparenleft}{\kern0pt}{\isacharparenleft}{\kern0pt}p\ {\isasymTurnstile}\ {\isasymphi}{\isacharparenright}{\kern0pt}\ {\isasymand}\ {\isasymnot}{\isacharparenleft}{\kern0pt}q\ {\isasymTurnstile}\ {\isasymphi}{\isacharparenright}{\kern0pt}{\isacharparenright}{\kern0pt}\ {\isasymor}\ {\isacharparenleft}{\kern0pt}{\isasymnot}{\isacharparenleft}{\kern0pt}p\ {\isasymTurnstile}\ {\isasymphi}{\isacharparenright}{\kern0pt}\ {\isasymand}\ {\isacharparenleft}{\kern0pt}q\ {\isasymTurnstile}\ {\isasymphi}{\isacharparenright}{\kern0pt}{\isacharparenright}{\kern0pt}{\isachardoublequoteclose}\isanewline
\ \ \ \ \isacommand{by}\isamarkupfalse%
\ blast\isanewline
\ \ \isacommand{then}\isamarkupfalse%
\ \isacommand{show}\isamarkupfalse%
\ {\isacharquery}{\kern0pt}thesis\isanewline
\ \ \isacommand{proof}\isamarkupfalse%
\isanewline
\ \ \ \ \isacommand{show}\isamarkupfalse%
\ {\isachardoublequoteopen}distinguishes\ {\isasymphi}\ p\ q\ {\isasymLongrightarrow}\ {\isasymexists}{\isasymphi}{\isachardot}{\kern0pt}\ distinguishes\ {\isasymphi}\ p\ q{\isachardoublequoteclose}\ \isacommand{by}\isamarkupfalse%
\ blast\isanewline
\ \ \isacommand{next}\isamarkupfalse%
\isanewline
\ \ \ \ \isacommand{assume}\isamarkupfalse%
\ assm{\isacharcolon}{\kern0pt}\ {\isachardoublequoteopen}{\isasymnot}\ p\ {\isasymTurnstile}\ {\isasymphi}\ {\isasymand}\ q\ {\isasymTurnstile}\ {\isasymphi}{\isachardoublequoteclose}\isanewline
\ \ \ \ \isacommand{define}\isamarkupfalse%
\ n{\isasymphi}\ \isakeyword{where}\ {\isachardoublequoteopen}n{\isasymphi}\ {\isasymequiv}{\isacharparenleft}{\kern0pt}hml{\isacharunderscore}{\kern0pt}conj\ {\isacharparenleft}{\kern0pt}{\isacharbraceleft}{\kern0pt}{\isacharbraceright}{\kern0pt}{\isacharcolon}{\kern0pt}{\isacharcolon}{\kern0pt}{\isacharprime}{\kern0pt}s\ set{\isacharparenright}{\kern0pt}\ \isanewline
\ \ \ \ \ \ \ \ \ \ \ \ \ \ \ \ \ \ \ \ \ \ \ \ \ \ {\isacharbraceleft}{\kern0pt}state{\isacharbraceright}{\kern0pt}\ \isanewline
\ \ \ \ \ \ \ \ \ \ \ \ \ \ \ \ \ \ \ \ \ \ \ \ \ \ {\isacharparenleft}{\kern0pt}{\isasymlambda}j{\isachardot}{\kern0pt}\ if\ j\ {\isacharequal}{\kern0pt}\ state\ then\ {\isasymphi}\ else\ undefined{\isacharparenright}{\kern0pt}{\isacharparenright}{\kern0pt}{\isachardoublequoteclose}\isanewline
\ \ \ \ \isacommand{have}\isamarkupfalse%
\ {\isachardoublequoteopen}p\ {\isasymTurnstile}\ n{\isasymphi}\ {\isasymand}\ {\isasymnot}\ q\ {\isasymTurnstile}\ n{\isasymphi}{\isachardoublequoteclose}\ \isanewline
\ \ \ \ \ \ \isacommand{unfolding}\isamarkupfalse%
\ n{\isasymphi}{\isacharunderscore}{\kern0pt}def\isanewline
\ \ \ \ \ \ \isacommand{using}\isamarkupfalse%
\ hml{\isacharunderscore}{\kern0pt}semantics{\isachardot}{\kern0pt}simps\ assm\isanewline
\ \ \ \ \ \ \isacommand{by}\isamarkupfalse%
\ force\isanewline
\ \ \ \ \isacommand{then}\isamarkupfalse%
\ \isacommand{show}\isamarkupfalse%
\ {\isacharquery}{\kern0pt}thesis\isanewline
\ \ \ \ \ \ \isacommand{by}\isamarkupfalse%
\ blast\isanewline
\ \ \isacommand{qed}\isamarkupfalse%
\isanewline
\isacommand{qed}\isamarkupfalse%
%
\endisatagproof
{\isafoldproof}%
%
\isadelimproof
\isanewline
%
\endisadelimproof
\isanewline
\isacommand{end}\isamarkupfalse%
%
\begin{isamarkuptext}%
We can now use HML to capture differences between $p_1$ and $q_1$ of (ref Example 1). The formula
$\langle a \rangle\bigwedge\{\lnot\langle c \rangle\}$ distinguishes $p_1$ from $q_1$ and $\bigwedge\{\lnot\langle a \rangle\bigwedge\{\lnot\langle c \rangle\}\}$ distinguishes
$q_1$ from $p_1$. The Hennessy–Milner Theorem implies that if a distinguishing formula exists, then $p_1$ and $q_1$ cannot be bisimilar.%
\end{isamarkuptext}\isamarkuptrue%
%
\isadelimvisible
%
\endisadelimvisible
%
\isatagvisible
%
\endisatagvisible
{\isafoldvisible}%
%
\isadelimvisible
%
\endisadelimvisible
%
\isadelimdocument
%
\endisadelimdocument
%
\isatagdocument
%
\endisatagdocument
{\isafolddocument}%
%
\isadelimdocument
%
\endisadelimdocument
%
\isadelimvisible
%
\endisadelimvisible
%
\isatagvisible
%
\endisatagvisible
{\isafoldvisible}%
%
\isadelimvisible
%
\endisadelimvisible
%
\isadelimtheory
%
\endisadelimtheory
%
\isatagtheory
%
\endisatagtheory
{\isafoldtheory}%
%
\isadelimtheory
%
\endisadelimtheory
%
\end{isabellebody}%
\endinput
%:%file=~/Documents/Isabelle_HOL/HML.thy%:%
%:%24=7%:%
%:%36=8%:%
%:%37=9%:%
%:%38=10%:%
%:%39=11%:%
%:%40=12%:%
%:%41=13%:%
%:%42=14%:%
%:%43=15%:%
%:%44=16%:%
%:%45=17%:%
%:%54=19%:%
%:%66=20%:%
%:%67=21%:%
%:%68=22%:%
%:%69=23%:%
%:%70=24%:%
%:%71=25%:%
%:%72=26%:%
%:%76=28%:%
%:%77=29%:%
%:%78=30%:%
%:%79=31%:%
%:%81=33%:%
%:%82=33%:%
%:%83=34%:%
%:%84=35%:%
%:%85=36%:%
%:%87=38%:%
%:%88=39%:%
%:%89=40%:%
%:%93=42%:%
%:%94=43%:%
%:%95=44%:%
%:%96=45%:%
%:%97=46%:%
%:%98=47%:%
%:%99=48%:%
%:%100=49%:%
%:%101=50%:%
%:%102=51%:%
%:%104=53%:%
%:%105=53%:%
%:%106=54%:%
%:%107=55%:%
%:%108=55%:%
%:%109=56%:%
%:%110=57%:%
%:%111=58%:%
%:%112=59%:%
%:%113=60%:%
%:%115=62%:%
%:%117=63%:%
%:%118=63%:%
%:%119=64%:%
%:%140=73%:%
%:%152=75%:%
%:%153=76%:%
%:%154=77%:%
%:%156=79%:%
%:%157=79%:%
%:%158=80%:%
%:%159=81%:%
%:%160=82%:%
%:%161=82%:%
%:%162=83%:%
%:%163=84%:%
%:%164=85%:%
%:%165=85%:%
%:%166=86%:%
%:%168=88%:%
%:%170=90%:%
%:%171=90%:%
%:%174=91%:%
%:%178=91%:%
%:%179=91%:%
%:%180=92%:%
%:%181=92%:%
%:%186=92%:%
%:%189=93%:%
%:%190=94%:%
%:%191=94%:%
%:%194=95%:%
%:%198=95%:%
%:%199=95%:%
%:%200=96%:%
%:%201=96%:%
%:%206=96%:%
%:%209=97%:%
%:%210=98%:%
%:%211=98%:%
%:%212=99%:%
%:%215=100%:%
%:%219=100%:%
%:%220=100%:%
%:%221=101%:%
%:%222=101%:%
%:%223=102%:%
%:%246=112%:%
%:%248=114%:%
%:%249=114%:%
%:%250=115%:%
%:%251=116%:%
%:%252=117%:%
%:%259=118%:%
%:%260=118%:%
%:%261=119%:%
%:%262=119%:%
%:%263=119%:%
%:%264=120%:%
%:%265=120%:%
%:%266=120%:%
%:%267=121%:%
%:%268=121%:%
%:%269=121%:%
%:%270=121%:%
%:%271=122%:%
%:%272=122%:%
%:%273=122%:%
%:%274=123%:%
%:%275=123%:%
%:%276=124%:%
%:%277=124%:%
%:%278=124%:%
%:%279=125%:%
%:%280=125%:%
%:%281=126%:%
%:%282=126%:%
%:%283=126%:%
%:%284=127%:%
%:%285=127%:%
%:%286=128%:%
%:%287=128%:%
%:%288=129%:%
%:%289=129%:%
%:%291=131%:%
%:%292=132%:%
%:%293=132%:%
%:%294=133%:%
%:%295=133%:%
%:%296=134%:%
%:%297=134%:%
%:%298=135%:%
%:%299=135%:%
%:%300=136%:%
%:%301=136%:%
%:%302=136%:%
%:%303=137%:%
%:%304=137%:%
%:%305=138%:%
%:%306=138%:%
%:%307=139%:%
%:%313=139%:%
%:%316=140%:%
%:%317=141%:%
%:%320=143%:%
%:%321=144%:%
%:%322=145%:%

%
\begin{isabellebody}%
\setisabellecontext{formula{\isacharunderscore}{\kern0pt}prices{\isacharunderscore}{\kern0pt}list}%
%
\isadelimtheory
%
\endisadelimtheory
%
\isatagtheory
%
\endisatagtheory
{\isafoldtheory}%
%
\isadelimtheory
%
\endisadelimtheory
%
\isadelimdocument
%
\endisadelimdocument
%
\isatagdocument
%
\isamarkupsection{Price Spectra of Behavioral Equivalences%
}
\isamarkuptrue%
%
\endisatagdocument
{\isafolddocument}%
%
\isadelimdocument
%
\endisadelimdocument
%
\begin{isamarkuptext}%
\cite{bisping2023process, bisping2022deciding} use a pricing system to measure the amount of HML-expressiveness used by a formula. By assigning an expressiveness price to each formula, the authors create a price lattice that allows for comparing distinguishing power of different formulas, where lower prices indicate less distinguishing power. This allows for a new way of defining HML-subsets. Instead of bounding the subsets by the structure of the included formulas, they are defined as sets of formulas whose prices are less than or equal to a given expressiveness price bound, or \textit{price coordinates}.
The value of each dimension of these price coordinates constrains different syntactic features of the formulas. The authors derive the linear-time--branching-time spectrum by assigning a price coordinate to every equivalence in the spectrum, see \cref{fig:1_1}. In this section, we introduce the definition of the expressiveness price function of (\cite{bisping2023process}, definition 5) and how that function is used to chart the spectrum.%
\end{isamarkuptext}\isamarkuptrue%
%
\begin{isamarkuptext}%
Unlike \cite{bisping2023process}, we define the price for every dimension $i$ as a seperate function, $\textsf{expr}_i : \text{HML}[\Sigma] \rightarrow (\mathbb{N \cup \{\infty\}})$ and combine them to the expressiveness function, $\textsf{expr} : \text{HML}[\Sigma] \rightarrow (\mathbb{N \cup \{\infty\}})^6$. Each function inductively traverses the syntax tree of a formula and increases its value when encountering the respective syntax feature.%
\end{isamarkuptext}\isamarkuptrue%
%
\isadelimdocument
%
\endisadelimdocument
%
\isatagdocument
%
\isamarkupsubsubsection{Definition 2.3.1 (Formula Prices)%
}
\isamarkuptrue%
%
\endisatagdocument
{\isafolddocument}%
%
\isadelimdocument
%
\endisadelimdocument
%
\begin{isamarkuptext}%
\textit{
(\textcolor{red}{1}) The \textnormal{modal depth} $\textsf{expr}_1$ measures the nesting depth of observations within a formula: $\text{HML}[\Sigma] \rightarrow (\mathbb{N} \cup \{\infty\})$ of a formula $\varphi$ is defined recursively by:}
\textit{
\begin{align*}
    \text{if } \varphi &= \langle a \rangle \psi \quad \text{ with } a \in \Sigma \\
    &\text{then } \textsf{expr}_1(\varphi) = 1 + \textsf{expr}_1(\psi) \\
    \text{if } \varphi &= \bigwedge_{i \in I} \{ \psi_1, \psi_2, \ldots \} \\
    &\text{then } \textsf{expr}_1(\varphi) = \sup(\textsf{expr}_1(\psi_i)) \\
    \text{if } \psi &= \neg \varphi \\
    &\text{then } \textsf{expr}_1(\psi) = \textsf{expr}_1(\varphi)
\end{align*}
}


\textit{(\textcolor{orange}{2}) The \textnormal{nesting depth of conjunctions} $\textsf{expr}_2$ measures the maximal number of conjunctions that are nested inside one another in a formula: $\text{HML}[\Sigma] \rightarrow (\mathbb{N} \cup \{\infty\})$ of a formula $\varphi$ is defined recursively by:}
\textit{
\begin{align*}
    \text{if } \varphi &= \langle a \rangle \psi \quad \text{ with } a \in \Sigma \\
    & \text{then } \textsf{expr}_2(\varphi) = \textsf{expr}_2(\psi) \\
    \text{if } \varphi &= \bigwedge_{i \in I} \{\psi_i \} \\
    & \text{then } \textsf{expr}_2(\varphi) = 1 + \sup(\textsf{expr}_2(\psi_i)) \\
    \text{if } \psi &= \neg \varphi \\
    & \text{then } \textsf{expr}_2(\psi) = \textsf{expr}_2(\varphi)
\end{align*}
}

\textit{(\textcolor{myyellow}{3}) The \textnormal{maximal modal depth of deepest positive clauses in conjunctions} $\textsf{expr}_3$ measures the deepest modal depth of the positive conjuncts of all conjunctions of a formula: $\text{HML}[\Sigma] \rightarrow (\mathbb{N} \cup \{\infty\})$ of a formula $\varphi$ is defined recursively by:}

\textit{\begin{align*}
    \text{if } \varphi &= \langle a \rangle \psi \quad \text{ with } a \in \Sigma \\
    & \text{then } \textsf{md}(\varphi) = \textsf{md}(\psi) \\
    \text{if } \varphi &= \bigwedge_{i \in I} \{ \psi_i \} \\
    & \text{then } \textsf{md}(\varphi) = \sup(\{\textsf{expr}_1(\psi_i) | i \in \text{Pos}\} \cup \{\textsf{expr}_3(\psi_i) | i \in I\}) \\
    \text{if } \psi &= \neg \varphi \\
    & \text{then } \textsf{expr}_3(\psi) = \textsf{expr}_3(\varphi)
\end{align*}
}

\textit{(\textcolor{green}{4}) The \textnormal{maximal modal depth of other positive clauses in conjunctions} $\textsf{expr}_4$ measures the deepest positive modal depth aside from the deepest positive clause: $\text{HML}[\Sigma] \rightarrow (\mathbb{N} \cup \{\infty\})$ of a formula $\varphi$ is defined recursively by:}
\textit{
\begin{align*}
    \text{if } \varphi &= \langle a \rangle \psi \quad \text{ with } a \in \Sigma \\
    & \text{then } \textsf{expr}_4(\varphi) = \textsf{expr}_4(\psi) \\
    \text{if } \varphi &= \bigwedge_{i \in I} \{\ \psi_i \} \\
    & \text{then } \text{md}(\varphi) = \sup(\{\textsf{expr}_1(\psi_i)|i\in\text{Pos}\backslash \mathcal{R}\}\cup\{\textsf{expr}_4(\psi_i) | i \in I\}) \\
    \text{if } \psi &= \neg \varphi \\
    & \text{then } \textsf{expr}_4(\psi) = \textsf{expr}_4(\varphi)
\end{align*}
}

\textit{
(\textcolor{cyan}{5}) The \textnormal{maximal modal depth of negative clauses in conjunctions} $\textsf{expr}_5$ measures the deepest modal depth of the negative conjuncts of all conjunctions of a formula: $\text{HML}[\Sigma] \rightarrow (\mathbb{N} \cup \{\infty\})$ of a formula $\varphi$ is defined recursively by:
}
\textit{
\begin{align*}
    \text{if } \varphi &= \langle a \rangle \psi \quad \text{ with } a \in \Sigma \\
    & \text{then } \textsf{expr}_5(\varphi) = \textsf{expr}_5(\psi) \\
    \text{if } \varphi &= \bigwedge_{i \in I} \{ \psi_i \} \\
    & \text{then } \textsf{expr}_5(\varphi) = \sup(\{\textsf{expr}_1(\psi_i)| i \in \text{Neg}\}\cup \{\textsf{expr}_5(\psi_i)|i \in I\}) \\
    \text{if } \psi &= \neg \varphi \\
    & \text{then } \textsf{expr}_5(\psi) = \textsf{expr}_5(\varphi)
\end{align*}
}
\textit{(\textcolor{violet}{6}) The \textnormal{nesting depth of negations} $\textsf{expr}_{6}$ measures the maximal number of negations when traversing the syntax tree of a formula: $\text{HML}[\Sigma] \rightarrow (\mathbb{N} \cup \{\infty\})$ of a formula $\varphi$ is defined recursively by:}
\textit{
\begin{align*}
    \text{if } \varphi &= \langle a \rangle \psi \quad \text{ with } a \in \Sigma \\
    & \text{then } \textsf{expr}_6(\varphi) = \textsf{expr}_6(\psi) \\
    \text{if } \varphi &= \bigwedge_{i \in I} \{ \psi_i \} \\
    & \text{then } \textsf{expr}_6(\varphi) = \sup(\{\textsf{expr}_6(\psi_i)| i \in I\}) \\
    \text{if } \psi &= \neg \varphi \\
    & \text{then } \textsf{expr}_6(\psi) = 1 + \textsf{expr}_6(\varphi)
\end{align*}
}
\textit{where:}

$\textit{Neg} := \{i \in I \, | \, \exists \varphi'_i. \psi_i = \neg \varphi'_i\}$

$\textit{Pos} := I \setminus \text{Neg}$

$\mathcal{R} := \left\{
\begin{aligned}
&\varnothing \textit{ if } \textit{Pos} = \varnothing, \\
&\{ r \} \textit{ for some } r \in \textit{Pos} \text{ where } \textit{expr}_1(\psi_r) \textit{ maximal for \textit{Pos}}
\end{aligned}
\right\}.$

\textit{We combine this to the expressiveness price $\textsf{expr}: \text{HML}[\Sigma] \rightarrow (\mathbb{N \cup \infty})^6$ of a formula $\varphi$:
\[
\text{expr}(\varphi) :=
\begin{pmatrix}
\text{expr}_1(\varphi) \\
\text{expr}_2(\varphi) \\
\text{expr}_3(\varphi) \\
\text{expr}_4(\varphi) \\
\text{expr}_5(\varphi) \\
\text{expr}_6(\varphi) \\
\end{pmatrix}
\]
}

We show that $\textsf{expr}$ defines the same function as (\cite{bisping2023process}, definition 5) in (appendix).%
\end{isamarkuptext}\isamarkuptrue%
%
\begin{isamarkuptext}%
The formalization closely follows the structure outlined in the definition. Neg and Pos can easily be derived using our formalization of Conjunctions. \\
The function \isa{pos{\isacharunderscore}{\kern0pt}r} formalizes the set $Pos - \mathcal{R}$. It invokes the axiom of choice by selecting and removing a formula with maximal modal depth from Pos using the Hilbert choice operator \isa{SOME}.%
\end{isamarkuptext}\isamarkuptrue%
\isacommand{primrec}\isamarkupfalse%
\ expr{\isacharunderscore}{\kern0pt}{\isadigit{1}}\ {\isacharcolon}{\kern0pt}{\isacharcolon}{\kern0pt}\ {\isachardoublequoteopen}{\isacharparenleft}{\kern0pt}{\isacharprime}{\kern0pt}a{\isacharcomma}{\kern0pt}\ {\isacharprime}{\kern0pt}s{\isacharparenright}{\kern0pt}hml\ {\isasymRightarrow}\ enat{\isachardoublequoteclose}\isanewline
\ \ \isakeyword{where}\isanewline
expr{\isacharunderscore}{\kern0pt}{\isadigit{1}}{\isacharunderscore}{\kern0pt}tt{\isacharcolon}{\kern0pt}\ {\isacartoucheopen}expr{\isacharunderscore}{\kern0pt}{\isadigit{1}}\ TT\ {\isacharequal}{\kern0pt}\ {\isadigit{0}}{\isacartoucheclose}\ {\isacharbar}{\kern0pt}\isanewline
expr{\isacharunderscore}{\kern0pt}{\isadigit{1}}{\isacharunderscore}{\kern0pt}conj{\isacharcolon}{\kern0pt}\ {\isacartoucheopen}expr{\isacharunderscore}{\kern0pt}{\isadigit{1}}\ {\isacharparenleft}{\kern0pt}hml{\isacharunderscore}{\kern0pt}conj\ I\ J\ {\isasymPhi}{\isacharparenright}{\kern0pt}\ {\isacharequal}{\kern0pt}\ Sup\ {\isacharparenleft}{\kern0pt}{\isacharparenleft}{\kern0pt}expr{\isacharunderscore}{\kern0pt}{\isadigit{1}}\ {\isasymcirc}\ {\isasymPhi}{\isacharparenright}{\kern0pt}\ {\isacharbackquote}{\kern0pt}\ I\ {\isasymunion}\ {\isacharparenleft}{\kern0pt}expr{\isacharunderscore}{\kern0pt}{\isadigit{1}}\ {\isasymcirc}\ {\isasymPhi}{\isacharparenright}{\kern0pt}\ {\isacharbackquote}{\kern0pt}\ J{\isacharparenright}{\kern0pt}{\isacartoucheclose}\ {\isacharbar}{\kern0pt}\isanewline
expr{\isacharunderscore}{\kern0pt}{\isadigit{1}}{\isacharunderscore}{\kern0pt}pos{\isacharcolon}{\kern0pt}\ {\isacartoucheopen}expr{\isacharunderscore}{\kern0pt}{\isadigit{1}}\ {\isacharparenleft}{\kern0pt}hml{\isacharunderscore}{\kern0pt}pos\ {\isasymalpha}\ {\isasymphi}{\isacharparenright}{\kern0pt}\ {\isacharequal}{\kern0pt}\ \isanewline
\ \ {\isadigit{1}}\ {\isacharplus}{\kern0pt}\ {\isacharparenleft}{\kern0pt}expr{\isacharunderscore}{\kern0pt}{\isadigit{1}}\ {\isasymphi}{\isacharparenright}{\kern0pt}{\isacartoucheclose}\isanewline
\isanewline
\isacommand{primrec}\isamarkupfalse%
\ expr{\isacharunderscore}{\kern0pt}{\isadigit{2}}\ {\isacharcolon}{\kern0pt}{\isacharcolon}{\kern0pt}\ {\isachardoublequoteopen}{\isacharparenleft}{\kern0pt}{\isacharprime}{\kern0pt}a{\isacharcomma}{\kern0pt}\ {\isacharprime}{\kern0pt}s{\isacharparenright}{\kern0pt}hml\ {\isasymRightarrow}\ enat{\isachardoublequoteclose}\isanewline
\ \ \isakeyword{where}\isanewline
expr{\isacharunderscore}{\kern0pt}{\isadigit{2}}{\isacharunderscore}{\kern0pt}tt{\isacharcolon}{\kern0pt}\ {\isacartoucheopen}expr{\isacharunderscore}{\kern0pt}{\isadigit{2}}\ TT\ {\isacharequal}{\kern0pt}\ {\isadigit{1}}{\isacartoucheclose}\ {\isacharbar}{\kern0pt}\isanewline
expr{\isacharunderscore}{\kern0pt}{\isadigit{2}}{\isacharunderscore}{\kern0pt}conj{\isacharcolon}{\kern0pt}\ {\isacartoucheopen}expr{\isacharunderscore}{\kern0pt}{\isadigit{2}}\ {\isacharparenleft}{\kern0pt}hml{\isacharunderscore}{\kern0pt}conj\ I\ J\ {\isasymPhi}{\isacharparenright}{\kern0pt}\ {\isacharequal}{\kern0pt}\ {\isadigit{1}}\ {\isacharplus}{\kern0pt}\ Sup\ {\isacharparenleft}{\kern0pt}{\isacharparenleft}{\kern0pt}expr{\isacharunderscore}{\kern0pt}{\isadigit{2}}\ {\isasymcirc}\ {\isasymPhi}{\isacharparenright}{\kern0pt}\ {\isacharbackquote}{\kern0pt}\ I\ {\isasymunion}\ {\isacharparenleft}{\kern0pt}expr{\isacharunderscore}{\kern0pt}{\isadigit{2}}\ {\isasymcirc}\ {\isasymPhi}{\isacharparenright}{\kern0pt}\ {\isacharbackquote}{\kern0pt}\ J{\isacharparenright}{\kern0pt}{\isacartoucheclose}\ {\isacharbar}{\kern0pt}\isanewline
expr{\isacharunderscore}{\kern0pt}{\isadigit{2}}{\isacharunderscore}{\kern0pt}pos{\isacharcolon}{\kern0pt}\ {\isacartoucheopen}expr{\isacharunderscore}{\kern0pt}{\isadigit{2}}\ {\isacharparenleft}{\kern0pt}hml{\isacharunderscore}{\kern0pt}pos\ {\isasymalpha}\ {\isasymphi}{\isacharparenright}{\kern0pt}\ {\isacharequal}{\kern0pt}\ expr{\isacharunderscore}{\kern0pt}{\isadigit{2}}\ {\isasymphi}{\isacartoucheclose}\isanewline
\isanewline
\isacommand{primrec}\isamarkupfalse%
\ expr{\isacharunderscore}{\kern0pt}{\isadigit{3}}\ {\isacharcolon}{\kern0pt}{\isacharcolon}{\kern0pt}\ {\isachardoublequoteopen}{\isacharparenleft}{\kern0pt}{\isacharprime}{\kern0pt}a{\isacharcomma}{\kern0pt}\ {\isacharprime}{\kern0pt}s{\isacharparenright}{\kern0pt}\ hml\ {\isasymRightarrow}\ enat{\isachardoublequoteclose}\isanewline
\ \ \isakeyword{where}\isanewline
expr{\isacharunderscore}{\kern0pt}{\isadigit{3}}{\isacharunderscore}{\kern0pt}tt{\isacharcolon}{\kern0pt}\ {\isacartoucheopen}expr{\isacharunderscore}{\kern0pt}{\isadigit{3}}\ TT\ {\isacharequal}{\kern0pt}\ {\isadigit{0}}{\isacartoucheclose}\ {\isacharbar}{\kern0pt}\isanewline
expr{\isacharunderscore}{\kern0pt}{\isadigit{3}}{\isacharunderscore}{\kern0pt}pos{\isacharcolon}{\kern0pt}\ {\isacartoucheopen}expr{\isacharunderscore}{\kern0pt}{\isadigit{3}}\ {\isacharparenleft}{\kern0pt}hml{\isacharunderscore}{\kern0pt}pos\ {\isasymalpha}\ {\isasymphi}{\isacharparenright}{\kern0pt}\ {\isacharequal}{\kern0pt}\ expr{\isacharunderscore}{\kern0pt}{\isadigit{3}}\ {\isasymphi}{\isacartoucheclose}\ {\isacharbar}{\kern0pt}\ \isanewline
expr{\isacharunderscore}{\kern0pt}{\isadigit{3}}{\isacharunderscore}{\kern0pt}conj{\isacharcolon}{\kern0pt}\ {\isacartoucheopen}expr{\isacharunderscore}{\kern0pt}{\isadigit{3}}\ {\isacharparenleft}{\kern0pt}hml{\isacharunderscore}{\kern0pt}conj\ I\ J\ {\isasymPhi}{\isacharparenright}{\kern0pt}\ {\isacharequal}{\kern0pt}\ {\isacharparenleft}{\kern0pt}Sup\ {\isacharparenleft}{\kern0pt}{\isacharparenleft}{\kern0pt}expr{\isacharunderscore}{\kern0pt}{\isadigit{1}}\ {\isasymcirc}\ {\isasymPhi}{\isacharparenright}{\kern0pt}\ {\isacharbackquote}{\kern0pt}\ I\ {\isasymunion}\ {\isacharparenleft}{\kern0pt}expr{\isacharunderscore}{\kern0pt}{\isadigit{3}}\ {\isasymcirc}\ {\isasymPhi}{\isacharparenright}{\kern0pt}\ {\isacharbackquote}{\kern0pt}\ I\ {\isasymunion}\ {\isacharparenleft}{\kern0pt}expr{\isacharunderscore}{\kern0pt}{\isadigit{3}}\ {\isasymcirc}\ {\isasymPhi}{\isacharparenright}{\kern0pt}\ {\isacharbackquote}{\kern0pt}\ J{\isacharparenright}{\kern0pt}{\isacharparenright}{\kern0pt}{\isacartoucheclose}\isanewline
\isanewline
\isacommand{fun}\isamarkupfalse%
\ pos{\isacharunderscore}{\kern0pt}r\ {\isacharcolon}{\kern0pt}{\isacharcolon}{\kern0pt}\ {\isachardoublequoteopen}{\isacharparenleft}{\kern0pt}{\isacharprime}{\kern0pt}a{\isacharcomma}{\kern0pt}\ {\isacharprime}{\kern0pt}s{\isacharparenright}{\kern0pt}hml\ set\ {\isasymRightarrow}\ {\isacharparenleft}{\kern0pt}{\isacharprime}{\kern0pt}a{\isacharcomma}{\kern0pt}\ {\isacharprime}{\kern0pt}s{\isacharparenright}{\kern0pt}hml\ set{\isachardoublequoteclose}\isanewline
\ \ \isakeyword{where}\isanewline
{\isachardoublequoteopen}pos{\isacharunderscore}{\kern0pt}r\ xs\ {\isacharequal}{\kern0pt}\ {\isacharparenleft}{\kern0pt}\isanewline
let\ max{\isacharunderscore}{\kern0pt}val\ {\isacharequal}{\kern0pt}\ {\isacharparenleft}{\kern0pt}Sup\ {\isacharparenleft}{\kern0pt}expr{\isacharunderscore}{\kern0pt}{\isadigit{1}}\ {\isacharbackquote}{\kern0pt}\ xs{\isacharparenright}{\kern0pt}{\isacharparenright}{\kern0pt}{\isacharsemicolon}{\kern0pt}\ \isanewline
\ \ max{\isacharunderscore}{\kern0pt}elem\ {\isacharequal}{\kern0pt}\ {\isacharparenleft}{\kern0pt}SOME\ {\isasympsi}{\isachardot}{\kern0pt}\ {\isasympsi}\ {\isasymin}\ xs\ {\isasymand}\ expr{\isacharunderscore}{\kern0pt}{\isadigit{1}}\ {\isasympsi}\ {\isacharequal}{\kern0pt}\ max{\isacharunderscore}{\kern0pt}val{\isacharparenright}{\kern0pt}{\isacharsemicolon}{\kern0pt}\isanewline
\ \ xs{\isacharunderscore}{\kern0pt}new\ {\isacharequal}{\kern0pt}\ xs\ {\isacharminus}{\kern0pt}\ {\isacharbraceleft}{\kern0pt}max{\isacharunderscore}{\kern0pt}elem{\isacharbraceright}{\kern0pt}\isanewline
in\ xs{\isacharunderscore}{\kern0pt}new{\isacharparenright}{\kern0pt}{\isachardoublequoteclose}\isanewline
\isanewline
\isacommand{primrec}\isamarkupfalse%
\ expr{\isacharunderscore}{\kern0pt}{\isadigit{4}}\ {\isacharcolon}{\kern0pt}{\isacharcolon}{\kern0pt}\ {\isachardoublequoteopen}{\isacharparenleft}{\kern0pt}{\isacharprime}{\kern0pt}a{\isacharcomma}{\kern0pt}\ {\isacharprime}{\kern0pt}s{\isacharparenright}{\kern0pt}hml\ {\isasymRightarrow}\ enat{\isachardoublequoteclose}\ \isanewline
\ \ \isakeyword{where}\isanewline
expr{\isacharunderscore}{\kern0pt}{\isadigit{4}}{\isacharunderscore}{\kern0pt}tt{\isacharcolon}{\kern0pt}\ {\isachardoublequoteopen}expr{\isacharunderscore}{\kern0pt}{\isadigit{4}}\ TT\ {\isacharequal}{\kern0pt}\ {\isadigit{0}}{\isachardoublequoteclose}\ {\isacharbar}{\kern0pt}\isanewline
expr{\isacharunderscore}{\kern0pt}{\isadigit{4}}{\isacharunderscore}{\kern0pt}pos{\isacharcolon}{\kern0pt}\ {\isachardoublequoteopen}expr{\isacharunderscore}{\kern0pt}{\isadigit{4}}\ {\isacharparenleft}{\kern0pt}hml{\isacharunderscore}{\kern0pt}pos\ a\ {\isasymphi}{\isacharparenright}{\kern0pt}\ {\isacharequal}{\kern0pt}\ expr{\isacharunderscore}{\kern0pt}{\isadigit{4}}\ {\isasymphi}{\isachardoublequoteclose}\ {\isacharbar}{\kern0pt}\isanewline
expr{\isacharunderscore}{\kern0pt}{\isadigit{4}}{\isacharunderscore}{\kern0pt}conj{\isacharcolon}{\kern0pt}\ {\isachardoublequoteopen}expr{\isacharunderscore}{\kern0pt}{\isadigit{4}}\ {\isacharparenleft}{\kern0pt}hml{\isacharunderscore}{\kern0pt}conj\ I\ J\ {\isasymPhi}{\isacharparenright}{\kern0pt}\ {\isacharequal}{\kern0pt}\ Sup\ {\isacharparenleft}{\kern0pt}{\isacharparenleft}{\kern0pt}expr{\isacharunderscore}{\kern0pt}{\isadigit{1}}\ {\isacharbackquote}{\kern0pt}\ {\isacharparenleft}{\kern0pt}pos{\isacharunderscore}{\kern0pt}r\ {\isacharparenleft}{\kern0pt}{\isasymPhi}\ {\isacharbackquote}{\kern0pt}\ I{\isacharparenright}{\kern0pt}{\isacharparenright}{\kern0pt}{\isacharparenright}{\kern0pt}\ \ {\isasymunion}\ {\isacharparenleft}{\kern0pt}expr{\isacharunderscore}{\kern0pt}{\isadigit{4}}\ {\isasymcirc}\ {\isasymPhi}{\isacharparenright}{\kern0pt}\ {\isacharbackquote}{\kern0pt}\ I\ {\isasymunion}\ {\isacharparenleft}{\kern0pt}expr{\isacharunderscore}{\kern0pt}{\isadigit{4}}\ {\isasymcirc}\ {\isasymPhi}{\isacharparenright}{\kern0pt}\ {\isacharbackquote}{\kern0pt}\ J{\isacharparenright}{\kern0pt}{\isachardoublequoteclose}\isanewline
\isanewline
\isacommand{primrec}\isamarkupfalse%
\ expr{\isacharunderscore}{\kern0pt}{\isadigit{5}}\ {\isacharcolon}{\kern0pt}{\isacharcolon}{\kern0pt}\ {\isachardoublequoteopen}{\isacharparenleft}{\kern0pt}{\isacharprime}{\kern0pt}a{\isacharcomma}{\kern0pt}\ {\isacharprime}{\kern0pt}s{\isacharparenright}{\kern0pt}hml\ {\isasymRightarrow}\ enat{\isachardoublequoteclose}\isanewline
\ \ \isakeyword{where}\isanewline
expr{\isacharunderscore}{\kern0pt}{\isadigit{5}}{\isacharunderscore}{\kern0pt}tt{\isacharcolon}{\kern0pt}\ {\isacartoucheopen}expr{\isacharunderscore}{\kern0pt}{\isadigit{5}}\ TT\ {\isacharequal}{\kern0pt}\ {\isadigit{0}}{\isacartoucheclose}\ {\isacharbar}{\kern0pt}\isanewline
expr{\isacharunderscore}{\kern0pt}{\isadigit{5}}{\isacharunderscore}{\kern0pt}pos{\isacharcolon}{\kern0pt}{\isacartoucheopen}expr{\isacharunderscore}{\kern0pt}{\isadigit{5}}\ {\isacharparenleft}{\kern0pt}hml{\isacharunderscore}{\kern0pt}pos\ {\isasymalpha}\ {\isasymphi}{\isacharparenright}{\kern0pt}\ {\isacharequal}{\kern0pt}\ expr{\isacharunderscore}{\kern0pt}{\isadigit{5}}\ {\isasymphi}{\isacartoucheclose}{\isacharbar}{\kern0pt}\isanewline
expr{\isacharunderscore}{\kern0pt}{\isadigit{5}}{\isacharunderscore}{\kern0pt}conj{\isacharcolon}{\kern0pt}\ {\isacartoucheopen}expr{\isacharunderscore}{\kern0pt}{\isadigit{5}}\ {\isacharparenleft}{\kern0pt}hml{\isacharunderscore}{\kern0pt}conj\ I\ J\ {\isasymPhi}{\isacharparenright}{\kern0pt}\ {\isacharequal}{\kern0pt}\ \isanewline
{\isacharparenleft}{\kern0pt}Sup\ {\isacharparenleft}{\kern0pt}{\isacharparenleft}{\kern0pt}expr{\isacharunderscore}{\kern0pt}{\isadigit{5}}\ {\isasymcirc}\ {\isasymPhi}{\isacharparenright}{\kern0pt}\ {\isacharbackquote}{\kern0pt}\ I\ {\isasymunion}\ {\isacharparenleft}{\kern0pt}expr{\isacharunderscore}{\kern0pt}{\isadigit{5}}\ {\isasymcirc}\ {\isasymPhi}{\isacharparenright}{\kern0pt}\ {\isacharbackquote}{\kern0pt}\ J\ {\isasymunion}\ {\isacharparenleft}{\kern0pt}expr{\isacharunderscore}{\kern0pt}{\isadigit{1}}\ {\isasymcirc}\ {\isasymPhi}{\isacharparenright}{\kern0pt}\ {\isacharbackquote}{\kern0pt}\ J{\isacharparenright}{\kern0pt}{\isacharparenright}{\kern0pt}{\isacartoucheclose}\isanewline
\isanewline
\isacommand{primrec}\isamarkupfalse%
\ expr{\isacharunderscore}{\kern0pt}{\isadigit{6}}\ {\isacharcolon}{\kern0pt}{\isacharcolon}{\kern0pt}\ {\isachardoublequoteopen}{\isacharparenleft}{\kern0pt}{\isacharprime}{\kern0pt}a{\isacharcomma}{\kern0pt}\ {\isacharprime}{\kern0pt}s{\isacharparenright}{\kern0pt}hml\ {\isasymRightarrow}\ enat{\isachardoublequoteclose}\isanewline
\ \ \isakeyword{where}\isanewline
expr{\isacharunderscore}{\kern0pt}{\isadigit{6}}{\isacharunderscore}{\kern0pt}tt{\isacharcolon}{\kern0pt}\ {\isacartoucheopen}expr{\isacharunderscore}{\kern0pt}{\isadigit{6}}\ TT\ {\isacharequal}{\kern0pt}\ {\isadigit{0}}{\isacartoucheclose}\ {\isacharbar}{\kern0pt}\isanewline
expr{\isacharunderscore}{\kern0pt}{\isadigit{6}}{\isacharunderscore}{\kern0pt}pos{\isacharcolon}{\kern0pt}\ {\isacartoucheopen}expr{\isacharunderscore}{\kern0pt}{\isadigit{6}}\ {\isacharparenleft}{\kern0pt}hml{\isacharunderscore}{\kern0pt}pos\ {\isasymalpha}\ {\isasymphi}{\isacharparenright}{\kern0pt}\ {\isacharequal}{\kern0pt}\ expr{\isacharunderscore}{\kern0pt}{\isadigit{6}}\ {\isasymphi}{\isacartoucheclose}{\isacharbar}{\kern0pt}\isanewline
expr{\isacharunderscore}{\kern0pt}{\isadigit{6}}{\isacharunderscore}{\kern0pt}conj{\isacharcolon}{\kern0pt}\ {\isacartoucheopen}expr{\isacharunderscore}{\kern0pt}{\isadigit{6}}\ {\isacharparenleft}{\kern0pt}hml{\isacharunderscore}{\kern0pt}conj\ I\ J\ {\isasymPhi}{\isacharparenright}{\kern0pt}\ {\isacharequal}{\kern0pt}\ \isanewline
{\isacharparenleft}{\kern0pt}Sup\ {\isacharparenleft}{\kern0pt}{\isacharparenleft}{\kern0pt}expr{\isacharunderscore}{\kern0pt}{\isadigit{6}}\ {\isasymcirc}\ {\isasymPhi}{\isacharparenright}{\kern0pt}\ {\isacharbackquote}{\kern0pt}\ I\ {\isasymunion}\ {\isacharparenleft}{\kern0pt}{\isacharparenleft}{\kern0pt}eSuc\ {\isasymcirc}\ expr{\isacharunderscore}{\kern0pt}{\isadigit{6}}\ {\isasymcirc}\ {\isasymPhi}{\isacharparenright}{\kern0pt}\ {\isacharbackquote}{\kern0pt}\ J{\isacharparenright}{\kern0pt}{\isacharparenright}{\kern0pt}{\isacharparenright}{\kern0pt}{\isacartoucheclose}\isanewline
\isanewline
\isacommand{fun}\isamarkupfalse%
\ expr\ {\isacharcolon}{\kern0pt}{\isacharcolon}{\kern0pt}\ {\isachardoublequoteopen}{\isacharparenleft}{\kern0pt}{\isacharprime}{\kern0pt}a{\isacharcomma}{\kern0pt}\ {\isacharprime}{\kern0pt}s{\isacharparenright}{\kern0pt}hml\ {\isasymRightarrow}\ enat\ {\isasymtimes}\ enat\ {\isasymtimes}\ enat\ {\isasymtimes}\ \ enat\ {\isasymtimes}\ enat\ {\isasymtimes}\ enat{\isachardoublequoteclose}\ \isanewline
\ \ \isakeyword{where}\isanewline
{\isacartoucheopen}expr\ {\isasymphi}\ {\isacharequal}{\kern0pt}\ {\isacharparenleft}{\kern0pt}expr{\isacharunderscore}{\kern0pt}{\isadigit{1}}\ {\isasymphi}{\isacharcomma}{\kern0pt}\ expr{\isacharunderscore}{\kern0pt}{\isadigit{2}}\ {\isasymphi}{\isacharcomma}{\kern0pt}\ expr{\isacharunderscore}{\kern0pt}{\isadigit{3}}\ {\isasymphi}{\isacharcomma}{\kern0pt}\ expr{\isacharunderscore}{\kern0pt}{\isadigit{4}}\ {\isasymphi}{\isacharcomma}{\kern0pt}\ expr{\isacharunderscore}{\kern0pt}{\isadigit{5}}\ {\isasymphi}{\isacharcomma}{\kern0pt}\ expr{\isacharunderscore}{\kern0pt}{\isadigit{6}}\ {\isasymphi}{\isacharparenright}{\kern0pt}{\isacartoucheclose}%
\begin{isamarkuptext}%
Prices are compared component wise, i.e., $(e_1, \ldots e_6) \leq (f_1 \ldots f_6)$ iff $e_i \leq f_i$ for each $i$.%
\end{isamarkuptext}\isamarkuptrue%
\isacommand{fun}\isamarkupfalse%
\ less{\isacharunderscore}{\kern0pt}eq{\isacharunderscore}{\kern0pt}t\ {\isacharcolon}{\kern0pt}{\isacharcolon}{\kern0pt}\ {\isachardoublequoteopen}{\isacharparenleft}{\kern0pt}enat\ {\isasymtimes}\ enat\ {\isasymtimes}\ enat\ {\isasymtimes}\ enat\ {\isasymtimes}\ enat\ {\isasymtimes}\ enat{\isacharparenright}{\kern0pt}\ {\isasymRightarrow}\ {\isacharparenleft}{\kern0pt}enat\ {\isasymtimes}\ enat\ {\isasymtimes}\ enat\ {\isasymtimes}\ enat\ {\isasymtimes}\ enat\ {\isasymtimes}\ enat{\isacharparenright}{\kern0pt}\ {\isasymRightarrow}\ bool{\isachardoublequoteclose}\isanewline
\ \ \isakeyword{where}\isanewline
{\isachardoublequoteopen}less{\isacharunderscore}{\kern0pt}eq{\isacharunderscore}{\kern0pt}t\ {\isacharparenleft}{\kern0pt}n{\isadigit{1}}{\isacharcomma}{\kern0pt}\ n{\isadigit{2}}{\isacharcomma}{\kern0pt}\ n{\isadigit{3}}{\isacharcomma}{\kern0pt}\ n{\isadigit{4}}{\isacharcomma}{\kern0pt}\ n{\isadigit{5}}{\isacharcomma}{\kern0pt}\ n{\isadigit{6}}{\isacharparenright}{\kern0pt}\ {\isacharparenleft}{\kern0pt}i{\isadigit{1}}{\isacharcomma}{\kern0pt}\ i{\isadigit{2}}{\isacharcomma}{\kern0pt}\ i{\isadigit{3}}{\isacharcomma}{\kern0pt}\ i{\isadigit{4}}{\isacharcomma}{\kern0pt}\ i{\isadigit{5}}{\isacharcomma}{\kern0pt}\ i{\isadigit{6}}{\isacharparenright}{\kern0pt}\ {\isacharequal}{\kern0pt}\isanewline
\ \ \ \ {\isacharparenleft}{\kern0pt}n{\isadigit{1}}\ {\isasymle}\ i{\isadigit{1}}\ {\isasymand}\ n{\isadigit{2}}\ {\isasymle}\ i{\isadigit{2}}\ {\isasymand}\ n{\isadigit{3}}\ {\isasymle}\ i{\isadigit{3}}\ {\isasymand}\ n{\isadigit{4}}\ {\isasymle}\ i{\isadigit{4}}\ {\isasymand}\ n{\isadigit{5}}\ {\isasymle}\ i{\isadigit{5}}\ {\isasymand}\ n{\isadigit{6}}\ {\isasymle}\ i{\isadigit{6}}{\isacharparenright}{\kern0pt}{\isachardoublequoteclose}\isanewline
\isanewline
\isacommand{definition}\isamarkupfalse%
\ less\ \isakeyword{where}\isanewline
{\isachardoublequoteopen}less\ x\ y\ {\isasymequiv}\ less{\isacharunderscore}{\kern0pt}eq{\isacharunderscore}{\kern0pt}t\ x\ y\ {\isasymand}\ {\isasymnot}\ {\isacharparenleft}{\kern0pt}less{\isacharunderscore}{\kern0pt}eq{\isacharunderscore}{\kern0pt}t\ y\ x{\isacharparenright}{\kern0pt}{\isachardoublequoteclose}\isanewline
%
\begin{isamarkuptext}%
\begin{figure}[t]
    \centering
\begin{multicols}{2}

\columnbreak
\begin{tikzpicture}[-, >=stealth, node distance=1.2cm, every node/.style={minimum size=0.3cm}]
    \node (p1) {$\langle a \rangle$};
    \node (p2) [below of=p1] {$\bigwedge$};
    \node (p3) [below left=of p2] {$\langle b \rangle$};
    \node (p4) [below right=of p2] {$\langle a \rangle$};
    \node (p5) [below of=p3] {$\bigwedge$};
    \node (p6) [below of=p4] {$\bigwedge$};
    \node (p7) [below left of=p6] {$\lnot$};
    \node (p8) [below right of=p6] {$\lnot$};
    \node (p9) [below of=p7] {$\langle a \rangle$};
    \node (p10) [below of=p9] {$\langle c \rangle$};
    \node (p11) [below of=p10] {$\bigwedge$};
    \node (p12) [below of=p8] {$\langle b \rangle$};
    \node (p13) [below of=p12] {$\bigwedge$};

    \path[-] (p1) edge (p2);
    \path[-] (p2) edge (p3);
    \path[-] (p2) edge (p4);
    \path[-] (p3) edge (p5);
    \path[-] (p4) edge (p6);
    \path[-] (p6) edge (p7);
    \path[-] (p6) edge (p8);
    \path[-] (p7) edge (p9);
    \path[-] (p9) edge (p10);
    \path[-] (p10) edge (p11);
    \path[-] (p8) edge (p12);
    \path[-] (p12) edge (p13);

  \node[circle, draw=red, fill=red, inner sep=0.3pt, minimum size=3pt] (v11) at (4,0) {};
  \node[circle, draw=red, fill=red, inner sep=0.3pt, minimum size=3pt] (v12) at (4,-2.67) {};
  \node[circle, draw=red, fill=red, inner sep=0.3pt, minimum size=3pt] (v12) at (4,-6) {};
  \node[circle, draw=red, fill=red, inner sep=0.3pt, minimum size=3pt] (v13) at (4,-7.2) {};
  \node (v14) at (4,-8.7) {};
  \node (v21) at (5,0.2) {};
  \node[circle, draw=orange, fill=orange, inner sep=0.3pt, minimum size=3pt] (v22) at (5,-1.2) {};
  \node[circle, draw=orange, fill=orange, inner sep=0.3pt, minimum size=3pt] (v23) at (5,-3.8) {};
  \node[circle, draw=orange, fill=orange, inner sep=0.3pt, minimum size=3pt] (v24) at (5,-7.2) {};
  \node (v25) at (5,-8.7) {};
  \node (v31) at (6,0.2) {};
  \node[circle, draw=myyellow, fill=myyellow, inner sep=0.3pt, minimum size=3pt] (v32) at (6,-2.67) {};
  \node[circle, draw=myyellow, fill=myyellow, inner sep=0.3pt, minimum size=3pt] (v32) at (6,-6) {};
  \node[circle, draw=myyellow, fill=myyellow, inner sep=0.3pt, minimum size=3pt] (v33) at (6,-7.2) {};
  \node (v34) at (6,-8.7) {};
  \node (v41) at (7,0.2) {};
  \node[circle, draw=green, fill=green, inner sep=0.3pt, minimum size=3pt] (v42) at (7,-2.67) {};
  \node (v43) at (7,-8.7) {};
  \node (v51) at (8,0.2) {};
  \node[circle, draw=cyan, fill=cyan, inner sep=0.3pt, minimum size=3pt] (v52) at (8,-6) {};
  \node[circle, draw=cyan, fill=cyan, inner sep=0.3pt, minimum size=3pt] (v53) at (8,-7.2) {};
  \node (v54) at (8,-8.7) {};
  \node (v61) at (9,0.2) {};
  \node[circle, draw=violet, fill=violet, inner sep=0.3pt, minimum size=3pt] (v62) at (9,-5.2) {};
  \node (v63) at (9,-8.7) {};

  \path[-, red] (v11) edge (v14);
  \path[-, orange] (v21) edge (v25);
  \path[-, myyellow] (v31) edge (v34);
  \path[-, green] (v41) edge (v43);
  \path[-, cyan] (v51) edge (v54);
  \path[-, violet] (v61) edge (v63);


\end{tikzpicture}
\end{multicols}
\caption{Pricing of formula $\langle a \rangle \bigwedge \{\langle b \rangle, \langle a \rangle \bigwedge \{\lnot \langle a \rangle \langle c \rangle, \lnot \langle b \rangle\}\}$}
    \label{fig:2_3}
\end{figure}%
\end{isamarkuptext}\isamarkuptrue%
\isacommand{lemma}\isamarkupfalse%
\ example{\isacharunderscore}{\kern0pt}{\isadigit{2}}{\isacharunderscore}{\kern0pt}{\isadigit{3}}{\isacharcolon}{\kern0pt}\isanewline
\ \ \isakeyword{fixes}\ s\ \isakeyword{and}\ t\ \isakeyword{and}\ a\ \isakeyword{and}\ b\ \isakeyword{and}\ c\isanewline
\ \ \isakeyword{assumes}\ {\isachardoublequoteopen}s\ {\isasymnoteq}\ t{\isachardoublequoteclose}\isanewline
\ \ \isakeyword{defines}\ {\isasymphi}{\isacharcolon}{\kern0pt}\ {\isachardoublequoteopen}{\isacharparenleft}{\kern0pt}{\isasymphi}{\isacharcolon}{\kern0pt}{\isacharcolon}{\kern0pt}{\isacharparenleft}{\kern0pt}{\isacharprime}{\kern0pt}a{\isacharcomma}{\kern0pt}\ {\isacharprime}{\kern0pt}s{\isacharparenright}{\kern0pt}hml{\isacharparenright}{\kern0pt}\ {\isasymequiv}\ \isanewline
\ \ {\isacharparenleft}{\kern0pt}hml{\isacharunderscore}{\kern0pt}pos\ a\ \isanewline
\ \ \ \ {\isacharparenleft}{\kern0pt}hml{\isacharunderscore}{\kern0pt}conj\ {\isacharbraceleft}{\kern0pt}s{\isacharcomma}{\kern0pt}\ t{\isacharbraceright}{\kern0pt}\ {\isacharbraceleft}{\kern0pt}{\isacharbraceright}{\kern0pt}\isanewline
\ \ \ \ \ \ {\isacharparenleft}{\kern0pt}{\isasymlambda}i{\isachardot}{\kern0pt}\ {\isacharparenleft}{\kern0pt}if\ i\ {\isacharequal}{\kern0pt}\ s\ \isanewline
\ \ \ \ \ \ \ \ \ \ \ \ then\ {\isacharparenleft}{\kern0pt}hml{\isacharunderscore}{\kern0pt}pos\ b\ TT{\isacharparenright}{\kern0pt}\ \isanewline
\ \ \ \ \ \ \ \ \ \ \ \ else\ \isanewline
\ \ \ \ \ \ \ \ \ \ \ \ \ \ {\isacharparenleft}{\kern0pt}if\ i\ {\isacharequal}{\kern0pt}\ t\ \ \isanewline
\ \ \ \ \ \ \ \ \ \ \ \ \ \ \ then\ {\isacharparenleft}{\kern0pt}hml{\isacharunderscore}{\kern0pt}pos\ a\ \isanewline
\ \ \ \ \ \ \ \ \ \ \ \ \ \ \ \ \ \ \ \ \ \ {\isacharparenleft}{\kern0pt}hml{\isacharunderscore}{\kern0pt}conj\ {\isacharbraceleft}{\kern0pt}{\isacharbraceright}{\kern0pt}\ {\isacharbraceleft}{\kern0pt}s{\isacharcomma}{\kern0pt}\ t{\isacharbraceright}{\kern0pt}\ \isanewline
\ \ \ \ \ \ \ \ \ \ \ \ \ \ \ \ \ \ \ \ \ \ \ \ {\isacharparenleft}{\kern0pt}{\isasymlambda}j{\isachardot}{\kern0pt}\ {\isacharparenleft}{\kern0pt}if\ j\ {\isacharequal}{\kern0pt}\ s\ \isanewline
\ \ \ \ \ \ \ \ \ \ \ \ \ \ \ \ \ \ \ \ \ \ \ \ \ \ \ \ \ \ then\ {\isacharparenleft}{\kern0pt}hml{\isacharunderscore}{\kern0pt}pos\ a\ \isanewline
\ \ \ \ \ \ \ \ \ \ \ \ \ \ \ \ \ \ \ \ \ \ \ \ \ \ \ \ \ \ \ \ \ \ \ \ \ {\isacharparenleft}{\kern0pt}hml{\isacharunderscore}{\kern0pt}pos\ c\ TT{\isacharparenright}{\kern0pt}{\isacharparenright}{\kern0pt}\ \isanewline
\ \ \ \ \ \ \ \ \ \ \ \ \ \ \ \ \ \ \ \ \ \ \ \ \ \ \ \ \ \ else\ \isanewline
\ \ \ \ \ \ \ \ \ \ \ \ \ \ \ \ \ \ \ \ \ \ \ \ \ \ \ \ \ \ \ \ {\isacharparenleft}{\kern0pt}if\ j\ {\isacharequal}{\kern0pt}\ t\ \isanewline
\ \ \ \ \ \ \ \ \ \ \ \ \ \ \ \ \ \ \ \ \ \ \ \ \ \ \ \ \ \ \ \ \ then\ {\isacharparenleft}{\kern0pt}hml{\isacharunderscore}{\kern0pt}pos\ b\ TT{\isacharparenright}{\kern0pt}\ \isanewline
\ \ \ \ \ \ \ \ \ \ \ \ \ \ \ \ \ \ \ \ \ \ \ \ \ \ \ \ \ \ \ \ \ else\ undefined{\isacharparenright}{\kern0pt}{\isacharparenright}{\kern0pt}{\isacharparenright}{\kern0pt}{\isacharparenright}{\kern0pt}{\isacharparenright}{\kern0pt}\isanewline
\ \ \ \ \ \ \ \ \ \ \ \ \ \ \ else\ undefined{\isacharparenright}{\kern0pt}{\isacharparenright}{\kern0pt}{\isacharparenright}{\kern0pt}{\isacharparenright}{\kern0pt}{\isacharparenright}{\kern0pt}{\isachardoublequoteclose}\isanewline
\isanewline
\isakeyword{shows}\ \isanewline
\ \ {\isachardoublequoteopen}expr{\isacharunderscore}{\kern0pt}{\isadigit{1}}\ {\isasymphi}\ {\isacharequal}{\kern0pt}\ {\isadigit{4}}{\isachardoublequoteclose}\isanewline
\ \ {\isachardoublequoteopen}expr{\isacharunderscore}{\kern0pt}{\isadigit{2}}\ {\isasymphi}\ {\isacharequal}{\kern0pt}\ {\isadigit{3}}{\isachardoublequoteclose}\isanewline
\ \ {\isachardoublequoteopen}expr{\isacharunderscore}{\kern0pt}{\isadigit{3}}\ {\isasymphi}\ {\isacharequal}{\kern0pt}\ {\isadigit{3}}{\isachardoublequoteclose}\isanewline
\ \ {\isachardoublequoteopen}expr{\isacharunderscore}{\kern0pt}{\isadigit{4}}\ {\isasymphi}\ {\isacharequal}{\kern0pt}\ {\isadigit{1}}{\isachardoublequoteclose}\isanewline
\ \ {\isachardoublequoteopen}expr{\isacharunderscore}{\kern0pt}{\isadigit{5}}\ {\isasymphi}\ {\isacharequal}{\kern0pt}\ {\isadigit{2}}{\isachardoublequoteclose}\isanewline
\ \ {\isachardoublequoteopen}expr{\isacharunderscore}{\kern0pt}{\isadigit{6}}\ {\isasymphi}\ {\isacharequal}{\kern0pt}\ {\isadigit{1}}{\isachardoublequoteclose}\isanewline
%
\isadelimproof
\ \ %
\endisadelimproof
%
\isatagproof
\isacommand{proof}\isamarkupfalse%
{\isacharminus}{\kern0pt}\ \isanewline
\ \ \ \ \isacommand{have}\isamarkupfalse%
\ {\isachardoublequoteopen}expr{\isacharunderscore}{\kern0pt}{\isadigit{1}}\ {\isacharparenleft}{\kern0pt}hml{\isacharunderscore}{\kern0pt}pos\ a\ {\isacharparenleft}{\kern0pt}hml{\isacharunderscore}{\kern0pt}pos\ c\ TT{\isacharparenright}{\kern0pt}{\isacharparenright}{\kern0pt}\ {\isacharequal}{\kern0pt}\ {\isadigit{2}}{\isachardoublequoteclose}\ {\isachardoublequoteopen}expr{\isacharunderscore}{\kern0pt}{\isadigit{1}}\ {\isacharparenleft}{\kern0pt}hml{\isacharunderscore}{\kern0pt}pos\ b\ TT{\isacharparenright}{\kern0pt}\ {\isacharequal}{\kern0pt}\ {\isadigit{1}}{\isachardoublequoteclose}\isanewline
\ \ \ \ \ \ \isacommand{by}\isamarkupfalse%
\ simp{\isacharplus}{\kern0pt}\isanewline
\ \ \ \ \isacommand{hence}\isamarkupfalse%
\ {\isachardoublequoteopen}Sup\ {\isacharparenleft}{\kern0pt}{\isacharparenleft}{\kern0pt}expr{\isacharunderscore}{\kern0pt}{\isadigit{1}}\ {\isasymcirc}\ {\isacharparenleft}{\kern0pt}{\isasymlambda}j{\isachardot}{\kern0pt}\ if\ j\ {\isacharequal}{\kern0pt}\ s\ then\ hml{\isacharunderscore}{\kern0pt}pos\ a\ {\isacharparenleft}{\kern0pt}hml{\isacharunderscore}{\kern0pt}pos\ c\ TT{\isacharparenright}{\kern0pt}\isanewline
\ \ \ \ \ \ \ \ \ \ \ \ \ \ \ \ \ \ \ \ \ \ \ \ \ \ \ \ \ \ \ \ \ \ \ \ else\ if\ j\ {\isacharequal}{\kern0pt}\ t\ then\ hml{\isacharunderscore}{\kern0pt}pos\ b\ TT\ else\ undefined{\isacharparenright}{\kern0pt}{\isacharparenright}{\kern0pt}\ {\isacharbackquote}{\kern0pt}\ {\isacharbraceleft}{\kern0pt}s{\isacharcomma}{\kern0pt}\ t{\isacharbraceright}{\kern0pt}{\isacharparenright}{\kern0pt}\ {\isacharequal}{\kern0pt}\ {\isadigit{2}}{\isachardoublequoteclose}\isanewline
\ \ \ \ \ \ \isacommand{by}\isamarkupfalse%
\ {\isacharparenleft}{\kern0pt}smt\ {\isacharparenleft}{\kern0pt}verit{\isacharcomma}{\kern0pt}\ best{\isacharparenright}{\kern0pt}\ SUP{\isacharunderscore}{\kern0pt}insert\ Sup{\isacharunderscore}{\kern0pt}empty\ assms{\isacharparenleft}{\kern0pt}{\isadigit{1}}{\isacharparenright}{\kern0pt}\ ile{\isacharunderscore}{\kern0pt}eSuc\ image{\isacharunderscore}{\kern0pt}is{\isacharunderscore}{\kern0pt}empty\ o{\isacharunderscore}{\kern0pt}apply\ one{\isacharunderscore}{\kern0pt}add{\isacharunderscore}{\kern0pt}one\ plus{\isacharunderscore}{\kern0pt}{\isadigit{1}}{\isacharunderscore}{\kern0pt}eSuc{\isacharparenleft}{\kern0pt}{\isadigit{1}}{\isacharparenright}{\kern0pt}\ sup{\isachardot}{\kern0pt}orderE\ sup{\isacharunderscore}{\kern0pt}bot{\isachardot}{\kern0pt}right{\isacharunderscore}{\kern0pt}neutral{\isacharparenright}{\kern0pt}\isanewline
\ \ \ \ \isacommand{hence}\isamarkupfalse%
\ {\isachardoublequoteopen}expr{\isacharunderscore}{\kern0pt}{\isadigit{1}}\ {\isacharparenleft}{\kern0pt}hml{\isacharunderscore}{\kern0pt}conj\ {\isacharbraceleft}{\kern0pt}{\isacharbraceright}{\kern0pt}\ {\isacharbraceleft}{\kern0pt}s{\isacharcomma}{\kern0pt}\ t{\isacharbraceright}{\kern0pt}\isanewline
\ \ \ \ \ \ \ \ \ \ \ \ \ \ \ \ \ \ \ \ \ \ \ \ \ \ \ \ \ \ {\isacharparenleft}{\kern0pt}{\isasymlambda}j{\isachardot}{\kern0pt}\ if\ j\ {\isacharequal}{\kern0pt}\ s\ then\ hml{\isacharunderscore}{\kern0pt}pos\ a\ {\isacharparenleft}{\kern0pt}hml{\isacharunderscore}{\kern0pt}pos\ c\ TT{\isacharparenright}{\kern0pt}\isanewline
\ \ \ \ \ \ \ \ \ \ \ \ \ \ \ \ \ \ \ \ \ \ \ \ \ \ \ \ \ \ \ \ \ \ \ \ else\ if\ j\ {\isacharequal}{\kern0pt}\ t\ then\ hml{\isacharunderscore}{\kern0pt}pos\ b\ TT\ else\ undefined{\isacharparenright}{\kern0pt}{\isacharparenright}{\kern0pt}\ {\isacharequal}{\kern0pt}\ {\isadigit{2}}{\isachardoublequoteclose}\isanewline
\ \ \ \ \ \ \isacommand{using}\isamarkupfalse%
\ expr{\isacharunderscore}{\kern0pt}{\isadigit{1}}{\isacharunderscore}{\kern0pt}conj\ \isanewline
\ \ \ \ \ \ \isacommand{by}\isamarkupfalse%
\ auto\isanewline
\ \ \ \ \isacommand{hence}\isamarkupfalse%
\ {\isadigit{1}}{\isacharcolon}{\kern0pt}\ {\isachardoublequoteopen}expr{\isacharunderscore}{\kern0pt}{\isadigit{1}}\ {\isacharparenleft}{\kern0pt}hml{\isacharunderscore}{\kern0pt}pos\ a\isanewline
\ \ \ \ \ \ \ \ \ \ \ \ \ \ \ \ \ \ \ \ \ \ \ \ \ \ \ \ {\isacharparenleft}{\kern0pt}hml{\isacharunderscore}{\kern0pt}conj\ {\isacharbraceleft}{\kern0pt}{\isacharbraceright}{\kern0pt}\ {\isacharbraceleft}{\kern0pt}s{\isacharcomma}{\kern0pt}\ t{\isacharbraceright}{\kern0pt}\isanewline
\ \ \ \ \ \ \ \ \ \ \ \ \ \ \ \ \ \ \ \ \ \ \ \ \ \ \ \ \ \ {\isacharparenleft}{\kern0pt}{\isasymlambda}j{\isachardot}{\kern0pt}\ if\ j\ {\isacharequal}{\kern0pt}\ s\ then\ hml{\isacharunderscore}{\kern0pt}pos\ a\ {\isacharparenleft}{\kern0pt}hml{\isacharunderscore}{\kern0pt}pos\ c\ TT{\isacharparenright}{\kern0pt}\isanewline
\ \ \ \ \ \ \ \ \ \ \ \ \ \ \ \ \ \ \ \ \ \ \ \ \ \ \ \ \ \ \ \ \ \ \ \ else\ if\ j\ {\isacharequal}{\kern0pt}\ t\ then\ hml{\isacharunderscore}{\kern0pt}pos\ b\ TT\ else\ undefined{\isacharparenright}{\kern0pt}{\isacharparenright}{\kern0pt}{\isacharparenright}{\kern0pt}\ {\isacharequal}{\kern0pt}\ {\isadigit{3}}{\isachardoublequoteclose}\isanewline
\ \ \ \ \ \ \isakeyword{and}\ {\isadigit{2}}{\isacharcolon}{\kern0pt}\ {\isachardoublequoteopen}expr{\isacharunderscore}{\kern0pt}{\isadigit{1}}\ {\isacharparenleft}{\kern0pt}\ hml{\isacharunderscore}{\kern0pt}pos\ b\ TT{\isacharparenright}{\kern0pt}\ {\isacharequal}{\kern0pt}\ {\isadigit{1}}{\isachardoublequoteclose}\isanewline
\ \ \ \ \ \ \isacommand{by}\isamarkupfalse%
\ force{\isacharplus}{\kern0pt}\isanewline
\ \ \ \ \isacommand{hence}\isamarkupfalse%
\ {\isachardoublequoteopen}Sup\ {\isacharparenleft}{\kern0pt}{\isacharparenleft}{\kern0pt}expr{\isacharunderscore}{\kern0pt}{\isadigit{1}}\ {\isasymcirc}\ {\isacharparenleft}{\kern0pt}{\isasymlambda}i{\isachardot}{\kern0pt}\ if\ i\ {\isacharequal}{\kern0pt}\ s\ then\ hml{\isacharunderscore}{\kern0pt}pos\ b\ TT\isanewline
\ \ \ \ \ \ \ \ \ \ \ \ \ \ \ \ \ else\ if\ i\ {\isacharequal}{\kern0pt}\ t\isanewline
\ \ \ \ \ \ \ \ \ \ \ \ \ \ \ \ \ \ \ \ \ \ then\ hml{\isacharunderscore}{\kern0pt}pos\ a\isanewline
\ \ \ \ \ \ \ \ \ \ \ \ \ \ \ \ \ \ \ \ \ \ \ \ \ \ \ \ {\isacharparenleft}{\kern0pt}hml{\isacharunderscore}{\kern0pt}conj\ {\isacharbraceleft}{\kern0pt}{\isacharbraceright}{\kern0pt}\ {\isacharbraceleft}{\kern0pt}s{\isacharcomma}{\kern0pt}\ t{\isacharbraceright}{\kern0pt}\isanewline
\ \ \ \ \ \ \ \ \ \ \ \ \ \ \ \ \ \ \ \ \ \ \ \ \ \ \ \ \ \ {\isacharparenleft}{\kern0pt}{\isasymlambda}j{\isachardot}{\kern0pt}\ if\ j\ {\isacharequal}{\kern0pt}\ s\ then\ hml{\isacharunderscore}{\kern0pt}pos\ a\ {\isacharparenleft}{\kern0pt}hml{\isacharunderscore}{\kern0pt}pos\ c\ TT{\isacharparenright}{\kern0pt}\isanewline
\ \ \ \ \ \ \ \ \ \ \ \ \ \ \ \ \ \ \ \ \ \ \ \ \ \ \ \ \ \ \ \ \ \ \ \ else\ if\ j\ {\isacharequal}{\kern0pt}\ t\ then\ hml{\isacharunderscore}{\kern0pt}pos\ b\ TT\ else\ undefined{\isacharparenright}{\kern0pt}{\isacharparenright}{\kern0pt}\isanewline
\ \ \ \ \ \ \ \ \ \ \ \ \ \ \ \ \ \ \ \ \ \ else\ undefined{\isacharparenright}{\kern0pt}{\isacharparenright}{\kern0pt}\ {\isacharbackquote}{\kern0pt}\ {\isacharbraceleft}{\kern0pt}s{\isacharcomma}{\kern0pt}\ t{\isacharbraceright}{\kern0pt}{\isacharparenright}{\kern0pt}\ {\isacharequal}{\kern0pt}\ {\isadigit{3}}{\isachardoublequoteclose}\isanewline
\ \ \ \ \ \ \isacommand{using}\isamarkupfalse%
\ {\isachardoublequoteopen}{\isadigit{1}}{\isachardoublequoteclose}\ SUP{\isacharunderscore}{\kern0pt}insert\ Sup{\isacharunderscore}{\kern0pt}empty\ assms{\isacharparenleft}{\kern0pt}{\isadigit{1}}{\isacharparenright}{\kern0pt}\ image{\isacharunderscore}{\kern0pt}is{\isacharunderscore}{\kern0pt}empty\ o{\isacharunderscore}{\kern0pt}apply\ one{\isacharunderscore}{\kern0pt}le{\isacharunderscore}{\kern0pt}numeral\ sup{\isachardot}{\kern0pt}orderE\ sup{\isacharunderscore}{\kern0pt}bot{\isachardot}{\kern0pt}right{\isacharunderscore}{\kern0pt}neutral\ sup{\isacharunderscore}{\kern0pt}commute\isanewline
\ \ \ \ \ \ \isacommand{by}\isamarkupfalse%
\ {\isacharparenleft}{\kern0pt}smt\ {\isacharparenleft}{\kern0pt}verit{\isacharcomma}{\kern0pt}\ del{\isacharunderscore}{\kern0pt}insts{\isacharparenright}{\kern0pt}\ ile{\isacharunderscore}{\kern0pt}eSuc\ numeral{\isacharunderscore}{\kern0pt}Bit{\isadigit{1}}\ numeral{\isacharunderscore}{\kern0pt}One\ one{\isacharunderscore}{\kern0pt}add{\isacharunderscore}{\kern0pt}one\ one{\isacharunderscore}{\kern0pt}plus{\isacharunderscore}{\kern0pt}numeral{\isacharunderscore}{\kern0pt}commute\ plus{\isacharunderscore}{\kern0pt}{\isadigit{1}}{\isacharunderscore}{\kern0pt}eSuc{\isacharparenleft}{\kern0pt}{\isadigit{1}}{\isacharparenright}{\kern0pt}{\isacharparenright}{\kern0pt}\isanewline
\ \ \ \ \isanewline
\ \ \ \ \isacommand{hence}\isamarkupfalse%
\ {\isachardoublequoteopen}expr{\isacharunderscore}{\kern0pt}{\isadigit{1}}\ {\isacharparenleft}{\kern0pt}hml{\isacharunderscore}{\kern0pt}conj\ {\isacharbraceleft}{\kern0pt}s{\isacharcomma}{\kern0pt}\ t{\isacharbraceright}{\kern0pt}\ {\isacharbraceleft}{\kern0pt}{\isacharbraceright}{\kern0pt}\isanewline
\ \ \ \ \ \ \ \ \ \ \ {\isacharparenleft}{\kern0pt}{\isasymlambda}i{\isachardot}{\kern0pt}\ if\ i\ {\isacharequal}{\kern0pt}\ s\ then\ hml{\isacharunderscore}{\kern0pt}pos\ b\ TT\isanewline
\ \ \ \ \ \ \ \ \ \ \ \ \ \ \ \ \ else\ if\ i\ {\isacharequal}{\kern0pt}\ t\isanewline
\ \ \ \ \ \ \ \ \ \ \ \ \ \ \ \ \ \ \ \ \ \ then\ hml{\isacharunderscore}{\kern0pt}pos\ a\isanewline
\ \ \ \ \ \ \ \ \ \ \ \ \ \ \ \ \ \ \ \ \ \ \ \ \ \ \ \ {\isacharparenleft}{\kern0pt}hml{\isacharunderscore}{\kern0pt}conj\ {\isacharbraceleft}{\kern0pt}{\isacharbraceright}{\kern0pt}\ {\isacharbraceleft}{\kern0pt}s{\isacharcomma}{\kern0pt}\ t{\isacharbraceright}{\kern0pt}\isanewline
\ \ \ \ \ \ \ \ \ \ \ \ \ \ \ \ \ \ \ \ \ \ \ \ \ \ \ \ \ \ {\isacharparenleft}{\kern0pt}{\isasymlambda}j{\isachardot}{\kern0pt}\ if\ j\ {\isacharequal}{\kern0pt}\ s\ then\ hml{\isacharunderscore}{\kern0pt}pos\ a\ {\isacharparenleft}{\kern0pt}hml{\isacharunderscore}{\kern0pt}pos\ c\ TT{\isacharparenright}{\kern0pt}\isanewline
\ \ \ \ \ \ \ \ \ \ \ \ \ \ \ \ \ \ \ \ \ \ \ \ \ \ \ \ \ \ \ \ \ \ \ \ else\ if\ j\ {\isacharequal}{\kern0pt}\ t\ then\ hml{\isacharunderscore}{\kern0pt}pos\ b\ TT\ else\ undefined{\isacharparenright}{\kern0pt}{\isacharparenright}{\kern0pt}\isanewline
\ \ \ \ \ \ \ \ \ \ \ \ \ \ \ \ \ \ \ \ \ \ else\ undefined{\isacharparenright}{\kern0pt}{\isacharparenright}{\kern0pt}\ {\isacharequal}{\kern0pt}\ {\isadigit{3}}{\isachardoublequoteclose}\ \isanewline
\ \ \ \ \ \ \isacommand{by}\isamarkupfalse%
\ simp\isanewline
\ \ \ \ \isacommand{thus}\isamarkupfalse%
\ e{\isadigit{1}}{\isacharcolon}{\kern0pt}\ {\isachardoublequoteopen}expr{\isacharunderscore}{\kern0pt}{\isadigit{1}}\ {\isasymphi}\ {\isacharequal}{\kern0pt}\ {\isadigit{4}}{\isachardoublequoteclose}\ \ \isanewline
\ \ \ \ \ \ \isacommand{unfolding}\isamarkupfalse%
\ {\isasymphi}\ \isanewline
\ \ \ \ \ \ \isacommand{by}\isamarkupfalse%
\ fastforce\isanewline
\isanewline
\ \ \ \ \isacommand{show}\isamarkupfalse%
\ e{\isadigit{2}}{\isacharcolon}{\kern0pt}\ {\isachardoublequoteopen}expr{\isacharunderscore}{\kern0pt}{\isadigit{2}}\ {\isasymphi}\ {\isacharequal}{\kern0pt}\ {\isadigit{3}}{\isachardoublequoteclose}\ \isanewline
\ \ \ \ \ \ \isacommand{unfolding}\isamarkupfalse%
\ {\isasymphi}\ \ \isanewline
\ \ \ \ \ \ \isacommand{by}\isamarkupfalse%
\ {\isacharparenleft}{\kern0pt}smt\ {\isacharparenleft}{\kern0pt}z{\isadigit{3}}{\isacharparenright}{\kern0pt}\ {\isachardoublequoteopen}{\isadigit{1}}{\isachardoublequoteclose}\ {\isachardoublequoteopen}{\isadigit{2}}{\isachardoublequoteclose}\ SUP{\isacharunderscore}{\kern0pt}empty\ SUP{\isacharunderscore}{\kern0pt}insert\ Un{\isacharunderscore}{\kern0pt}insert{\isacharunderscore}{\kern0pt}right\ {\isacartoucheopen}Sup\ {\isacharparenleft}{\kern0pt}{\isacharparenleft}{\kern0pt}expr{\isacharunderscore}{\kern0pt}{\isadigit{1}}\ {\isasymcirc}\ {\isacharparenleft}{\kern0pt}{\isasymlambda}j{\isachardot}{\kern0pt}\ if\ j\ {\isacharequal}{\kern0pt}\ s\ then\ hml{\isacharunderscore}{\kern0pt}pos\ a\ {\isacharparenleft}{\kern0pt}hml{\isacharunderscore}{\kern0pt}pos\ c\ TT{\isacharparenright}{\kern0pt}\ else\ if\ j\ {\isacharequal}{\kern0pt}\ t\ then\ hml{\isacharunderscore}{\kern0pt}pos\ b\ TT\ else\ undefined{\isacharparenright}{\kern0pt}{\isacharparenright}{\kern0pt}\ {\isacharbackquote}{\kern0pt}\ {\isacharbraceleft}{\kern0pt}s{\isacharcomma}{\kern0pt}\ t{\isacharbraceright}{\kern0pt}{\isacharparenright}{\kern0pt}\ {\isacharequal}{\kern0pt}\ {\isadigit{2}}{\isacartoucheclose}\ {\isacartoucheopen}expr{\isacharunderscore}{\kern0pt}{\isadigit{1}}\ {\isacharparenleft}{\kern0pt}hml{\isacharunderscore}{\kern0pt}conj\ {\isacharbraceleft}{\kern0pt}{\isacharbraceright}{\kern0pt}\ {\isacharbraceleft}{\kern0pt}s{\isacharcomma}{\kern0pt}\ t{\isacharbraceright}{\kern0pt}\ {\isacharparenleft}{\kern0pt}{\isasymlambda}j{\isachardot}{\kern0pt}\ if\ j\ {\isacharequal}{\kern0pt}\ s\ then\ hml{\isacharunderscore}{\kern0pt}pos\ a\ {\isacharparenleft}{\kern0pt}hml{\isacharunderscore}{\kern0pt}pos\ c\ TT{\isacharparenright}{\kern0pt}\ else\ if\ j\ {\isacharequal}{\kern0pt}\ t\ then\ hml{\isacharunderscore}{\kern0pt}pos\ b\ TT\ else\ undefined{\isacharparenright}{\kern0pt}{\isacharparenright}{\kern0pt}\ {\isacharequal}{\kern0pt}\ {\isadigit{2}}{\isacartoucheclose}\ assms{\isacharparenleft}{\kern0pt}{\isadigit{1}}{\isacharparenright}{\kern0pt}\ comp{\isacharunderscore}{\kern0pt}def\ expr{\isacharunderscore}{\kern0pt}{\isadigit{1}}{\isacharunderscore}{\kern0pt}pos\ expr{\isacharunderscore}{\kern0pt}{\isadigit{2}}{\isacharunderscore}{\kern0pt}conj\ expr{\isacharunderscore}{\kern0pt}{\isadigit{2}}{\isacharunderscore}{\kern0pt}pos\ expr{\isacharunderscore}{\kern0pt}{\isadigit{2}}{\isacharunderscore}{\kern0pt}tt\ image{\isacharunderscore}{\kern0pt}Un\ insert{\isacharunderscore}{\kern0pt}is{\isacharunderscore}{\kern0pt}Un\ numeral{\isacharunderscore}{\kern0pt}One\ one{\isacharunderscore}{\kern0pt}le{\isacharunderscore}{\kern0pt}numeral\ sup{\isachardot}{\kern0pt}order{\isacharunderscore}{\kern0pt}iff\ sup{\isacharunderscore}{\kern0pt}bot{\isacharunderscore}{\kern0pt}right{\isacharparenright}{\kern0pt}\isanewline
\isanewline
\ \ \ \ \isacommand{show}\isamarkupfalse%
\ e{\isadigit{3}}{\isacharcolon}{\kern0pt}\ {\isachardoublequoteopen}expr{\isacharunderscore}{\kern0pt}{\isadigit{3}}\ {\isasymphi}\ {\isacharequal}{\kern0pt}\ {\isadigit{3}}{\isachardoublequoteclose}\isanewline
\ \ \ \ \isacommand{proof}\isamarkupfalse%
{\isacharminus}{\kern0pt}\isanewline
\ \ \ \ \ \ \isacommand{have}\isamarkupfalse%
\ e{\isadigit{3}}{\isacharunderscore}{\kern0pt}{\isadigit{1}}{\isacharcolon}{\kern0pt}\ {\isachardoublequoteopen}expr{\isacharunderscore}{\kern0pt}{\isadigit{3}}\ {\isacharparenleft}{\kern0pt}hml{\isacharunderscore}{\kern0pt}conj\ {\isacharbraceleft}{\kern0pt}{\isacharbraceright}{\kern0pt}\ {\isacharbraceleft}{\kern0pt}s{\isacharcomma}{\kern0pt}\ t{\isacharbraceright}{\kern0pt}\ \isanewline
\ \ \ \ \ \ \ \ \ \ \ \ \ \ \ \ \ \ \ \ \ \ \ \ {\isacharparenleft}{\kern0pt}{\isasymlambda}j{\isachardot}{\kern0pt}\ {\isacharparenleft}{\kern0pt}if\ j\ {\isacharequal}{\kern0pt}\ s\ \isanewline
\ \ \ \ \ \ \ \ \ \ \ \ \ \ \ \ \ \ \ \ \ \ \ \ \ \ \ \ \ \ then\ {\isacharparenleft}{\kern0pt}hml{\isacharunderscore}{\kern0pt}pos\ a\ \isanewline
\ \ \ \ \ \ \ \ \ \ \ \ \ \ \ \ \ \ \ \ \ \ \ \ \ \ \ \ \ \ \ \ \ \ \ \ \ {\isacharparenleft}{\kern0pt}hml{\isacharunderscore}{\kern0pt}pos\ c\ TT{\isacharparenright}{\kern0pt}{\isacharparenright}{\kern0pt}\ \isanewline
\ \ \ \ \ \ \ \ \ \ \ \ \ \ \ \ \ \ \ \ \ \ \ \ \ \ \ \ \ \ else\ \isanewline
\ \ \ \ \ \ \ \ \ \ \ \ \ \ \ \ \ \ \ \ \ \ \ \ \ \ \ \ \ \ \ \ {\isacharparenleft}{\kern0pt}if\ j\ {\isacharequal}{\kern0pt}\ t\ \isanewline
\ \ \ \ \ \ \ \ \ \ \ \ \ \ \ \ \ \ \ \ \ \ \ \ \ \ \ \ \ \ \ \ \ then\ {\isacharparenleft}{\kern0pt}hml{\isacharunderscore}{\kern0pt}pos\ b\ TT{\isacharparenright}{\kern0pt}\ \isanewline
\ \ \ \ \ \ \ \ \ \ \ \ \ \ \ \ \ \ \ \ \ \ \ \ \ \ \ \ \ \ \ \ \ else\ undefined{\isacharparenright}{\kern0pt}{\isacharparenright}{\kern0pt}{\isacharparenright}{\kern0pt}{\isacharparenright}{\kern0pt}\ {\isacharequal}{\kern0pt}\ {\isadigit{0}}{\isachardoublequoteclose}\isanewline
\ \ \ \ \ \ \ \ \isacommand{by}\isamarkupfalse%
\ simp\isanewline
\ \ \ \ \ \ \isacommand{hence}\isamarkupfalse%
\ e{\isadigit{1}}{\isacharunderscore}{\kern0pt}{\isadigit{1}}{\isacharcolon}{\kern0pt}\ {\isachardoublequoteopen}{\isacharparenleft}{\kern0pt}Sup\ {\isacharparenleft}{\kern0pt}{\isacharparenleft}{\kern0pt}expr{\isacharunderscore}{\kern0pt}{\isadigit{1}}\ {\isasymcirc}\ {\isacharparenleft}{\kern0pt}{\isasymlambda}i{\isachardot}{\kern0pt}\ {\isacharparenleft}{\kern0pt}if\ i\ {\isacharequal}{\kern0pt}\ s\ \isanewline
\ \ \ \ \ \ \ \ \ \ \ \ then\ {\isacharparenleft}{\kern0pt}hml{\isacharunderscore}{\kern0pt}pos\ b\ TT{\isacharparenright}{\kern0pt}\ \isanewline
\ \ \ \ \ \ \ \ \ \ \ \ else\ \isanewline
\ \ \ \ \ \ \ \ \ \ \ \ \ \ {\isacharparenleft}{\kern0pt}if\ i\ {\isacharequal}{\kern0pt}\ t\ \ \isanewline
\ \ \ \ \ \ \ \ \ \ \ \ \ \ \ then\ {\isacharparenleft}{\kern0pt}hml{\isacharunderscore}{\kern0pt}pos\ a\ \isanewline
\ \ \ \ \ \ \ \ \ \ \ \ \ \ \ \ \ \ \ \ \ \ {\isacharparenleft}{\kern0pt}hml{\isacharunderscore}{\kern0pt}conj\ {\isacharbraceleft}{\kern0pt}{\isacharbraceright}{\kern0pt}\ {\isacharbraceleft}{\kern0pt}s{\isacharcomma}{\kern0pt}\ t{\isacharbraceright}{\kern0pt}\ \isanewline
\ \ \ \ \ \ \ \ \ \ \ \ \ \ \ \ \ \ \ \ \ \ \ \ {\isacharparenleft}{\kern0pt}{\isasymlambda}j{\isachardot}{\kern0pt}\ {\isacharparenleft}{\kern0pt}if\ j\ {\isacharequal}{\kern0pt}\ s\ \isanewline
\ \ \ \ \ \ \ \ \ \ \ \ \ \ \ \ \ \ \ \ \ \ \ \ \ \ \ \ \ \ then\ {\isacharparenleft}{\kern0pt}hml{\isacharunderscore}{\kern0pt}pos\ a\ \isanewline
\ \ \ \ \ \ \ \ \ \ \ \ \ \ \ \ \ \ \ \ \ \ \ \ \ \ \ \ \ \ \ \ \ \ \ \ \ {\isacharparenleft}{\kern0pt}hml{\isacharunderscore}{\kern0pt}pos\ c\ TT{\isacharparenright}{\kern0pt}{\isacharparenright}{\kern0pt}\ \isanewline
\ \ \ \ \ \ \ \ \ \ \ \ \ \ \ \ \ \ \ \ \ \ \ \ \ \ \ \ \ \ else\ \isanewline
\ \ \ \ \ \ \ \ \ \ \ \ \ \ \ \ \ \ \ \ \ \ \ \ \ \ \ \ \ \ \ \ {\isacharparenleft}{\kern0pt}if\ j\ {\isacharequal}{\kern0pt}\ t\ \isanewline
\ \ \ \ \ \ \ \ \ \ \ \ \ \ \ \ \ \ \ \ \ \ \ \ \ \ \ \ \ \ \ \ \ then\ {\isacharparenleft}{\kern0pt}hml{\isacharunderscore}{\kern0pt}pos\ b\ TT{\isacharparenright}{\kern0pt}\ \isanewline
\ \ \ \ \ \ \ \ \ \ \ \ \ \ \ \ \ \ \ \ \ \ \ \ \ \ \ \ \ \ \ \ \ else\ undefined{\isacharparenright}{\kern0pt}{\isacharparenright}{\kern0pt}{\isacharparenright}{\kern0pt}{\isacharparenright}{\kern0pt}{\isacharparenright}{\kern0pt}\isanewline
\ \ \ \ \ \ \ \ \ \ \ \ \ \ \ else\ undefined{\isacharparenright}{\kern0pt}{\isacharparenright}{\kern0pt}{\isacharparenright}{\kern0pt}{\isacharparenright}{\kern0pt}\ {\isacharbackquote}{\kern0pt}\ {\isacharbraceleft}{\kern0pt}s{\isacharcomma}{\kern0pt}\ t{\isacharbraceright}{\kern0pt}{\isacharparenright}{\kern0pt}{\isacharparenright}{\kern0pt}\ {\isacharequal}{\kern0pt}\ {\isadigit{3}}{\isachardoublequoteclose}\isanewline
\ \ \ \ \ \ \ \ \isacommand{using}\isamarkupfalse%
\ {\isacartoucheopen}Sup\ {\isacharparenleft}{\kern0pt}{\isacharparenleft}{\kern0pt}expr{\isacharunderscore}{\kern0pt}{\isadigit{1}}\ {\isasymcirc}\ {\isacharparenleft}{\kern0pt}{\isasymlambda}i{\isachardot}{\kern0pt}\ if\ i\ {\isacharequal}{\kern0pt}\ s\ then\ hml{\isacharunderscore}{\kern0pt}pos\ b\ TT\ else\ if\ i\ {\isacharequal}{\kern0pt}\ t\ then\ hml{\isacharunderscore}{\kern0pt}pos\ a\ {\isacharparenleft}{\kern0pt}hml{\isacharunderscore}{\kern0pt}conj\ {\isacharbraceleft}{\kern0pt}{\isacharbraceright}{\kern0pt}\ {\isacharbraceleft}{\kern0pt}s{\isacharcomma}{\kern0pt}\ t{\isacharbraceright}{\kern0pt}\ {\isacharparenleft}{\kern0pt}{\isasymlambda}j{\isachardot}{\kern0pt}\ if\ j\ {\isacharequal}{\kern0pt}\ s\ then\ hml{\isacharunderscore}{\kern0pt}pos\ a\ {\isacharparenleft}{\kern0pt}hml{\isacharunderscore}{\kern0pt}pos\ c\ TT{\isacharparenright}{\kern0pt}\ else\ if\ j\ {\isacharequal}{\kern0pt}\ t\ then\ hml{\isacharunderscore}{\kern0pt}pos\ b\ TT\ else\ undefined{\isacharparenright}{\kern0pt}{\isacharparenright}{\kern0pt}\ else\ undefined{\isacharparenright}{\kern0pt}{\isacharparenright}{\kern0pt}\ {\isacharbackquote}{\kern0pt}\ {\isacharbraceleft}{\kern0pt}s{\isacharcomma}{\kern0pt}\ t{\isacharbraceright}{\kern0pt}{\isacharparenright}{\kern0pt}\ {\isacharequal}{\kern0pt}\ {\isadigit{3}}{\isacartoucheclose}\ \isacommand{by}\isamarkupfalse%
\ blast\isanewline
\ \ \ \ \ \ \isacommand{have}\isamarkupfalse%
\ {\isachardoublequoteopen}{\isacharparenleft}{\kern0pt}Sup\ {\isacharparenleft}{\kern0pt}{\isacharparenleft}{\kern0pt}expr{\isacharunderscore}{\kern0pt}{\isadigit{3}}\ {\isasymcirc}\ {\isacharparenleft}{\kern0pt}{\isasymlambda}i{\isachardot}{\kern0pt}\ {\isacharparenleft}{\kern0pt}if\ i\ {\isacharequal}{\kern0pt}\ s\ \isanewline
\ \ \ \ \ \ \ \ \ \ \ \ then\ {\isacharparenleft}{\kern0pt}hml{\isacharunderscore}{\kern0pt}pos\ b\ TT{\isacharparenright}{\kern0pt}\ \isanewline
\ \ \ \ \ \ \ \ \ \ \ \ else\ \isanewline
\ \ \ \ \ \ \ \ \ \ \ \ \ \ {\isacharparenleft}{\kern0pt}if\ i\ {\isacharequal}{\kern0pt}\ t\ \ \isanewline
\ \ \ \ \ \ \ \ \ \ \ \ \ \ \ then\ {\isacharparenleft}{\kern0pt}hml{\isacharunderscore}{\kern0pt}pos\ a\ \isanewline
\ \ \ \ \ \ \ \ \ \ \ \ \ \ \ \ \ \ \ \ \ \ {\isacharparenleft}{\kern0pt}hml{\isacharunderscore}{\kern0pt}conj\ {\isacharbraceleft}{\kern0pt}{\isacharbraceright}{\kern0pt}\ {\isacharbraceleft}{\kern0pt}s{\isacharcomma}{\kern0pt}\ t{\isacharbraceright}{\kern0pt}\ \isanewline
\ \ \ \ \ \ \ \ \ \ \ \ \ \ \ \ \ \ \ \ \ \ \ \ {\isacharparenleft}{\kern0pt}{\isasymlambda}j{\isachardot}{\kern0pt}\ {\isacharparenleft}{\kern0pt}if\ j\ {\isacharequal}{\kern0pt}\ s\ \isanewline
\ \ \ \ \ \ \ \ \ \ \ \ \ \ \ \ \ \ \ \ \ \ \ \ \ \ \ \ \ \ then\ {\isacharparenleft}{\kern0pt}hml{\isacharunderscore}{\kern0pt}pos\ a\ \isanewline
\ \ \ \ \ \ \ \ \ \ \ \ \ \ \ \ \ \ \ \ \ \ \ \ \ \ \ \ \ \ \ \ \ \ \ \ \ {\isacharparenleft}{\kern0pt}hml{\isacharunderscore}{\kern0pt}pos\ c\ TT{\isacharparenright}{\kern0pt}{\isacharparenright}{\kern0pt}\ \isanewline
\ \ \ \ \ \ \ \ \ \ \ \ \ \ \ \ \ \ \ \ \ \ \ \ \ \ \ \ \ \ else\ \isanewline
\ \ \ \ \ \ \ \ \ \ \ \ \ \ \ \ \ \ \ \ \ \ \ \ \ \ \ \ \ \ \ \ {\isacharparenleft}{\kern0pt}if\ j\ {\isacharequal}{\kern0pt}\ t\ \isanewline
\ \ \ \ \ \ \ \ \ \ \ \ \ \ \ \ \ \ \ \ \ \ \ \ \ \ \ \ \ \ \ \ \ then\ {\isacharparenleft}{\kern0pt}hml{\isacharunderscore}{\kern0pt}pos\ b\ TT{\isacharparenright}{\kern0pt}\ \isanewline
\ \ \ \ \ \ \ \ \ \ \ \ \ \ \ \ \ \ \ \ \ \ \ \ \ \ \ \ \ \ \ \ \ else\ undefined{\isacharparenright}{\kern0pt}{\isacharparenright}{\kern0pt}{\isacharparenright}{\kern0pt}{\isacharparenright}{\kern0pt}{\isacharparenright}{\kern0pt}\isanewline
\ \ \ \ \ \ \ \ \ \ \ \ \ \ \ else\ undefined{\isacharparenright}{\kern0pt}{\isacharparenright}{\kern0pt}{\isacharparenright}{\kern0pt}{\isacharparenright}{\kern0pt}\ {\isacharbackquote}{\kern0pt}\ {\isacharbraceleft}{\kern0pt}s{\isacharcomma}{\kern0pt}\ t{\isacharbraceright}{\kern0pt}{\isacharparenright}{\kern0pt}{\isacharparenright}{\kern0pt}\ {\isacharequal}{\kern0pt}\ {\isadigit{0}}{\isachardoublequoteclose}\isanewline
\ \ \ \ \ \ \ \ \isanewline
\ \ \ \ \ \ \ \ \isacommand{by}\isamarkupfalse%
\ simp\isanewline
\ \ \ \ \ \ \isacommand{hence}\isamarkupfalse%
\ {\isachardoublequoteopen}expr{\isacharunderscore}{\kern0pt}{\isadigit{3}}\ \ \isanewline
\ \ \ \ {\isacharparenleft}{\kern0pt}hml{\isacharunderscore}{\kern0pt}conj\ {\isacharbraceleft}{\kern0pt}s{\isacharcomma}{\kern0pt}\ t{\isacharbraceright}{\kern0pt}\ {\isacharbraceleft}{\kern0pt}{\isacharbraceright}{\kern0pt}\isanewline
\ \ \ \ \ \ {\isacharparenleft}{\kern0pt}{\isasymlambda}i{\isachardot}{\kern0pt}\ {\isacharparenleft}{\kern0pt}if\ i\ {\isacharequal}{\kern0pt}\ s\ \isanewline
\ \ \ \ \ \ \ \ \ \ \ \ then\ {\isacharparenleft}{\kern0pt}hml{\isacharunderscore}{\kern0pt}pos\ b\ TT{\isacharparenright}{\kern0pt}\ \isanewline
\ \ \ \ \ \ \ \ \ \ \ \ else\ \isanewline
\ \ \ \ \ \ \ \ \ \ \ \ \ \ {\isacharparenleft}{\kern0pt}if\ i\ {\isacharequal}{\kern0pt}\ t\ \ \isanewline
\ \ \ \ \ \ \ \ \ \ \ \ \ \ \ then\ {\isacharparenleft}{\kern0pt}hml{\isacharunderscore}{\kern0pt}pos\ a\ \isanewline
\ \ \ \ \ \ \ \ \ \ \ \ \ \ \ \ \ \ \ \ \ \ {\isacharparenleft}{\kern0pt}hml{\isacharunderscore}{\kern0pt}conj\ {\isacharbraceleft}{\kern0pt}{\isacharbraceright}{\kern0pt}\ {\isacharbraceleft}{\kern0pt}s{\isacharcomma}{\kern0pt}\ t{\isacharbraceright}{\kern0pt}\ \isanewline
\ \ \ \ \ \ \ \ \ \ \ \ \ \ \ \ \ \ \ \ \ \ \ \ {\isacharparenleft}{\kern0pt}{\isasymlambda}j{\isachardot}{\kern0pt}\ {\isacharparenleft}{\kern0pt}if\ j\ {\isacharequal}{\kern0pt}\ s\ \isanewline
\ \ \ \ \ \ \ \ \ \ \ \ \ \ \ \ \ \ \ \ \ \ \ \ \ \ \ \ \ \ then\ {\isacharparenleft}{\kern0pt}hml{\isacharunderscore}{\kern0pt}pos\ a\ \isanewline
\ \ \ \ \ \ \ \ \ \ \ \ \ \ \ \ \ \ \ \ \ \ \ \ \ \ \ \ \ \ \ \ \ \ \ \ \ {\isacharparenleft}{\kern0pt}hml{\isacharunderscore}{\kern0pt}pos\ c\ TT{\isacharparenright}{\kern0pt}{\isacharparenright}{\kern0pt}\ \isanewline
\ \ \ \ \ \ \ \ \ \ \ \ \ \ \ \ \ \ \ \ \ \ \ \ \ \ \ \ \ \ else\ \isanewline
\ \ \ \ \ \ \ \ \ \ \ \ \ \ \ \ \ \ \ \ \ \ \ \ \ \ \ \ \ \ \ \ {\isacharparenleft}{\kern0pt}if\ j\ {\isacharequal}{\kern0pt}\ t\ \isanewline
\ \ \ \ \ \ \ \ \ \ \ \ \ \ \ \ \ \ \ \ \ \ \ \ \ \ \ \ \ \ \ \ \ then\ {\isacharparenleft}{\kern0pt}hml{\isacharunderscore}{\kern0pt}pos\ b\ TT{\isacharparenright}{\kern0pt}\ \isanewline
\ \ \ \ \ \ \ \ \ \ \ \ \ \ \ \ \ \ \ \ \ \ \ \ \ \ \ \ \ \ \ \ \ else\ undefined{\isacharparenright}{\kern0pt}{\isacharparenright}{\kern0pt}{\isacharparenright}{\kern0pt}{\isacharparenright}{\kern0pt}{\isacharparenright}{\kern0pt}\isanewline
\ \ \ \ \ \ \ \ \ \ \ \ \ \ \ else\ undefined{\isacharparenright}{\kern0pt}{\isacharparenright}{\kern0pt}{\isacharparenright}{\kern0pt}{\isacharparenright}{\kern0pt}\ {\isacharequal}{\kern0pt}\ {\isadigit{3}}{\isachardoublequoteclose}\isanewline
\ \ \ \ \ \ \ \ \isacommand{using}\isamarkupfalse%
\ e{\isadigit{1}}{\isacharunderscore}{\kern0pt}{\isadigit{1}}\isanewline
\ \ \ \ \ \ \ \ \isacommand{by}\isamarkupfalse%
\ {\isacharparenleft}{\kern0pt}smt\ {\isacharparenleft}{\kern0pt}verit{\isacharcomma}{\kern0pt}\ best{\isacharparenright}{\kern0pt}\ Sup{\isacharunderscore}{\kern0pt}union{\isacharunderscore}{\kern0pt}distrib\ bot{\isacharunderscore}{\kern0pt}enat{\isacharunderscore}{\kern0pt}def\ expr{\isacharunderscore}{\kern0pt}{\isadigit{3}}{\isacharunderscore}{\kern0pt}conj\ image{\isacharunderscore}{\kern0pt}empty\ sup{\isacharunderscore}{\kern0pt}bot{\isachardot}{\kern0pt}right{\isacharunderscore}{\kern0pt}neutral{\isacharparenright}{\kern0pt}\isanewline
\ \ \ \ \ \ \isacommand{thus}\isamarkupfalse%
\ {\isacharquery}{\kern0pt}thesis\isanewline
\ \ \ \ \ \ \ \ \isacommand{unfolding}\isamarkupfalse%
\ {\isasymphi}\ \isanewline
\ \ \ \ \ \ \ \ \isacommand{by}\isamarkupfalse%
\ fastforce\isanewline
\ \ \ \ \isacommand{qed}\isamarkupfalse%
\isanewline
\isanewline
\ \ \ \ \isacommand{show}\isamarkupfalse%
\ e{\isadigit{5}}{\isacharcolon}{\kern0pt}\ {\isachardoublequoteopen}expr{\isacharunderscore}{\kern0pt}{\isadigit{5}}\ {\isasymphi}\ {\isacharequal}{\kern0pt}\ {\isadigit{2}}{\isachardoublequoteclose}\isanewline
\ \ \ \ \isacommand{proof}\isamarkupfalse%
{\isacharminus}{\kern0pt}\isanewline
\ \ \ \ \ \ \isacommand{have}\isamarkupfalse%
\ {\isachardoublequoteopen}expr{\isacharunderscore}{\kern0pt}{\isadigit{5}}\ {\isacharparenleft}{\kern0pt}hml{\isacharunderscore}{\kern0pt}pos\ b\ TT{\isacharparenright}{\kern0pt}\ {\isacharequal}{\kern0pt}\ {\isadigit{0}}{\isachardoublequoteclose}\isanewline
\ \ \ \ \ \ \ \ \isacommand{by}\isamarkupfalse%
\ fastforce\isanewline
\ \ \ \ \ \ \isacommand{have}\isamarkupfalse%
\ {\isachardoublequoteopen}expr{\isacharunderscore}{\kern0pt}{\isadigit{5}}\ {\isacharparenleft}{\kern0pt}hml{\isacharunderscore}{\kern0pt}conj\ {\isacharbraceleft}{\kern0pt}{\isacharbraceright}{\kern0pt}\ {\isacharbraceleft}{\kern0pt}s{\isacharcomma}{\kern0pt}\ t{\isacharbraceright}{\kern0pt}\ \isanewline
\ \ \ \ \ \ \ \ \ \ \ \ \ \ \ \ \ \ \ \ \ \ \ \ {\isacharparenleft}{\kern0pt}{\isasymlambda}j{\isachardot}{\kern0pt}\ {\isacharparenleft}{\kern0pt}if\ j\ {\isacharequal}{\kern0pt}\ s\ \isanewline
\ \ \ \ \ \ \ \ \ \ \ \ \ \ \ \ \ \ \ \ \ \ \ \ \ \ \ \ \ \ then\ {\isacharparenleft}{\kern0pt}hml{\isacharunderscore}{\kern0pt}pos\ a\ \isanewline
\ \ \ \ \ \ \ \ \ \ \ \ \ \ \ \ \ \ \ \ \ \ \ \ \ \ \ \ \ \ \ \ \ \ \ \ \ {\isacharparenleft}{\kern0pt}hml{\isacharunderscore}{\kern0pt}pos\ c\ TT{\isacharparenright}{\kern0pt}{\isacharparenright}{\kern0pt}\ \isanewline
\ \ \ \ \ \ \ \ \ \ \ \ \ \ \ \ \ \ \ \ \ \ \ \ \ \ \ \ \ \ else\ \isanewline
\ \ \ \ \ \ \ \ \ \ \ \ \ \ \ \ \ \ \ \ \ \ \ \ \ \ \ \ \ \ \ \ {\isacharparenleft}{\kern0pt}if\ j\ {\isacharequal}{\kern0pt}\ t\ \isanewline
\ \ \ \ \ \ \ \ \ \ \ \ \ \ \ \ \ \ \ \ \ \ \ \ \ \ \ \ \ \ \ \ \ then\ {\isacharparenleft}{\kern0pt}hml{\isacharunderscore}{\kern0pt}pos\ b\ TT{\isacharparenright}{\kern0pt}\ \isanewline
\ \ \ \ \ \ \ \ \ \ \ \ \ \ \ \ \ \ \ \ \ \ \ \ \ \ \ \ \ \ \ \ \ else\ undefined{\isacharparenright}{\kern0pt}{\isacharparenright}{\kern0pt}{\isacharparenright}{\kern0pt}{\isacharparenright}{\kern0pt}\ {\isacharequal}{\kern0pt}\ {\isadigit{2}}{\isachardoublequoteclose}\isanewline
\ \ \ \ \ \ \ \ \isacommand{using}\isamarkupfalse%
\ {\isacartoucheopen}Sup\ {\isacharparenleft}{\kern0pt}{\isacharparenleft}{\kern0pt}expr{\isacharunderscore}{\kern0pt}{\isadigit{1}}\ {\isasymcirc}\ {\isacharparenleft}{\kern0pt}{\isasymlambda}j{\isachardot}{\kern0pt}\ if\ j\ {\isacharequal}{\kern0pt}\ s\ then\ hml{\isacharunderscore}{\kern0pt}pos\ a\ {\isacharparenleft}{\kern0pt}hml{\isacharunderscore}{\kern0pt}pos\ c\ TT{\isacharparenright}{\kern0pt}\ else\ if\ j\ {\isacharequal}{\kern0pt}\ t\ then\ hml{\isacharunderscore}{\kern0pt}pos\ b\ TT\ else\ undefined{\isacharparenright}{\kern0pt}{\isacharparenright}{\kern0pt}\ {\isacharbackquote}{\kern0pt}\ {\isacharbraceleft}{\kern0pt}s{\isacharcomma}{\kern0pt}\ t{\isacharbraceright}{\kern0pt}{\isacharparenright}{\kern0pt}\ {\isacharequal}{\kern0pt}\ {\isadigit{2}}{\isacartoucheclose}\ bot{\isacharunderscore}{\kern0pt}enat{\isacharunderscore}{\kern0pt}def\ \isacommand{by}\isamarkupfalse%
\ auto\isanewline
\isanewline
\ \ \ \ \ \ \isacommand{hence}\isamarkupfalse%
\ {\isachardoublequoteopen}expr{\isacharunderscore}{\kern0pt}{\isadigit{5}}\ {\isacharparenleft}{\kern0pt}hml{\isacharunderscore}{\kern0pt}conj\ {\isacharbraceleft}{\kern0pt}s{\isacharcomma}{\kern0pt}\ t{\isacharbraceright}{\kern0pt}\ {\isacharbraceleft}{\kern0pt}{\isacharbraceright}{\kern0pt}\isanewline
\ \ \ \ \ \ {\isacharparenleft}{\kern0pt}{\isasymlambda}i{\isachardot}{\kern0pt}\ {\isacharparenleft}{\kern0pt}if\ i\ {\isacharequal}{\kern0pt}\ s\ \isanewline
\ \ \ \ \ \ \ \ \ \ \ \ then\ {\isacharparenleft}{\kern0pt}hml{\isacharunderscore}{\kern0pt}pos\ b\ TT{\isacharparenright}{\kern0pt}\ \isanewline
\ \ \ \ \ \ \ \ \ \ \ \ else\ \isanewline
\ \ \ \ \ \ \ \ \ \ \ \ \ \ {\isacharparenleft}{\kern0pt}if\ i\ {\isacharequal}{\kern0pt}\ t\ \ \isanewline
\ \ \ \ \ \ \ \ \ \ \ \ \ \ \ then\ {\isacharparenleft}{\kern0pt}hml{\isacharunderscore}{\kern0pt}pos\ a\ \isanewline
\ \ \ \ \ \ \ \ \ \ \ \ \ \ \ \ \ \ \ \ \ \ {\isacharparenleft}{\kern0pt}hml{\isacharunderscore}{\kern0pt}conj\ {\isacharbraceleft}{\kern0pt}{\isacharbraceright}{\kern0pt}\ {\isacharbraceleft}{\kern0pt}s{\isacharcomma}{\kern0pt}\ t{\isacharbraceright}{\kern0pt}\ \isanewline
\ \ \ \ \ \ \ \ \ \ \ \ \ \ \ \ \ \ \ \ \ \ \ \ {\isacharparenleft}{\kern0pt}{\isasymlambda}j{\isachardot}{\kern0pt}\ {\isacharparenleft}{\kern0pt}if\ j\ {\isacharequal}{\kern0pt}\ s\ \isanewline
\ \ \ \ \ \ \ \ \ \ \ \ \ \ \ \ \ \ \ \ \ \ \ \ \ \ \ \ \ \ then\ {\isacharparenleft}{\kern0pt}hml{\isacharunderscore}{\kern0pt}pos\ a\ \isanewline
\ \ \ \ \ \ \ \ \ \ \ \ \ \ \ \ \ \ \ \ \ \ \ \ \ \ \ \ \ \ \ \ \ \ \ \ \ {\isacharparenleft}{\kern0pt}hml{\isacharunderscore}{\kern0pt}pos\ c\ TT{\isacharparenright}{\kern0pt}{\isacharparenright}{\kern0pt}\ \isanewline
\ \ \ \ \ \ \ \ \ \ \ \ \ \ \ \ \ \ \ \ \ \ \ \ \ \ \ \ \ \ else\ \isanewline
\ \ \ \ \ \ \ \ \ \ \ \ \ \ \ \ \ \ \ \ \ \ \ \ \ \ \ \ \ \ \ \ {\isacharparenleft}{\kern0pt}if\ j\ {\isacharequal}{\kern0pt}\ t\ \isanewline
\ \ \ \ \ \ \ \ \ \ \ \ \ \ \ \ \ \ \ \ \ \ \ \ \ \ \ \ \ \ \ \ \ then\ {\isacharparenleft}{\kern0pt}hml{\isacharunderscore}{\kern0pt}pos\ b\ TT{\isacharparenright}{\kern0pt}\ \isanewline
\ \ \ \ \ \ \ \ \ \ \ \ \ \ \ \ \ \ \ \ \ \ \ \ \ \ \ \ \ \ \ \ \ else\ undefined{\isacharparenright}{\kern0pt}{\isacharparenright}{\kern0pt}{\isacharparenright}{\kern0pt}{\isacharparenright}{\kern0pt}{\isacharparenright}{\kern0pt}\isanewline
\ \ \ \ \ \ \ \ \ \ \ \ \ \ \ else\ undefined{\isacharparenright}{\kern0pt}{\isacharparenright}{\kern0pt}{\isacharparenright}{\kern0pt}{\isacharparenright}{\kern0pt}\ {\isacharequal}{\kern0pt}\ {\isadigit{2}}{\isachardoublequoteclose}\isanewline
\ \ \ \ \ \ \ \ \isacommand{using}\isamarkupfalse%
\ {\isacartoucheopen}expr{\isacharunderscore}{\kern0pt}{\isadigit{5}}\ {\isacharparenleft}{\kern0pt}hml{\isacharunderscore}{\kern0pt}pos\ b\ TT{\isacharparenright}{\kern0pt}\ {\isacharequal}{\kern0pt}\ {\isadigit{0}}{\isacartoucheclose}\ image{\isacharunderscore}{\kern0pt}insert\ SUP{\isacharunderscore}{\kern0pt}insert\ Sup{\isacharunderscore}{\kern0pt}union{\isacharunderscore}{\kern0pt}distrib\ Un{\isacharunderscore}{\kern0pt}insert{\isacharunderscore}{\kern0pt}left\ Un{\isacharunderscore}{\kern0pt}insert{\isacharunderscore}{\kern0pt}right\ {\isacartoucheopen}Sup\ {\isacharparenleft}{\kern0pt}{\isacharparenleft}{\kern0pt}expr{\isacharunderscore}{\kern0pt}{\isadigit{1}}\ {\isasymcirc}\ {\isacharparenleft}{\kern0pt}{\isasymlambda}j{\isachardot}{\kern0pt}\ if\ j\ {\isacharequal}{\kern0pt}\ s\ then\ hml{\isacharunderscore}{\kern0pt}pos\ a\ {\isacharparenleft}{\kern0pt}hml{\isacharunderscore}{\kern0pt}pos\ c\ TT{\isacharparenright}{\kern0pt}\ else\ if\ j\ {\isacharequal}{\kern0pt}\ t\ then\ hml{\isacharunderscore}{\kern0pt}pos\ b\ TT\ else\ undefined{\isacharparenright}{\kern0pt}{\isacharparenright}{\kern0pt}\ {\isacharbackquote}{\kern0pt}\ {\isacharbraceleft}{\kern0pt}s{\isacharcomma}{\kern0pt}\ t{\isacharbraceright}{\kern0pt}{\isacharparenright}{\kern0pt}\ {\isacharequal}{\kern0pt}\ {\isadigit{2}}{\isacartoucheclose}\ assms{\isacharparenleft}{\kern0pt}{\isadigit{1}}{\isacharparenright}{\kern0pt}\ comp{\isacharunderscore}{\kern0pt}apply\ expr{\isacharunderscore}{\kern0pt}{\isadigit{5}}{\isacharunderscore}{\kern0pt}conj\ expr{\isacharunderscore}{\kern0pt}{\isadigit{5}}{\isacharunderscore}{\kern0pt}pos\ image{\isacharunderscore}{\kern0pt}Un\ image{\isacharunderscore}{\kern0pt}empty\ insert{\isacharunderscore}{\kern0pt}absorb\ insert{\isacharunderscore}{\kern0pt}iff\ insert{\isacharunderscore}{\kern0pt}is{\isacharunderscore}{\kern0pt}Un\isanewline
\ \ \ \ \ \ \ \ \isanewline
\ \ \ \ \ \ \isacommand{proof}\isamarkupfalse%
\ {\isacharminus}{\kern0pt}\isanewline
\ \ \ \ \ \ \ \ \isacommand{assume}\isamarkupfalse%
\ {\isachardoublequoteopen}Sup\ {\isacharparenleft}{\kern0pt}{\isacharparenleft}{\kern0pt}expr{\isacharunderscore}{\kern0pt}{\isadigit{1}}\ {\isasymcirc}\ {\isacharparenleft}{\kern0pt}{\isasymlambda}j{\isachardot}{\kern0pt}\ if\ j\ {\isacharequal}{\kern0pt}\ s\ then\ hml{\isacharunderscore}{\kern0pt}pos\ a\ {\isacharparenleft}{\kern0pt}hml{\isacharunderscore}{\kern0pt}pos\ c\ TT{\isacharparenright}{\kern0pt}\ else\ if\ j\ {\isacharequal}{\kern0pt}\ t\ then\ hml{\isacharunderscore}{\kern0pt}pos\ b\ TT\ else\ undefined{\isacharparenright}{\kern0pt}{\isacharparenright}{\kern0pt}\ {\isacharbackquote}{\kern0pt}\ {\isacharbraceleft}{\kern0pt}s{\isacharcomma}{\kern0pt}\ t{\isacharbraceright}{\kern0pt}{\isacharparenright}{\kern0pt}\ {\isacharequal}{\kern0pt}\ {\isadigit{2}}{\isachardoublequoteclose}\isanewline
\ \ \ \ \ \ \ \ \isacommand{have}\isamarkupfalse%
\ f{\isadigit{1}}{\isacharcolon}{\kern0pt}\ {\isachardoublequoteopen}{\isacharparenleft}{\kern0pt}expr{\isacharunderscore}{\kern0pt}{\isadigit{5}}\ {\isasymcirc}\ {\isacharparenleft}{\kern0pt}{\isasymlambda}sa{\isachardot}{\kern0pt}\ if\ sa\ {\isacharequal}{\kern0pt}\ s\ then\ hml{\isacharunderscore}{\kern0pt}pos\ a\ {\isacharparenleft}{\kern0pt}hml{\isacharunderscore}{\kern0pt}pos\ c\ {\isacharparenleft}{\kern0pt}TT{\isacharcolon}{\kern0pt}{\isacharcolon}{\kern0pt}{\isacharparenleft}{\kern0pt}{\isacharprime}{\kern0pt}a{\isacharcomma}{\kern0pt}\ {\isacharprime}{\kern0pt}s{\isacharparenright}{\kern0pt}\ hml{\isacharparenright}{\kern0pt}{\isacharparenright}{\kern0pt}\ else\ if\ sa\ {\isacharequal}{\kern0pt}\ t\ then\ hml{\isacharunderscore}{\kern0pt}pos\ b\ TT\ else\ undefined{\isacharparenright}{\kern0pt}{\isacharparenright}{\kern0pt}\ {\isacharbackquote}{\kern0pt}\ {\isacharbraceleft}{\kern0pt}{\isacharbraceright}{\kern0pt}\ {\isasymunion}\ {\isacharparenleft}{\kern0pt}expr{\isacharunderscore}{\kern0pt}{\isadigit{5}}\ {\isasymcirc}\ {\isacharparenleft}{\kern0pt}{\isasymlambda}sa{\isachardot}{\kern0pt}\ if\ sa\ {\isacharequal}{\kern0pt}\ s\ then\ hml{\isacharunderscore}{\kern0pt}pos\ a\ {\isacharparenleft}{\kern0pt}hml{\isacharunderscore}{\kern0pt}pos\ c\ {\isacharparenleft}{\kern0pt}TT{\isacharcolon}{\kern0pt}{\isacharcolon}{\kern0pt}{\isacharparenleft}{\kern0pt}{\isacharprime}{\kern0pt}a{\isacharcomma}{\kern0pt}\ {\isacharprime}{\kern0pt}s{\isacharparenright}{\kern0pt}\ hml{\isacharparenright}{\kern0pt}{\isacharparenright}{\kern0pt}\ else\ if\ sa\ {\isacharequal}{\kern0pt}\ t\ then\ hml{\isacharunderscore}{\kern0pt}pos\ b\ TT\ else\ undefined{\isacharparenright}{\kern0pt}{\isacharparenright}{\kern0pt}\ {\isacharbackquote}{\kern0pt}\ {\isacharbraceleft}{\kern0pt}s{\isacharcomma}{\kern0pt}\ t{\isacharbraceright}{\kern0pt}\ {\isasymunion}\ {\isacharparenleft}{\kern0pt}expr{\isacharunderscore}{\kern0pt}{\isadigit{1}}\ {\isasymcirc}\ {\isacharparenleft}{\kern0pt}{\isasymlambda}sa{\isachardot}{\kern0pt}\ if\ sa\ {\isacharequal}{\kern0pt}\ s\ then\ hml{\isacharunderscore}{\kern0pt}pos\ a\ {\isacharparenleft}{\kern0pt}hml{\isacharunderscore}{\kern0pt}pos\ c\ {\isacharparenleft}{\kern0pt}TT{\isacharcolon}{\kern0pt}{\isacharcolon}{\kern0pt}{\isacharparenleft}{\kern0pt}{\isacharprime}{\kern0pt}a{\isacharcomma}{\kern0pt}\ {\isacharprime}{\kern0pt}s{\isacharparenright}{\kern0pt}\ hml{\isacharparenright}{\kern0pt}{\isacharparenright}{\kern0pt}\ else\ if\ sa\ {\isacharequal}{\kern0pt}\ t\ then\ hml{\isacharunderscore}{\kern0pt}pos\ b\ TT\ else\ undefined{\isacharparenright}{\kern0pt}{\isacharparenright}{\kern0pt}\ {\isacharbackquote}{\kern0pt}\ {\isacharbraceleft}{\kern0pt}s{\isacharcomma}{\kern0pt}\ t{\isacharbraceright}{\kern0pt}\ {\isacharequal}{\kern0pt}\ {\isacharparenleft}{\kern0pt}expr{\isacharunderscore}{\kern0pt}{\isadigit{5}}\ {\isasymcirc}\ {\isacharparenleft}{\kern0pt}{\isasymlambda}sa{\isachardot}{\kern0pt}\ if\ sa\ {\isacharequal}{\kern0pt}\ s\ then\ hml{\isacharunderscore}{\kern0pt}pos\ a\ {\isacharparenleft}{\kern0pt}hml{\isacharunderscore}{\kern0pt}pos\ c\ {\isacharparenleft}{\kern0pt}TT{\isacharcolon}{\kern0pt}{\isacharcolon}{\kern0pt}{\isacharparenleft}{\kern0pt}{\isacharprime}{\kern0pt}a{\isacharcomma}{\kern0pt}\ {\isacharprime}{\kern0pt}s{\isacharparenright}{\kern0pt}\ hml{\isacharparenright}{\kern0pt}{\isacharparenright}{\kern0pt}\ else\ if\ sa\ {\isacharequal}{\kern0pt}\ t\ then\ hml{\isacharunderscore}{\kern0pt}pos\ b\ TT\ else\ undefined{\isacharparenright}{\kern0pt}{\isacharparenright}{\kern0pt}\ {\isacharbackquote}{\kern0pt}\ {\isacharbraceleft}{\kern0pt}{\isacharbraceright}{\kern0pt}\ {\isasymunion}\ {\isacharparenleft}{\kern0pt}expr{\isacharunderscore}{\kern0pt}{\isadigit{5}}\ {\isasymcirc}\ {\isacharparenleft}{\kern0pt}{\isasymlambda}sa{\isachardot}{\kern0pt}\ if\ sa\ {\isacharequal}{\kern0pt}\ s\ then\ hml{\isacharunderscore}{\kern0pt}pos\ a\ {\isacharparenleft}{\kern0pt}hml{\isacharunderscore}{\kern0pt}pos\ c\ {\isacharparenleft}{\kern0pt}TT{\isacharcolon}{\kern0pt}{\isacharcolon}{\kern0pt}{\isacharparenleft}{\kern0pt}{\isacharprime}{\kern0pt}a{\isacharcomma}{\kern0pt}\ {\isacharprime}{\kern0pt}s{\isacharparenright}{\kern0pt}\ hml{\isacharparenright}{\kern0pt}{\isacharparenright}{\kern0pt}\ else\ if\ sa\ {\isacharequal}{\kern0pt}\ t\ then\ hml{\isacharunderscore}{\kern0pt}pos\ b\ TT\ else\ undefined{\isacharparenright}{\kern0pt}{\isacharparenright}{\kern0pt}\ {\isacharbackquote}{\kern0pt}\ {\isacharbraceleft}{\kern0pt}t{\isacharcomma}{\kern0pt}\ s{\isacharbraceright}{\kern0pt}\ {\isasymunion}\ insert\ {\isacharparenleft}{\kern0pt}{\isacharparenleft}{\kern0pt}expr{\isacharunderscore}{\kern0pt}{\isadigit{1}}\ {\isasymcirc}\ {\isacharparenleft}{\kern0pt}{\isasymlambda}sa{\isachardot}{\kern0pt}\ if\ sa\ {\isacharequal}{\kern0pt}\ s\ then\ hml{\isacharunderscore}{\kern0pt}pos\ a\ {\isacharparenleft}{\kern0pt}hml{\isacharunderscore}{\kern0pt}pos\ c\ {\isacharparenleft}{\kern0pt}TT{\isacharcolon}{\kern0pt}{\isacharcolon}{\kern0pt}{\isacharparenleft}{\kern0pt}{\isacharprime}{\kern0pt}a{\isacharcomma}{\kern0pt}\ {\isacharprime}{\kern0pt}s{\isacharparenright}{\kern0pt}\ hml{\isacharparenright}{\kern0pt}{\isacharparenright}{\kern0pt}\ else\ if\ sa\ {\isacharequal}{\kern0pt}\ t\ then\ hml{\isacharunderscore}{\kern0pt}pos\ b\ TT\ else\ undefined{\isacharparenright}{\kern0pt}{\isacharparenright}{\kern0pt}\ t{\isacharparenright}{\kern0pt}\ {\isacharparenleft}{\kern0pt}{\isacharparenleft}{\kern0pt}expr{\isacharunderscore}{\kern0pt}{\isadigit{1}}\ {\isasymcirc}\ {\isacharparenleft}{\kern0pt}{\isasymlambda}sa{\isachardot}{\kern0pt}\ if\ sa\ {\isacharequal}{\kern0pt}\ s\ then\ hml{\isacharunderscore}{\kern0pt}pos\ a\ {\isacharparenleft}{\kern0pt}hml{\isacharunderscore}{\kern0pt}pos\ c\ {\isacharparenleft}{\kern0pt}TT{\isacharcolon}{\kern0pt}{\isacharcolon}{\kern0pt}{\isacharparenleft}{\kern0pt}{\isacharprime}{\kern0pt}a{\isacharcomma}{\kern0pt}\ {\isacharprime}{\kern0pt}s{\isacharparenright}{\kern0pt}\ hml{\isacharparenright}{\kern0pt}{\isacharparenright}{\kern0pt}\ else\ if\ sa\ {\isacharequal}{\kern0pt}\ t\ then\ hml{\isacharunderscore}{\kern0pt}pos\ b\ TT\ else\ undefined{\isacharparenright}{\kern0pt}{\isacharparenright}{\kern0pt}\ {\isacharbackquote}{\kern0pt}\ {\isacharbraceleft}{\kern0pt}s{\isacharbraceright}{\kern0pt}{\isacharparenright}{\kern0pt}{\isachardoublequoteclose}\isanewline
\ \ \ \ \ \ \ \ \ \ \isacommand{by}\isamarkupfalse%
\ {\isacharparenleft}{\kern0pt}simp\ add{\isacharcolon}{\kern0pt}\ insert{\isacharunderscore}{\kern0pt}commute{\isacharparenright}{\kern0pt}\isanewline
\ \ \ \ \ \ \ \ \isacommand{have}\isamarkupfalse%
\ f{\isadigit{2}}{\isacharcolon}{\kern0pt}\ {\isachardoublequoteopen}expr{\isacharunderscore}{\kern0pt}{\isadigit{5}}\ {\isacharparenleft}{\kern0pt}TT{\isacharcolon}{\kern0pt}{\isacharcolon}{\kern0pt}{\isacharparenleft}{\kern0pt}{\isacharprime}{\kern0pt}a{\isacharcomma}{\kern0pt}\ {\isacharprime}{\kern0pt}s{\isacharparenright}{\kern0pt}\ hml{\isacharparenright}{\kern0pt}\ {\isacharequal}{\kern0pt}\ expr{\isacharunderscore}{\kern0pt}{\isadigit{5}}\ {\isacharparenleft}{\kern0pt}if\ t\ {\isacharequal}{\kern0pt}\ s\ then\ hml{\isacharunderscore}{\kern0pt}pos\ a\ {\isacharparenleft}{\kern0pt}hml{\isacharunderscore}{\kern0pt}pos\ c\ {\isacharparenleft}{\kern0pt}TT{\isacharcolon}{\kern0pt}{\isacharcolon}{\kern0pt}{\isacharparenleft}{\kern0pt}{\isacharprime}{\kern0pt}a{\isacharcomma}{\kern0pt}\ {\isacharprime}{\kern0pt}s{\isacharparenright}{\kern0pt}\ hml{\isacharparenright}{\kern0pt}{\isacharparenright}{\kern0pt}\ else\ if\ t\ {\isacharequal}{\kern0pt}\ t\ then\ hml{\isacharunderscore}{\kern0pt}pos\ b\ TT\ else\ undefined{\isacharparenright}{\kern0pt}{\isachardoublequoteclose}\isanewline
\ \ \ \ \ \ \ \ \ \ \isacommand{by}\isamarkupfalse%
\ auto\isanewline
\ \ \ \ \ \ \ \ \isacommand{have}\isamarkupfalse%
\ f{\isadigit{3}}{\isacharcolon}{\kern0pt}\ {\isachardoublequoteopen}insert\ {\isacharparenleft}{\kern0pt}expr{\isacharunderscore}{\kern0pt}{\isadigit{5}}\ {\isacharparenleft}{\kern0pt}if\ t\ {\isacharequal}{\kern0pt}\ s\ then\ hml{\isacharunderscore}{\kern0pt}pos\ a\ {\isacharparenleft}{\kern0pt}hml{\isacharunderscore}{\kern0pt}pos\ c\ {\isacharparenleft}{\kern0pt}TT{\isacharcolon}{\kern0pt}{\isacharcolon}{\kern0pt}{\isacharparenleft}{\kern0pt}{\isacharprime}{\kern0pt}a{\isacharcomma}{\kern0pt}\ {\isacharprime}{\kern0pt}s{\isacharparenright}{\kern0pt}\ hml{\isacharparenright}{\kern0pt}{\isacharparenright}{\kern0pt}\ else\ if\ t\ {\isacharequal}{\kern0pt}\ t\ then\ hml{\isacharunderscore}{\kern0pt}pos\ b\ TT\ else\ undefined{\isacharparenright}{\kern0pt}{\isacharparenright}{\kern0pt}\ {\isacharparenleft}{\kern0pt}insert\ {\isacharparenleft}{\kern0pt}{\isacharparenleft}{\kern0pt}expr{\isacharunderscore}{\kern0pt}{\isadigit{5}}\ {\isasymcirc}\ {\isacharparenleft}{\kern0pt}{\isasymlambda}sa{\isachardot}{\kern0pt}\ if\ sa\ {\isacharequal}{\kern0pt}\ s\ then\ hml{\isacharunderscore}{\kern0pt}pos\ a\ {\isacharparenleft}{\kern0pt}hml{\isacharunderscore}{\kern0pt}pos\ c\ {\isacharparenleft}{\kern0pt}TT{\isacharcolon}{\kern0pt}{\isacharcolon}{\kern0pt}{\isacharparenleft}{\kern0pt}{\isacharprime}{\kern0pt}a{\isacharcomma}{\kern0pt}\ {\isacharprime}{\kern0pt}s{\isacharparenright}{\kern0pt}\ hml{\isacharparenright}{\kern0pt}{\isacharparenright}{\kern0pt}\ else\ if\ sa\ {\isacharequal}{\kern0pt}\ t\ then\ hml{\isacharunderscore}{\kern0pt}pos\ b\ TT\ else\ undefined{\isacharparenright}{\kern0pt}{\isacharparenright}{\kern0pt}\ s{\isacharparenright}{\kern0pt}\ {\isacharparenleft}{\kern0pt}{\isacharparenleft}{\kern0pt}expr{\isacharunderscore}{\kern0pt}{\isadigit{5}}\ {\isasymcirc}\ {\isacharparenleft}{\kern0pt}{\isasymlambda}sa{\isachardot}{\kern0pt}\ if\ sa\ {\isacharequal}{\kern0pt}\ s\ then\ hml{\isacharunderscore}{\kern0pt}pos\ a\ {\isacharparenleft}{\kern0pt}hml{\isacharunderscore}{\kern0pt}pos\ c\ {\isacharparenleft}{\kern0pt}TT{\isacharcolon}{\kern0pt}{\isacharcolon}{\kern0pt}{\isacharparenleft}{\kern0pt}{\isacharprime}{\kern0pt}a{\isacharcomma}{\kern0pt}\ {\isacharprime}{\kern0pt}s{\isacharparenright}{\kern0pt}\ hml{\isacharparenright}{\kern0pt}{\isacharparenright}{\kern0pt}\ else\ if\ sa\ {\isacharequal}{\kern0pt}\ t\ then\ hml{\isacharunderscore}{\kern0pt}pos\ b\ TT\ else\ undefined{\isacharparenright}{\kern0pt}{\isacharparenright}{\kern0pt}\ {\isacharbackquote}{\kern0pt}\ {\isacharbraceleft}{\kern0pt}{\isacharbraceright}{\kern0pt}{\isacharparenright}{\kern0pt}{\isacharparenright}{\kern0pt}\ {\isacharequal}{\kern0pt}\ {\isacharparenleft}{\kern0pt}expr{\isacharunderscore}{\kern0pt}{\isadigit{5}}\ {\isasymcirc}\ {\isacharparenleft}{\kern0pt}{\isasymlambda}sa{\isachardot}{\kern0pt}\ if\ sa\ {\isacharequal}{\kern0pt}\ s\ then\ hml{\isacharunderscore}{\kern0pt}pos\ a\ {\isacharparenleft}{\kern0pt}hml{\isacharunderscore}{\kern0pt}pos\ c\ {\isacharparenleft}{\kern0pt}TT{\isacharcolon}{\kern0pt}{\isacharcolon}{\kern0pt}{\isacharparenleft}{\kern0pt}{\isacharprime}{\kern0pt}a{\isacharcomma}{\kern0pt}\ {\isacharprime}{\kern0pt}s{\isacharparenright}{\kern0pt}\ hml{\isacharparenright}{\kern0pt}{\isacharparenright}{\kern0pt}\ else\ if\ sa\ {\isacharequal}{\kern0pt}\ t\ then\ hml{\isacharunderscore}{\kern0pt}pos\ b\ TT\ else\ undefined{\isacharparenright}{\kern0pt}{\isacharparenright}{\kern0pt}\ {\isacharbackquote}{\kern0pt}\ {\isacharbraceleft}{\kern0pt}{\isacharbraceright}{\kern0pt}\ {\isasymunion}\ {\isacharparenleft}{\kern0pt}expr{\isacharunderscore}{\kern0pt}{\isadigit{5}}\ {\isasymcirc}\ {\isacharparenleft}{\kern0pt}{\isasymlambda}sa{\isachardot}{\kern0pt}\ if\ sa\ {\isacharequal}{\kern0pt}\ s\ then\ hml{\isacharunderscore}{\kern0pt}pos\ a\ {\isacharparenleft}{\kern0pt}hml{\isacharunderscore}{\kern0pt}pos\ c\ {\isacharparenleft}{\kern0pt}TT{\isacharcolon}{\kern0pt}{\isacharcolon}{\kern0pt}{\isacharparenleft}{\kern0pt}{\isacharprime}{\kern0pt}a{\isacharcomma}{\kern0pt}\ {\isacharprime}{\kern0pt}s{\isacharparenright}{\kern0pt}\ hml{\isacharparenright}{\kern0pt}{\isacharparenright}{\kern0pt}\ else\ if\ sa\ {\isacharequal}{\kern0pt}\ t\ then\ hml{\isacharunderscore}{\kern0pt}pos\ b\ TT\ else\ undefined{\isacharparenright}{\kern0pt}{\isacharparenright}{\kern0pt}\ {\isacharbackquote}{\kern0pt}\ {\isacharbraceleft}{\kern0pt}t{\isacharcomma}{\kern0pt}\ s{\isacharbraceright}{\kern0pt}{\isachardoublequoteclose}\isanewline
\ \ \ \ \ \ \ \ \ \ \isacommand{by}\isamarkupfalse%
\ auto\isanewline
\ \ \ \ \ \ \ \ \isacommand{have}\isamarkupfalse%
\ {\isachardoublequoteopen}expr{\isacharunderscore}{\kern0pt}{\isadigit{5}}\ {\isacharparenleft}{\kern0pt}hml{\isacharunderscore}{\kern0pt}conj\ {\isacharbraceleft}{\kern0pt}s{\isacharcomma}{\kern0pt}\ t{\isacharbraceright}{\kern0pt}\ {\isacharbraceleft}{\kern0pt}{\isacharbraceright}{\kern0pt}\ {\isacharparenleft}{\kern0pt}{\isasymlambda}sa{\isachardot}{\kern0pt}\ if\ sa\ {\isacharequal}{\kern0pt}\ s\ then\ hml{\isacharunderscore}{\kern0pt}pos\ b\ TT\ else\ if\ sa\ {\isacharequal}{\kern0pt}\ t\ then\ hml{\isacharunderscore}{\kern0pt}pos\ a\ {\isacharparenleft}{\kern0pt}hml{\isacharunderscore}{\kern0pt}conj\ {\isacharbraceleft}{\kern0pt}{\isacharbraceright}{\kern0pt}\ {\isacharbraceleft}{\kern0pt}s{\isacharcomma}{\kern0pt}\ t{\isacharbraceright}{\kern0pt}\ {\isacharparenleft}{\kern0pt}{\isasymlambda}sa{\isachardot}{\kern0pt}\ if\ sa\ {\isacharequal}{\kern0pt}\ s\ then\ hml{\isacharunderscore}{\kern0pt}pos\ a\ {\isacharparenleft}{\kern0pt}hml{\isacharunderscore}{\kern0pt}pos\ c\ TT{\isacharparenright}{\kern0pt}\ else\ if\ sa\ {\isacharequal}{\kern0pt}\ t\ then\ hml{\isacharunderscore}{\kern0pt}pos\ b\ TT\ else\ undefined{\isacharparenright}{\kern0pt}{\isacharparenright}{\kern0pt}\ else\ undefined{\isacharparenright}{\kern0pt}{\isacharparenright}{\kern0pt}\ {\isacharequal}{\kern0pt}\ Sup\ {\isacharparenleft}{\kern0pt}insert\ {\isacharparenleft}{\kern0pt}{\isacharparenleft}{\kern0pt}expr{\isacharunderscore}{\kern0pt}{\isadigit{5}}\ {\isasymcirc}\ {\isacharparenleft}{\kern0pt}{\isasymlambda}sa{\isachardot}{\kern0pt}\ if\ sa\ {\isacharequal}{\kern0pt}\ s\ then\ hml{\isacharunderscore}{\kern0pt}pos\ a\ {\isacharparenleft}{\kern0pt}hml{\isacharunderscore}{\kern0pt}pos\ c\ {\isacharparenleft}{\kern0pt}TT{\isacharcolon}{\kern0pt}{\isacharcolon}{\kern0pt}{\isacharparenleft}{\kern0pt}{\isacharprime}{\kern0pt}a{\isacharcomma}{\kern0pt}\ {\isacharprime}{\kern0pt}s{\isacharparenright}{\kern0pt}\ hml{\isacharparenright}{\kern0pt}{\isacharparenright}{\kern0pt}\ else\ if\ sa\ {\isacharequal}{\kern0pt}\ t\ then\ hml{\isacharunderscore}{\kern0pt}pos\ b\ TT\ else\ undefined{\isacharparenright}{\kern0pt}{\isacharparenright}{\kern0pt}\ s{\isacharparenright}{\kern0pt}\ {\isacharparenleft}{\kern0pt}{\isacharparenleft}{\kern0pt}expr{\isacharunderscore}{\kern0pt}{\isadigit{5}}\ {\isasymcirc}\ {\isacharparenleft}{\kern0pt}{\isasymlambda}sa{\isachardot}{\kern0pt}\ if\ sa\ {\isacharequal}{\kern0pt}\ s\ then\ hml{\isacharunderscore}{\kern0pt}pos\ a\ {\isacharparenleft}{\kern0pt}hml{\isacharunderscore}{\kern0pt}pos\ c\ {\isacharparenleft}{\kern0pt}TT{\isacharcolon}{\kern0pt}{\isacharcolon}{\kern0pt}{\isacharparenleft}{\kern0pt}{\isacharprime}{\kern0pt}a{\isacharcomma}{\kern0pt}\ {\isacharprime}{\kern0pt}s{\isacharparenright}{\kern0pt}\ hml{\isacharparenright}{\kern0pt}{\isacharparenright}{\kern0pt}\ else\ if\ sa\ {\isacharequal}{\kern0pt}\ t\ then\ hml{\isacharunderscore}{\kern0pt}pos\ b\ TT\ else\ undefined{\isacharparenright}{\kern0pt}{\isacharparenright}{\kern0pt}\ {\isacharbackquote}{\kern0pt}\ {\isacharbraceleft}{\kern0pt}{\isacharbraceright}{\kern0pt}{\isacharparenright}{\kern0pt}\ {\isasymunion}\ insert\ {\isacharparenleft}{\kern0pt}{\isacharparenleft}{\kern0pt}expr{\isacharunderscore}{\kern0pt}{\isadigit{1}}\ {\isasymcirc}\ {\isacharparenleft}{\kern0pt}{\isasymlambda}sa{\isachardot}{\kern0pt}\ if\ sa\ {\isacharequal}{\kern0pt}\ s\ then\ hml{\isacharunderscore}{\kern0pt}pos\ a\ {\isacharparenleft}{\kern0pt}hml{\isacharunderscore}{\kern0pt}pos\ c\ {\isacharparenleft}{\kern0pt}TT{\isacharcolon}{\kern0pt}{\isacharcolon}{\kern0pt}{\isacharparenleft}{\kern0pt}{\isacharprime}{\kern0pt}a{\isacharcomma}{\kern0pt}\ {\isacharprime}{\kern0pt}s{\isacharparenright}{\kern0pt}\ hml{\isacharparenright}{\kern0pt}{\isacharparenright}{\kern0pt}\ else\ if\ sa\ {\isacharequal}{\kern0pt}\ t\ then\ hml{\isacharunderscore}{\kern0pt}pos\ b\ TT\ else\ undefined{\isacharparenright}{\kern0pt}{\isacharparenright}{\kern0pt}\ t{\isacharparenright}{\kern0pt}\ {\isacharparenleft}{\kern0pt}{\isacharparenleft}{\kern0pt}expr{\isacharunderscore}{\kern0pt}{\isadigit{1}}\ {\isasymcirc}\ {\isacharparenleft}{\kern0pt}{\isasymlambda}sa{\isachardot}{\kern0pt}\ if\ sa\ {\isacharequal}{\kern0pt}\ s\ then\ hml{\isacharunderscore}{\kern0pt}pos\ a\ {\isacharparenleft}{\kern0pt}hml{\isacharunderscore}{\kern0pt}pos\ c\ {\isacharparenleft}{\kern0pt}TT{\isacharcolon}{\kern0pt}{\isacharcolon}{\kern0pt}{\isacharparenleft}{\kern0pt}{\isacharprime}{\kern0pt}a{\isacharcomma}{\kern0pt}\ {\isacharprime}{\kern0pt}s{\isacharparenright}{\kern0pt}\ hml{\isacharparenright}{\kern0pt}{\isacharparenright}{\kern0pt}\ else\ if\ sa\ {\isacharequal}{\kern0pt}\ t\ then\ hml{\isacharunderscore}{\kern0pt}pos\ b\ TT\ else\ undefined{\isacharparenright}{\kern0pt}{\isacharparenright}{\kern0pt}\ {\isacharbackquote}{\kern0pt}\ {\isacharbraceleft}{\kern0pt}s{\isacharbraceright}{\kern0pt}{\isacharparenright}{\kern0pt}\ {\isasymunion}\ insert\ {\isacharparenleft}{\kern0pt}{\isacharparenleft}{\kern0pt}expr{\isacharunderscore}{\kern0pt}{\isadigit{5}}\ {\isasymcirc}\ {\isacharparenleft}{\kern0pt}{\isasymlambda}sa{\isachardot}{\kern0pt}\ if\ sa\ {\isacharequal}{\kern0pt}\ s\ then\ hml{\isacharunderscore}{\kern0pt}pos\ a\ {\isacharparenleft}{\kern0pt}hml{\isacharunderscore}{\kern0pt}pos\ c\ {\isacharparenleft}{\kern0pt}TT{\isacharcolon}{\kern0pt}{\isacharcolon}{\kern0pt}{\isacharparenleft}{\kern0pt}{\isacharprime}{\kern0pt}a{\isacharcomma}{\kern0pt}\ {\isacharprime}{\kern0pt}s{\isacharparenright}{\kern0pt}\ hml{\isacharparenright}{\kern0pt}{\isacharparenright}{\kern0pt}\ else\ if\ sa\ {\isacharequal}{\kern0pt}\ t\ then\ hml{\isacharunderscore}{\kern0pt}pos\ b\ TT\ else\ undefined{\isacharparenright}{\kern0pt}{\isacharparenright}{\kern0pt}\ s{\isacharparenright}{\kern0pt}\ {\isacharparenleft}{\kern0pt}{\isacharparenleft}{\kern0pt}expr{\isacharunderscore}{\kern0pt}{\isadigit{5}}\ {\isasymcirc}\ {\isacharparenleft}{\kern0pt}{\isasymlambda}sa{\isachardot}{\kern0pt}\ if\ sa\ {\isacharequal}{\kern0pt}\ s\ then\ hml{\isacharunderscore}{\kern0pt}pos\ a\ {\isacharparenleft}{\kern0pt}hml{\isacharunderscore}{\kern0pt}pos\ c\ {\isacharparenleft}{\kern0pt}TT{\isacharcolon}{\kern0pt}{\isacharcolon}{\kern0pt}{\isacharparenleft}{\kern0pt}{\isacharprime}{\kern0pt}a{\isacharcomma}{\kern0pt}\ {\isacharprime}{\kern0pt}s{\isacharparenright}{\kern0pt}\ hml{\isacharparenright}{\kern0pt}{\isacharparenright}{\kern0pt}\ else\ if\ sa\ {\isacharequal}{\kern0pt}\ t\ then\ hml{\isacharunderscore}{\kern0pt}pos\ b\ TT\ else\ undefined{\isacharparenright}{\kern0pt}{\isacharparenright}{\kern0pt}\ {\isacharbackquote}{\kern0pt}\ {\isacharbraceleft}{\kern0pt}{\isacharbraceright}{\kern0pt}{\isacharparenright}{\kern0pt}{\isacharparenright}{\kern0pt}{\isachardoublequoteclose}\isanewline
\ \ \ \ \ \ \ \ \ \ \isacommand{using}\isamarkupfalse%
\ f{\isadigit{1}}\ {\isacartoucheopen}s\ {\isasymnoteq}\ t{\isacartoucheclose}\ \isacommand{by}\isamarkupfalse%
\ auto\isanewline
\ \ \ \ \ \ \ \ \isacommand{then}\isamarkupfalse%
\ \isacommand{show}\isamarkupfalse%
\ {\isacharquery}{\kern0pt}thesis\isanewline
\ \ \ \ \ \ \ \ \ \ \isacommand{using}\isamarkupfalse%
\ f{\isadigit{3}}\ f{\isadigit{2}}\ \isacommand{by}\isamarkupfalse%
\ {\isacharparenleft}{\kern0pt}smt\ {\isacharparenleft}{\kern0pt}z{\isadigit{3}}{\isacharparenright}{\kern0pt}\ {\isacartoucheopen}{\isasymAnd}a\ A{\isachardot}{\kern0pt}\ insert\ a\ A\ {\isacharequal}{\kern0pt}\ {\isacharbraceleft}{\kern0pt}a{\isacharbraceright}{\kern0pt}\ {\isasymunion}\ A{\isacartoucheclose}\ {\isacartoucheopen}{\isasymAnd}a\ C\ B{\isachardot}{\kern0pt}\ insert\ a\ B\ {\isasymunion}\ C\ {\isacharequal}{\kern0pt}\ insert\ a\ {\isacharparenleft}{\kern0pt}B\ {\isasymunion}\ C{\isacharparenright}{\kern0pt}{\isacartoucheclose}\ {\isacartoucheopen}{\isasymAnd}f\ B\ A{\isachardot}{\kern0pt}\ f\ {\isacharbackquote}{\kern0pt}\ {\isacharparenleft}{\kern0pt}A\ {\isasymunion}\ B{\isacharparenright}{\kern0pt}\ {\isacharequal}{\kern0pt}\ f\ {\isacharbackquote}{\kern0pt}\ A\ {\isasymunion}\ f\ {\isacharbackquote}{\kern0pt}\ B{\isacartoucheclose}\ {\isacartoucheopen}{\isasymAnd}f\ a\ B{\isachardot}{\kern0pt}\ f\ {\isacharbackquote}{\kern0pt}\ insert\ a\ B\ {\isacharequal}{\kern0pt}\ insert\ {\isacharparenleft}{\kern0pt}f\ a{\isacharparenright}{\kern0pt}\ {\isacharparenleft}{\kern0pt}f\ {\isacharbackquote}{\kern0pt}\ B{\isacharparenright}{\kern0pt}{\isacartoucheclose}\ {\isacartoucheopen}{\isasymAnd}x\ g\ f{\isachardot}{\kern0pt}\ {\isacharparenleft}{\kern0pt}f\ {\isasymcirc}\ g{\isacharparenright}{\kern0pt}\ x\ {\isacharequal}{\kern0pt}\ f\ {\isacharparenleft}{\kern0pt}g\ x{\isacharparenright}{\kern0pt}{\isacartoucheclose}\ {\isacartoucheopen}expr{\isacharunderscore}{\kern0pt}{\isadigit{5}}\ {\isacharparenleft}{\kern0pt}hml{\isacharunderscore}{\kern0pt}conj\ {\isacharbraceleft}{\kern0pt}{\isacharbraceright}{\kern0pt}\ {\isacharbraceleft}{\kern0pt}s{\isacharcomma}{\kern0pt}\ t{\isacharbraceright}{\kern0pt}\ {\isacharparenleft}{\kern0pt}{\isasymlambda}j{\isachardot}{\kern0pt}\ if\ j\ {\isacharequal}{\kern0pt}\ s\ then\ hml{\isacharunderscore}{\kern0pt}pos\ a\ {\isacharparenleft}{\kern0pt}hml{\isacharunderscore}{\kern0pt}pos\ c\ TT{\isacharparenright}{\kern0pt}\ else\ if\ j\ {\isacharequal}{\kern0pt}\ t\ then\ hml{\isacharunderscore}{\kern0pt}pos\ b\ TT\ else\ undefined{\isacharparenright}{\kern0pt}{\isacharparenright}{\kern0pt}\ {\isacharequal}{\kern0pt}\ {\isadigit{2}}{\isacartoucheclose}\ empty{\isacharunderscore}{\kern0pt}is{\isacharunderscore}{\kern0pt}image\ expr{\isacharunderscore}{\kern0pt}{\isadigit{5}}{\isacharunderscore}{\kern0pt}conj\ expr{\isacharunderscore}{\kern0pt}{\isadigit{5}}{\isacharunderscore}{\kern0pt}pos\ insert{\isacharunderscore}{\kern0pt}commute{\isacharparenright}{\kern0pt}\isanewline
\ \ \ \ \ \ \isacommand{qed}\isamarkupfalse%
\isanewline
\ \ \ \ \ \ \isacommand{thus}\isamarkupfalse%
\ {\isacharquery}{\kern0pt}thesis\isanewline
\ \ \ \ \ \ \ \ \isacommand{unfolding}\isamarkupfalse%
\ {\isasymphi}\isanewline
\ \ \ \ \ \ \ \ \isacommand{by}\isamarkupfalse%
\ force\isanewline
\ \ \ \ \isacommand{qed}\isamarkupfalse%
\isanewline
\isanewline
\ \ \ \ \isacommand{show}\isamarkupfalse%
\ e{\isadigit{6}}{\isacharcolon}{\kern0pt}\ {\isachardoublequoteopen}expr{\isacharunderscore}{\kern0pt}{\isadigit{6}}\ {\isasymphi}\ {\isacharequal}{\kern0pt}\ {\isadigit{1}}{\isachardoublequoteclose}\isanewline
\ \ \ \ \isacommand{proof}\isamarkupfalse%
{\isacharminus}{\kern0pt}\isanewline
\ \ \ \ \ \ \isacommand{have}\isamarkupfalse%
\ e{\isadigit{6}}{\isacharunderscore}{\kern0pt}{\isadigit{1}}{\isacharcolon}{\kern0pt}\ {\isachardoublequoteopen}Sup\ {\isacharparenleft}{\kern0pt}{\isacharparenleft}{\kern0pt}expr{\isacharunderscore}{\kern0pt}{\isadigit{6}}\ {\isasymcirc}\ {\isacharparenleft}{\kern0pt}{\isasymlambda}j{\isachardot}{\kern0pt}\ if\ j\ {\isacharequal}{\kern0pt}\ s\ then\ hml{\isacharunderscore}{\kern0pt}pos\ a\ {\isacharparenleft}{\kern0pt}hml{\isacharunderscore}{\kern0pt}pos\ c\ TT{\isacharparenright}{\kern0pt}\isanewline
\ \ \ \ \ \ \ \ \ \ \ \ \ \ \ \ \ \ \ \ \ \ \ \ \ \ \ \ \ \ \ \ \ \ else\ if\ j\ {\isacharequal}{\kern0pt}\ t\ then\ hml{\isacharunderscore}{\kern0pt}pos\ b\ TT\ else\ undefined{\isacharparenright}{\kern0pt}{\isacharparenright}{\kern0pt}\ {\isacharbackquote}{\kern0pt}\ {\isacharbraceleft}{\kern0pt}{\isacharbraceright}{\kern0pt}{\isacharparenright}{\kern0pt}\ {\isacharequal}{\kern0pt}\ {\isadigit{0}}{\isachardoublequoteclose}\isanewline
\ \ \ \ \ \ \ \ \isacommand{using}\isamarkupfalse%
\ bot{\isacharunderscore}{\kern0pt}enat{\isacharunderscore}{\kern0pt}def\ \isacommand{by}\isamarkupfalse%
\ force\ \isanewline
\ \ \ \ \ \ \isacommand{have}\isamarkupfalse%
\ {\isachardoublequoteopen}Sup\ {\isacharparenleft}{\kern0pt}{\isacharparenleft}{\kern0pt}{\isacharparenleft}{\kern0pt}expr{\isacharunderscore}{\kern0pt}{\isadigit{6}}\ {\isasymcirc}\ {\isacharparenleft}{\kern0pt}{\isasymlambda}j{\isachardot}{\kern0pt}\ if\ j\ {\isacharequal}{\kern0pt}\ s\ then\ hml{\isacharunderscore}{\kern0pt}pos\ a\ {\isacharparenleft}{\kern0pt}hml{\isacharunderscore}{\kern0pt}pos\ c\ TT{\isacharparenright}{\kern0pt}\isanewline
\ \ \ \ \ \ \ \ \ \ \ \ \ \ \ \ \ \ \ \ \ \ \ \ \ \ \ \ \ \ \ \ \ \ else\ if\ j\ {\isacharequal}{\kern0pt}\ t\ then\ hml{\isacharunderscore}{\kern0pt}pos\ b\ TT\ else\ undefined{\isacharparenright}{\kern0pt}{\isacharparenright}{\kern0pt}\ {\isacharbackquote}{\kern0pt}\ {\isacharbraceleft}{\kern0pt}s{\isacharcomma}{\kern0pt}\ t{\isacharbraceright}{\kern0pt}{\isacharparenright}{\kern0pt}{\isacharparenright}{\kern0pt}\ {\isacharequal}{\kern0pt}\ {\isadigit{0}}{\isachardoublequoteclose}\isanewline
\ \ \ \ \ \ \ \ \isacommand{by}\isamarkupfalse%
\ simp\isanewline
\ \ \ \ \ \ \isacommand{hence}\isamarkupfalse%
\ e{\isadigit{6}}{\isacharunderscore}{\kern0pt}{\isadigit{2}}{\isacharcolon}{\kern0pt}\ {\isachardoublequoteopen}Sup\ {\isacharparenleft}{\kern0pt}{\isacharparenleft}{\kern0pt}{\isacharparenleft}{\kern0pt}eSuc\ {\isasymcirc}\ expr{\isacharunderscore}{\kern0pt}{\isadigit{6}}\ {\isasymcirc}\ {\isacharparenleft}{\kern0pt}{\isasymlambda}j{\isachardot}{\kern0pt}\ if\ j\ {\isacharequal}{\kern0pt}\ s\ then\ hml{\isacharunderscore}{\kern0pt}pos\ a\ {\isacharparenleft}{\kern0pt}hml{\isacharunderscore}{\kern0pt}pos\ c\ TT{\isacharparenright}{\kern0pt}\isanewline
\ \ \ \ \ \ \ \ \ \ \ \ \ \ \ \ \ \ \ \ \ \ \ \ \ \ \ \ \ \ \ \ \ \ else\ if\ j\ {\isacharequal}{\kern0pt}\ t\ then\ hml{\isacharunderscore}{\kern0pt}pos\ b\ TT\ else\ undefined{\isacharparenright}{\kern0pt}{\isacharparenright}{\kern0pt}\ {\isacharbackquote}{\kern0pt}\ {\isacharbraceleft}{\kern0pt}s{\isacharcomma}{\kern0pt}\ t{\isacharbraceright}{\kern0pt}{\isacharparenright}{\kern0pt}{\isacharparenright}{\kern0pt}\ {\isacharequal}{\kern0pt}\ {\isadigit{1}}{\isachardoublequoteclose}\isanewline
\ \ \ \ \ \ \ \ \isacommand{using}\isamarkupfalse%
\ eSuc{\isacharunderscore}{\kern0pt}def\ \isanewline
\ \ \ \ \ \ \ \ \isacommand{by}\isamarkupfalse%
\ {\isacharparenleft}{\kern0pt}simp\ add{\isacharcolon}{\kern0pt}\ one{\isacharunderscore}{\kern0pt}eSuc{\isacharparenright}{\kern0pt}\isanewline
\ \ \ \ \ \ \isacommand{thus}\isamarkupfalse%
\ {\isachardoublequoteopen}expr{\isacharunderscore}{\kern0pt}{\isadigit{6}}\ {\isasymphi}\ {\isacharequal}{\kern0pt}\ {\isadigit{1}}{\isachardoublequoteclose}\isanewline
\ \ \ \ \ \ \ \ \isacommand{unfolding}\isamarkupfalse%
\ {\isasymphi}\isanewline
\ \ \ \ \ \ \ \ \isacommand{using}\isamarkupfalse%
\ e{\isadigit{6}}{\isacharunderscore}{\kern0pt}{\isadigit{1}}\ \isanewline
\ \ \ \ \ \ \ \ \isacommand{by}\isamarkupfalse%
\ {\isacharparenleft}{\kern0pt}smt\ {\isacharparenleft}{\kern0pt}z{\isadigit{3}}{\isacharparenright}{\kern0pt}\ SUP{\isacharunderscore}{\kern0pt}insert\ assms{\isacharparenleft}{\kern0pt}{\isadigit{1}}{\isacharparenright}{\kern0pt}\ ccSUP{\isacharunderscore}{\kern0pt}empty\ comp{\isacharunderscore}{\kern0pt}def\ empty{\isacharunderscore}{\kern0pt}is{\isacharunderscore}{\kern0pt}image\ expr{\isacharunderscore}{\kern0pt}{\isadigit{6}}{\isachardot}{\kern0pt}simps{\isacharparenleft}{\kern0pt}{\isadigit{1}}{\isacharparenright}{\kern0pt}\ expr{\isacharunderscore}{\kern0pt}{\isadigit{6}}{\isachardot}{\kern0pt}simps{\isacharparenleft}{\kern0pt}{\isadigit{2}}{\isacharparenright}{\kern0pt}\ expr{\isacharunderscore}{\kern0pt}{\isadigit{6}}{\isacharunderscore}{\kern0pt}conj\ sup{\isacharunderscore}{\kern0pt}bot{\isachardot}{\kern0pt}left{\isacharunderscore}{\kern0pt}neutral\ sup{\isacharunderscore}{\kern0pt}bot{\isacharunderscore}{\kern0pt}right{\isacharparenright}{\kern0pt}\isanewline
\ \ \ \ \isacommand{qed}\isamarkupfalse%
\isanewline
\isanewline
\ \ \ \ \ \ \isacommand{have}\isamarkupfalse%
\ {\isachardoublequoteopen}{\isacharparenleft}{\kern0pt}expr{\isacharunderscore}{\kern0pt}{\isadigit{1}}\ {\isacharbackquote}{\kern0pt}\ {\isacharparenleft}{\kern0pt}pos{\isacharunderscore}{\kern0pt}r\ {\isacharparenleft}{\kern0pt}{\isacharparenleft}{\kern0pt}{\isasymlambda}j{\isachardot}{\kern0pt}\ if\ j\ {\isacharequal}{\kern0pt}\ s\ then\ hml{\isacharunderscore}{\kern0pt}pos\ a\ {\isacharparenleft}{\kern0pt}hml{\isacharunderscore}{\kern0pt}pos\ c\ TT{\isacharparenright}{\kern0pt}\isanewline
\ \ \ \ \ \ \ \ \ \ \ \ \ \ \ \ \ \ \ \ \ \ \ \ \ \ \ \ \ \ \ \ \ \ \ \ else\ if\ j\ {\isacharequal}{\kern0pt}\ t\ then\ hml{\isacharunderscore}{\kern0pt}pos\ b\ TT\ else\ undefined{\isacharparenright}{\kern0pt}\ {\isacharbackquote}{\kern0pt}\ {\isacharbraceleft}{\kern0pt}{\isacharbraceright}{\kern0pt}{\isacharparenright}{\kern0pt}{\isacharparenright}{\kern0pt}{\isacharparenright}{\kern0pt}\ {\isacharequal}{\kern0pt}\ {\isacharbraceleft}{\kern0pt}{\isacharbraceright}{\kern0pt}{\isachardoublequoteclose}\isanewline
\ \ {\isachardoublequoteopen}{\isacharparenleft}{\kern0pt}{\isacharparenleft}{\kern0pt}expr{\isacharunderscore}{\kern0pt}{\isadigit{4}}\ {\isasymcirc}\ {\isacharparenleft}{\kern0pt}{\isasymlambda}j{\isachardot}{\kern0pt}\ if\ j\ {\isacharequal}{\kern0pt}\ s\ then\ hml{\isacharunderscore}{\kern0pt}pos\ a\ {\isacharparenleft}{\kern0pt}hml{\isacharunderscore}{\kern0pt}pos\ c\ TT{\isacharparenright}{\kern0pt}\isanewline
\ \ \ \ \ \ \ \ \ \ \ \ \ \ \ \ \ \ \ \ \ \ \ \ \ \ \ \ \ \ \ \ \ \ \ \ else\ if\ j\ {\isacharequal}{\kern0pt}\ t\ then\ hml{\isacharunderscore}{\kern0pt}pos\ b\ TT\ else\ undefined{\isacharparenright}{\kern0pt}{\isacharparenright}{\kern0pt}\ {\isacharbackquote}{\kern0pt}\ {\isacharbraceleft}{\kern0pt}{\isacharbraceright}{\kern0pt}{\isacharparenright}{\kern0pt}\ {\isacharequal}{\kern0pt}\ {\isacharbraceleft}{\kern0pt}{\isacharbraceright}{\kern0pt}{\isachardoublequoteclose}\isanewline
\ \ \ \ \ \ \isacommand{by}\isamarkupfalse%
\ auto\isanewline
\ \ \ \ \isacommand{hence}\isamarkupfalse%
\ {\isachardoublequoteopen}Sup\ {\isacharparenleft}{\kern0pt}expr{\isacharunderscore}{\kern0pt}{\isadigit{1}}\ {\isacharbackquote}{\kern0pt}\ {\isacharparenleft}{\kern0pt}pos{\isacharunderscore}{\kern0pt}r\ {\isacharparenleft}{\kern0pt}{\isacharparenleft}{\kern0pt}{\isasymlambda}j{\isachardot}{\kern0pt}\ if\ j\ {\isacharequal}{\kern0pt}\ s\ then\ hml{\isacharunderscore}{\kern0pt}pos\ a\ {\isacharparenleft}{\kern0pt}hml{\isacharunderscore}{\kern0pt}pos\ c\ TT{\isacharparenright}{\kern0pt}\isanewline
\ \ \ \ \ \ \ \ \ \ \ \ \ \ \ \ \ \ \ \ \ \ \ \ \ \ \ \ \ \ \ \ \ \ \ \ else\ if\ j\ {\isacharequal}{\kern0pt}\ t\ then\ hml{\isacharunderscore}{\kern0pt}pos\ b\ TT\ else\ undefined{\isacharparenright}{\kern0pt}\ {\isacharbackquote}{\kern0pt}\ {\isacharbraceleft}{\kern0pt}{\isacharbraceright}{\kern0pt}{\isacharparenright}{\kern0pt}{\isacharparenright}{\kern0pt}{\isacharparenright}{\kern0pt}\ {\isacharequal}{\kern0pt}\ {\isadigit{0}}{\isachardoublequoteclose}\isanewline
\ \ \ \ \ \ \ \ \ \ {\isachardoublequoteopen}Sup{\isacharparenleft}{\kern0pt}{\isacharparenleft}{\kern0pt}expr{\isacharunderscore}{\kern0pt}{\isadigit{4}}\ {\isasymcirc}\ {\isacharparenleft}{\kern0pt}{\isasymlambda}j{\isachardot}{\kern0pt}\ if\ j\ {\isacharequal}{\kern0pt}\ s\ then\ hml{\isacharunderscore}{\kern0pt}pos\ a\ {\isacharparenleft}{\kern0pt}hml{\isacharunderscore}{\kern0pt}pos\ c\ TT{\isacharparenright}{\kern0pt}\isanewline
\ \ \ \ \ \ \ \ \ \ \ \ \ \ \ \ \ \ \ \ \ \ \ \ \ \ \ \ \ \ \ \ \ \ \ \ else\ if\ j\ {\isacharequal}{\kern0pt}\ t\ then\ hml{\isacharunderscore}{\kern0pt}pos\ b\ TT\ else\ undefined{\isacharparenright}{\kern0pt}{\isacharparenright}{\kern0pt}\ {\isacharbackquote}{\kern0pt}\ {\isacharbraceleft}{\kern0pt}{\isacharbraceright}{\kern0pt}{\isacharparenright}{\kern0pt}\ {\isacharequal}{\kern0pt}\ {\isadigit{0}}{\isachardoublequoteclose}\isanewline
\ \ \ \ \ \ \ \ \ \ {\isachardoublequoteopen}Sup{\isacharparenleft}{\kern0pt}{\isacharparenleft}{\kern0pt}expr{\isacharunderscore}{\kern0pt}{\isadigit{4}}\ {\isasymcirc}\ {\isacharparenleft}{\kern0pt}{\isasymlambda}j{\isachardot}{\kern0pt}\ if\ j\ {\isacharequal}{\kern0pt}\ s\ then\ hml{\isacharunderscore}{\kern0pt}pos\ a\ {\isacharparenleft}{\kern0pt}hml{\isacharunderscore}{\kern0pt}pos\ c\ TT{\isacharparenright}{\kern0pt}\isanewline
\ \ \ \ \ \ \ \ \ \ \ \ \ \ \ \ \ \ \ \ \ \ \ \ \ \ \ \ \ \ \ \ \ \ \ \ else\ if\ j\ {\isacharequal}{\kern0pt}\ t\ then\ hml{\isacharunderscore}{\kern0pt}pos\ b\ TT\ else\ undefined{\isacharparenright}{\kern0pt}{\isacharparenright}{\kern0pt}\ {\isacharbackquote}{\kern0pt}\ {\isacharbraceleft}{\kern0pt}s{\isacharcomma}{\kern0pt}\ t{\isacharbraceright}{\kern0pt}{\isacharparenright}{\kern0pt}\ {\isacharequal}{\kern0pt}\ {\isadigit{0}}{\isachardoublequoteclose}\isanewline
\ \ \ \ \ \ \isacommand{using}\isamarkupfalse%
\ bot{\isacharunderscore}{\kern0pt}enat{\isacharunderscore}{\kern0pt}def\ \isacommand{by}\isamarkupfalse%
\ fastforce{\isacharplus}{\kern0pt}\isanewline
\ \ \ \ \isacommand{hence}\isamarkupfalse%
\ {\isachardoublequoteopen}Sup\ {\isacharparenleft}{\kern0pt}{\isacharparenleft}{\kern0pt}expr{\isacharunderscore}{\kern0pt}{\isadigit{1}}\ {\isacharbackquote}{\kern0pt}\ {\isacharparenleft}{\kern0pt}pos{\isacharunderscore}{\kern0pt}r\ {\isacharparenleft}{\kern0pt}{\isacharparenleft}{\kern0pt}{\isasymlambda}j{\isachardot}{\kern0pt}\ if\ j\ {\isacharequal}{\kern0pt}\ s\ then\ hml{\isacharunderscore}{\kern0pt}pos\ a\ {\isacharparenleft}{\kern0pt}hml{\isacharunderscore}{\kern0pt}pos\ c\ TT{\isacharparenright}{\kern0pt}\isanewline
\ \ \ \ \ \ \ \ \ \ \ \ \ \ \ \ \ \ \ \ \ \ \ \ \ \ \ \ \ \ \ \ \ \ \ \ else\ if\ j\ {\isacharequal}{\kern0pt}\ t\ then\ hml{\isacharunderscore}{\kern0pt}pos\ b\ TT\ else\ undefined{\isacharparenright}{\kern0pt}\ {\isacharbackquote}{\kern0pt}\ {\isacharbraceleft}{\kern0pt}{\isacharbraceright}{\kern0pt}{\isacharparenright}{\kern0pt}{\isacharparenright}{\kern0pt}{\isacharparenright}{\kern0pt}\ \ {\isasymunion}\ {\isacharparenleft}{\kern0pt}expr{\isacharunderscore}{\kern0pt}{\isadigit{4}}\ {\isasymcirc}\ {\isacharparenleft}{\kern0pt}{\isasymlambda}j{\isachardot}{\kern0pt}\ if\ j\ {\isacharequal}{\kern0pt}\ s\ then\ hml{\isacharunderscore}{\kern0pt}pos\ a\ {\isacharparenleft}{\kern0pt}hml{\isacharunderscore}{\kern0pt}pos\ c\ TT{\isacharparenright}{\kern0pt}\isanewline
\ \ \ \ \ \ \ \ \ \ \ \ \ \ \ \ \ \ \ \ \ \ \ \ \ \ \ \ \ \ \ \ \ \ \ \ else\ if\ j\ {\isacharequal}{\kern0pt}\ t\ then\ hml{\isacharunderscore}{\kern0pt}pos\ b\ TT\ else\ undefined{\isacharparenright}{\kern0pt}{\isacharparenright}{\kern0pt}\ {\isacharbackquote}{\kern0pt}\ {\isacharbraceleft}{\kern0pt}{\isacharbraceright}{\kern0pt}\ {\isasymunion}\ {\isacharparenleft}{\kern0pt}expr{\isacharunderscore}{\kern0pt}{\isadigit{4}}\ {\isasymcirc}\ {\isacharparenleft}{\kern0pt}{\isasymlambda}j{\isachardot}{\kern0pt}\ if\ j\ {\isacharequal}{\kern0pt}\ s\ then\ hml{\isacharunderscore}{\kern0pt}pos\ a\ {\isacharparenleft}{\kern0pt}hml{\isacharunderscore}{\kern0pt}pos\ c\ TT{\isacharparenright}{\kern0pt}\isanewline
\ \ \ \ \ \ \ \ \ \ \ \ \ \ \ \ \ \ \ \ \ \ \ \ \ \ \ \ \ \ \ \ \ \ \ \ else\ if\ j\ {\isacharequal}{\kern0pt}\ t\ then\ hml{\isacharunderscore}{\kern0pt}pos\ b\ TT\ else\ undefined{\isacharparenright}{\kern0pt}{\isacharparenright}{\kern0pt}\ {\isacharbackquote}{\kern0pt}\ {\isacharbraceleft}{\kern0pt}s{\isacharcomma}{\kern0pt}\ t{\isacharbraceright}{\kern0pt}{\isacharparenright}{\kern0pt}\ {\isacharequal}{\kern0pt}\ {\isadigit{0}}{\isachardoublequoteclose}\ \isacommand{by}\isamarkupfalse%
\ simp\isanewline
\ \ \ \ \isacommand{hence}\isamarkupfalse%
\ {\isachardoublequoteopen}expr{\isacharunderscore}{\kern0pt}{\isadigit{4}}\ {\isacharparenleft}{\kern0pt}hml{\isacharunderscore}{\kern0pt}pos\ a\isanewline
\ \ \ \ \ \ \ \ \ \ \ \ \ \ \ \ \ \ \ \ \ \ \ \ \ \ \ \ {\isacharparenleft}{\kern0pt}hml{\isacharunderscore}{\kern0pt}conj\ {\isacharbraceleft}{\kern0pt}{\isacharbraceright}{\kern0pt}\ {\isacharbraceleft}{\kern0pt}s{\isacharcomma}{\kern0pt}\ t{\isacharbraceright}{\kern0pt}\isanewline
\ \ \ \ \ \ \ \ \ \ \ \ \ \ \ \ \ \ \ \ \ \ \ \ \ \ \ \ \ \ {\isacharparenleft}{\kern0pt}{\isasymlambda}j{\isachardot}{\kern0pt}\ if\ j\ {\isacharequal}{\kern0pt}\ s\ then\ hml{\isacharunderscore}{\kern0pt}pos\ a\ {\isacharparenleft}{\kern0pt}hml{\isacharunderscore}{\kern0pt}pos\ c\ TT{\isacharparenright}{\kern0pt}\isanewline
\ \ \ \ \ \ \ \ \ \ \ \ \ \ \ \ \ \ \ \ \ \ \ \ \ \ \ \ \ \ \ \ \ \ \ \ else\ if\ j\ {\isacharequal}{\kern0pt}\ t\ then\ hml{\isacharunderscore}{\kern0pt}pos\ b\ TT\ else\ undefined{\isacharparenright}{\kern0pt}{\isacharparenright}{\kern0pt}{\isacharparenright}{\kern0pt}\ {\isacharequal}{\kern0pt}\ {\isadigit{0}}{\isachardoublequoteclose}\isanewline
\ \ \ \ \ \ \isacommand{by}\isamarkupfalse%
\ simp\isanewline
\ \ \ \ \isacommand{have}\isamarkupfalse%
\ {\isachardoublequoteopen}{\isacharparenleft}{\kern0pt}pos{\isacharunderscore}{\kern0pt}r\ {\isacharparenleft}{\kern0pt}{\isacharparenleft}{\kern0pt}{\isasymlambda}i{\isachardot}{\kern0pt}\ if\ i\ {\isacharequal}{\kern0pt}\ s\ then\ hml{\isacharunderscore}{\kern0pt}pos\ b\ TT\isanewline
\ \ \ \ \ \ \ \ \ \ \ \ \ \ \ \ \ else\ if\ i\ {\isacharequal}{\kern0pt}\ t\isanewline
\ \ \ \ \ \ \ \ \ \ \ \ \ \ \ \ \ \ \ \ \ \ then\ hml{\isacharunderscore}{\kern0pt}pos\ a\isanewline
\ \ \ \ \ \ \ \ \ \ \ \ \ \ \ \ \ \ \ \ \ \ \ \ \ \ \ \ {\isacharparenleft}{\kern0pt}hml{\isacharunderscore}{\kern0pt}conj\ {\isacharbraceleft}{\kern0pt}{\isacharbraceright}{\kern0pt}\ {\isacharbraceleft}{\kern0pt}s{\isacharcomma}{\kern0pt}\ t{\isacharbraceright}{\kern0pt}\isanewline
\ \ \ \ \ \ \ \ \ \ \ \ \ \ \ \ \ \ \ \ \ \ \ \ \ \ \ \ \ \ {\isacharparenleft}{\kern0pt}{\isasymlambda}j{\isachardot}{\kern0pt}\ if\ j\ {\isacharequal}{\kern0pt}\ s\ then\ hml{\isacharunderscore}{\kern0pt}pos\ a\ {\isacharparenleft}{\kern0pt}hml{\isacharunderscore}{\kern0pt}pos\ c\ TT{\isacharparenright}{\kern0pt}\isanewline
\ \ \ \ \ \ \ \ \ \ \ \ \ \ \ \ \ \ \ \ \ \ \ \ \ \ \ \ \ \ \ \ \ \ \ \ else\ if\ j\ {\isacharequal}{\kern0pt}\ t\ then\ hml{\isacharunderscore}{\kern0pt}pos\ b\ TT\ else\ undefined{\isacharparenright}{\kern0pt}{\isacharparenright}{\kern0pt}\isanewline
\ \ \ \ \ \ \ \ \ \ \ \ \ \ \ \ \ \ \ \ \ \ else\ undefined{\isacharparenright}{\kern0pt}\ {\isacharbackquote}{\kern0pt}\ {\isacharbraceleft}{\kern0pt}s{\isacharcomma}{\kern0pt}\ t{\isacharbraceright}{\kern0pt}{\isacharparenright}{\kern0pt}{\isacharparenright}{\kern0pt}\ {\isacharequal}{\kern0pt}\ \isanewline
\ \ \ \ \ \ \ \ \ \ \ \ \ \ \ \ \ \ \ \ \ \ \ \ \ \ {\isacharbraceleft}{\kern0pt}hml{\isacharunderscore}{\kern0pt}pos\ b\ TT{\isacharbraceright}{\kern0pt}{\isachardoublequoteclose}\isanewline
\ \ \ \ \isacommand{proof}\isamarkupfalse%
{\isacharminus}{\kern0pt}\isanewline
\ \ \ \ \ \ \isacommand{define}\isamarkupfalse%
\ xs\ \isakeyword{where}\ {\isachardoublequoteopen}xs\ {\isasymequiv}\ {\isacharparenleft}{\kern0pt}{\isacharparenleft}{\kern0pt}{\isasymlambda}i{\isachardot}{\kern0pt}\ if\ i\ {\isacharequal}{\kern0pt}\ s\ then\ hml{\isacharunderscore}{\kern0pt}pos\ b\ TT\isanewline
\ \ \ \ \ \ \ \ \ \ \ \ \ \ \ \ \ else\ if\ i\ {\isacharequal}{\kern0pt}\ t\isanewline
\ \ \ \ \ \ \ \ \ \ \ \ \ \ \ \ \ \ \ \ \ \ then\ hml{\isacharunderscore}{\kern0pt}pos\ a\isanewline
\ \ \ \ \ \ \ \ \ \ \ \ \ \ \ \ \ \ \ \ \ \ \ \ \ \ \ \ {\isacharparenleft}{\kern0pt}hml{\isacharunderscore}{\kern0pt}conj\ {\isacharbraceleft}{\kern0pt}{\isacharbraceright}{\kern0pt}\ {\isacharbraceleft}{\kern0pt}s{\isacharcomma}{\kern0pt}\ t{\isacharbraceright}{\kern0pt}\isanewline
\ \ \ \ \ \ \ \ \ \ \ \ \ \ \ \ \ \ \ \ \ \ \ \ \ \ \ \ \ \ {\isacharparenleft}{\kern0pt}{\isasymlambda}j{\isachardot}{\kern0pt}\ if\ j\ {\isacharequal}{\kern0pt}\ s\ then\ hml{\isacharunderscore}{\kern0pt}pos\ a\ {\isacharparenleft}{\kern0pt}hml{\isacharunderscore}{\kern0pt}pos\ c\ TT{\isacharparenright}{\kern0pt}\isanewline
\ \ \ \ \ \ \ \ \ \ \ \ \ \ \ \ \ \ \ \ \ \ \ \ \ \ \ \ \ \ \ \ \ \ \ \ else\ if\ j\ {\isacharequal}{\kern0pt}\ t\ then\ hml{\isacharunderscore}{\kern0pt}pos\ b\ TT\ else\ undefined{\isacharparenright}{\kern0pt}{\isacharparenright}{\kern0pt}\isanewline
\ \ \ \ \ \ \ \ \ \ \ \ \ \ \ \ \ \ \ \ \ \ else\ undefined{\isacharparenright}{\kern0pt}\ {\isacharbackquote}{\kern0pt}\ {\isacharbraceleft}{\kern0pt}s{\isacharcomma}{\kern0pt}\ t{\isacharbraceright}{\kern0pt}{\isacharparenright}{\kern0pt}{\isachardoublequoteclose}\isanewline
\ \ \ \ \ \ \isacommand{have}\isamarkupfalse%
\ {\isachardoublequoteopen}{\isacharparenleft}{\kern0pt}Sup\ {\isacharparenleft}{\kern0pt}expr{\isacharunderscore}{\kern0pt}{\isadigit{1}}\ {\isacharbackquote}{\kern0pt}\ xs{\isacharparenright}{\kern0pt}{\isacharparenright}{\kern0pt}\ {\isacharequal}{\kern0pt}\ {\isadigit{3}}{\isachardoublequoteclose}\ \isanewline
\ \ \ \ \ \ \ \ \isacommand{unfolding}\isamarkupfalse%
\ xs{\isacharunderscore}{\kern0pt}def\isanewline
\ \ \ \ \ \ \ \ \isacommand{by}\isamarkupfalse%
\ {\isacharparenleft}{\kern0pt}metis\ Sup{\isachardot}{\kern0pt}SUP{\isacharunderscore}{\kern0pt}image\ {\isacartoucheopen}Sup\ {\isacharparenleft}{\kern0pt}{\isacharparenleft}{\kern0pt}expr{\isacharunderscore}{\kern0pt}{\isadigit{1}}\ {\isasymcirc}\ {\isacharparenleft}{\kern0pt}{\isasymlambda}i{\isachardot}{\kern0pt}\ if\ i\ {\isacharequal}{\kern0pt}\ s\ then\ hml{\isacharunderscore}{\kern0pt}pos\ b\ TT\ else\ if\ i\ {\isacharequal}{\kern0pt}\ t\ then\ hml{\isacharunderscore}{\kern0pt}pos\ a\ {\isacharparenleft}{\kern0pt}hml{\isacharunderscore}{\kern0pt}conj\ {\isacharbraceleft}{\kern0pt}{\isacharbraceright}{\kern0pt}\ {\isacharbraceleft}{\kern0pt}s{\isacharcomma}{\kern0pt}\ t{\isacharbraceright}{\kern0pt}\ {\isacharparenleft}{\kern0pt}{\isasymlambda}j{\isachardot}{\kern0pt}\ if\ j\ {\isacharequal}{\kern0pt}\ s\ then\ hml{\isacharunderscore}{\kern0pt}pos\ a\ {\isacharparenleft}{\kern0pt}hml{\isacharunderscore}{\kern0pt}pos\ c\ TT{\isacharparenright}{\kern0pt}\ else\ if\ j\ {\isacharequal}{\kern0pt}\ t\ then\ hml{\isacharunderscore}{\kern0pt}pos\ b\ TT\ else\ undefined{\isacharparenright}{\kern0pt}{\isacharparenright}{\kern0pt}\ else\ undefined{\isacharparenright}{\kern0pt}{\isacharparenright}{\kern0pt}\ {\isacharbackquote}{\kern0pt}\ {\isacharbraceleft}{\kern0pt}s{\isacharcomma}{\kern0pt}\ t{\isacharbraceright}{\kern0pt}{\isacharparenright}{\kern0pt}\ {\isacharequal}{\kern0pt}\ {\isadigit{3}}{\isacartoucheclose}{\isacharparenright}{\kern0pt}\isanewline
\ \ \ \ \ \ \isacommand{hence}\isamarkupfalse%
\ {\isachardoublequoteopen}{\isacharbraceleft}{\kern0pt}{\isasympsi}\ {\isacharbar}{\kern0pt}\ {\isasympsi}{\isachardot}{\kern0pt}\ {\isasympsi}\ {\isasymin}\ xs\ {\isasymand}\ {\isacharparenleft}{\kern0pt}expr{\isacharunderscore}{\kern0pt}{\isadigit{1}}\ {\isasympsi}\ {\isacharequal}{\kern0pt}\ {\isacharparenleft}{\kern0pt}Sup\ {\isacharparenleft}{\kern0pt}expr{\isacharunderscore}{\kern0pt}{\isadigit{1}}\ {\isacharbackquote}{\kern0pt}\ xs{\isacharparenright}{\kern0pt}{\isacharparenright}{\kern0pt}{\isacharparenright}{\kern0pt}{\isacharbraceright}{\kern0pt}\ {\isacharequal}{\kern0pt}\isanewline
\ \ \ \ \ \ \ \ \ \ \ \ \ \ {\isacharbraceleft}{\kern0pt}{\isacharparenleft}{\kern0pt}hml{\isacharunderscore}{\kern0pt}pos\ a\isanewline
\ \ \ \ \ \ \ \ \ \ \ \ \ \ \ \ \ \ \ \ \ \ \ \ \ \ \ \ {\isacharparenleft}{\kern0pt}hml{\isacharunderscore}{\kern0pt}conj\ {\isacharbraceleft}{\kern0pt}{\isacharbraceright}{\kern0pt}\ {\isacharbraceleft}{\kern0pt}s{\isacharcomma}{\kern0pt}\ t{\isacharbraceright}{\kern0pt}\isanewline
\ \ \ \ \ \ \ \ \ \ \ \ \ \ \ \ \ \ \ \ \ \ \ \ \ \ \ \ \ \ {\isacharparenleft}{\kern0pt}{\isasymlambda}j{\isachardot}{\kern0pt}\ if\ j\ {\isacharequal}{\kern0pt}\ s\ then\ hml{\isacharunderscore}{\kern0pt}pos\ a\ {\isacharparenleft}{\kern0pt}hml{\isacharunderscore}{\kern0pt}pos\ c\ TT{\isacharparenright}{\kern0pt}\isanewline
\ \ \ \ \ \ \ \ \ \ \ \ \ \ \ \ \ \ \ \ \ \ \ \ \ \ \ \ \ \ \ \ \ \ \ \ else\ if\ j\ {\isacharequal}{\kern0pt}\ t\ then\ hml{\isacharunderscore}{\kern0pt}pos\ b\ TT\ else\ undefined{\isacharparenright}{\kern0pt}{\isacharparenright}{\kern0pt}{\isacharparenright}{\kern0pt}{\isacharbraceright}{\kern0pt}{\isachardoublequoteclose}\isanewline
\ \ \ \ \ \ \ \ \isacommand{using}\isamarkupfalse%
\ {\isadigit{1}}\ {\isadigit{2}}\ assms{\isacharparenleft}{\kern0pt}{\isadigit{1}}{\isacharparenright}{\kern0pt}\ xs{\isacharunderscore}{\kern0pt}def\ \isacommand{by}\isamarkupfalse%
\ force\isanewline
\ \ \ \ \ \ \isacommand{hence}\isamarkupfalse%
\ {\isachardoublequoteopen}{\isacharparenleft}{\kern0pt}SOME\ {\isasympsi}{\isachardot}{\kern0pt}\ {\isasympsi}\ {\isasymin}\ xs\ {\isasymand}\ expr{\isacharunderscore}{\kern0pt}{\isadigit{1}}\ {\isasympsi}\ {\isacharequal}{\kern0pt}\ {\isacharparenleft}{\kern0pt}Sup\ {\isacharparenleft}{\kern0pt}expr{\isacharunderscore}{\kern0pt}{\isadigit{1}}\ {\isacharbackquote}{\kern0pt}\ xs{\isacharparenright}{\kern0pt}{\isacharparenright}{\kern0pt}{\isacharparenright}{\kern0pt}\ {\isacharequal}{\kern0pt}\isanewline
\ \ \ \ \ \ \ \ \ \ \ \ \ \ {\isacharparenleft}{\kern0pt}hml{\isacharunderscore}{\kern0pt}pos\ a\isanewline
\ \ \ \ \ \ \ \ \ \ \ \ \ \ \ \ \ \ \ \ \ \ \ \ \ \ \ \ {\isacharparenleft}{\kern0pt}hml{\isacharunderscore}{\kern0pt}conj\ {\isacharbraceleft}{\kern0pt}{\isacharbraceright}{\kern0pt}\ {\isacharbraceleft}{\kern0pt}s{\isacharcomma}{\kern0pt}\ t{\isacharbraceright}{\kern0pt}\isanewline
\ \ \ \ \ \ \ \ \ \ \ \ \ \ \ \ \ \ \ \ \ \ \ \ \ \ \ \ \ \ {\isacharparenleft}{\kern0pt}{\isasymlambda}j{\isachardot}{\kern0pt}\ if\ j\ {\isacharequal}{\kern0pt}\ s\ then\ hml{\isacharunderscore}{\kern0pt}pos\ a\ {\isacharparenleft}{\kern0pt}hml{\isacharunderscore}{\kern0pt}pos\ c\ TT{\isacharparenright}{\kern0pt}\isanewline
\ \ \ \ \ \ \ \ \ \ \ \ \ \ \ \ \ \ \ \ \ \ \ \ \ \ \ \ \ \ \ \ \ \ \ \ else\ if\ j\ {\isacharequal}{\kern0pt}\ t\ then\ hml{\isacharunderscore}{\kern0pt}pos\ b\ TT\ else\ undefined{\isacharparenright}{\kern0pt}{\isacharparenright}{\kern0pt}{\isacharparenright}{\kern0pt}{\isachardoublequoteclose}\isanewline
\ \ \ \ \ \ \ \ \isacommand{unfolding}\isamarkupfalse%
\ xs{\isacharunderscore}{\kern0pt}def\isanewline
\ \ \ \ \ \ \ \ \isacommand{using}\isamarkupfalse%
\ pos{\isacharunderscore}{\kern0pt}r{\isachardot}{\kern0pt}simps\ {\isadigit{2}}\ {\isadigit{1}}\ assms{\isacharparenleft}{\kern0pt}{\isadigit{1}}{\isacharparenright}{\kern0pt}\isanewline
\ \ \ \ \ \ \ \ \isacommand{by}\isamarkupfalse%
\ {\isacharparenleft}{\kern0pt}smt\ {\isacharparenleft}{\kern0pt}verit{\isacharcomma}{\kern0pt}\ ccfv{\isacharunderscore}{\kern0pt}threshold{\isacharparenright}{\kern0pt}\ mem{\isacharunderscore}{\kern0pt}Collect{\isacharunderscore}{\kern0pt}eq\ singleton{\isacharunderscore}{\kern0pt}iff\ someI{\isacharunderscore}{\kern0pt}ex{\isacharparenright}{\kern0pt}\isanewline
\ \ \ \ \ \ \isacommand{then}\isamarkupfalse%
\ \isacommand{show}\isamarkupfalse%
\ {\isacharquery}{\kern0pt}thesis\ \isanewline
\ \ \ \ \ \ \ \ \isacommand{unfolding}\isamarkupfalse%
\ xs{\isacharunderscore}{\kern0pt}def\isanewline
\ \ \ \ \ \ \ \ \isacommand{using}\isamarkupfalse%
\ pos{\isacharunderscore}{\kern0pt}r{\isachardot}{\kern0pt}simps\ assms{\isacharparenleft}{\kern0pt}{\isadigit{1}}{\isacharparenright}{\kern0pt}\ \isacommand{by}\isamarkupfalse%
\ auto\isanewline
\ \ \ \ \isacommand{qed}\isamarkupfalse%
\isanewline
\ \ \ \ \isacommand{hence}\isamarkupfalse%
\ {\isachardoublequoteopen}Sup\ {\isacharparenleft}{\kern0pt}{\isacharparenleft}{\kern0pt}expr{\isacharunderscore}{\kern0pt}{\isadigit{1}}\ {\isacharbackquote}{\kern0pt}\ {\isacharparenleft}{\kern0pt}pos{\isacharunderscore}{\kern0pt}r\ {\isacharparenleft}{\kern0pt}{\isacharparenleft}{\kern0pt}{\isasymlambda}i{\isachardot}{\kern0pt}\ if\ i\ {\isacharequal}{\kern0pt}\ s\ then\ hml{\isacharunderscore}{\kern0pt}pos\ b\ TT\isanewline
\ \ \ \ \ \ \ \ \ \ \ \ \ \ \ \ \ else\ if\ i\ {\isacharequal}{\kern0pt}\ t\isanewline
\ \ \ \ \ \ \ \ \ \ \ \ \ \ \ \ \ \ \ \ \ \ then\ hml{\isacharunderscore}{\kern0pt}pos\ a\isanewline
\ \ \ \ \ \ \ \ \ \ \ \ \ \ \ \ \ \ \ \ \ \ \ \ \ \ \ \ {\isacharparenleft}{\kern0pt}hml{\isacharunderscore}{\kern0pt}conj\ {\isacharbraceleft}{\kern0pt}{\isacharbraceright}{\kern0pt}\ {\isacharbraceleft}{\kern0pt}s{\isacharcomma}{\kern0pt}\ t{\isacharbraceright}{\kern0pt}\isanewline
\ \ \ \ \ \ \ \ \ \ \ \ \ \ \ \ \ \ \ \ \ \ \ \ \ \ \ \ \ \ {\isacharparenleft}{\kern0pt}{\isasymlambda}j{\isachardot}{\kern0pt}\ if\ j\ {\isacharequal}{\kern0pt}\ s\ then\ hml{\isacharunderscore}{\kern0pt}pos\ a\ {\isacharparenleft}{\kern0pt}hml{\isacharunderscore}{\kern0pt}pos\ c\ TT{\isacharparenright}{\kern0pt}\isanewline
\ \ \ \ \ \ \ \ \ \ \ \ \ \ \ \ \ \ \ \ \ \ \ \ \ \ \ \ \ \ \ \ \ \ \ \ else\ if\ j\ {\isacharequal}{\kern0pt}\ t\ then\ hml{\isacharunderscore}{\kern0pt}pos\ b\ TT\ else\ undefined{\isacharparenright}{\kern0pt}{\isacharparenright}{\kern0pt}\isanewline
\ \ \ \ \ \ \ \ \ \ \ \ \ \ \ \ \ \ \ \ \ \ else\ undefined{\isacharparenright}{\kern0pt}\ {\isacharbackquote}{\kern0pt}\ {\isacharbraceleft}{\kern0pt}s{\isacharcomma}{\kern0pt}\ t{\isacharbraceright}{\kern0pt}{\isacharparenright}{\kern0pt}{\isacharparenright}{\kern0pt}{\isacharparenright}{\kern0pt}{\isacharparenright}{\kern0pt}\ {\isacharequal}{\kern0pt}\ {\isadigit{1}}{\isachardoublequoteclose}\ \isanewline
\ \ \ \ \ \ \isacommand{using}\isamarkupfalse%
\ {\isachardoublequoteopen}{\isadigit{1}}{\isachardoublequoteclose}\ \isacommand{by}\isamarkupfalse%
\ auto\isanewline
\ \ \ \ \isacommand{then}\isamarkupfalse%
\ \isacommand{show}\isamarkupfalse%
\ {\isachardoublequoteopen}expr{\isacharunderscore}{\kern0pt}{\isadigit{4}}\ {\isasymphi}\ {\isacharequal}{\kern0pt}\ {\isadigit{1}}{\isachardoublequoteclose}\ \isanewline
\ \ \ \ \ \ \isacommand{unfolding}\isamarkupfalse%
\ {\isasymphi}\isanewline
\ \ \ \ \ \ \isacommand{by}\isamarkupfalse%
\ {\isacharparenleft}{\kern0pt}smt\ {\isacharparenleft}{\kern0pt}verit{\isacharcomma}{\kern0pt}\ del{\isacharunderscore}{\kern0pt}insts{\isacharparenright}{\kern0pt}\ SUP{\isacharunderscore}{\kern0pt}insert\ Sup{\isacharunderscore}{\kern0pt}union{\isacharunderscore}{\kern0pt}distrib\ {\isacartoucheopen}expr{\isacharunderscore}{\kern0pt}{\isadigit{4}}\ {\isacharparenleft}{\kern0pt}hml{\isacharunderscore}{\kern0pt}pos\ a\ {\isacharparenleft}{\kern0pt}hml{\isacharunderscore}{\kern0pt}conj\ {\isacharbraceleft}{\kern0pt}{\isacharbraceright}{\kern0pt}\ {\isacharbraceleft}{\kern0pt}s{\isacharcomma}{\kern0pt}\ t{\isacharbraceright}{\kern0pt}\ {\isacharparenleft}{\kern0pt}{\isasymlambda}j{\isachardot}{\kern0pt}\ if\ j\ {\isacharequal}{\kern0pt}\ s\ then\ hml{\isacharunderscore}{\kern0pt}pos\ a\ {\isacharparenleft}{\kern0pt}hml{\isacharunderscore}{\kern0pt}pos\ c\ TT{\isacharparenright}{\kern0pt}\ else\ if\ j\ {\isacharequal}{\kern0pt}\ t\ then\ hml{\isacharunderscore}{\kern0pt}pos\ b\ TT\ else\ undefined{\isacharparenright}{\kern0pt}{\isacharparenright}{\kern0pt}{\isacharparenright}{\kern0pt}\ {\isacharequal}{\kern0pt}\ {\isadigit{0}}{\isacartoucheclose}\ bot{\isacharunderscore}{\kern0pt}enat{\isacharunderscore}{\kern0pt}def\ ccSUP{\isacharunderscore}{\kern0pt}empty\ comp{\isacharunderscore}{\kern0pt}apply\ expr{\isacharunderscore}{\kern0pt}{\isadigit{4}}{\isacharunderscore}{\kern0pt}conj\ expr{\isacharunderscore}{\kern0pt}{\isadigit{4}}{\isacharunderscore}{\kern0pt}pos\ expr{\isacharunderscore}{\kern0pt}{\isadigit{4}}{\isacharunderscore}{\kern0pt}tt\ sup{\isacharunderscore}{\kern0pt}bot{\isachardot}{\kern0pt}right{\isacharunderscore}{\kern0pt}neutral{\isacharparenright}{\kern0pt}\isanewline
\ \ \isacommand{qed}\isamarkupfalse%
%
\endisatagproof
{\isafoldproof}%
%
\isadelimproof
%
\endisadelimproof
%
\begin{isamarkuptext}%
\textit{Example 2}: This example illustrates how the prices of a formula are calculated. In Figure \ref{fig:2_3}, you can see the pricing process for the formula $\langle a \rangle \bigwedge \{\langle b \rangle, \langle a \rangle \bigwedge \{\lnot \langle a \rangle \langle c \rangle, \lnot \langle b \rangle\}\}$. Each line to the right of the syntax tree represents the price of a specific dimension. The circles of each line represent an increase in the price of that dimension. The colors of these lines correspond to those defined in (ref definition 2.3.1) and indicate the dimension they represent. Note the finishing empty conjunction, which increases the conjunction depth by one.
We can use the function to calculate the prices of the formulas in Example 1. The price of $\varphi_1 :=\langle a \rangle\bigwedge\{\lnot\langle c \rangle\}$ is $expr(\varphi_1) = (2, 2, 0, 0, 1, 1)$. For $\varphi_2 := \bigwedge\{\lnot\langle a \rangle\bigwedge\{\lnot\langle c \rangle\}\}$, $expr(\varphi_2) = (2, 3, 0, 0, 2, 2)$.%
\end{isamarkuptext}\isamarkuptrue%
%
\begin{isamarkuptext}%
\textbf{Proposition} The expressiveness function is monotonic. Specifically, for any formula $\langle\alpha\rangle\varphi$, is the expressiveness of the subformula $\varphi$ less than or equal to the expressiveness of $\langle\alpha\rangle\varphi$.
Similarly, for any conjunctive formula $\bigwedge_{i\in I}\psi_i$, the expressiveness of every conjunct $\psi_i$ is less than or equal to the expressiveness of $\bigwedge_{i\in I}\psi_i$.%
\end{isamarkuptext}\isamarkuptrue%
\isacommand{lemma}\isamarkupfalse%
\ mon{\isacharunderscore}{\kern0pt}pos{\isacharcolon}{\kern0pt}\isanewline
\ \ \isakeyword{fixes}\ n{\isadigit{1}}\ \isakeyword{and}\ n{\isadigit{2}}\ \isakeyword{and}\ n{\isadigit{3}}\ \isakeyword{and}\ n{\isadigit{4}}{\isacharcolon}{\kern0pt}{\isacharcolon}{\kern0pt}enat\ \isakeyword{and}\ n{\isadigit{5}}\ \isakeyword{and}\ n{\isadigit{6}}\ \isakeyword{and}\ {\isasymalpha}\isanewline
\ \ \isakeyword{assumes}\ A{\isadigit{1}}{\isacharcolon}{\kern0pt}\ {\isachardoublequoteopen}less{\isacharunderscore}{\kern0pt}eq{\isacharunderscore}{\kern0pt}t\ {\isacharparenleft}{\kern0pt}expr\ {\isacharparenleft}{\kern0pt}hml{\isacharunderscore}{\kern0pt}pos\ {\isasymalpha}\ {\isasymphi}{\isacharparenright}{\kern0pt}{\isacharparenright}{\kern0pt}\ {\isacharparenleft}{\kern0pt}n{\isadigit{1}}{\isacharcomma}{\kern0pt}\ n{\isadigit{2}}{\isacharcomma}{\kern0pt}\ n{\isadigit{3}}{\isacharcomma}{\kern0pt}\ n{\isadigit{4}}{\isacharcomma}{\kern0pt}\ n{\isadigit{5}}{\isacharcomma}{\kern0pt}\ n{\isadigit{6}}{\isacharparenright}{\kern0pt}{\isachardoublequoteclose}\isanewline
\ \ \isakeyword{shows}\ {\isachardoublequoteopen}less{\isacharunderscore}{\kern0pt}eq{\isacharunderscore}{\kern0pt}t\ {\isacharparenleft}{\kern0pt}expr\ {\isasymphi}{\isacharparenright}{\kern0pt}\ {\isacharparenleft}{\kern0pt}n{\isadigit{1}}{\isacharcomma}{\kern0pt}\ n{\isadigit{2}}{\isacharcomma}{\kern0pt}\ n{\isadigit{3}}{\isacharcomma}{\kern0pt}\ n{\isadigit{4}}{\isacharcomma}{\kern0pt}\ n{\isadigit{5}}{\isacharcomma}{\kern0pt}\ n{\isadigit{6}}{\isacharparenright}{\kern0pt}{\isachardoublequoteclose}\ \isanewline
%
\isadelimproof
%
\endisadelimproof
%
\isatagproof
\isacommand{proof}\isamarkupfalse%
{\isacharminus}{\kern0pt}\isanewline
\ \ \isacommand{from}\isamarkupfalse%
\ A{\isadigit{1}}\ \isacommand{have}\isamarkupfalse%
\ E{\isacharunderscore}{\kern0pt}rest{\isacharcolon}{\kern0pt}\ \isanewline
{\isachardoublequoteopen}expr{\isacharunderscore}{\kern0pt}{\isadigit{2}}\ {\isasymphi}\ {\isasymle}\ n{\isadigit{2}}\ {\isasymand}\ expr{\isacharunderscore}{\kern0pt}{\isadigit{3}}\ {\isasymphi}\ {\isasymle}\ n{\isadigit{3}}\ {\isasymand}\ expr{\isacharunderscore}{\kern0pt}{\isadigit{4}}\ {\isasymphi}\ {\isasymle}\ n{\isadigit{4}}\ {\isasymand}\ expr{\isacharunderscore}{\kern0pt}{\isadigit{5}}\ {\isasymphi}\ {\isasymle}\ n{\isadigit{5}}\ {\isasymand}expr{\isacharunderscore}{\kern0pt}{\isadigit{6}}\ {\isasymphi}\ {\isasymle}\ n{\isadigit{6}}{\isachardoublequoteclose}\ \isanewline
\ \ \ \ \isacommand{using}\isamarkupfalse%
\ expr{\isachardot}{\kern0pt}simps\ \isanewline
\ \ \ \ \isacommand{by}\isamarkupfalse%
\ simp\isanewline
\ \ \isacommand{from}\isamarkupfalse%
\ A{\isadigit{1}}\ \isacommand{have}\isamarkupfalse%
\ {\isachardoublequoteopen}{\isadigit{1}}\ {\isacharplus}{\kern0pt}\ expr{\isacharunderscore}{\kern0pt}{\isadigit{1}}\ {\isasymphi}\ {\isasymle}\ n{\isadigit{1}}{\isachardoublequoteclose}\isanewline
\ \ \ \ \isacommand{using}\isamarkupfalse%
\ expr{\isacharunderscore}{\kern0pt}{\isadigit{1}}{\isachardot}{\kern0pt}simps{\isacharparenleft}{\kern0pt}{\isadigit{1}}{\isacharparenright}{\kern0pt}\ \isacommand{by}\isamarkupfalse%
\ simp\isanewline
\ \ \isacommand{hence}\isamarkupfalse%
\ {\isachardoublequoteopen}expr{\isacharunderscore}{\kern0pt}{\isadigit{1}}\ {\isasymphi}\ {\isasymle}\ n{\isadigit{1}}{\isachardoublequoteclose}\ \isanewline
\ \ \ \ \isacommand{using}\isamarkupfalse%
\ ile{\isacharunderscore}{\kern0pt}eSuc\ plus{\isacharunderscore}{\kern0pt}{\isadigit{1}}{\isacharunderscore}{\kern0pt}eSuc{\isacharparenleft}{\kern0pt}{\isadigit{1}}{\isacharparenright}{\kern0pt}\ dual{\isacharunderscore}{\kern0pt}order{\isachardot}{\kern0pt}trans\ \isacommand{by}\isamarkupfalse%
\ fastforce\isanewline
\ \ \isacommand{with}\isamarkupfalse%
\ E{\isacharunderscore}{\kern0pt}rest\ \isacommand{show}\isamarkupfalse%
\ {\isacharquery}{\kern0pt}thesis\ \isacommand{by}\isamarkupfalse%
\ simp\isanewline
\isacommand{qed}\isamarkupfalse%
%
\endisatagproof
{\isafoldproof}%
%
\isadelimproof
\isanewline
%
\endisadelimproof
\isanewline
\isacommand{lemma}\isamarkupfalse%
\ mon{\isacharunderscore}{\kern0pt}conj{\isacharcolon}{\kern0pt}\isanewline
\ \ \isakeyword{fixes}\ n{\isadigit{1}}\ \isakeyword{and}\ n{\isadigit{2}}\ \isakeyword{and}\ n{\isadigit{3}}\ \isakeyword{and}\ n{\isadigit{4}}\ \isakeyword{and}\ n{\isadigit{5}}\ \isakeyword{and}\ n{\isadigit{6}}\ \isakeyword{and}\ xs\ \isakeyword{and}\ ys\isanewline
\ \ \isakeyword{assumes}\ {\isachardoublequoteopen}less{\isacharunderscore}{\kern0pt}eq{\isacharunderscore}{\kern0pt}t\ {\isacharparenleft}{\kern0pt}expr\ {\isacharparenleft}{\kern0pt}hml{\isacharunderscore}{\kern0pt}conj\ I\ J\ {\isasymPhi}{\isacharparenright}{\kern0pt}{\isacharparenright}{\kern0pt}\ {\isacharparenleft}{\kern0pt}n{\isadigit{1}}{\isacharcomma}{\kern0pt}\ n{\isadigit{2}}{\isacharcomma}{\kern0pt}\ n{\isadigit{3}}{\isacharcomma}{\kern0pt}\ n{\isadigit{4}}{\isacharcomma}{\kern0pt}\ n{\isadigit{5}}{\isacharcomma}{\kern0pt}\ n{\isadigit{6}}{\isacharparenright}{\kern0pt}{\isachardoublequoteclose}\isanewline
\ \ \isakeyword{shows}\ {\isachardoublequoteopen}{\isacharparenleft}{\kern0pt}{\isasymforall}x\ {\isasymin}\ {\isacharparenleft}{\kern0pt}{\isasymPhi}\ {\isacharbackquote}{\kern0pt}\ I{\isacharparenright}{\kern0pt}{\isachardot}{\kern0pt}\ less{\isacharunderscore}{\kern0pt}eq{\isacharunderscore}{\kern0pt}t\ {\isacharparenleft}{\kern0pt}expr\ x{\isacharparenright}{\kern0pt}\ {\isacharparenleft}{\kern0pt}n{\isadigit{1}}{\isacharcomma}{\kern0pt}\ n{\isadigit{2}}{\isacharcomma}{\kern0pt}\ n{\isadigit{3}}{\isacharcomma}{\kern0pt}\ n{\isadigit{4}}{\isacharcomma}{\kern0pt}\ n{\isadigit{5}}{\isacharcomma}{\kern0pt}\ n{\isadigit{6}}{\isacharparenright}{\kern0pt}{\isacharparenright}{\kern0pt}{\isachardoublequoteclose}\ \isanewline
{\isachardoublequoteopen}{\isacharparenleft}{\kern0pt}{\isasymforall}y\ {\isasymin}\ {\isacharparenleft}{\kern0pt}{\isasymPhi}\ {\isacharbackquote}{\kern0pt}\ J{\isacharparenright}{\kern0pt}{\isachardot}{\kern0pt}\ less{\isacharunderscore}{\kern0pt}eq{\isacharunderscore}{\kern0pt}t\ {\isacharparenleft}{\kern0pt}expr\ y{\isacharparenright}{\kern0pt}\ {\isacharparenleft}{\kern0pt}n{\isadigit{1}}{\isacharcomma}{\kern0pt}\ n{\isadigit{2}}{\isacharcomma}{\kern0pt}\ n{\isadigit{3}}{\isacharcomma}{\kern0pt}\ n{\isadigit{4}}{\isacharcomma}{\kern0pt}\ n{\isadigit{5}}{\isacharcomma}{\kern0pt}\ n{\isadigit{6}}{\isacharparenright}{\kern0pt}{\isacharparenright}{\kern0pt}{\isachardoublequoteclose}\isanewline
%
\isadelimproof
%
\endisadelimproof
%
\isatagproof
\isacommand{proof}\isamarkupfalse%
{\isacharminus}{\kern0pt}\isanewline
\ \ \isacommand{have}\isamarkupfalse%
\ e{\isadigit{1}}{\isacharunderscore}{\kern0pt}eq{\isacharcolon}{\kern0pt}\ {\isachardoublequoteopen}expr{\isacharunderscore}{\kern0pt}{\isadigit{1}}\ {\isacharparenleft}{\kern0pt}hml{\isacharunderscore}{\kern0pt}conj\ I\ J\ {\isasymPhi}{\isacharparenright}{\kern0pt}\ {\isacharequal}{\kern0pt}\ Sup\ {\isacharparenleft}{\kern0pt}{\isacharparenleft}{\kern0pt}expr{\isacharunderscore}{\kern0pt}{\isadigit{1}}\ {\isasymcirc}\ {\isasymPhi}{\isacharparenright}{\kern0pt}\ {\isacharbackquote}{\kern0pt}\ I\ {\isasymunion}\ {\isacharparenleft}{\kern0pt}expr{\isacharunderscore}{\kern0pt}{\isadigit{1}}\ {\isasymcirc}\ {\isasymPhi}{\isacharparenright}{\kern0pt}\ {\isacharbackquote}{\kern0pt}\ J{\isacharparenright}{\kern0pt}{\isachardoublequoteclose}\isanewline
\ \ \ \ \isacommand{using}\isamarkupfalse%
\ expr{\isacharunderscore}{\kern0pt}{\isadigit{1}}{\isacharunderscore}{\kern0pt}conj\ \isacommand{by}\isamarkupfalse%
\ blast\isanewline
\ \ \isacommand{have}\isamarkupfalse%
\ e{\isadigit{2}}{\isacharunderscore}{\kern0pt}eq{\isacharcolon}{\kern0pt}\ {\isachardoublequoteopen}expr{\isacharunderscore}{\kern0pt}{\isadigit{2}}\ {\isacharparenleft}{\kern0pt}hml{\isacharunderscore}{\kern0pt}conj\ I\ J\ {\isasymPhi}{\isacharparenright}{\kern0pt}\ {\isacharequal}{\kern0pt}\ {\isadigit{1}}\ {\isacharplus}{\kern0pt}\ Sup\ {\isacharparenleft}{\kern0pt}{\isacharparenleft}{\kern0pt}expr{\isacharunderscore}{\kern0pt}{\isadigit{2}}\ {\isasymcirc}\ {\isasymPhi}{\isacharparenright}{\kern0pt}\ {\isacharbackquote}{\kern0pt}\ I\ {\isasymunion}\ {\isacharparenleft}{\kern0pt}expr{\isacharunderscore}{\kern0pt}{\isadigit{2}}\ {\isasymcirc}\ {\isasymPhi}{\isacharparenright}{\kern0pt}\ {\isacharbackquote}{\kern0pt}\ J{\isacharparenright}{\kern0pt}{\isachardoublequoteclose}\isanewline
\ \ \ \ \isacommand{using}\isamarkupfalse%
\ expr{\isacharunderscore}{\kern0pt}{\isadigit{2}}{\isacharunderscore}{\kern0pt}conj\ \isacommand{by}\isamarkupfalse%
\ blast\isanewline
\ \ \isacommand{have}\isamarkupfalse%
\ e{\isadigit{3}}{\isacharunderscore}{\kern0pt}eq{\isacharcolon}{\kern0pt}\ {\isachardoublequoteopen}expr{\isacharunderscore}{\kern0pt}{\isadigit{3}}\ {\isacharparenleft}{\kern0pt}hml{\isacharunderscore}{\kern0pt}conj\ I\ J\ {\isasymPhi}{\isacharparenright}{\kern0pt}\ {\isacharequal}{\kern0pt}\ {\isacharparenleft}{\kern0pt}Sup\ {\isacharparenleft}{\kern0pt}{\isacharparenleft}{\kern0pt}expr{\isacharunderscore}{\kern0pt}{\isadigit{1}}\ {\isasymcirc}\ {\isasymPhi}{\isacharparenright}{\kern0pt}\ {\isacharbackquote}{\kern0pt}\ I\ {\isasymunion}\ {\isacharparenleft}{\kern0pt}expr{\isacharunderscore}{\kern0pt}{\isadigit{3}}\ {\isasymcirc}\ {\isasymPhi}{\isacharparenright}{\kern0pt}\ {\isacharbackquote}{\kern0pt}\ I\ {\isasymunion}\ {\isacharparenleft}{\kern0pt}expr{\isacharunderscore}{\kern0pt}{\isadigit{3}}\ {\isasymcirc}\ {\isasymPhi}{\isacharparenright}{\kern0pt}\ {\isacharbackquote}{\kern0pt}\ J{\isacharparenright}{\kern0pt}{\isacharparenright}{\kern0pt}{\isachardoublequoteclose}\isanewline
\ \ \ \ \isacommand{using}\isamarkupfalse%
\ expr{\isacharunderscore}{\kern0pt}{\isadigit{3}}{\isacharunderscore}{\kern0pt}conj\ \isacommand{by}\isamarkupfalse%
\ blast\isanewline
\ \ \isacommand{have}\isamarkupfalse%
\ e{\isadigit{4}}{\isacharunderscore}{\kern0pt}eq{\isacharcolon}{\kern0pt}\ {\isachardoublequoteopen}expr{\isacharunderscore}{\kern0pt}{\isadigit{4}}\ {\isacharparenleft}{\kern0pt}hml{\isacharunderscore}{\kern0pt}conj\ I\ J\ {\isasymPhi}{\isacharparenright}{\kern0pt}\ {\isacharequal}{\kern0pt}\ Sup\ {\isacharparenleft}{\kern0pt}{\isacharparenleft}{\kern0pt}expr{\isacharunderscore}{\kern0pt}{\isadigit{1}}\ {\isacharbackquote}{\kern0pt}\ {\isacharparenleft}{\kern0pt}pos{\isacharunderscore}{\kern0pt}r\ {\isacharparenleft}{\kern0pt}{\isasymPhi}\ {\isacharbackquote}{\kern0pt}\ I{\isacharparenright}{\kern0pt}{\isacharparenright}{\kern0pt}{\isacharparenright}{\kern0pt}\ \ {\isasymunion}\ {\isacharparenleft}{\kern0pt}expr{\isacharunderscore}{\kern0pt}{\isadigit{4}}\ {\isasymcirc}\ {\isasymPhi}{\isacharparenright}{\kern0pt}\ {\isacharbackquote}{\kern0pt}\ I\ {\isasymunion}\ {\isacharparenleft}{\kern0pt}expr{\isacharunderscore}{\kern0pt}{\isadigit{4}}\ {\isasymcirc}\ {\isasymPhi}{\isacharparenright}{\kern0pt}\ {\isacharbackquote}{\kern0pt}\ J{\isacharparenright}{\kern0pt}{\isachardoublequoteclose}\isanewline
\ \ \ \ \isacommand{using}\isamarkupfalse%
\ expr{\isacharunderscore}{\kern0pt}{\isadigit{4}}{\isacharunderscore}{\kern0pt}conj\ \isacommand{by}\isamarkupfalse%
\ blast\isanewline
\ \ \isacommand{have}\isamarkupfalse%
\ e{\isadigit{5}}{\isacharunderscore}{\kern0pt}eq{\isacharcolon}{\kern0pt}\ {\isachardoublequoteopen}expr{\isacharunderscore}{\kern0pt}{\isadigit{5}}\ {\isacharparenleft}{\kern0pt}hml{\isacharunderscore}{\kern0pt}conj\ I\ J\ {\isasymPhi}{\isacharparenright}{\kern0pt}\ {\isacharequal}{\kern0pt}\ {\isacharparenleft}{\kern0pt}Sup\ {\isacharparenleft}{\kern0pt}{\isacharparenleft}{\kern0pt}expr{\isacharunderscore}{\kern0pt}{\isadigit{5}}\ {\isasymcirc}\ {\isasymPhi}{\isacharparenright}{\kern0pt}\ {\isacharbackquote}{\kern0pt}\ I\ {\isasymunion}\ {\isacharparenleft}{\kern0pt}expr{\isacharunderscore}{\kern0pt}{\isadigit{5}}\ {\isasymcirc}\ {\isasymPhi}{\isacharparenright}{\kern0pt}\ {\isacharbackquote}{\kern0pt}\ J\ {\isasymunion}\ {\isacharparenleft}{\kern0pt}expr{\isacharunderscore}{\kern0pt}{\isadigit{1}}\ {\isasymcirc}\ {\isasymPhi}{\isacharparenright}{\kern0pt}\ {\isacharbackquote}{\kern0pt}\ J{\isacharparenright}{\kern0pt}{\isacharparenright}{\kern0pt}{\isachardoublequoteclose}\isanewline
\ \ \ \ \isacommand{using}\isamarkupfalse%
\ expr{\isacharunderscore}{\kern0pt}{\isadigit{5}}{\isacharunderscore}{\kern0pt}conj\ \isacommand{by}\isamarkupfalse%
\ blast\isanewline
\ \ \isacommand{have}\isamarkupfalse%
\ e{\isadigit{6}}{\isacharunderscore}{\kern0pt}eq{\isacharcolon}{\kern0pt}\ {\isachardoublequoteopen}expr{\isacharunderscore}{\kern0pt}{\isadigit{6}}\ {\isacharparenleft}{\kern0pt}hml{\isacharunderscore}{\kern0pt}conj\ I\ J\ {\isasymPhi}{\isacharparenright}{\kern0pt}\ {\isacharequal}{\kern0pt}\ {\isacharparenleft}{\kern0pt}Sup\ {\isacharparenleft}{\kern0pt}{\isacharparenleft}{\kern0pt}expr{\isacharunderscore}{\kern0pt}{\isadigit{6}}\ {\isasymcirc}\ {\isasymPhi}{\isacharparenright}{\kern0pt}\ {\isacharbackquote}{\kern0pt}\ I\ {\isasymunion}\ {\isacharparenleft}{\kern0pt}{\isacharparenleft}{\kern0pt}eSuc\ {\isasymcirc}\ expr{\isacharunderscore}{\kern0pt}{\isadigit{6}}\ {\isasymcirc}\ {\isasymPhi}{\isacharparenright}{\kern0pt}\ {\isacharbackquote}{\kern0pt}\ J{\isacharparenright}{\kern0pt}{\isacharparenright}{\kern0pt}{\isacharparenright}{\kern0pt}{\isachardoublequoteclose}\isanewline
\ \ \ \ \isacommand{using}\isamarkupfalse%
\ expr{\isacharunderscore}{\kern0pt}{\isadigit{6}}{\isacharunderscore}{\kern0pt}conj\ \isacommand{by}\isamarkupfalse%
\ blast\isanewline
\isanewline
\ \ \isacommand{have}\isamarkupfalse%
\ e{\isadigit{1}}{\isacharunderscore}{\kern0pt}le{\isacharcolon}{\kern0pt}\ {\isachardoublequoteopen}expr{\isacharunderscore}{\kern0pt}{\isadigit{1}}\ {\isacharparenleft}{\kern0pt}hml{\isacharunderscore}{\kern0pt}conj\ I\ J\ {\isasymPhi}{\isacharparenright}{\kern0pt}\ {\isasymle}\ n{\isadigit{1}}{\isachardoublequoteclose}\ \isakeyword{and}\isanewline
e{\isadigit{2}}{\isacharunderscore}{\kern0pt}le{\isacharcolon}{\kern0pt}\ {\isachardoublequoteopen}expr{\isacharunderscore}{\kern0pt}{\isadigit{2}}\ {\isacharparenleft}{\kern0pt}hml{\isacharunderscore}{\kern0pt}conj\ I\ J\ {\isasymPhi}{\isacharparenright}{\kern0pt}\ {\isasymle}\ n{\isadigit{2}}{\isachardoublequoteclose}\ \isakeyword{and}\isanewline
e{\isadigit{3}}{\isacharunderscore}{\kern0pt}le{\isacharcolon}{\kern0pt}\ {\isachardoublequoteopen}expr{\isacharunderscore}{\kern0pt}{\isadigit{3}}\ {\isacharparenleft}{\kern0pt}hml{\isacharunderscore}{\kern0pt}conj\ I\ J\ {\isasymPhi}{\isacharparenright}{\kern0pt}\ {\isasymle}\ n{\isadigit{3}}{\isachardoublequoteclose}\ \isakeyword{and}\isanewline
e{\isadigit{4}}{\isacharunderscore}{\kern0pt}le{\isacharcolon}{\kern0pt}\ {\isachardoublequoteopen}expr{\isacharunderscore}{\kern0pt}{\isadigit{4}}\ {\isacharparenleft}{\kern0pt}hml{\isacharunderscore}{\kern0pt}conj\ I\ J\ {\isasymPhi}{\isacharparenright}{\kern0pt}\ {\isasymle}\ n{\isadigit{4}}{\isachardoublequoteclose}\ \isakeyword{and}\isanewline
e{\isadigit{5}}{\isacharunderscore}{\kern0pt}le{\isacharcolon}{\kern0pt}\ {\isachardoublequoteopen}expr{\isacharunderscore}{\kern0pt}{\isadigit{5}}\ {\isacharparenleft}{\kern0pt}hml{\isacharunderscore}{\kern0pt}conj\ I\ J\ {\isasymPhi}{\isacharparenright}{\kern0pt}\ {\isasymle}\ n{\isadigit{5}}{\isachardoublequoteclose}\ \isakeyword{and}\isanewline
e{\isadigit{6}}{\isacharunderscore}{\kern0pt}le{\isacharcolon}{\kern0pt}\ {\isachardoublequoteopen}expr{\isacharunderscore}{\kern0pt}{\isadigit{6}}\ {\isacharparenleft}{\kern0pt}hml{\isacharunderscore}{\kern0pt}conj\ I\ J\ {\isasymPhi}{\isacharparenright}{\kern0pt}\ {\isasymle}\ n{\isadigit{6}}{\isachardoublequoteclose}\isanewline
\ \ \ \ \isacommand{using}\isamarkupfalse%
\ less{\isacharunderscore}{\kern0pt}eq{\isacharunderscore}{\kern0pt}t{\isachardot}{\kern0pt}simps\ expr{\isachardot}{\kern0pt}simps\ assms\isanewline
\ \ \ \ \isacommand{by}\isamarkupfalse%
\ metis{\isacharplus}{\kern0pt}\isanewline
\isanewline
\ \ \isacommand{from}\isamarkupfalse%
\ e{\isadigit{1}}{\isacharunderscore}{\kern0pt}eq\ e{\isadigit{1}}{\isacharunderscore}{\kern0pt}le\ \isacommand{have}\isamarkupfalse%
\ e{\isadigit{1}}{\isacharunderscore}{\kern0pt}pos{\isacharcolon}{\kern0pt}\ {\isachardoublequoteopen}Sup\ {\isacharparenleft}{\kern0pt}{\isacharparenleft}{\kern0pt}expr{\isacharunderscore}{\kern0pt}{\isadigit{1}}\ {\isasymcirc}\ {\isasymPhi}{\isacharparenright}{\kern0pt}\ {\isacharbackquote}{\kern0pt}\ I{\isacharparenright}{\kern0pt}\ {\isasymle}\ n{\isadigit{1}}{\isachardoublequoteclose}\isanewline
\isakeyword{and}\ e{\isadigit{1}}{\isacharunderscore}{\kern0pt}neg{\isacharcolon}{\kern0pt}\ {\isachardoublequoteopen}Sup\ {\isacharparenleft}{\kern0pt}{\isacharparenleft}{\kern0pt}expr{\isacharunderscore}{\kern0pt}{\isadigit{1}}\ {\isasymcirc}\ {\isasymPhi}{\isacharparenright}{\kern0pt}\ {\isacharbackquote}{\kern0pt}\ J{\isacharparenright}{\kern0pt}\ {\isasymle}\ n{\isadigit{1}}{\isachardoublequoteclose}\isanewline
\ \ \ \ \isacommand{using}\isamarkupfalse%
\ Sup{\isacharunderscore}{\kern0pt}union{\isacharunderscore}{\kern0pt}distrib\ le{\isacharunderscore}{\kern0pt}sup{\isacharunderscore}{\kern0pt}iff\ sup{\isacharunderscore}{\kern0pt}enat{\isacharunderscore}{\kern0pt}def\isanewline
\ \ \ \ \isacommand{by}\isamarkupfalse%
\ metis{\isacharplus}{\kern0pt}\isanewline
\ \ \isacommand{hence}\isamarkupfalse%
\ e{\isadigit{1}}{\isacharunderscore}{\kern0pt}le{\isacharunderscore}{\kern0pt}pos{\isacharcolon}{\kern0pt}\ {\isachardoublequoteopen}{\isasymforall}x{\isasymin}{\isasymPhi}\ {\isacharbackquote}{\kern0pt}\ I{\isachardot}{\kern0pt}\ expr{\isacharunderscore}{\kern0pt}{\isadigit{1}}\ x\ {\isasymle}\ n{\isadigit{1}}{\isachardoublequoteclose}\isanewline
\isakeyword{and}\ e{\isadigit{1}}{\isacharunderscore}{\kern0pt}le{\isacharunderscore}{\kern0pt}neg{\isacharcolon}{\kern0pt}\ {\isachardoublequoteopen}{\isasymforall}y{\isasymin}{\isasymPhi}\ {\isacharbackquote}{\kern0pt}\ J{\isachardot}{\kern0pt}\ expr{\isacharunderscore}{\kern0pt}{\isadigit{1}}\ y\ {\isasymle}\ n{\isadigit{1}}{\isachardoublequoteclose}\isanewline
\ \ \ \ \isacommand{by}\isamarkupfalse%
\ {\isacharparenleft}{\kern0pt}simp\ add{\isacharcolon}{\kern0pt}\ Sup{\isacharunderscore}{\kern0pt}le{\isacharunderscore}{\kern0pt}iff{\isacharparenright}{\kern0pt}{\isacharplus}{\kern0pt}\isanewline
\isanewline
\ \ \isacommand{from}\isamarkupfalse%
\ e{\isadigit{2}}{\isacharunderscore}{\kern0pt}eq\ e{\isadigit{2}}{\isacharunderscore}{\kern0pt}le\ \isacommand{have}\isamarkupfalse%
\ e{\isadigit{2}}{\isacharunderscore}{\kern0pt}pos{\isacharcolon}{\kern0pt}\ {\isachardoublequoteopen}Sup\ {\isacharparenleft}{\kern0pt}{\isacharparenleft}{\kern0pt}expr{\isacharunderscore}{\kern0pt}{\isadigit{2}}\ {\isasymcirc}\ {\isasymPhi}{\isacharparenright}{\kern0pt}\ {\isacharbackquote}{\kern0pt}\ I{\isacharparenright}{\kern0pt}\ {\isacharless}{\kern0pt}{\isacharequal}{\kern0pt}\ n{\isadigit{2}}{\isachardoublequoteclose}\isanewline
\isakeyword{and}\ e{\isadigit{2}}{\isacharunderscore}{\kern0pt}neg{\isacharcolon}{\kern0pt}\ {\isachardoublequoteopen}Sup\ {\isacharparenleft}{\kern0pt}{\isacharparenleft}{\kern0pt}expr{\isacharunderscore}{\kern0pt}{\isadigit{2}}\ {\isasymcirc}\ {\isasymPhi}{\isacharparenright}{\kern0pt}\ {\isacharbackquote}{\kern0pt}\ J{\isacharparenright}{\kern0pt}\ {\isasymle}\ n{\isadigit{2}}{\isachardoublequoteclose}\isanewline
\ \ \ \ \isacommand{using}\isamarkupfalse%
\ Sup{\isacharunderscore}{\kern0pt}union{\isacharunderscore}{\kern0pt}distrib\ le{\isacharunderscore}{\kern0pt}sup{\isacharunderscore}{\kern0pt}iff\ sup{\isacharunderscore}{\kern0pt}enat{\isacharunderscore}{\kern0pt}def\isanewline
\ \ \ \ \isacommand{by}\isamarkupfalse%
\ {\isacharparenleft}{\kern0pt}metis\ {\isacharparenleft}{\kern0pt}no{\isacharunderscore}{\kern0pt}types{\isacharcomma}{\kern0pt}\ lifting{\isacharparenright}{\kern0pt}\ dual{\isacharunderscore}{\kern0pt}order{\isachardot}{\kern0pt}trans\ ile{\isacharunderscore}{\kern0pt}eSuc\ plus{\isacharunderscore}{\kern0pt}{\isadigit{1}}{\isacharunderscore}{\kern0pt}eSuc{\isacharparenleft}{\kern0pt}{\isadigit{1}}{\isacharparenright}{\kern0pt}{\isacharparenright}{\kern0pt}{\isacharplus}{\kern0pt}\isanewline
\isanewline
\ \ \isacommand{from}\isamarkupfalse%
\ e{\isadigit{3}}{\isacharunderscore}{\kern0pt}eq\ e{\isadigit{3}}{\isacharunderscore}{\kern0pt}le\ \isacommand{have}\isamarkupfalse%
\ e{\isadigit{3}}{\isacharunderscore}{\kern0pt}pos{\isacharcolon}{\kern0pt}\ {\isachardoublequoteopen}Sup\ {\isacharparenleft}{\kern0pt}{\isacharparenleft}{\kern0pt}expr{\isacharunderscore}{\kern0pt}{\isadigit{3}}\ {\isasymcirc}\ {\isasymPhi}{\isacharparenright}{\kern0pt}\ {\isacharbackquote}{\kern0pt}\ I{\isacharparenright}{\kern0pt}\ {\isacharless}{\kern0pt}{\isacharequal}{\kern0pt}\ n{\isadigit{3}}{\isachardoublequoteclose}\isanewline
\isakeyword{and}\ e{\isadigit{3}}{\isacharunderscore}{\kern0pt}neg{\isacharcolon}{\kern0pt}\ {\isachardoublequoteopen}Sup\ {\isacharparenleft}{\kern0pt}{\isacharparenleft}{\kern0pt}expr{\isacharunderscore}{\kern0pt}{\isadigit{3}}\ {\isasymcirc}\ {\isasymPhi}{\isacharparenright}{\kern0pt}\ {\isacharbackquote}{\kern0pt}\ J{\isacharparenright}{\kern0pt}\ {\isacharless}{\kern0pt}{\isacharequal}{\kern0pt}\ n{\isadigit{3}}{\isachardoublequoteclose}\isanewline
\ \ \ \ \isacommand{using}\isamarkupfalse%
\ Sup{\isacharunderscore}{\kern0pt}union{\isacharunderscore}{\kern0pt}distrib\ le{\isacharunderscore}{\kern0pt}sup{\isacharunderscore}{\kern0pt}iff\ sup{\isacharunderscore}{\kern0pt}enat{\isacharunderscore}{\kern0pt}def\isanewline
\ \ \ \ \isacommand{by}\isamarkupfalse%
\ {\isacharparenleft}{\kern0pt}simp\ add{\isacharcolon}{\kern0pt}\ Sup{\isacharunderscore}{\kern0pt}le{\isacharunderscore}{\kern0pt}iff{\isacharparenright}{\kern0pt}{\isacharplus}{\kern0pt}\isanewline
\isanewline
\ \ \isacommand{from}\isamarkupfalse%
\ e{\isadigit{4}}{\isacharunderscore}{\kern0pt}eq\ e{\isadigit{4}}{\isacharunderscore}{\kern0pt}le\ \isacommand{have}\isamarkupfalse%
\ e{\isadigit{4}}{\isacharunderscore}{\kern0pt}pos{\isacharcolon}{\kern0pt}\ {\isachardoublequoteopen}Sup\ {\isacharparenleft}{\kern0pt}{\isacharparenleft}{\kern0pt}expr{\isacharunderscore}{\kern0pt}{\isadigit{4}}\ {\isasymcirc}\ {\isasymPhi}{\isacharparenright}{\kern0pt}\ {\isacharbackquote}{\kern0pt}\ I{\isacharparenright}{\kern0pt}\ {\isasymle}\ n{\isadigit{4}}{\isachardoublequoteclose}\isanewline
\isakeyword{and}\ e{\isadigit{4}}{\isacharunderscore}{\kern0pt}neg{\isacharcolon}{\kern0pt}\ {\isachardoublequoteopen}Sup\ {\isacharparenleft}{\kern0pt}{\isacharparenleft}{\kern0pt}expr{\isacharunderscore}{\kern0pt}{\isadigit{4}}\ {\isasymcirc}\ {\isasymPhi}{\isacharparenright}{\kern0pt}\ {\isacharbackquote}{\kern0pt}\ J{\isacharparenright}{\kern0pt}\ {\isasymle}\ n{\isadigit{4}}{\isachardoublequoteclose}\isanewline
\ \ \ \ \isacommand{using}\isamarkupfalse%
\ Sup{\isacharunderscore}{\kern0pt}union{\isacharunderscore}{\kern0pt}distrib\ le{\isacharunderscore}{\kern0pt}sup{\isacharunderscore}{\kern0pt}iff\ sup{\isacharunderscore}{\kern0pt}enat{\isacharunderscore}{\kern0pt}def\isanewline
\ \ \ \ \isacommand{by}\isamarkupfalse%
\ {\isacharparenleft}{\kern0pt}simp\ add{\isacharcolon}{\kern0pt}\ Sup{\isacharunderscore}{\kern0pt}le{\isacharunderscore}{\kern0pt}iff{\isacharparenright}{\kern0pt}{\isacharplus}{\kern0pt}\isanewline
\isanewline
\ \ \isacommand{from}\isamarkupfalse%
\ e{\isadigit{5}}{\isacharunderscore}{\kern0pt}eq\ e{\isadigit{5}}{\isacharunderscore}{\kern0pt}le\ \isacommand{have}\isamarkupfalse%
\ e{\isadigit{5}}{\isacharunderscore}{\kern0pt}pos{\isacharcolon}{\kern0pt}\ {\isachardoublequoteopen}Sup\ {\isacharparenleft}{\kern0pt}{\isacharparenleft}{\kern0pt}expr{\isacharunderscore}{\kern0pt}{\isadigit{5}}\ {\isasymcirc}\ {\isasymPhi}{\isacharparenright}{\kern0pt}\ {\isacharbackquote}{\kern0pt}\ I{\isacharparenright}{\kern0pt}\ {\isacharless}{\kern0pt}{\isacharequal}{\kern0pt}\ n{\isadigit{5}}{\isachardoublequoteclose}\isanewline
\isakeyword{and}\ e{\isadigit{5}}{\isacharunderscore}{\kern0pt}neg{\isacharcolon}{\kern0pt}\ {\isachardoublequoteopen}Sup\ {\isacharparenleft}{\kern0pt}{\isacharparenleft}{\kern0pt}expr{\isacharunderscore}{\kern0pt}{\isadigit{5}}\ {\isasymcirc}\ {\isasymPhi}{\isacharparenright}{\kern0pt}\ {\isacharbackquote}{\kern0pt}\ J{\isacharparenright}{\kern0pt}\ {\isacharless}{\kern0pt}{\isacharequal}{\kern0pt}\ n{\isadigit{5}}{\isachardoublequoteclose}\isanewline
\ \ \ \ \isacommand{using}\isamarkupfalse%
\ Sup{\isacharunderscore}{\kern0pt}union{\isacharunderscore}{\kern0pt}distrib\ le{\isacharunderscore}{\kern0pt}sup{\isacharunderscore}{\kern0pt}iff\ sup{\isacharunderscore}{\kern0pt}enat{\isacharunderscore}{\kern0pt}def\isanewline
\ \ \ \ \isacommand{by}\isamarkupfalse%
\ {\isacharparenleft}{\kern0pt}simp\ add{\isacharcolon}{\kern0pt}\ Sup{\isacharunderscore}{\kern0pt}le{\isacharunderscore}{\kern0pt}iff{\isacharparenright}{\kern0pt}{\isacharplus}{\kern0pt}\isanewline
\isanewline
\ \ \isacommand{from}\isamarkupfalse%
\ e{\isadigit{6}}{\isacharunderscore}{\kern0pt}eq\ e{\isadigit{6}}{\isacharunderscore}{\kern0pt}le\ \isacommand{have}\isamarkupfalse%
\ e{\isadigit{6}}{\isacharunderscore}{\kern0pt}pos{\isacharcolon}{\kern0pt}\ {\isachardoublequoteopen}Sup\ {\isacharparenleft}{\kern0pt}{\isacharparenleft}{\kern0pt}expr{\isacharunderscore}{\kern0pt}{\isadigit{6}}\ {\isasymcirc}\ {\isasymPhi}{\isacharparenright}{\kern0pt}\ {\isacharbackquote}{\kern0pt}\ I{\isacharparenright}{\kern0pt}\ {\isasymle}\ n{\isadigit{6}}{\isachardoublequoteclose}\isanewline
\isakeyword{and}\ e{\isadigit{6}}{\isacharunderscore}{\kern0pt}neg{\isacharcolon}{\kern0pt}\ {\isachardoublequoteopen}Sup\ {\isacharparenleft}{\kern0pt}{\isacharparenleft}{\kern0pt}eSuc\ {\isasymcirc}\ expr{\isacharunderscore}{\kern0pt}{\isadigit{6}}\ {\isasymcirc}\ {\isasymPhi}{\isacharparenright}{\kern0pt}\ {\isacharbackquote}{\kern0pt}\ J{\isacharparenright}{\kern0pt}\ {\isasymle}\ n{\isadigit{6}}{\isachardoublequoteclose}\isanewline
\ \ \ \ \isacommand{using}\isamarkupfalse%
\ Sup{\isacharunderscore}{\kern0pt}union{\isacharunderscore}{\kern0pt}distrib\ le{\isacharunderscore}{\kern0pt}sup{\isacharunderscore}{\kern0pt}iff\ sup{\isacharunderscore}{\kern0pt}enat{\isacharunderscore}{\kern0pt}def\isanewline
\ \ \ \ \ \isacommand{apply}\isamarkupfalse%
\ {\isacharparenleft}{\kern0pt}simp\ add{\isacharcolon}{\kern0pt}\ Sup{\isacharunderscore}{\kern0pt}le{\isacharunderscore}{\kern0pt}iff{\isacharparenright}{\kern0pt}\isanewline
\ \ \ \ \isacommand{using}\isamarkupfalse%
\ Sup{\isacharunderscore}{\kern0pt}union{\isacharunderscore}{\kern0pt}distrib\ le{\isacharunderscore}{\kern0pt}sup{\isacharunderscore}{\kern0pt}iff\ sup{\isacharunderscore}{\kern0pt}enat{\isacharunderscore}{\kern0pt}def\ e{\isadigit{6}}{\isacharunderscore}{\kern0pt}eq\ e{\isadigit{6}}{\isacharunderscore}{\kern0pt}le\isanewline
\ \ \ \ \isacommand{by}\isamarkupfalse%
\ metis\isanewline
\isanewline
\ \ \isacommand{from}\isamarkupfalse%
\ e{\isadigit{6}}{\isacharunderscore}{\kern0pt}neg\ \isacommand{have}\isamarkupfalse%
\ e{\isadigit{6}}{\isacharunderscore}{\kern0pt}neg{\isacharcolon}{\kern0pt}\ {\isachardoublequoteopen}Sup\ {\isacharparenleft}{\kern0pt}{\isacharparenleft}{\kern0pt}expr{\isacharunderscore}{\kern0pt}{\isadigit{6}}\ {\isasymcirc}\ {\isasymPhi}{\isacharparenright}{\kern0pt}\ {\isacharbackquote}{\kern0pt}\ J{\isacharparenright}{\kern0pt}\ {\isasymle}\ n{\isadigit{6}}{\isachardoublequoteclose}\isanewline
\ \ \ \ \isacommand{using}\isamarkupfalse%
\ Sup{\isacharunderscore}{\kern0pt}enat{\isacharunderscore}{\kern0pt}def\ eSuc{\isacharunderscore}{\kern0pt}def\isanewline
\ \ \ \ \isacommand{by}\isamarkupfalse%
\ {\isacharparenleft}{\kern0pt}metis\ dual{\isacharunderscore}{\kern0pt}order{\isachardot}{\kern0pt}trans\ eSuc{\isacharunderscore}{\kern0pt}Sup\ i{\isadigit{0}}{\isacharunderscore}{\kern0pt}lb\ ile{\isacharunderscore}{\kern0pt}eSuc\ image{\isacharunderscore}{\kern0pt}comp{\isacharparenright}{\kern0pt}\isanewline
\isanewline
\isanewline
\ \ \isacommand{show}\isamarkupfalse%
\ {\isachardoublequoteopen}{\isasymforall}x{\isasymin}{\isasymPhi}\ {\isacharbackquote}{\kern0pt}\ I{\isachardot}{\kern0pt}\ less{\isacharunderscore}{\kern0pt}eq{\isacharunderscore}{\kern0pt}t\ {\isacharparenleft}{\kern0pt}expr\ x{\isacharparenright}{\kern0pt}\ {\isacharparenleft}{\kern0pt}n{\isadigit{1}}{\isacharcomma}{\kern0pt}\ n{\isadigit{2}}{\isacharcomma}{\kern0pt}\ n{\isadigit{3}}{\isacharcomma}{\kern0pt}\ n{\isadigit{4}}{\isacharcomma}{\kern0pt}\ n{\isadigit{5}}{\isacharcomma}{\kern0pt}\ n{\isadigit{6}}{\isacharparenright}{\kern0pt}{\isachardoublequoteclose}\isanewline
\ \ \ \ \isacommand{using}\isamarkupfalse%
\ e{\isadigit{1}}{\isacharunderscore}{\kern0pt}pos\ e{\isadigit{2}}{\isacharunderscore}{\kern0pt}pos\ e{\isadigit{3}}{\isacharunderscore}{\kern0pt}pos\ e{\isadigit{4}}{\isacharunderscore}{\kern0pt}pos\ e{\isadigit{5}}{\isacharunderscore}{\kern0pt}pos\ e{\isadigit{6}}{\isacharunderscore}{\kern0pt}pos\isanewline
expr{\isachardot}{\kern0pt}simps\ less{\isacharunderscore}{\kern0pt}eq{\isacharunderscore}{\kern0pt}t{\isachardot}{\kern0pt}simps\isanewline
\ \ \ \ \isacommand{by}\isamarkupfalse%
\ {\isacharparenleft}{\kern0pt}simp\ add{\isacharcolon}{\kern0pt}\ Sup{\isacharunderscore}{\kern0pt}le{\isacharunderscore}{\kern0pt}iff{\isacharparenright}{\kern0pt}\isanewline
\isanewline
\ \ \isacommand{show}\isamarkupfalse%
\ {\isachardoublequoteopen}{\isasymforall}y{\isasymin}{\isasymPhi}\ {\isacharbackquote}{\kern0pt}\ J{\isachardot}{\kern0pt}\ less{\isacharunderscore}{\kern0pt}eq{\isacharunderscore}{\kern0pt}t\ {\isacharparenleft}{\kern0pt}expr\ y{\isacharparenright}{\kern0pt}\ {\isacharparenleft}{\kern0pt}n{\isadigit{1}}{\isacharcomma}{\kern0pt}\ n{\isadigit{2}}{\isacharcomma}{\kern0pt}\ n{\isadigit{3}}{\isacharcomma}{\kern0pt}\ n{\isadigit{4}}{\isacharcomma}{\kern0pt}\ n{\isadigit{5}}{\isacharcomma}{\kern0pt}\ n{\isadigit{6}}{\isacharparenright}{\kern0pt}{\isachardoublequoteclose}\isanewline
\ \ \ \ \isacommand{using}\isamarkupfalse%
\ e{\isadigit{1}}{\isacharunderscore}{\kern0pt}neg\ e{\isadigit{2}}{\isacharunderscore}{\kern0pt}neg\ e{\isadigit{3}}{\isacharunderscore}{\kern0pt}neg\ e{\isadigit{4}}{\isacharunderscore}{\kern0pt}neg\ e{\isadigit{5}}{\isacharunderscore}{\kern0pt}neg\ e{\isadigit{6}}{\isacharunderscore}{\kern0pt}neg\isanewline
expr{\isachardot}{\kern0pt}simps\ less{\isacharunderscore}{\kern0pt}eq{\isacharunderscore}{\kern0pt}t{\isachardot}{\kern0pt}simps\isanewline
\ \ \ \ \isacommand{by}\isamarkupfalse%
\ {\isacharparenleft}{\kern0pt}simp\ add{\isacharcolon}{\kern0pt}\ Sup{\isacharunderscore}{\kern0pt}le{\isacharunderscore}{\kern0pt}iff{\isacharparenright}{\kern0pt}\isanewline
\isacommand{qed}\isamarkupfalse%
\isanewline
\isanewline
\isanewline
%
\endisatagproof
{\isafoldproof}%
%
\isadelimproof
%
\endisadelimproof
%
\isadelimtheory
%
\endisadelimtheory
%
\isatagtheory
%
\endisatagtheory
{\isafoldtheory}%
%
\isadelimtheory
%
\endisadelimtheory
%
\end{isabellebody}%
\endinput
%:%file=~/Documents/Isabelle_HOL/formula_prices_list.thy%:%
%:%24=9%:%
%:%36=11%:%
%:%37=12%:%
%:%41=14%:%
%:%50=16%:%
%:%62=17%:%
%:%63=18%:%
%:%64=19%:%
%:%65=20%:%
%:%66=21%:%
%:%67=22%:%
%:%68=23%:%
%:%69=24%:%
%:%70=25%:%
%:%71=26%:%
%:%72=27%:%
%:%73=28%:%
%:%74=29%:%
%:%75=30%:%
%:%76=31%:%
%:%77=32%:%
%:%78=33%:%
%:%79=34%:%
%:%80=35%:%
%:%81=36%:%
%:%82=37%:%
%:%83=38%:%
%:%84=39%:%
%:%85=40%:%
%:%86=41%:%
%:%87=42%:%
%:%88=43%:%
%:%89=44%:%
%:%90=45%:%
%:%91=46%:%
%:%92=47%:%
%:%93=48%:%
%:%94=49%:%
%:%95=50%:%
%:%96=51%:%
%:%97=52%:%
%:%98=53%:%
%:%99=54%:%
%:%100=55%:%
%:%101=56%:%
%:%102=57%:%
%:%103=58%:%
%:%104=59%:%
%:%105=60%:%
%:%106=61%:%
%:%107=62%:%
%:%108=63%:%
%:%109=64%:%
%:%110=65%:%
%:%111=66%:%
%:%112=67%:%
%:%113=68%:%
%:%114=69%:%
%:%115=70%:%
%:%116=71%:%
%:%117=72%:%
%:%118=73%:%
%:%119=74%:%
%:%120=75%:%
%:%121=76%:%
%:%122=77%:%
%:%123=78%:%
%:%124=79%:%
%:%125=80%:%
%:%126=81%:%
%:%127=82%:%
%:%128=83%:%
%:%129=84%:%
%:%130=85%:%
%:%131=86%:%
%:%132=87%:%
%:%133=88%:%
%:%134=89%:%
%:%135=90%:%
%:%136=91%:%
%:%137=92%:%
%:%138=93%:%
%:%139=94%:%
%:%140=95%:%
%:%141=96%:%
%:%142=97%:%
%:%143=98%:%
%:%144=99%:%
%:%145=100%:%
%:%146=101%:%
%:%147=102%:%
%:%148=103%:%
%:%149=104%:%
%:%150=105%:%
%:%151=106%:%
%:%152=107%:%
%:%153=108%:%
%:%154=109%:%
%:%155=110%:%
%:%156=111%:%
%:%157=112%:%
%:%158=113%:%
%:%159=114%:%
%:%160=115%:%
%:%161=116%:%
%:%162=117%:%
%:%163=118%:%
%:%167=121%:%
%:%168=122%:%
%:%170=124%:%
%:%171=124%:%
%:%172=125%:%
%:%173=126%:%
%:%174=127%:%
%:%175=128%:%
%:%176=129%:%
%:%177=130%:%
%:%178=131%:%
%:%179=131%:%
%:%180=132%:%
%:%181=133%:%
%:%182=134%:%
%:%183=135%:%
%:%184=136%:%
%:%185=137%:%
%:%186=137%:%
%:%187=138%:%
%:%188=139%:%
%:%189=140%:%
%:%190=141%:%
%:%191=142%:%
%:%192=143%:%
%:%193=143%:%
%:%194=144%:%
%:%195=145%:%
%:%199=149%:%
%:%200=150%:%
%:%201=151%:%
%:%202=151%:%
%:%203=152%:%
%:%204=153%:%
%:%205=154%:%
%:%206=155%:%
%:%207=156%:%
%:%208=157%:%
%:%209=157%:%
%:%210=158%:%
%:%211=159%:%
%:%212=160%:%
%:%213=161%:%
%:%214=162%:%
%:%215=163%:%
%:%216=164%:%
%:%217=164%:%
%:%218=165%:%
%:%219=166%:%
%:%220=167%:%
%:%221=168%:%
%:%222=169%:%
%:%223=170%:%
%:%224=171%:%
%:%225=171%:%
%:%226=172%:%
%:%227=173%:%
%:%229=175%:%
%:%231=177%:%
%:%232=177%:%
%:%233=178%:%
%:%234=179%:%
%:%235=180%:%
%:%236=181%:%
%:%237=182%:%
%:%238=182%:%
%:%239=183%:%
%:%242=193%:%
%:%243=194%:%
%:%244=195%:%
%:%245=196%:%
%:%246=197%:%
%:%247=198%:%
%:%248=199%:%
%:%249=200%:%
%:%250=201%:%
%:%251=202%:%
%:%252=203%:%
%:%253=204%:%
%:%254=205%:%
%:%255=206%:%
%:%256=207%:%
%:%257=208%:%
%:%258=209%:%
%:%259=210%:%
%:%260=211%:%
%:%261=212%:%
%:%262=213%:%
%:%263=214%:%
%:%264=215%:%
%:%265=216%:%
%:%266=217%:%
%:%267=218%:%
%:%268=219%:%
%:%269=220%:%
%:%270=221%:%
%:%271=222%:%
%:%272=223%:%
%:%273=224%:%
%:%274=225%:%
%:%275=226%:%
%:%276=227%:%
%:%277=228%:%
%:%278=229%:%
%:%279=230%:%
%:%280=231%:%
%:%281=232%:%
%:%282=233%:%
%:%283=234%:%
%:%284=235%:%
%:%285=236%:%
%:%286=237%:%
%:%287=238%:%
%:%288=239%:%
%:%289=240%:%
%:%290=241%:%
%:%291=242%:%
%:%292=243%:%
%:%293=244%:%
%:%294=245%:%
%:%295=246%:%
%:%296=247%:%
%:%297=248%:%
%:%298=249%:%
%:%299=250%:%
%:%300=251%:%
%:%301=252%:%
%:%302=253%:%
%:%303=254%:%
%:%304=255%:%
%:%305=256%:%
%:%306=257%:%
%:%307=258%:%
%:%308=259%:%
%:%309=260%:%
%:%310=261%:%
%:%311=262%:%
%:%312=263%:%
%:%313=264%:%
%:%315=266%:%
%:%316=266%:%
%:%317=267%:%
%:%318=268%:%
%:%319=269%:%
%:%335=285%:%
%:%336=286%:%
%:%337=287%:%
%:%338=288%:%
%:%339=289%:%
%:%340=290%:%
%:%341=291%:%
%:%342=292%:%
%:%343=293%:%
%:%346=294%:%
%:%350=294%:%
%:%351=294%:%
%:%352=295%:%
%:%353=295%:%
%:%354=296%:%
%:%355=296%:%
%:%356=297%:%
%:%357=297%:%
%:%358=298%:%
%:%359=299%:%
%:%360=299%:%
%:%361=300%:%
%:%362=300%:%
%:%364=302%:%
%:%365=303%:%
%:%366=303%:%
%:%367=304%:%
%:%368=304%:%
%:%369=305%:%
%:%370=305%:%
%:%373=308%:%
%:%374=309%:%
%:%375=310%:%
%:%376=310%:%
%:%377=311%:%
%:%378=311%:%
%:%384=317%:%
%:%385=318%:%
%:%386=318%:%
%:%387=319%:%
%:%388=319%:%
%:%389=320%:%
%:%390=321%:%
%:%391=321%:%
%:%398=328%:%
%:%399=329%:%
%:%400=329%:%
%:%401=330%:%
%:%402=330%:%
%:%403=331%:%
%:%404=331%:%
%:%405=332%:%
%:%406=332%:%
%:%407=333%:%
%:%408=334%:%
%:%409=334%:%
%:%410=335%:%
%:%411=335%:%
%:%412=336%:%
%:%413=336%:%
%:%414=337%:%
%:%415=338%:%
%:%416=338%:%
%:%417=339%:%
%:%418=339%:%
%:%419=340%:%
%:%420=340%:%
%:%427=347%:%
%:%428=348%:%
%:%429=348%:%
%:%430=349%:%
%:%431=349%:%
%:%444=362%:%
%:%445=363%:%
%:%446=363%:%
%:%447=363%:%
%:%448=364%:%
%:%449=364%:%
%:%462=377%:%
%:%463=378%:%
%:%464=379%:%
%:%465=379%:%
%:%466=380%:%
%:%467=380%:%
%:%482=395%:%
%:%483=396%:%
%:%484=396%:%
%:%485=397%:%
%:%486=397%:%
%:%487=398%:%
%:%488=398%:%
%:%489=399%:%
%:%490=399%:%
%:%491=400%:%
%:%492=400%:%
%:%493=401%:%
%:%494=401%:%
%:%495=402%:%
%:%496=403%:%
%:%497=403%:%
%:%498=404%:%
%:%499=404%:%
%:%500=405%:%
%:%501=405%:%
%:%502=406%:%
%:%503=406%:%
%:%504=407%:%
%:%505=407%:%
%:%512=414%:%
%:%513=415%:%
%:%514=415%:%
%:%515=415%:%
%:%516=416%:%
%:%517=417%:%
%:%518=417%:%
%:%532=431%:%
%:%533=432%:%
%:%534=432%:%
%:%535=433%:%
%:%536=434%:%
%:%537=434%:%
%:%538=435%:%
%:%539=435%:%
%:%540=436%:%
%:%541=436%:%
%:%542=437%:%
%:%543=437%:%
%:%544=438%:%
%:%545=438%:%
%:%546=439%:%
%:%547=439%:%
%:%548=440%:%
%:%549=440%:%
%:%550=441%:%
%:%551=441%:%
%:%552=442%:%
%:%553=442%:%
%:%554=443%:%
%:%555=443%:%
%:%556=443%:%
%:%557=444%:%
%:%558=444%:%
%:%559=444%:%
%:%560=445%:%
%:%561=445%:%
%:%562=445%:%
%:%563=446%:%
%:%564=446%:%
%:%565=447%:%
%:%566=447%:%
%:%567=448%:%
%:%568=448%:%
%:%569=449%:%
%:%570=449%:%
%:%571=450%:%
%:%572=450%:%
%:%573=451%:%
%:%574=452%:%
%:%575=452%:%
%:%576=453%:%
%:%577=453%:%
%:%578=454%:%
%:%579=454%:%
%:%580=455%:%
%:%581=456%:%
%:%582=456%:%
%:%583=456%:%
%:%584=457%:%
%:%585=457%:%
%:%586=458%:%
%:%587=459%:%
%:%588=459%:%
%:%589=460%:%
%:%590=460%:%
%:%591=461%:%
%:%592=462%:%
%:%593=462%:%
%:%594=463%:%
%:%595=463%:%
%:%596=464%:%
%:%597=464%:%
%:%598=465%:%
%:%599=465%:%
%:%600=466%:%
%:%601=466%:%
%:%602=467%:%
%:%603=467%:%
%:%604=468%:%
%:%605=468%:%
%:%606=469%:%
%:%607=470%:%
%:%608=470%:%
%:%609=471%:%
%:%610=472%:%
%:%611=473%:%
%:%612=474%:%
%:%613=474%:%
%:%614=475%:%
%:%615=475%:%
%:%616=476%:%
%:%617=477%:%
%:%618=478%:%
%:%619=479%:%
%:%620=480%:%
%:%621=481%:%
%:%622=481%:%
%:%623=481%:%
%:%624=482%:%
%:%625=482%:%
%:%628=485%:%
%:%629=485%:%
%:%630=486%:%
%:%631=486%:%
%:%634=489%:%
%:%635=490%:%
%:%636=490%:%
%:%637=491%:%
%:%638=491%:%
%:%645=498%:%
%:%646=499%:%
%:%647=499%:%
%:%648=500%:%
%:%649=500%:%
%:%655=506%:%
%:%656=507%:%
%:%657=507%:%
%:%658=508%:%
%:%659=508%:%
%:%660=509%:%
%:%661=509%:%
%:%662=510%:%
%:%663=510%:%
%:%667=514%:%
%:%668=515%:%
%:%669=515%:%
%:%670=515%:%
%:%671=516%:%
%:%672=516%:%
%:%676=520%:%
%:%677=521%:%
%:%678=521%:%
%:%679=522%:%
%:%680=522%:%
%:%681=523%:%
%:%682=523%:%
%:%683=524%:%
%:%684=524%:%
%:%685=524%:%
%:%686=525%:%
%:%687=525%:%
%:%688=526%:%
%:%689=526%:%
%:%690=526%:%
%:%691=527%:%
%:%692=527%:%
%:%693=528%:%
%:%694=528%:%
%:%700=534%:%
%:%701=535%:%
%:%702=535%:%
%:%703=535%:%
%:%704=536%:%
%:%705=536%:%
%:%706=536%:%
%:%707=537%:%
%:%708=537%:%
%:%709=538%:%
%:%710=538%:%
%:%711=539%:%
%:%721=541%:%
%:%722=542%:%
%:%726=544%:%
%:%727=545%:%
%:%729=547%:%
%:%730=547%:%
%:%731=548%:%
%:%732=549%:%
%:%733=550%:%
%:%740=551%:%
%:%741=551%:%
%:%742=552%:%
%:%743=552%:%
%:%744=552%:%
%:%745=553%:%
%:%746=554%:%
%:%747=554%:%
%:%748=555%:%
%:%749=555%:%
%:%750=556%:%
%:%751=556%:%
%:%752=556%:%
%:%753=557%:%
%:%754=557%:%
%:%755=557%:%
%:%756=558%:%
%:%757=558%:%
%:%758=559%:%
%:%759=559%:%
%:%760=559%:%
%:%761=560%:%
%:%762=560%:%
%:%763=560%:%
%:%764=560%:%
%:%765=561%:%
%:%771=561%:%
%:%774=562%:%
%:%775=563%:%
%:%776=563%:%
%:%777=564%:%
%:%778=565%:%
%:%779=566%:%
%:%780=567%:%
%:%787=568%:%
%:%788=568%:%
%:%789=569%:%
%:%790=569%:%
%:%791=570%:%
%:%792=570%:%
%:%793=570%:%
%:%794=571%:%
%:%795=571%:%
%:%796=572%:%
%:%797=572%:%
%:%798=572%:%
%:%799=573%:%
%:%800=573%:%
%:%801=574%:%
%:%802=574%:%
%:%803=574%:%
%:%804=575%:%
%:%805=575%:%
%:%806=576%:%
%:%807=576%:%
%:%808=576%:%
%:%809=577%:%
%:%810=577%:%
%:%811=578%:%
%:%812=578%:%
%:%813=578%:%
%:%814=579%:%
%:%815=579%:%
%:%816=580%:%
%:%817=580%:%
%:%818=580%:%
%:%819=581%:%
%:%820=582%:%
%:%821=582%:%
%:%822=583%:%
%:%823=584%:%
%:%824=585%:%
%:%825=586%:%
%:%826=587%:%
%:%827=588%:%
%:%828=588%:%
%:%829=589%:%
%:%830=589%:%
%:%831=590%:%
%:%832=591%:%
%:%833=591%:%
%:%834=591%:%
%:%835=592%:%
%:%836=593%:%
%:%837=593%:%
%:%838=594%:%
%:%839=594%:%
%:%840=595%:%
%:%841=595%:%
%:%842=596%:%
%:%843=597%:%
%:%844=597%:%
%:%845=598%:%
%:%846=599%:%
%:%847=599%:%
%:%848=599%:%
%:%849=600%:%
%:%850=601%:%
%:%851=601%:%
%:%852=602%:%
%:%853=602%:%
%:%854=603%:%
%:%855=604%:%
%:%856=604%:%
%:%857=604%:%
%:%858=605%:%
%:%859=606%:%
%:%860=606%:%
%:%861=607%:%
%:%862=607%:%
%:%863=608%:%
%:%864=609%:%
%:%865=609%:%
%:%866=609%:%
%:%867=610%:%
%:%868=611%:%
%:%869=611%:%
%:%870=612%:%
%:%871=612%:%
%:%872=613%:%
%:%873=614%:%
%:%874=614%:%
%:%875=614%:%
%:%876=615%:%
%:%877=616%:%
%:%878=616%:%
%:%879=617%:%
%:%880=617%:%
%:%881=618%:%
%:%882=619%:%
%:%883=619%:%
%:%884=619%:%
%:%885=620%:%
%:%886=621%:%
%:%887=621%:%
%:%888=622%:%
%:%889=622%:%
%:%890=623%:%
%:%891=623%:%
%:%892=624%:%
%:%893=624%:%
%:%894=625%:%
%:%895=626%:%
%:%896=626%:%
%:%897=626%:%
%:%898=627%:%
%:%899=627%:%
%:%900=628%:%
%:%901=628%:%
%:%902=629%:%
%:%903=630%:%
%:%904=631%:%
%:%905=631%:%
%:%906=632%:%
%:%907=632%:%
%:%908=633%:%
%:%909=634%:%
%:%910=634%:%
%:%911=635%:%
%:%912=636%:%
%:%913=636%:%
%:%914=637%:%
%:%915=637%:%
%:%916=638%:%
%:%917=639%:%
%:%918=639%:%
%:%919=640%:%
%:%920=640%:%
%:%921=641%:%
%:%922=642%:%

%
\begin{isabellebody}%
\setisabellecontext{HML{\isacharunderscore}{\kern0pt}list}%
%
\isadelimtheory
%
\endisadelimtheory
%
\isatagtheory
%
\endisatagtheory
{\isafoldtheory}%
%
\isadelimtheory
%
\endisadelimtheory
%
\isadelimdocument
%
\endisadelimdocument
%
\isatagdocument
%
\isamarkupsubsection{Hennessy--Milner logic%
}
\isamarkuptrue%
%
\endisatagdocument
{\isafolddocument}%
%
\isadelimdocument
%
\endisadelimdocument
%
\begin{isamarkuptext}%
Hennessy--Milner logic, first introduced by Matthew Hennessy and Robin Milner (citation), is a modal logic for expressing properties of systems described by LTS.
Intuitively, HML describes observations on an LTS and two processes are considered equivalent under HML when there exists no observation that distinguishes between them.
(citation) defined the modal-logical language as consisting of (finite) conjunctions, negations and a (modal) possibility operator:
$$\varphi ::= t\!t \mid \varphi_1 \;\wedge\; \varphi_2 \mid \neg\varphi \mid \langle\alpha\rangle\varphi$$
(where $\alpha$ ranges over the set of actions. The paper also proves that this characterization of strong bisimilarity
corresponds to a relational definition that is effectively the same as in (...). Their result can be expressed as follows:
for image-finite LTSs, two processes are strongly bisimilar iff they satisfy the same set of HML formulas. We call this the modal characterisation of
strong bisimilarity. By allowing for conjunction of arbitrary width (infinitary HML), the modal characterization of strong bisimilarity can be proved for arbitrary LTS. This is done in (...)%
\end{isamarkuptext}\isamarkuptrue%
%
\begin{isamarkuptext}%
\textbf{Hennessy--Milner logic}
The syntax of Hennessy--Milner logic over a set $\Sigma$ of actions, (HML) - richtige font!!!!![$\Sigma$], is defined by the grammar:
\begin{align*}
    \varphi &::= \langle a \rangle \varphi && \text{with } a \in \Sigma \\
            &\quad | \, \bigwedge_{i \in I} \psi_i \\
    \psi &::= \neg \varphi \, | \, \varphi.
\end{align*}%
\end{isamarkuptext}\isamarkuptrue%
%
\begin{isamarkuptext}%
The data type \isa{{\isacharparenleft}{\kern0pt}{\isacharprime}{\kern0pt}a{\isacharcomma}{\kern0pt}\ {\isacharprime}{\kern0pt}i{\isacharparenright}{\kern0pt}hml} formalizes the definition of HML formulas above. It is parameterized by the type of actions \isa{{\isacharprime}{\kern0pt}a} for $\Sigma$
and an index type \isa{{\isacharprime}{\kern0pt}i}. We use an index sets of arbitrary type \isa{I\ {\isacharcolon}{\kern0pt}{\isacharcolon}{\kern0pt}\ {\isacharprime}{\kern0pt}i\ set} and a mapping \isa{F\ {\isacharcolon}{\kern0pt}{\isacharcolon}{\kern0pt}\ {\isacharprime}{\kern0pt}i\ {\isasymRightarrow}\ {\isacharparenleft}{\kern0pt}{\isacharprime}{\kern0pt}a{\isacharcomma}{\kern0pt}\ {\isacharprime}{\kern0pt}i{\isacharparenright}{\kern0pt}\ hml} to formalize
conjunctions so that each element of \isa{I} is mapped to a formula%
\footnote{Note that the formalization via an arbitrary set (...) does not yield a valid type, since \isa{set} is not a bounded natural functor.}%.%
\end{isamarkuptext}\isamarkuptrue%
\isacommand{datatype}\isamarkupfalse%
\ {\isacharparenleft}{\kern0pt}{\isacharprime}{\kern0pt}a{\isacharcomma}{\kern0pt}\ {\isacharprime}{\kern0pt}i{\isacharparenright}{\kern0pt}hml\ {\isacharequal}{\kern0pt}\isanewline
TT\ {\isacharbar}{\kern0pt}\isanewline
hml{\isacharunderscore}{\kern0pt}pos\ {\isacartoucheopen}{\isacharprime}{\kern0pt}a{\isacartoucheclose}\ {\isacartoucheopen}{\isacharparenleft}{\kern0pt}{\isacharprime}{\kern0pt}a{\isacharcomma}{\kern0pt}\ {\isacharprime}{\kern0pt}i{\isacharparenright}{\kern0pt}hml{\isacartoucheclose}\ {\isacharbar}{\kern0pt}\isanewline
hml{\isacharunderscore}{\kern0pt}conj\ {\isachardoublequoteopen}{\isacharprime}{\kern0pt}i\ set{\isachardoublequoteclose}\ {\isachardoublequoteopen}{\isacharprime}{\kern0pt}i\ set{\isachardoublequoteclose}\ {\isachardoublequoteopen}{\isacharprime}{\kern0pt}i\ {\isasymRightarrow}\ {\isacharparenleft}{\kern0pt}{\isacharprime}{\kern0pt}a{\isacharcomma}{\kern0pt}\ {\isacharprime}{\kern0pt}i{\isacharparenright}{\kern0pt}\ hml{\isachardoublequoteclose}%
\begin{isamarkuptext}%
Note that in canonical definitions of HML \isa{TT} is not usually part of the syntax,
but is instead synonymous to \isa{{\isasymAnd}{\isacharbraceleft}{\kern0pt}{\isacharbraceright}{\kern0pt}}.
We include \isa{TT} in the definition to enable Isabelle to infer that the type \isa{hml} is not empty..
This formalization allows for conjunctions of arbitrary - even of infinite - width and has been
taken from \cite{Pohlmann2021ReducingReactive} (Appendix B).%
\end{isamarkuptext}\isamarkuptrue%
\isacommand{inductive}\isamarkupfalse%
\ TT{\isacharunderscore}{\kern0pt}like\ {\isacharcolon}{\kern0pt}{\isacharcolon}{\kern0pt}\ {\isachardoublequoteopen}{\isacharparenleft}{\kern0pt}{\isacharprime}{\kern0pt}a{\isacharcomma}{\kern0pt}\ {\isacharprime}{\kern0pt}i{\isacharparenright}{\kern0pt}\ hml\ {\isasymRightarrow}\ bool{\isachardoublequoteclose}\isanewline
\ \ \isakeyword{where}\isanewline
{\isachardoublequoteopen}TT{\isacharunderscore}{\kern0pt}like\ TT{\isachardoublequoteclose}\ {\isacharbar}{\kern0pt}\isanewline
{\isachardoublequoteopen}TT{\isacharunderscore}{\kern0pt}like\ {\isacharparenleft}{\kern0pt}hml{\isacharunderscore}{\kern0pt}conj\ I\ J\ {\isasymPhi}{\isacharparenright}{\kern0pt}{\isachardoublequoteclose}\ \isakeyword{if}\ {\isachardoublequoteopen}{\isacharparenleft}{\kern0pt}{\isasymPhi}\ {\isacharbackquote}{\kern0pt}I{\isacharparenright}{\kern0pt}\ {\isacharequal}{\kern0pt}\ {\isacharbraceleft}{\kern0pt}{\isacharbraceright}{\kern0pt}{\isachardoublequoteclose}\ {\isachardoublequoteopen}{\isacharparenleft}{\kern0pt}{\isasymPhi}\ {\isacharbackquote}{\kern0pt}\ J{\isacharparenright}{\kern0pt}\ {\isacharequal}{\kern0pt}\ {\isacharbraceleft}{\kern0pt}{\isacharbraceright}{\kern0pt}{\isachardoublequoteclose}\isanewline
\isanewline
\isacommand{inductive}\isamarkupfalse%
\ nested{\isacharunderscore}{\kern0pt}empty{\isacharunderscore}{\kern0pt}pos{\isacharunderscore}{\kern0pt}conj\ {\isacharcolon}{\kern0pt}{\isacharcolon}{\kern0pt}\ {\isachardoublequoteopen}{\isacharparenleft}{\kern0pt}{\isacharprime}{\kern0pt}a{\isacharcomma}{\kern0pt}\ {\isacharprime}{\kern0pt}i{\isacharparenright}{\kern0pt}\ hml\ {\isasymRightarrow}\ bool{\isachardoublequoteclose}\isanewline
\ \ \isakeyword{where}\isanewline
{\isachardoublequoteopen}nested{\isacharunderscore}{\kern0pt}empty{\isacharunderscore}{\kern0pt}pos{\isacharunderscore}{\kern0pt}conj\ TT{\isachardoublequoteclose}\ {\isacharbar}{\kern0pt}\isanewline
{\isachardoublequoteopen}nested{\isacharunderscore}{\kern0pt}empty{\isacharunderscore}{\kern0pt}pos{\isacharunderscore}{\kern0pt}conj\ {\isacharparenleft}{\kern0pt}hml{\isacharunderscore}{\kern0pt}conj\ I\ J\ {\isasymPhi}{\isacharparenright}{\kern0pt}{\isachardoublequoteclose}\ \isanewline
\isakeyword{if}\ {\isachardoublequoteopen}{\isasymforall}x\ {\isasymin}\ {\isacharparenleft}{\kern0pt}{\isasymPhi}\ {\isacharbackquote}{\kern0pt}I{\isacharparenright}{\kern0pt}{\isachardot}{\kern0pt}\ nested{\isacharunderscore}{\kern0pt}empty{\isacharunderscore}{\kern0pt}pos{\isacharunderscore}{\kern0pt}conj\ x{\isachardoublequoteclose}\ {\isachardoublequoteopen}{\isacharparenleft}{\kern0pt}{\isasymPhi}\ {\isacharbackquote}{\kern0pt}\ J{\isacharparenright}{\kern0pt}\ {\isacharequal}{\kern0pt}\ {\isacharbraceleft}{\kern0pt}{\isacharbraceright}{\kern0pt}{\isachardoublequoteclose}\isanewline
\isanewline
\isacommand{inductive}\isamarkupfalse%
\ nested{\isacharunderscore}{\kern0pt}empty{\isacharunderscore}{\kern0pt}conj\ {\isacharcolon}{\kern0pt}{\isacharcolon}{\kern0pt}\ {\isachardoublequoteopen}{\isacharparenleft}{\kern0pt}{\isacharprime}{\kern0pt}a{\isacharcomma}{\kern0pt}\ {\isacharprime}{\kern0pt}i{\isacharparenright}{\kern0pt}\ hml\ {\isasymRightarrow}\ bool{\isachardoublequoteclose}\isanewline
\ \ \isakeyword{where}\isanewline
{\isachardoublequoteopen}nested{\isacharunderscore}{\kern0pt}empty{\isacharunderscore}{\kern0pt}conj\ TT{\isachardoublequoteclose}\ {\isacharbar}{\kern0pt}\isanewline
{\isachardoublequoteopen}nested{\isacharunderscore}{\kern0pt}empty{\isacharunderscore}{\kern0pt}conj\ {\isacharparenleft}{\kern0pt}hml{\isacharunderscore}{\kern0pt}conj\ I\ J\ {\isasymPhi}{\isacharparenright}{\kern0pt}{\isachardoublequoteclose}\isanewline
\isakeyword{if}\ {\isachardoublequoteopen}{\isasymforall}x\ {\isasymin}\ {\isacharparenleft}{\kern0pt}{\isasymPhi}\ {\isacharbackquote}{\kern0pt}I{\isacharparenright}{\kern0pt}{\isachardot}{\kern0pt}\ nested{\isacharunderscore}{\kern0pt}empty{\isacharunderscore}{\kern0pt}conj\ x{\isachardoublequoteclose}\ {\isachardoublequoteopen}{\isasymforall}x\ {\isasymin}\ {\isacharparenleft}{\kern0pt}{\isasymPhi}\ {\isacharbackquote}{\kern0pt}J{\isacharparenright}{\kern0pt}{\isachardot}{\kern0pt}\ nested{\isacharunderscore}{\kern0pt}empty{\isacharunderscore}{\kern0pt}pos{\isacharunderscore}{\kern0pt}conj\ x{\isachardoublequoteclose}\isanewline
\isanewline
\isacommand{inductive}\isamarkupfalse%
\ stacked{\isacharunderscore}{\kern0pt}pos{\isacharunderscore}{\kern0pt}conj{\isacharunderscore}{\kern0pt}pos\ {\isacharcolon}{\kern0pt}{\isacharcolon}{\kern0pt}\ {\isachardoublequoteopen}{\isacharparenleft}{\kern0pt}{\isacharprime}{\kern0pt}a{\isacharcomma}{\kern0pt}\ {\isacharprime}{\kern0pt}i{\isacharparenright}{\kern0pt}\ hml\ {\isasymRightarrow}\ bool{\isachardoublequoteclose}\isanewline
\ \ \isakeyword{where}\isanewline
{\isachardoublequoteopen}stacked{\isacharunderscore}{\kern0pt}pos{\isacharunderscore}{\kern0pt}conj{\isacharunderscore}{\kern0pt}pos\ TT{\isachardoublequoteclose}\ {\isacharbar}{\kern0pt}\isanewline
{\isachardoublequoteopen}stacked{\isacharunderscore}{\kern0pt}pos{\isacharunderscore}{\kern0pt}conj{\isacharunderscore}{\kern0pt}pos\ {\isacharparenleft}{\kern0pt}hml{\isacharunderscore}{\kern0pt}pos\ {\isacharunderscore}{\kern0pt}\ {\isasympsi}{\isacharparenright}{\kern0pt}{\isachardoublequoteclose}\ \isakeyword{if}\ {\isachardoublequoteopen}nested{\isacharunderscore}{\kern0pt}empty{\isacharunderscore}{\kern0pt}pos{\isacharunderscore}{\kern0pt}conj\ {\isasympsi}{\isachardoublequoteclose}\ {\isacharbar}{\kern0pt}\isanewline
{\isachardoublequoteopen}stacked{\isacharunderscore}{\kern0pt}pos{\isacharunderscore}{\kern0pt}conj{\isacharunderscore}{\kern0pt}pos\ {\isacharparenleft}{\kern0pt}hml{\isacharunderscore}{\kern0pt}conj\ I\ J\ {\isasymPhi}{\isacharparenright}{\kern0pt}{\isachardoublequoteclose}\isanewline
\isakeyword{if}\ {\isachardoublequoteopen}{\isacharparenleft}{\kern0pt}{\isacharparenleft}{\kern0pt}{\isasymexists}{\isasymphi}\ {\isasymin}\ {\isacharparenleft}{\kern0pt}{\isasymPhi}\ {\isacharbackquote}{\kern0pt}\ I{\isacharparenright}{\kern0pt}{\isachardot}{\kern0pt}\ {\isacharparenleft}{\kern0pt}{\isacharparenleft}{\kern0pt}stacked{\isacharunderscore}{\kern0pt}pos{\isacharunderscore}{\kern0pt}conj{\isacharunderscore}{\kern0pt}pos\ {\isasymphi}{\isacharparenright}{\kern0pt}\ {\isasymand}\ \isanewline
\ \ \ \ \ \ \ \ \ \ \ \ \ \ \ \ \ \ \ \ \ {\isacharparenleft}{\kern0pt}{\isasymforall}{\isasympsi}\ {\isasymin}\ {\isacharparenleft}{\kern0pt}{\isasymPhi}\ {\isacharbackquote}{\kern0pt}\ I{\isacharparenright}{\kern0pt}{\isachardot}{\kern0pt}\ {\isasympsi}\ {\isasymnoteq}\ {\isasymphi}\ {\isasymlongrightarrow}\ nested{\isacharunderscore}{\kern0pt}empty{\isacharunderscore}{\kern0pt}pos{\isacharunderscore}{\kern0pt}conj\ {\isasympsi}{\isacharparenright}{\kern0pt}{\isacharparenright}{\kern0pt}{\isacharparenright}{\kern0pt}\ {\isasymor}\isanewline
\ \ \ {\isacharparenleft}{\kern0pt}{\isasymforall}{\isasympsi}\ {\isasymin}\ {\isacharparenleft}{\kern0pt}{\isasymPhi}\ {\isacharbackquote}{\kern0pt}\ I{\isacharparenright}{\kern0pt}{\isachardot}{\kern0pt}\ nested{\isacharunderscore}{\kern0pt}empty{\isacharunderscore}{\kern0pt}pos{\isacharunderscore}{\kern0pt}conj\ {\isasympsi}{\isacharparenright}{\kern0pt}{\isacharparenright}{\kern0pt}{\isachardoublequoteclose}\isanewline
{\isachardoublequoteopen}{\isacharparenleft}{\kern0pt}{\isasymPhi}\ {\isacharbackquote}{\kern0pt}\ J{\isacharparenright}{\kern0pt}\ {\isacharequal}{\kern0pt}\ {\isacharbraceleft}{\kern0pt}{\isacharbraceright}{\kern0pt}{\isachardoublequoteclose}\isanewline
\isanewline
\isacommand{inductive}\isamarkupfalse%
\ stacked{\isacharunderscore}{\kern0pt}pos{\isacharunderscore}{\kern0pt}conj\ {\isacharcolon}{\kern0pt}{\isacharcolon}{\kern0pt}\ {\isachardoublequoteopen}{\isacharparenleft}{\kern0pt}{\isacharprime}{\kern0pt}a{\isacharcomma}{\kern0pt}\ {\isacharprime}{\kern0pt}i{\isacharparenright}{\kern0pt}\ hml\ {\isasymRightarrow}\ bool{\isachardoublequoteclose}\isanewline
\ \ \isakeyword{where}\ \isanewline
{\isachardoublequoteopen}stacked{\isacharunderscore}{\kern0pt}pos{\isacharunderscore}{\kern0pt}conj\ TT{\isachardoublequoteclose}\ {\isacharbar}{\kern0pt}\isanewline
{\isachardoublequoteopen}stacked{\isacharunderscore}{\kern0pt}pos{\isacharunderscore}{\kern0pt}conj\ {\isacharparenleft}{\kern0pt}hml{\isacharunderscore}{\kern0pt}pos\ {\isacharunderscore}{\kern0pt}\ {\isasympsi}{\isacharparenright}{\kern0pt}{\isachardoublequoteclose}\ \isakeyword{if}\ {\isachardoublequoteopen}nested{\isacharunderscore}{\kern0pt}empty{\isacharunderscore}{\kern0pt}pos{\isacharunderscore}{\kern0pt}conj\ {\isasympsi}{\isachardoublequoteclose}\ {\isacharbar}{\kern0pt}\isanewline
{\isachardoublequoteopen}stacked{\isacharunderscore}{\kern0pt}pos{\isacharunderscore}{\kern0pt}conj\ {\isacharparenleft}{\kern0pt}hml{\isacharunderscore}{\kern0pt}conj\ I\ J\ {\isasymPhi}{\isacharparenright}{\kern0pt}{\isachardoublequoteclose}\isanewline
\isakeyword{if}\ {\isachardoublequoteopen}{\isasymforall}{\isasymphi}\ {\isasymin}\ {\isacharparenleft}{\kern0pt}{\isasymPhi}\ {\isacharbackquote}{\kern0pt}\ I{\isacharparenright}{\kern0pt}{\isachardot}{\kern0pt}\ {\isacharparenleft}{\kern0pt}{\isacharparenleft}{\kern0pt}stacked{\isacharunderscore}{\kern0pt}pos{\isacharunderscore}{\kern0pt}conj\ {\isasymphi}{\isacharparenright}{\kern0pt}\ {\isasymor}\ nested{\isacharunderscore}{\kern0pt}empty{\isacharunderscore}{\kern0pt}conj\ {\isasymphi}{\isacharparenright}{\kern0pt}{\isachardoublequoteclose}\isanewline
{\isachardoublequoteopen}{\isacharparenleft}{\kern0pt}{\isasymforall}{\isasympsi}\ {\isasymin}\ {\isacharparenleft}{\kern0pt}{\isasymPhi}\ {\isacharbackquote}{\kern0pt}\ J{\isacharparenright}{\kern0pt}{\isachardot}{\kern0pt}\ nested{\isacharunderscore}{\kern0pt}empty{\isacharunderscore}{\kern0pt}conj\ {\isasympsi}{\isacharparenright}{\kern0pt}{\isachardoublequoteclose}\isanewline
\isanewline
\isacommand{inductive}\isamarkupfalse%
\ stacked{\isacharunderscore}{\kern0pt}pos{\isacharunderscore}{\kern0pt}conj{\isacharunderscore}{\kern0pt}J{\isacharunderscore}{\kern0pt}empty\ {\isacharcolon}{\kern0pt}{\isacharcolon}{\kern0pt}\ {\isachardoublequoteopen}{\isacharparenleft}{\kern0pt}{\isacharprime}{\kern0pt}a{\isacharcomma}{\kern0pt}\ {\isacharprime}{\kern0pt}i{\isacharparenright}{\kern0pt}\ hml\ {\isasymRightarrow}\ bool{\isachardoublequoteclose}\isanewline
\ \ \isakeyword{where}\isanewline
{\isachardoublequoteopen}stacked{\isacharunderscore}{\kern0pt}pos{\isacharunderscore}{\kern0pt}conj{\isacharunderscore}{\kern0pt}J{\isacharunderscore}{\kern0pt}empty\ TT{\isachardoublequoteclose}\ {\isacharbar}{\kern0pt}\isanewline
{\isachardoublequoteopen}stacked{\isacharunderscore}{\kern0pt}pos{\isacharunderscore}{\kern0pt}conj{\isacharunderscore}{\kern0pt}J{\isacharunderscore}{\kern0pt}empty\ {\isacharparenleft}{\kern0pt}hml{\isacharunderscore}{\kern0pt}pos\ {\isacharunderscore}{\kern0pt}\ {\isasympsi}{\isacharparenright}{\kern0pt}{\isachardoublequoteclose}\ \isakeyword{if}\ {\isachardoublequoteopen}stacked{\isacharunderscore}{\kern0pt}pos{\isacharunderscore}{\kern0pt}conj{\isacharunderscore}{\kern0pt}J{\isacharunderscore}{\kern0pt}empty\ {\isasympsi}{\isachardoublequoteclose}\ {\isacharbar}{\kern0pt}\isanewline
{\isachardoublequoteopen}stacked{\isacharunderscore}{\kern0pt}pos{\isacharunderscore}{\kern0pt}conj{\isacharunderscore}{\kern0pt}J{\isacharunderscore}{\kern0pt}empty\ {\isacharparenleft}{\kern0pt}hml{\isacharunderscore}{\kern0pt}conj\ I\ J\ {\isasymPhi}{\isacharparenright}{\kern0pt}{\isachardoublequoteclose}\isanewline
\isakeyword{if}\ {\isachardoublequoteopen}{\isasymforall}{\isasymphi}\ {\isasymin}\ {\isacharparenleft}{\kern0pt}{\isasymPhi}\ {\isacharbackquote}{\kern0pt}\ I{\isacharparenright}{\kern0pt}{\isachardot}{\kern0pt}\ {\isacharparenleft}{\kern0pt}stacked{\isacharunderscore}{\kern0pt}pos{\isacharunderscore}{\kern0pt}conj{\isacharunderscore}{\kern0pt}J{\isacharunderscore}{\kern0pt}empty\ {\isasymphi}{\isacharparenright}{\kern0pt}{\isachardoublequoteclose}\ {\isachardoublequoteopen}{\isasymPhi}\ {\isacharbackquote}{\kern0pt}\ J\ {\isacharequal}{\kern0pt}\ {\isacharbraceleft}{\kern0pt}{\isacharbraceright}{\kern0pt}{\isachardoublequoteclose}\isanewline
\isanewline
\isacommand{inductive}\isamarkupfalse%
\ single{\isacharunderscore}{\kern0pt}pos{\isacharunderscore}{\kern0pt}pos\ {\isacharcolon}{\kern0pt}{\isacharcolon}{\kern0pt}\ {\isachardoublequoteopen}{\isacharparenleft}{\kern0pt}{\isacharprime}{\kern0pt}a{\isacharcomma}{\kern0pt}\ {\isacharprime}{\kern0pt}i{\isacharparenright}{\kern0pt}\ hml\ {\isasymRightarrow}\ bool{\isachardoublequoteclose}\isanewline
\ \ \isakeyword{where}\isanewline
{\isachardoublequoteopen}single{\isacharunderscore}{\kern0pt}pos{\isacharunderscore}{\kern0pt}pos\ TT{\isachardoublequoteclose}\ {\isacharbar}{\kern0pt}\isanewline
{\isachardoublequoteopen}single{\isacharunderscore}{\kern0pt}pos{\isacharunderscore}{\kern0pt}pos\ {\isacharparenleft}{\kern0pt}hml{\isacharunderscore}{\kern0pt}pos\ {\isacharunderscore}{\kern0pt}\ {\isasympsi}{\isacharparenright}{\kern0pt}{\isachardoublequoteclose}\ \isakeyword{if}\ {\isachardoublequoteopen}nested{\isacharunderscore}{\kern0pt}empty{\isacharunderscore}{\kern0pt}pos{\isacharunderscore}{\kern0pt}conj\ {\isasympsi}{\isachardoublequoteclose}\ {\isacharbar}{\kern0pt}\isanewline
{\isachardoublequoteopen}single{\isacharunderscore}{\kern0pt}pos{\isacharunderscore}{\kern0pt}pos\ {\isacharparenleft}{\kern0pt}hml{\isacharunderscore}{\kern0pt}conj\ I\ J\ {\isasymPhi}{\isacharparenright}{\kern0pt}{\isachardoublequoteclose}\ \isakeyword{if}\ \isanewline
{\isachardoublequoteopen}{\isacharparenleft}{\kern0pt}{\isasymforall}{\isasymphi}\ {\isasymin}\ {\isacharparenleft}{\kern0pt}{\isasymPhi}\ {\isacharbackquote}{\kern0pt}I{\isacharparenright}{\kern0pt}{\isachardot}{\kern0pt}\ {\isacharparenleft}{\kern0pt}single{\isacharunderscore}{\kern0pt}pos{\isacharunderscore}{\kern0pt}pos\ {\isasymphi}{\isacharparenright}{\kern0pt}{\isacharparenright}{\kern0pt}{\isachardoublequoteclose}\isanewline
{\isachardoublequoteopen}{\isacharparenleft}{\kern0pt}{\isasymPhi}\ {\isacharbackquote}{\kern0pt}\ J{\isacharparenright}{\kern0pt}\ {\isacharequal}{\kern0pt}\ {\isacharbraceleft}{\kern0pt}{\isacharbraceright}{\kern0pt}{\isachardoublequoteclose}\isanewline
\isanewline
\isacommand{inductive}\isamarkupfalse%
\ single{\isacharunderscore}{\kern0pt}pos\ {\isacharcolon}{\kern0pt}{\isacharcolon}{\kern0pt}\ {\isachardoublequoteopen}{\isacharparenleft}{\kern0pt}{\isacharprime}{\kern0pt}a{\isacharcomma}{\kern0pt}\ {\isacharprime}{\kern0pt}i{\isacharparenright}{\kern0pt}\ hml\ {\isasymRightarrow}\ bool{\isachardoublequoteclose}\isanewline
\ \ \isakeyword{where}\isanewline
{\isachardoublequoteopen}single{\isacharunderscore}{\kern0pt}pos\ TT{\isachardoublequoteclose}\ {\isacharbar}{\kern0pt}\isanewline
{\isachardoublequoteopen}single{\isacharunderscore}{\kern0pt}pos\ {\isacharparenleft}{\kern0pt}hml{\isacharunderscore}{\kern0pt}pos\ {\isacharunderscore}{\kern0pt}\ {\isasympsi}{\isacharparenright}{\kern0pt}{\isachardoublequoteclose}\ \isakeyword{if}\ {\isachardoublequoteopen}nested{\isacharunderscore}{\kern0pt}empty{\isacharunderscore}{\kern0pt}conj\ {\isasympsi}{\isachardoublequoteclose}\ {\isacharbar}{\kern0pt}\isanewline
{\isachardoublequoteopen}single{\isacharunderscore}{\kern0pt}pos\ {\isacharparenleft}{\kern0pt}hml{\isacharunderscore}{\kern0pt}conj\ I\ J\ {\isasymPhi}{\isacharparenright}{\kern0pt}{\isachardoublequoteclose}\isanewline
\isakeyword{if}\ {\isachardoublequoteopen}{\isasymforall}{\isasymphi}\ {\isasymin}\ {\isacharparenleft}{\kern0pt}{\isasymPhi}\ {\isacharbackquote}{\kern0pt}\ I{\isacharparenright}{\kern0pt}{\isachardot}{\kern0pt}\ {\isacharparenleft}{\kern0pt}single{\isacharunderscore}{\kern0pt}pos\ {\isasymphi}{\isacharparenright}{\kern0pt}{\isachardoublequoteclose}\isanewline
{\isachardoublequoteopen}{\isasymforall}{\isasymphi}\ {\isasymin}\ {\isacharparenleft}{\kern0pt}{\isasymPhi}\ {\isacharbackquote}{\kern0pt}\ J{\isacharparenright}{\kern0pt}{\isachardot}{\kern0pt}\ single{\isacharunderscore}{\kern0pt}pos{\isacharunderscore}{\kern0pt}pos\ {\isasymphi}{\isachardoublequoteclose}\isanewline
\isanewline
\isacommand{context}\isamarkupfalse%
\ lts\ \isakeyword{begin}\isanewline
\isanewline
\isacommand{primrec}\isamarkupfalse%
\ hml{\isacharunderscore}{\kern0pt}semantics\ {\isacharcolon}{\kern0pt}{\isacharcolon}{\kern0pt}\ {\isacartoucheopen}{\isacharprime}{\kern0pt}s\ {\isasymRightarrow}\ {\isacharparenleft}{\kern0pt}{\isacharprime}{\kern0pt}a{\isacharcomma}{\kern0pt}\ {\isacharprime}{\kern0pt}s{\isacharparenright}{\kern0pt}hml\ {\isasymRightarrow}\ bool{\isacartoucheclose}\isanewline
{\isacharparenleft}{\kern0pt}{\isacartoucheopen}{\isacharunderscore}{\kern0pt}\ {\isasymTurnstile}\ {\isacharunderscore}{\kern0pt}{\isacartoucheclose}\ {\isacharbrackleft}{\kern0pt}{\isadigit{5}}{\isadigit{0}}{\isacharcomma}{\kern0pt}\ {\isadigit{5}}{\isadigit{0}}{\isacharbrackright}{\kern0pt}\ {\isadigit{5}}{\isadigit{0}}{\isacharparenright}{\kern0pt}\isanewline
\isakeyword{where}\isanewline
hml{\isacharunderscore}{\kern0pt}sem{\isacharunderscore}{\kern0pt}tt{\isacharcolon}{\kern0pt}\ {\isacartoucheopen}{\isacharparenleft}{\kern0pt}{\isacharunderscore}{\kern0pt}\ {\isasymTurnstile}\ TT{\isacharparenright}{\kern0pt}\ {\isacharequal}{\kern0pt}\ True{\isacartoucheclose}\ {\isacharbar}{\kern0pt}\isanewline
hml{\isacharunderscore}{\kern0pt}sem{\isacharunderscore}{\kern0pt}pos{\isacharcolon}{\kern0pt}\ {\isacartoucheopen}{\isacharparenleft}{\kern0pt}p\ {\isasymTurnstile}\ {\isacharparenleft}{\kern0pt}hml{\isacharunderscore}{\kern0pt}pos\ {\isasymalpha}\ {\isasymphi}{\isacharparenright}{\kern0pt}{\isacharparenright}{\kern0pt}\ {\isacharequal}{\kern0pt}\ {\isacharparenleft}{\kern0pt}{\isasymexists}\ q{\isachardot}{\kern0pt}\ {\isacharparenleft}{\kern0pt}p\ {\isasymmapsto}{\isasymalpha}\ q{\isacharparenright}{\kern0pt}\ {\isasymand}\ q\ {\isasymTurnstile}\ {\isasymphi}{\isacharparenright}{\kern0pt}{\isacartoucheclose}\ {\isacharbar}{\kern0pt}\isanewline
hml{\isacharunderscore}{\kern0pt}sem{\isacharunderscore}{\kern0pt}conj{\isacharcolon}{\kern0pt}\ {\isacartoucheopen}{\isacharparenleft}{\kern0pt}p\ {\isasymTurnstile}\ {\isacharparenleft}{\kern0pt}hml{\isacharunderscore}{\kern0pt}conj\ I\ J\ {\isasympsi}s{\isacharparenright}{\kern0pt}{\isacharparenright}{\kern0pt}\ {\isacharequal}{\kern0pt}\ {\isacharparenleft}{\kern0pt}{\isacharparenleft}{\kern0pt}{\isasymforall}i\ {\isasymin}\ I{\isachardot}{\kern0pt}\ p\ {\isasymTurnstile}\ {\isacharparenleft}{\kern0pt}{\isasympsi}s\ i{\isacharparenright}{\kern0pt}{\isacharparenright}{\kern0pt}\ {\isasymand}\ {\isacharparenleft}{\kern0pt}{\isasymforall}j\ {\isasymin}\ J{\isachardot}{\kern0pt}\ {\isasymnot}{\isacharparenleft}{\kern0pt}p\ {\isasymTurnstile}\ {\isacharparenleft}{\kern0pt}{\isasympsi}s\ j{\isacharparenright}{\kern0pt}{\isacharparenright}{\kern0pt}{\isacharparenright}{\kern0pt}{\isacharparenright}{\kern0pt}{\isacartoucheclose}\isanewline
\isanewline
\isacommand{lemma}\isamarkupfalse%
\ index{\isacharunderscore}{\kern0pt}sets{\isacharunderscore}{\kern0pt}conj{\isacharunderscore}{\kern0pt}disjunct{\isacharcolon}{\kern0pt}\isanewline
\ \ \isakeyword{assumes}\ {\isachardoublequoteopen}I\ {\isasyminter}\ J\ {\isasymnoteq}\ {\isacharbraceleft}{\kern0pt}{\isacharbraceright}{\kern0pt}{\isachardoublequoteclose}\isanewline
\ \ \isakeyword{shows}\ {\isachardoublequoteopen}{\isasymforall}s{\isachardot}{\kern0pt}\ {\isasymnot}\ {\isacharparenleft}{\kern0pt}s\ {\isasymTurnstile}\ {\isacharparenleft}{\kern0pt}hml{\isacharunderscore}{\kern0pt}conj\ I\ J\ {\isasymPhi}{\isacharparenright}{\kern0pt}{\isacharparenright}{\kern0pt}{\isachardoublequoteclose}\isanewline
%
\isadelimproof
%
\endisadelimproof
%
\isatagproof
\isacommand{proof}\isamarkupfalse%
{\isacharparenleft}{\kern0pt}safe{\isacharparenright}{\kern0pt}\isanewline
\ \ \isacommand{fix}\isamarkupfalse%
\ s\isanewline
\ \ \isacommand{assume}\isamarkupfalse%
\ {\isachardoublequoteopen}s\ {\isasymTurnstile}\ hml{\isacharunderscore}{\kern0pt}conj\ I\ J\ {\isasymPhi}{\isachardoublequoteclose}\isanewline
\ \ \isacommand{from}\isamarkupfalse%
\ assms\ \isacommand{obtain}\isamarkupfalse%
\ i\ \isakeyword{where}\ {\isachardoublequoteopen}i\ {\isasymin}\ I\ {\isasyminter}\ J{\isachardoublequoteclose}\ \isacommand{by}\isamarkupfalse%
\ blast\isanewline
\ \ \isacommand{with}\isamarkupfalse%
\ {\isacartoucheopen}s\ {\isasymTurnstile}\ hml{\isacharunderscore}{\kern0pt}conj\ I\ J\ {\isasymPhi}{\isacartoucheclose}\ \isacommand{have}\isamarkupfalse%
\ {\isachardoublequoteopen}{\isacharparenleft}{\kern0pt}{\isacharparenleft}{\kern0pt}s\ {\isasymTurnstile}\ {\isacharparenleft}{\kern0pt}{\isasymPhi}\ i{\isacharparenright}{\kern0pt}{\isacharparenright}{\kern0pt}\ {\isasymand}\ {\isacharparenleft}{\kern0pt}{\isasymnot}{\isacharparenleft}{\kern0pt}s\ {\isasymTurnstile}\ {\isacharparenleft}{\kern0pt}{\isasymPhi}\ i{\isacharparenright}{\kern0pt}{\isacharparenright}{\kern0pt}{\isacharparenright}{\kern0pt}{\isacharparenright}{\kern0pt}{\isachardoublequoteclose}\isanewline
\ \ \ \ \isacommand{by}\isamarkupfalse%
\ auto\isanewline
\ \ \isacommand{then}\isamarkupfalse%
\ \isacommand{show}\isamarkupfalse%
\ False\ \isacommand{by}\isamarkupfalse%
\ blast\isanewline
\isacommand{qed}\isamarkupfalse%
%
\endisatagproof
{\isafoldproof}%
%
\isadelimproof
\isanewline
%
\endisadelimproof
\isanewline
\isacommand{definition}\isamarkupfalse%
\ HML{\isacharunderscore}{\kern0pt}true\ \isakeyword{where}\isanewline
{\isachardoublequoteopen}HML{\isacharunderscore}{\kern0pt}true\ {\isasymphi}\ {\isasymequiv}\ {\isasymforall}s{\isachardot}{\kern0pt}\ s\ {\isasymTurnstile}\ {\isasymphi}{\isachardoublequoteclose}\isanewline
\isanewline
\isacommand{lemma}\isamarkupfalse%
\ \isanewline
\ \ \isakeyword{fixes}\ s{\isacharcolon}{\kern0pt}{\isacharcolon}{\kern0pt}{\isacharprime}{\kern0pt}s\isanewline
\ \ \isakeyword{assumes}\ {\isachardoublequoteopen}HML{\isacharunderscore}{\kern0pt}true\ {\isacharparenleft}{\kern0pt}hml{\isacharunderscore}{\kern0pt}conj\ I\ J\ {\isasymPhi}{\isacharparenright}{\kern0pt}{\isachardoublequoteclose}\isanewline
\ \ \isakeyword{shows}\ {\isachardoublequoteopen}{\isasymforall}{\isasymphi}\ {\isasymin}\ {\isasymPhi}\ {\isacharbackquote}{\kern0pt}\ I{\isachardot}{\kern0pt}\ HML{\isacharunderscore}{\kern0pt}true\ {\isasymphi}{\isachardoublequoteclose}\isanewline
%
\isadelimproof
\ \ %
\endisadelimproof
%
\isatagproof
\isacommand{using}\isamarkupfalse%
\ HML{\isacharunderscore}{\kern0pt}true{\isacharunderscore}{\kern0pt}def\ assms\ \isacommand{by}\isamarkupfalse%
\ auto%
\endisatagproof
{\isafoldproof}%
%
\isadelimproof
\isanewline
%
\endisadelimproof
\isanewline
\isacommand{lemma}\isamarkupfalse%
\ HML{\isacharunderscore}{\kern0pt}true{\isacharunderscore}{\kern0pt}TT{\isacharunderscore}{\kern0pt}like{\isacharcolon}{\kern0pt}\isanewline
\ \ \isakeyword{assumes}\ {\isachardoublequoteopen}TT{\isacharunderscore}{\kern0pt}like\ {\isasymphi}{\isachardoublequoteclose}\isanewline
\ \ \isakeyword{shows}\ {\isachardoublequoteopen}HML{\isacharunderscore}{\kern0pt}true\ {\isasymphi}{\isachardoublequoteclose}\isanewline
%
\isadelimproof
\ \ %
\endisadelimproof
%
\isatagproof
\isacommand{using}\isamarkupfalse%
\ assms\isanewline
\ \ \isacommand{unfolding}\isamarkupfalse%
\ HML{\isacharunderscore}{\kern0pt}true{\isacharunderscore}{\kern0pt}def\isanewline
\ \ \isacommand{apply}\isamarkupfalse%
\ {\isacharparenleft}{\kern0pt}induction\ {\isasymphi}\ rule{\isacharcolon}{\kern0pt}\ TT{\isacharunderscore}{\kern0pt}like{\isachardot}{\kern0pt}induct{\isacharparenright}{\kern0pt}\isanewline
\ \ \isacommand{by}\isamarkupfalse%
\ simp{\isacharplus}{\kern0pt}%
\endisatagproof
{\isafoldproof}%
%
\isadelimproof
\isanewline
%
\endisadelimproof
\isanewline
\isacommand{lemma}\isamarkupfalse%
\ HML{\isacharunderscore}{\kern0pt}true{\isacharunderscore}{\kern0pt}nested{\isacharunderscore}{\kern0pt}empty{\isacharunderscore}{\kern0pt}pos{\isacharunderscore}{\kern0pt}conj{\isacharcolon}{\kern0pt}\isanewline
\ \ \isakeyword{assumes}\ {\isachardoublequoteopen}nested{\isacharunderscore}{\kern0pt}empty{\isacharunderscore}{\kern0pt}pos{\isacharunderscore}{\kern0pt}conj\ {\isasymphi}{\isachardoublequoteclose}\isanewline
\ \ \isakeyword{shows}\ {\isachardoublequoteopen}HML{\isacharunderscore}{\kern0pt}true\ {\isasymphi}{\isachardoublequoteclose}\isanewline
%
\isadelimproof
\ \ %
\endisadelimproof
%
\isatagproof
\isacommand{using}\isamarkupfalse%
\ assms\isanewline
\ \ \isacommand{unfolding}\isamarkupfalse%
\ HML{\isacharunderscore}{\kern0pt}true{\isacharunderscore}{\kern0pt}def\isanewline
\ \ \isacommand{apply}\isamarkupfalse%
\ {\isacharparenleft}{\kern0pt}induction\ {\isasymphi}\ rule{\isacharcolon}{\kern0pt}\ nested{\isacharunderscore}{\kern0pt}empty{\isacharunderscore}{\kern0pt}pos{\isacharunderscore}{\kern0pt}conj{\isachardot}{\kern0pt}induct{\isacharparenright}{\kern0pt}\isanewline
\ \ \isacommand{by}\isamarkupfalse%
\ {\isacharparenleft}{\kern0pt}simp{\isacharcomma}{\kern0pt}\ force{\isacharparenright}{\kern0pt}%
\endisatagproof
{\isafoldproof}%
%
\isadelimproof
%
\endisadelimproof
%
\begin{isamarkuptext}%
Two states are HML equivalent if they satisfy the same formula.%
\end{isamarkuptext}\isamarkuptrue%
\isacommand{definition}\isamarkupfalse%
\ HML{\isacharunderscore}{\kern0pt}equivalent\ {\isacharcolon}{\kern0pt}{\isacharcolon}{\kern0pt}\ {\isacartoucheopen}{\isacharprime}{\kern0pt}s\ {\isasymRightarrow}\ {\isacharprime}{\kern0pt}s\ {\isasymRightarrow}\ bool{\isacartoucheclose}\ \isakeyword{where}\isanewline
\ \ {\isacartoucheopen}HML{\isacharunderscore}{\kern0pt}equivalent\ p\ q\ {\isasymequiv}\ {\isacharparenleft}{\kern0pt}{\isasymforall}\ {\isasymphi}{\isacharcolon}{\kern0pt}{\isacharcolon}{\kern0pt}{\isacharparenleft}{\kern0pt}{\isacharprime}{\kern0pt}a{\isacharcomma}{\kern0pt}\ {\isacharprime}{\kern0pt}s{\isacharparenright}{\kern0pt}\ hml{\isachardot}{\kern0pt}\ {\isacharparenleft}{\kern0pt}p\ {\isasymTurnstile}\ {\isasymphi}{\isacharparenright}{\kern0pt}\ {\isasymlongleftrightarrow}\ {\isacharparenleft}{\kern0pt}q\ {\isasymTurnstile}\ {\isasymphi}{\isacharparenright}{\kern0pt}{\isacharparenright}{\kern0pt}{\isacartoucheclose}%
\begin{isamarkuptext}%
An HML formula \isa{{\isasymphi}l} implies another (\isa{{\isasymphi}r}) if the fact that some process \isa{p} satisfies \isa{{\isasymphi}l}
implies that \isa{p} must also satisfy \isa{{\isasymphi}r}, no matter the process \isa{p}.%
\end{isamarkuptext}\isamarkuptrue%
\isacommand{definition}\isamarkupfalse%
\ hml{\isacharunderscore}{\kern0pt}impl\ {\isacharcolon}{\kern0pt}{\isacharcolon}{\kern0pt}\ {\isachardoublequoteopen}{\isacharparenleft}{\kern0pt}{\isacharprime}{\kern0pt}a{\isacharcomma}{\kern0pt}\ {\isacharprime}{\kern0pt}s{\isacharparenright}{\kern0pt}\ hml\ {\isasymRightarrow}\ {\isacharparenleft}{\kern0pt}{\isacharprime}{\kern0pt}a{\isacharcomma}{\kern0pt}\ {\isacharprime}{\kern0pt}s{\isacharparenright}{\kern0pt}\ hml\ {\isasymRightarrow}\ bool{\isachardoublequoteclose}\ {\isacharparenleft}{\kern0pt}\isakeyword{infix}\ {\isachardoublequoteopen}{\isasymRrightarrow}{\isachardoublequoteclose}\ {\isadigit{6}}{\isadigit{0}}{\isacharparenright}{\kern0pt}\ \ \isakeyword{where}\isanewline
\ \ {\isachardoublequoteopen}{\isasymphi}l\ {\isasymRrightarrow}\ {\isasymphi}r\ {\isasymequiv}\ {\isacharparenleft}{\kern0pt}{\isasymforall}p{\isachardot}{\kern0pt}\ {\isacharparenleft}{\kern0pt}p\ {\isasymTurnstile}\ {\isasymphi}l{\isacharparenright}{\kern0pt}\ {\isasymlongrightarrow}\ {\isacharparenleft}{\kern0pt}p\ {\isasymTurnstile}\ {\isasymphi}r{\isacharparenright}{\kern0pt}{\isacharparenright}{\kern0pt}{\isachardoublequoteclose}\isanewline
\isanewline
\isacommand{lemma}\isamarkupfalse%
\ hml{\isacharunderscore}{\kern0pt}impl{\isacharunderscore}{\kern0pt}iffI{\isacharcolon}{\kern0pt}\ {\isachardoublequoteopen}{\isasymphi}l\ {\isasymRrightarrow}\ {\isasymphi}r\ {\isacharequal}{\kern0pt}\ {\isacharparenleft}{\kern0pt}{\isasymforall}p{\isachardot}{\kern0pt}\ {\isacharparenleft}{\kern0pt}p\ {\isasymTurnstile}\ {\isasymphi}l{\isacharparenright}{\kern0pt}\ {\isasymlongrightarrow}\ {\isacharparenleft}{\kern0pt}p\ {\isasymTurnstile}\ {\isasymphi}r{\isacharparenright}{\kern0pt}{\isacharparenright}{\kern0pt}{\isachardoublequoteclose}\isanewline
%
\isadelimproof
\ \ %
\endisadelimproof
%
\isatagproof
\isacommand{using}\isamarkupfalse%
\ hml{\isacharunderscore}{\kern0pt}impl{\isacharunderscore}{\kern0pt}def\ \isacommand{by}\isamarkupfalse%
\ force%
\endisatagproof
{\isafoldproof}%
%
\isadelimproof
%
\endisadelimproof
%
\isadelimdocument
%
\endisadelimdocument
%
\isatagdocument
%
\isamarkupsubsection{Equivalence%
}
\isamarkuptrue%
%
\endisatagdocument
{\isafolddocument}%
%
\isadelimdocument
%
\endisadelimdocument
%
\begin{isamarkuptext}%
A HML formula \isa{{\isasymphi}l} is said to be equivalent to some other HML formula \isa{{\isasymphi}r} (written \isa{{\isasymphi}l\ {\isasymLleftarrow}{\isasymRrightarrow}\ {\isasymphi}r})
iff process \isa{p} satisfies \isa{{\isasymphi}l} iff it also satisfies \isa{{\isasymphi}r}, no matter the process \isa{p}.

We have chosen to define this equivalence by appealing to HML formula implication (c.f. pre-order).%
\end{isamarkuptext}\isamarkuptrue%
\isacommand{definition}\isamarkupfalse%
\ hml{\isacharunderscore}{\kern0pt}formula{\isacharunderscore}{\kern0pt}eq\ {\isacharcolon}{\kern0pt}{\isacharcolon}{\kern0pt}\ {\isachardoublequoteopen}{\isacharparenleft}{\kern0pt}{\isacharprime}{\kern0pt}a{\isacharcomma}{\kern0pt}\ {\isacharprime}{\kern0pt}s{\isacharparenright}{\kern0pt}\ hml\ {\isasymRightarrow}\ {\isacharparenleft}{\kern0pt}{\isacharprime}{\kern0pt}a{\isacharcomma}{\kern0pt}\ {\isacharprime}{\kern0pt}s{\isacharparenright}{\kern0pt}\ hml\ {\isasymRightarrow}\ bool{\isachardoublequoteclose}\ {\isacharparenleft}{\kern0pt}\isakeyword{infix}\ {\isachardoublequoteopen}{\isasymLleftarrow}{\isasymRrightarrow}{\isachardoublequoteclose}\ {\isadigit{6}}{\isadigit{0}}{\isacharparenright}{\kern0pt}\ \ \isakeyword{where}\isanewline
\ \ {\isachardoublequoteopen}{\isasymphi}l\ {\isasymLleftarrow}{\isasymRrightarrow}\ {\isasymphi}r\ {\isasymequiv}\ {\isasymphi}l\ {\isasymRrightarrow}\ {\isasymphi}r\ {\isasymand}\ {\isasymphi}r\ {\isasymRrightarrow}\ {\isasymphi}l{\isachardoublequoteclose}%
\begin{isamarkuptext}%
\isa{{\isasymLleftarrow}{\isasymRrightarrow}} is truly an equivalence relation.%
\end{isamarkuptext}\isamarkuptrue%
\isacommand{lemma}\isamarkupfalse%
\ hml{\isacharunderscore}{\kern0pt}eq{\isacharunderscore}{\kern0pt}equiv{\isacharcolon}{\kern0pt}\ {\isachardoublequoteopen}equivp\ {\isacharparenleft}{\kern0pt}{\isasymLleftarrow}{\isasymRrightarrow}{\isacharparenright}{\kern0pt}{\isachardoublequoteclose}\isanewline
%
\isadelimproof
\ \ %
\endisadelimproof
%
\isatagproof
\isacommand{by}\isamarkupfalse%
\ {\isacharparenleft}{\kern0pt}smt\ {\isacharparenleft}{\kern0pt}verit{\isacharcomma}{\kern0pt}\ del{\isacharunderscore}{\kern0pt}insts{\isacharparenright}{\kern0pt}\ equivpI\ hml{\isacharunderscore}{\kern0pt}formula{\isacharunderscore}{\kern0pt}eq{\isacharunderscore}{\kern0pt}def\ hml{\isacharunderscore}{\kern0pt}impl{\isacharunderscore}{\kern0pt}def\ reflpI\ sympI\ transpI{\isacharparenright}{\kern0pt}%
\endisatagproof
{\isafoldproof}%
%
\isadelimproof
\isanewline
%
\endisadelimproof
\isanewline
\isacommand{lemma}\isamarkupfalse%
\ equiv{\isacharunderscore}{\kern0pt}der{\isacharcolon}{\kern0pt}\isanewline
\ \ \isakeyword{assumes}\ {\isachardoublequoteopen}HML{\isacharunderscore}{\kern0pt}equivalent\ p\ q{\isachardoublequoteclose}\ {\isachardoublequoteopen}{\isasymexists}p{\isacharprime}{\kern0pt}{\isachardot}{\kern0pt}\ p\ {\isasymmapsto}{\isasymalpha}\ p{\isacharprime}{\kern0pt}{\isachardoublequoteclose}\isanewline
\ \ \isakeyword{shows}\ {\isachardoublequoteopen}{\isasymexists}p{\isacharprime}{\kern0pt}\ q{\isacharprime}{\kern0pt}{\isachardot}{\kern0pt}\ {\isacharparenleft}{\kern0pt}HML{\isacharunderscore}{\kern0pt}equivalent\ p{\isacharprime}{\kern0pt}\ q{\isacharprime}{\kern0pt}{\isacharparenright}{\kern0pt}\ {\isasymand}\ q\ {\isasymmapsto}{\isasymalpha}\ q{\isacharprime}{\kern0pt}{\isachardoublequoteclose}\isanewline
%
\isadelimproof
\ \ %
\endisadelimproof
%
\isatagproof
\isacommand{using}\isamarkupfalse%
\ assms\ hml{\isacharunderscore}{\kern0pt}semantics{\isachardot}{\kern0pt}simps\isanewline
\ \ \isacommand{unfolding}\isamarkupfalse%
\ HML{\isacharunderscore}{\kern0pt}equivalent{\isacharunderscore}{\kern0pt}def\ \isanewline
\ \ \isacommand{by}\isamarkupfalse%
\ metis%
\endisatagproof
{\isafoldproof}%
%
\isadelimproof
\isanewline
%
\endisadelimproof
\isanewline
\isanewline
\isacommand{lemma}\isamarkupfalse%
\ equiv{\isacharunderscore}{\kern0pt}trans{\isacharcolon}{\kern0pt}\ {\isachardoublequoteopen}transp\ HML{\isacharunderscore}{\kern0pt}equivalent{\isachardoublequoteclose}\isanewline
%
\isadelimproof
\ \ %
\endisadelimproof
%
\isatagproof
\isacommand{by}\isamarkupfalse%
\ {\isacharparenleft}{\kern0pt}simp\ add{\isacharcolon}{\kern0pt}\ HML{\isacharunderscore}{\kern0pt}equivalent{\isacharunderscore}{\kern0pt}def\ transp{\isacharunderscore}{\kern0pt}def{\isacharparenright}{\kern0pt}%
\endisatagproof
{\isafoldproof}%
%
\isadelimproof
%
\endisadelimproof
%
\begin{isamarkuptext}%
A formula distinguishes one state from another if its true for the
  first and false for the second.%
\end{isamarkuptext}\isamarkuptrue%
\isacommand{abbreviation}\isamarkupfalse%
\ distinguishes\ {\isacharcolon}{\kern0pt}{\isacharcolon}{\kern0pt}\ \ {\isacartoucheopen}{\isacharparenleft}{\kern0pt}{\isacharprime}{\kern0pt}a{\isacharcomma}{\kern0pt}\ {\isacharprime}{\kern0pt}s{\isacharparenright}{\kern0pt}\ hml\ {\isasymRightarrow}\ {\isacharprime}{\kern0pt}s\ {\isasymRightarrow}\ {\isacharprime}{\kern0pt}s\ {\isasymRightarrow}\ bool{\isacartoucheclose}\ \isakeyword{where}\isanewline
\ \ \ {\isacartoucheopen}distinguishes\ {\isasymphi}\ p\ q\ {\isasymequiv}\ p\ {\isasymTurnstile}\ {\isasymphi}\ {\isasymand}\ {\isasymnot}\ q\ {\isasymTurnstile}\ {\isasymphi}{\isacartoucheclose}\isanewline
\isanewline
\isacommand{lemma}\isamarkupfalse%
\ hml{\isacharunderscore}{\kern0pt}equiv{\isacharunderscore}{\kern0pt}sym{\isacharcolon}{\kern0pt}\isanewline
\ \ \isakeyword{shows}\ {\isacartoucheopen}symp\ HML{\isacharunderscore}{\kern0pt}equivalent{\isacartoucheclose}\isanewline
%
\isadelimproof
%
\endisadelimproof
%
\isatagproof
\isacommand{unfolding}\isamarkupfalse%
\ HML{\isacharunderscore}{\kern0pt}equivalent{\isacharunderscore}{\kern0pt}def\ symp{\isacharunderscore}{\kern0pt}def\ \isacommand{by}\isamarkupfalse%
\ simp%
\endisatagproof
{\isafoldproof}%
%
\isadelimproof
%
\endisadelimproof
%
\begin{isamarkuptext}%
If two states are not HML equivalent then there must be a
  distinguishing formula.%
\end{isamarkuptext}\isamarkuptrue%
\isacommand{lemma}\isamarkupfalse%
\ hml{\isacharunderscore}{\kern0pt}distinctions{\isacharcolon}{\kern0pt}\isanewline
\ \ \isakeyword{fixes}\ state{\isacharcolon}{\kern0pt}{\isacharcolon}{\kern0pt}{\isachardoublequoteopen}{\isacharprime}{\kern0pt}s{\isachardoublequoteclose}\isanewline
\ \ \isakeyword{assumes}\ {\isacartoucheopen}{\isasymnot}\ HML{\isacharunderscore}{\kern0pt}equivalent\ p\ q{\isacartoucheclose}\isanewline
\ \ \isakeyword{shows}\ {\isacartoucheopen}{\isasymexists}{\isasymphi}{\isachardot}{\kern0pt}\ distinguishes\ {\isasymphi}\ p\ q{\isacartoucheclose}\isanewline
%
\isadelimproof
%
\endisadelimproof
%
\isatagproof
\isacommand{proof}\isamarkupfalse%
{\isacharminus}{\kern0pt}\isanewline
\ \ \isacommand{from}\isamarkupfalse%
\ assms\ \isacommand{have}\isamarkupfalse%
\ {\isachardoublequoteopen}{\isasymnot}\ {\isacharparenleft}{\kern0pt}{\isasymforall}\ {\isasymphi}{\isacharcolon}{\kern0pt}{\isacharcolon}{\kern0pt}{\isacharparenleft}{\kern0pt}{\isacharprime}{\kern0pt}a{\isacharcomma}{\kern0pt}\ {\isacharprime}{\kern0pt}s{\isacharparenright}{\kern0pt}\ hml{\isachardot}{\kern0pt}\ {\isacharparenleft}{\kern0pt}p\ {\isasymTurnstile}\ {\isasymphi}{\isacharparenright}{\kern0pt}\ {\isasymlongleftrightarrow}\ {\isacharparenleft}{\kern0pt}q\ {\isasymTurnstile}\ {\isasymphi}{\isacharparenright}{\kern0pt}{\isacharparenright}{\kern0pt}{\isachardoublequoteclose}\ \isanewline
\ \ \ \ \isacommand{unfolding}\isamarkupfalse%
\ HML{\isacharunderscore}{\kern0pt}equivalent{\isacharunderscore}{\kern0pt}def\ \isacommand{by}\isamarkupfalse%
\ blast\isanewline
\ \ \isacommand{then}\isamarkupfalse%
\ \isacommand{obtain}\isamarkupfalse%
\ {\isasymphi}{\isacharcolon}{\kern0pt}{\isacharcolon}{\kern0pt}{\isachardoublequoteopen}{\isacharparenleft}{\kern0pt}{\isacharprime}{\kern0pt}a{\isacharcomma}{\kern0pt}\ {\isacharprime}{\kern0pt}s{\isacharparenright}{\kern0pt}\ hml{\isachardoublequoteclose}\ \isakeyword{where}\ {\isachardoublequoteopen}{\isacharparenleft}{\kern0pt}p\ {\isasymTurnstile}\ {\isasymphi}{\isacharparenright}{\kern0pt}\ {\isasymnoteq}\ {\isacharparenleft}{\kern0pt}q\ {\isasymTurnstile}\ {\isasymphi}{\isacharparenright}{\kern0pt}{\isachardoublequoteclose}\ \isacommand{by}\isamarkupfalse%
\ blast\isanewline
\ \ \isacommand{then}\isamarkupfalse%
\ \isacommand{have}\isamarkupfalse%
\ {\isachardoublequoteopen}{\isacharparenleft}{\kern0pt}{\isacharparenleft}{\kern0pt}p\ {\isasymTurnstile}\ {\isasymphi}{\isacharparenright}{\kern0pt}\ {\isasymand}\ {\isasymnot}{\isacharparenleft}{\kern0pt}q\ {\isasymTurnstile}\ {\isasymphi}{\isacharparenright}{\kern0pt}{\isacharparenright}{\kern0pt}\ {\isasymor}\ {\isacharparenleft}{\kern0pt}{\isasymnot}{\isacharparenleft}{\kern0pt}p\ {\isasymTurnstile}\ {\isasymphi}{\isacharparenright}{\kern0pt}\ {\isasymand}\ {\isacharparenleft}{\kern0pt}q\ {\isasymTurnstile}\ {\isasymphi}{\isacharparenright}{\kern0pt}{\isacharparenright}{\kern0pt}{\isachardoublequoteclose}\isanewline
\ \ \ \ \isacommand{by}\isamarkupfalse%
\ blast\isanewline
\ \ \isacommand{then}\isamarkupfalse%
\ \isacommand{show}\isamarkupfalse%
\ {\isacharquery}{\kern0pt}thesis\isanewline
\ \ \isacommand{proof}\isamarkupfalse%
\isanewline
\ \ \ \ \isacommand{show}\isamarkupfalse%
\ {\isachardoublequoteopen}distinguishes\ {\isasymphi}\ p\ q\ {\isasymLongrightarrow}\ {\isasymexists}{\isasymphi}{\isachardot}{\kern0pt}\ distinguishes\ {\isasymphi}\ p\ q{\isachardoublequoteclose}\ \isacommand{by}\isamarkupfalse%
\ blast\isanewline
\ \ \isacommand{next}\isamarkupfalse%
\isanewline
\ \ \ \ \isacommand{assume}\isamarkupfalse%
\ assm{\isacharcolon}{\kern0pt}\ {\isachardoublequoteopen}{\isasymnot}\ p\ {\isasymTurnstile}\ {\isasymphi}\ {\isasymand}\ q\ {\isasymTurnstile}\ {\isasymphi}{\isachardoublequoteclose}\isanewline
\ \ \ \ \isacommand{define}\isamarkupfalse%
\ n{\isasymphi}\ \isakeyword{where}\ {\isachardoublequoteopen}n{\isasymphi}\ {\isasymequiv}{\isacharparenleft}{\kern0pt}hml{\isacharunderscore}{\kern0pt}conj\ {\isacharparenleft}{\kern0pt}{\isacharbraceleft}{\kern0pt}{\isacharbraceright}{\kern0pt}{\isacharcolon}{\kern0pt}{\isacharcolon}{\kern0pt}{\isacharprime}{\kern0pt}s\ set{\isacharparenright}{\kern0pt}\ \isanewline
\ \ \ \ \ \ \ \ \ \ \ \ \ \ \ \ \ \ \ \ \ \ \ \ \ \ {\isacharbraceleft}{\kern0pt}state{\isacharbraceright}{\kern0pt}\ \isanewline
\ \ \ \ \ \ \ \ \ \ \ \ \ \ \ \ \ \ \ \ \ \ \ \ \ \ {\isacharparenleft}{\kern0pt}{\isasymlambda}j{\isachardot}{\kern0pt}\ if\ j\ {\isacharequal}{\kern0pt}\ state\ then\ {\isasymphi}\ else\ undefined{\isacharparenright}{\kern0pt}{\isacharparenright}{\kern0pt}{\isachardoublequoteclose}\isanewline
\ \ \ \ \isacommand{have}\isamarkupfalse%
\ {\isachardoublequoteopen}p\ {\isasymTurnstile}\ n{\isasymphi}\ {\isasymand}\ {\isasymnot}\ q\ {\isasymTurnstile}\ n{\isasymphi}{\isachardoublequoteclose}\ \isanewline
\ \ \ \ \ \ \isacommand{unfolding}\isamarkupfalse%
\ n{\isasymphi}{\isacharunderscore}{\kern0pt}def\isanewline
\ \ \ \ \ \ \isacommand{using}\isamarkupfalse%
\ hml{\isacharunderscore}{\kern0pt}semantics{\isachardot}{\kern0pt}simps\ assm\isanewline
\ \ \ \ \ \ \isacommand{by}\isamarkupfalse%
\ force\isanewline
\ \ \ \ \isacommand{then}\isamarkupfalse%
\ \isacommand{show}\isamarkupfalse%
\ {\isacharquery}{\kern0pt}thesis\isanewline
\ \ \ \ \ \ \isacommand{by}\isamarkupfalse%
\ blast\isanewline
\ \ \isacommand{qed}\isamarkupfalse%
\isanewline
\isacommand{qed}\isamarkupfalse%
%
\endisatagproof
{\isafoldproof}%
%
\isadelimproof
\isanewline
%
\endisadelimproof
\isanewline
\isacommand{end}\isamarkupfalse%
\ \isanewline
\isanewline
\isanewline
\isanewline
\isacommand{inductive}\isamarkupfalse%
\ HML{\isacharunderscore}{\kern0pt}trace\ {\isacharcolon}{\kern0pt}{\isacharcolon}{\kern0pt}\ {\isachardoublequoteopen}{\isacharparenleft}{\kern0pt}{\isacharprime}{\kern0pt}a{\isacharcomma}{\kern0pt}\ {\isacharprime}{\kern0pt}s{\isacharparenright}{\kern0pt}hml\ {\isasymRightarrow}\ bool{\isachardoublequoteclose}\isanewline
\ \ \isakeyword{where}\isanewline
trace{\isacharunderscore}{\kern0pt}tt\ {\isacharcolon}{\kern0pt}\ {\isachardoublequoteopen}HML{\isacharunderscore}{\kern0pt}trace\ TT{\isachardoublequoteclose}\ {\isacharbar}{\kern0pt}\isanewline
trace{\isacharunderscore}{\kern0pt}conj{\isacharcolon}{\kern0pt}\ {\isachardoublequoteopen}HML{\isacharunderscore}{\kern0pt}trace\ {\isacharparenleft}{\kern0pt}hml{\isacharunderscore}{\kern0pt}conj\ {\isacharbraceleft}{\kern0pt}{\isacharbraceright}{\kern0pt}\ {\isacharbraceleft}{\kern0pt}{\isacharbraceright}{\kern0pt}\ {\isasympsi}s{\isacharparenright}{\kern0pt}{\isachardoublequoteclose}{\isacharbar}{\kern0pt}\isanewline
trace{\isacharunderscore}{\kern0pt}pos{\isacharcolon}{\kern0pt}\ {\isachardoublequoteopen}HML{\isacharunderscore}{\kern0pt}trace\ {\isacharparenleft}{\kern0pt}hml{\isacharunderscore}{\kern0pt}pos\ {\isasymalpha}\ {\isasymphi}{\isacharparenright}{\kern0pt}{\isachardoublequoteclose}\ \isakeyword{if}\ {\isachardoublequoteopen}HML{\isacharunderscore}{\kern0pt}trace\ {\isasymphi}{\isachardoublequoteclose}\isanewline
\isanewline
\isacommand{definition}\isamarkupfalse%
\ HML{\isacharunderscore}{\kern0pt}trace{\isacharunderscore}{\kern0pt}formulas\ \isakeyword{where}\isanewline
{\isachardoublequoteopen}HML{\isacharunderscore}{\kern0pt}trace{\isacharunderscore}{\kern0pt}formulas\ {\isasymequiv}\ {\isacharbraceleft}{\kern0pt}{\isasymphi}{\isachardot}{\kern0pt}\ HML{\isacharunderscore}{\kern0pt}trace\ {\isasymphi}{\isacharbraceright}{\kern0pt}{\isachardoublequoteclose}%
\begin{isamarkuptext}%
translation of a trace to a formula%
\end{isamarkuptext}\isamarkuptrue%
\isacommand{fun}\isamarkupfalse%
\ trace{\isacharunderscore}{\kern0pt}to{\isacharunderscore}{\kern0pt}formula\ {\isacharcolon}{\kern0pt}{\isacharcolon}{\kern0pt}\ {\isachardoublequoteopen}{\isacharprime}{\kern0pt}a\ list\ {\isasymRightarrow}\ {\isacharparenleft}{\kern0pt}{\isacharprime}{\kern0pt}a{\isacharcomma}{\kern0pt}\ {\isacharprime}{\kern0pt}s{\isacharparenright}{\kern0pt}hml{\isachardoublequoteclose}\isanewline
\ \ \isakeyword{where}\isanewline
{\isachardoublequoteopen}trace{\isacharunderscore}{\kern0pt}to{\isacharunderscore}{\kern0pt}formula\ {\isacharbrackleft}{\kern0pt}{\isacharbrackright}{\kern0pt}\ {\isacharequal}{\kern0pt}\ TT{\isachardoublequoteclose}\ {\isacharbar}{\kern0pt}\isanewline
{\isachardoublequoteopen}trace{\isacharunderscore}{\kern0pt}to{\isacharunderscore}{\kern0pt}formula\ {\isacharparenleft}{\kern0pt}a{\isacharhash}{\kern0pt}xs{\isacharparenright}{\kern0pt}\ {\isacharequal}{\kern0pt}\ hml{\isacharunderscore}{\kern0pt}pos\ a\ {\isacharparenleft}{\kern0pt}trace{\isacharunderscore}{\kern0pt}to{\isacharunderscore}{\kern0pt}formula\ xs{\isacharparenright}{\kern0pt}{\isachardoublequoteclose}\isanewline
\isanewline
\isacommand{inductive}\isamarkupfalse%
\ HML{\isacharunderscore}{\kern0pt}failure\ {\isacharcolon}{\kern0pt}{\isacharcolon}{\kern0pt}\ {\isachardoublequoteopen}{\isacharparenleft}{\kern0pt}{\isacharprime}{\kern0pt}a{\isacharcomma}{\kern0pt}\ {\isacharprime}{\kern0pt}s{\isacharparenright}{\kern0pt}hml\ {\isasymRightarrow}\ bool{\isachardoublequoteclose}\isanewline
\ \ \isakeyword{where}\isanewline
failure{\isacharunderscore}{\kern0pt}tt{\isacharcolon}{\kern0pt}\ {\isachardoublequoteopen}HML{\isacharunderscore}{\kern0pt}failure\ TT{\isachardoublequoteclose}\ {\isacharbar}{\kern0pt}\isanewline
failure{\isacharunderscore}{\kern0pt}pos{\isacharcolon}{\kern0pt}\ {\isachardoublequoteopen}HML{\isacharunderscore}{\kern0pt}failure\ {\isacharparenleft}{\kern0pt}hml{\isacharunderscore}{\kern0pt}pos\ {\isasymalpha}\ {\isasymphi}{\isacharparenright}{\kern0pt}{\isachardoublequoteclose}\ \isakeyword{if}\ {\isachardoublequoteopen}HML{\isacharunderscore}{\kern0pt}failure\ {\isasymphi}{\isachardoublequoteclose}\ {\isacharbar}{\kern0pt}\isanewline
failure{\isacharunderscore}{\kern0pt}conj{\isacharcolon}{\kern0pt}\ {\isachardoublequoteopen}HML{\isacharunderscore}{\kern0pt}failure\ {\isacharparenleft}{\kern0pt}hml{\isacharunderscore}{\kern0pt}conj\ I\ J\ {\isasympsi}s{\isacharparenright}{\kern0pt}{\isachardoublequoteclose}\ \isanewline
\isakeyword{if}\ {\isachardoublequoteopen}{\isacharparenleft}{\kern0pt}{\isasymforall}i\ {\isasymin}\ I{\isachardot}{\kern0pt}\ TT{\isacharunderscore}{\kern0pt}like\ {\isacharparenleft}{\kern0pt}{\isasympsi}s\ i{\isacharparenright}{\kern0pt}{\isacharparenright}{\kern0pt}\ {\isasymand}\ {\isacharparenleft}{\kern0pt}{\isasymforall}j\ {\isasymin}\ J{\isachardot}{\kern0pt}\ {\isacharparenleft}{\kern0pt}TT{\isacharunderscore}{\kern0pt}like\ {\isacharparenleft}{\kern0pt}{\isasympsi}s\ j{\isacharparenright}{\kern0pt}{\isacharparenright}{\kern0pt}\ {\isasymor}\ {\isacharparenleft}{\kern0pt}{\isasymexists}{\isasymalpha}\ {\isasymchi}{\isachardot}{\kern0pt}\ {\isacharparenleft}{\kern0pt}{\isacharparenleft}{\kern0pt}{\isasympsi}s\ j{\isacharparenright}{\kern0pt}\ {\isacharequal}{\kern0pt}\ hml{\isacharunderscore}{\kern0pt}pos\ {\isasymalpha}\ {\isasymchi}\ {\isasymand}\ {\isacharparenleft}{\kern0pt}TT{\isacharunderscore}{\kern0pt}like\ {\isasymchi}{\isacharparenright}{\kern0pt}{\isacharparenright}{\kern0pt}{\isacharparenright}{\kern0pt}{\isacharparenright}{\kern0pt}{\isachardoublequoteclose}\ \isanewline
\isanewline
\isacommand{inductive}\isamarkupfalse%
\ HML{\isacharunderscore}{\kern0pt}simulation\ {\isacharcolon}{\kern0pt}{\isacharcolon}{\kern0pt}\ {\isachardoublequoteopen}{\isacharparenleft}{\kern0pt}{\isacharprime}{\kern0pt}a{\isacharcomma}{\kern0pt}\ {\isacharprime}{\kern0pt}s{\isacharparenright}{\kern0pt}hml\ {\isasymRightarrow}\ bool{\isachardoublequoteclose}\isanewline
\ \ \isakeyword{where}\isanewline
sim{\isacharunderscore}{\kern0pt}tt{\isacharcolon}{\kern0pt}\ {\isachardoublequoteopen}HML{\isacharunderscore}{\kern0pt}simulation\ TT{\isachardoublequoteclose}\ {\isacharbar}{\kern0pt}\isanewline
sim{\isacharunderscore}{\kern0pt}pos{\isacharcolon}{\kern0pt}\ {\isachardoublequoteopen}HML{\isacharunderscore}{\kern0pt}simulation\ {\isacharparenleft}{\kern0pt}hml{\isacharunderscore}{\kern0pt}pos\ {\isasymalpha}\ {\isasymphi}{\isacharparenright}{\kern0pt}{\isachardoublequoteclose}\ \isakeyword{if}\ {\isachardoublequoteopen}HML{\isacharunderscore}{\kern0pt}simulation\ {\isasymphi}{\isachardoublequoteclose}{\isacharbar}{\kern0pt}\isanewline
sim{\isacharunderscore}{\kern0pt}conj{\isacharcolon}{\kern0pt}\ {\isachardoublequoteopen}HML{\isacharunderscore}{\kern0pt}simulation\ {\isacharparenleft}{\kern0pt}hml{\isacharunderscore}{\kern0pt}conj\ I\ J\ {\isasympsi}s{\isacharparenright}{\kern0pt}\ {\isachardoublequoteclose}\ \isanewline
\isakeyword{if}\ {\isachardoublequoteopen}{\isacharparenleft}{\kern0pt}{\isasymforall}x\ {\isasymin}\ {\isacharparenleft}{\kern0pt}{\isasympsi}s\ {\isacharbackquote}{\kern0pt}\ I{\isacharparenright}{\kern0pt}{\isachardot}{\kern0pt}\ HML{\isacharunderscore}{\kern0pt}simulation\ x{\isacharparenright}{\kern0pt}\ {\isasymand}\ {\isacharparenleft}{\kern0pt}{\isasympsi}s\ {\isacharbackquote}{\kern0pt}\ J\ {\isacharequal}{\kern0pt}\ {\isacharbraceleft}{\kern0pt}{\isacharbraceright}{\kern0pt}{\isacharparenright}{\kern0pt}{\isachardoublequoteclose}\isanewline
\isanewline
\isacommand{definition}\isamarkupfalse%
\ HML{\isacharunderscore}{\kern0pt}simulation{\isacharunderscore}{\kern0pt}formulas\ \isakeyword{where}\isanewline
{\isachardoublequoteopen}HML{\isacharunderscore}{\kern0pt}simulation{\isacharunderscore}{\kern0pt}formulas\ {\isasymequiv}\ {\isacharbraceleft}{\kern0pt}{\isasymphi}{\isachardot}{\kern0pt}\ HML{\isacharunderscore}{\kern0pt}simulation\ {\isasymphi}{\isacharbraceright}{\kern0pt}{\isachardoublequoteclose}\isanewline
\isanewline
\isacommand{inductive}\isamarkupfalse%
\ HML{\isacharunderscore}{\kern0pt}readiness\ {\isacharcolon}{\kern0pt}{\isacharcolon}{\kern0pt}\ {\isachardoublequoteopen}{\isacharparenleft}{\kern0pt}{\isacharprime}{\kern0pt}a{\isacharcomma}{\kern0pt}\ {\isacharprime}{\kern0pt}s{\isacharparenright}{\kern0pt}hml\ {\isasymRightarrow}\ bool{\isachardoublequoteclose}\isanewline
\ \ \isakeyword{where}\isanewline
read{\isacharunderscore}{\kern0pt}tt{\isacharcolon}{\kern0pt}\ {\isachardoublequoteopen}HML{\isacharunderscore}{\kern0pt}readiness\ TT{\isachardoublequoteclose}\ {\isacharbar}{\kern0pt}\isanewline
read{\isacharunderscore}{\kern0pt}pos{\isacharcolon}{\kern0pt}\ {\isachardoublequoteopen}HML{\isacharunderscore}{\kern0pt}readiness\ {\isacharparenleft}{\kern0pt}hml{\isacharunderscore}{\kern0pt}pos\ {\isasymalpha}\ {\isasymphi}{\isacharparenright}{\kern0pt}{\isachardoublequoteclose}\ \isakeyword{if}\ {\isachardoublequoteopen}HML{\isacharunderscore}{\kern0pt}readiness\ {\isasymphi}{\isachardoublequoteclose}{\isacharbar}{\kern0pt}\isanewline
read{\isacharunderscore}{\kern0pt}conj{\isacharcolon}{\kern0pt}\ {\isachardoublequoteopen}HML{\isacharunderscore}{\kern0pt}readiness\ {\isacharparenleft}{\kern0pt}hml{\isacharunderscore}{\kern0pt}conj\ I\ J\ {\isasymPhi}{\isacharparenright}{\kern0pt}{\isachardoublequoteclose}\ \isanewline
\isakeyword{if}\ {\isachardoublequoteopen}{\isacharparenleft}{\kern0pt}{\isasymforall}x\ {\isasymin}\ {\isacharparenleft}{\kern0pt}{\isasymPhi}\ {\isacharbackquote}{\kern0pt}\ {\isacharparenleft}{\kern0pt}I\ {\isasymunion}\ J{\isacharparenright}{\kern0pt}{\isacharparenright}{\kern0pt}{\isachardot}{\kern0pt}\ TT{\isacharunderscore}{\kern0pt}like\ x\ {\isasymor}\ {\isacharparenleft}{\kern0pt}{\isasymexists}{\isasymalpha}\ {\isasymchi}{\isachardot}{\kern0pt}\ x\ {\isacharequal}{\kern0pt}\ hml{\isacharunderscore}{\kern0pt}pos\ {\isasymalpha}\ {\isasymchi}\ {\isasymand}\ TT{\isacharunderscore}{\kern0pt}like\ {\isasymchi}{\isacharparenright}{\kern0pt}{\isacharparenright}{\kern0pt}{\isachardoublequoteclose}\isanewline
\isanewline
\isacommand{inductive}\isamarkupfalse%
\ HML{\isacharunderscore}{\kern0pt}impossible{\isacharunderscore}{\kern0pt}futures\ {\isacharcolon}{\kern0pt}{\isacharcolon}{\kern0pt}\ \ {\isachardoublequoteopen}{\isacharparenleft}{\kern0pt}{\isacharprime}{\kern0pt}a{\isacharcomma}{\kern0pt}\ {\isacharprime}{\kern0pt}s{\isacharparenright}{\kern0pt}hml\ {\isasymRightarrow}\ bool{\isachardoublequoteclose}\isanewline
\ \ \isakeyword{where}\isanewline
\ \ if{\isacharunderscore}{\kern0pt}tt{\isacharcolon}{\kern0pt}\ {\isachardoublequoteopen}HML{\isacharunderscore}{\kern0pt}impossible{\isacharunderscore}{\kern0pt}futures\ TT{\isachardoublequoteclose}\ {\isacharbar}{\kern0pt}\isanewline
\ \ if{\isacharunderscore}{\kern0pt}pos{\isacharcolon}{\kern0pt}\ {\isachardoublequoteopen}HML{\isacharunderscore}{\kern0pt}impossible{\isacharunderscore}{\kern0pt}futures\ {\isacharparenleft}{\kern0pt}hml{\isacharunderscore}{\kern0pt}pos\ {\isasymalpha}\ {\isasymphi}{\isacharparenright}{\kern0pt}{\isachardoublequoteclose}\ \isakeyword{if}\ {\isachardoublequoteopen}HML{\isacharunderscore}{\kern0pt}impossible{\isacharunderscore}{\kern0pt}futures\ {\isasymphi}{\isachardoublequoteclose}\ {\isacharbar}{\kern0pt}\isanewline
if{\isacharunderscore}{\kern0pt}conj{\isacharcolon}{\kern0pt}\ {\isachardoublequoteopen}HML{\isacharunderscore}{\kern0pt}impossible{\isacharunderscore}{\kern0pt}futures\ {\isacharparenleft}{\kern0pt}hml{\isacharunderscore}{\kern0pt}conj\ I\ J\ {\isasymPhi}{\isacharparenright}{\kern0pt}{\isachardoublequoteclose}\isanewline
\isakeyword{if}\ {\isachardoublequoteopen}{\isasymforall}x\ {\isasymin}\ {\isacharparenleft}{\kern0pt}{\isasymPhi}\ {\isacharbackquote}{\kern0pt}\ I{\isacharparenright}{\kern0pt}{\isachardot}{\kern0pt}\ TT{\isacharunderscore}{\kern0pt}like\ x{\isachardoublequoteclose}\ {\isachardoublequoteopen}{\isasymforall}x\ {\isasymin}\ {\isacharparenleft}{\kern0pt}{\isasymPhi}\ {\isacharbackquote}{\kern0pt}\ J{\isacharparenright}{\kern0pt}{\isachardot}{\kern0pt}\ {\isacharparenleft}{\kern0pt}HML{\isacharunderscore}{\kern0pt}trace\ x{\isacharparenright}{\kern0pt}{\isachardoublequoteclose}\isanewline
\isanewline
\isacommand{inductive}\isamarkupfalse%
\ HML{\isacharunderscore}{\kern0pt}possible{\isacharunderscore}{\kern0pt}futures\ {\isacharcolon}{\kern0pt}{\isacharcolon}{\kern0pt}\ {\isachardoublequoteopen}{\isacharparenleft}{\kern0pt}{\isacharprime}{\kern0pt}a{\isacharcomma}{\kern0pt}\ {\isacharprime}{\kern0pt}s{\isacharparenright}{\kern0pt}hml\ {\isasymRightarrow}\ bool{\isachardoublequoteclose}\isanewline
\ \ \isakeyword{where}\isanewline
pf{\isacharunderscore}{\kern0pt}tt{\isacharcolon}{\kern0pt}\ {\isachardoublequoteopen}HML{\isacharunderscore}{\kern0pt}possible{\isacharunderscore}{\kern0pt}futures\ TT{\isachardoublequoteclose}\ {\isacharbar}{\kern0pt}\isanewline
pf{\isacharunderscore}{\kern0pt}pos{\isacharcolon}{\kern0pt}\ {\isachardoublequoteopen}HML{\isacharunderscore}{\kern0pt}possible{\isacharunderscore}{\kern0pt}futures\ {\isacharparenleft}{\kern0pt}hml{\isacharunderscore}{\kern0pt}pos\ {\isasymalpha}\ {\isasymphi}{\isacharparenright}{\kern0pt}{\isachardoublequoteclose}\ \isakeyword{if}\ {\isachardoublequoteopen}HML{\isacharunderscore}{\kern0pt}possible{\isacharunderscore}{\kern0pt}futures\ {\isasymphi}{\isachardoublequoteclose}\ {\isacharbar}{\kern0pt}\isanewline
pf{\isacharunderscore}{\kern0pt}conj{\isacharcolon}{\kern0pt}\ {\isachardoublequoteopen}HML{\isacharunderscore}{\kern0pt}possible{\isacharunderscore}{\kern0pt}futures\ {\isacharparenleft}{\kern0pt}hml{\isacharunderscore}{\kern0pt}conj\ I\ J\ {\isasymPhi}{\isacharparenright}{\kern0pt}{\isachardoublequoteclose}\ \isanewline
\isakeyword{if}\ {\isachardoublequoteopen}{\isasymforall}x\ {\isasymin}\ {\isacharparenleft}{\kern0pt}{\isasymPhi}\ {\isacharbackquote}{\kern0pt}\ {\isacharparenleft}{\kern0pt}I{\isasymunion}\ J{\isacharparenright}{\kern0pt}{\isacharparenright}{\kern0pt}{\isachardot}{\kern0pt}\ {\isacharparenleft}{\kern0pt}HML{\isacharunderscore}{\kern0pt}trace\ x{\isacharparenright}{\kern0pt}{\isachardoublequoteclose}\isanewline
\isanewline
\isacommand{definition}\isamarkupfalse%
\ HML{\isacharunderscore}{\kern0pt}possible{\isacharunderscore}{\kern0pt}futures{\isacharunderscore}{\kern0pt}formulas\ \isakeyword{where}\isanewline
{\isachardoublequoteopen}HML{\isacharunderscore}{\kern0pt}possible{\isacharunderscore}{\kern0pt}futures{\isacharunderscore}{\kern0pt}formulas\ {\isasymequiv}\ {\isacharbraceleft}{\kern0pt}{\isasymphi}{\isachardot}{\kern0pt}\ HML{\isacharunderscore}{\kern0pt}possible{\isacharunderscore}{\kern0pt}futures\ {\isasymphi}{\isacharbraceright}{\kern0pt}{\isachardoublequoteclose}\isanewline
\isanewline
\isacommand{inductive}\isamarkupfalse%
\ HML{\isacharunderscore}{\kern0pt}failure{\isacharunderscore}{\kern0pt}trace\ {\isacharcolon}{\kern0pt}{\isacharcolon}{\kern0pt}\ {\isachardoublequoteopen}{\isacharparenleft}{\kern0pt}{\isacharprime}{\kern0pt}a{\isacharcomma}{\kern0pt}\ {\isacharprime}{\kern0pt}s{\isacharparenright}{\kern0pt}hml\ {\isasymRightarrow}\ bool{\isachardoublequoteclose}\isanewline
\ \ \isakeyword{where}\isanewline
f{\isacharunderscore}{\kern0pt}trace{\isacharunderscore}{\kern0pt}tt{\isacharcolon}{\kern0pt}\ {\isachardoublequoteopen}HML{\isacharunderscore}{\kern0pt}failure{\isacharunderscore}{\kern0pt}trace\ TT{\isachardoublequoteclose}\ {\isacharbar}{\kern0pt}\isanewline
f{\isacharunderscore}{\kern0pt}trace{\isacharunderscore}{\kern0pt}pos{\isacharcolon}{\kern0pt}\ {\isachardoublequoteopen}HML{\isacharunderscore}{\kern0pt}failure{\isacharunderscore}{\kern0pt}trace\ {\isacharparenleft}{\kern0pt}hml{\isacharunderscore}{\kern0pt}pos\ {\isasymalpha}\ {\isasymphi}{\isacharparenright}{\kern0pt}{\isachardoublequoteclose}\ \isakeyword{if}\ {\isachardoublequoteopen}HML{\isacharunderscore}{\kern0pt}failure{\isacharunderscore}{\kern0pt}trace\ {\isasymphi}{\isachardoublequoteclose}{\isacharbar}{\kern0pt}\isanewline
f{\isacharunderscore}{\kern0pt}trace{\isacharunderscore}{\kern0pt}conj{\isacharcolon}{\kern0pt}\ {\isachardoublequoteopen}HML{\isacharunderscore}{\kern0pt}failure{\isacharunderscore}{\kern0pt}trace\ {\isacharparenleft}{\kern0pt}hml{\isacharunderscore}{\kern0pt}conj\ I\ J\ {\isasymPhi}{\isacharparenright}{\kern0pt}{\isachardoublequoteclose}\isanewline
\isakeyword{if}\ {\isachardoublequoteopen}{\isacharparenleft}{\kern0pt}{\isacharparenleft}{\kern0pt}{\isasymexists}{\isasympsi}\ {\isasymin}\ {\isacharparenleft}{\kern0pt}{\isasymPhi}\ {\isacharbackquote}{\kern0pt}\ I{\isacharparenright}{\kern0pt}{\isachardot}{\kern0pt}\ {\isacharparenleft}{\kern0pt}HML{\isacharunderscore}{\kern0pt}failure{\isacharunderscore}{\kern0pt}trace\ {\isasympsi}{\isacharparenright}{\kern0pt}\ {\isasymand}\ {\isacharparenleft}{\kern0pt}{\isasymforall}y\ {\isasymin}\ {\isacharparenleft}{\kern0pt}{\isasymPhi}\ {\isacharbackquote}{\kern0pt}\ I{\isacharparenright}{\kern0pt}{\isachardot}{\kern0pt}\ {\isasympsi}\ {\isasymnoteq}\ y\ {\isasymlongrightarrow}\ nested{\isacharunderscore}{\kern0pt}empty{\isacharunderscore}{\kern0pt}conj\ y{\isacharparenright}{\kern0pt}{\isacharparenright}{\kern0pt}\ {\isasymor}\ \isanewline
{\isacharparenleft}{\kern0pt}{\isasymforall}y\ {\isasymin}\ {\isacharparenleft}{\kern0pt}{\isasymPhi}\ {\isacharbackquote}{\kern0pt}\ I{\isacharparenright}{\kern0pt}{\isachardot}{\kern0pt}\ nested{\isacharunderscore}{\kern0pt}empty{\isacharunderscore}{\kern0pt}conj\ y{\isacharparenright}{\kern0pt}{\isacharparenright}{\kern0pt}\ {\isasymand}\isanewline
{\isacharparenleft}{\kern0pt}{\isasymforall}y\ {\isasymin}\ {\isacharparenleft}{\kern0pt}{\isasymPhi}\ {\isacharbackquote}{\kern0pt}\ J{\isacharparenright}{\kern0pt}{\isachardot}{\kern0pt}\ stacked{\isacharunderscore}{\kern0pt}pos{\isacharunderscore}{\kern0pt}conj{\isacharunderscore}{\kern0pt}pos\ y{\isacharparenright}{\kern0pt}{\isachardoublequoteclose}\isanewline
\isanewline
\isanewline
\isacommand{inductive}\isamarkupfalse%
\ HML{\isacharunderscore}{\kern0pt}ready{\isacharunderscore}{\kern0pt}trace\ {\isacharcolon}{\kern0pt}{\isacharcolon}{\kern0pt}\ {\isachardoublequoteopen}{\isacharparenleft}{\kern0pt}{\isacharprime}{\kern0pt}a{\isacharcomma}{\kern0pt}\ {\isacharprime}{\kern0pt}s{\isacharparenright}{\kern0pt}hml\ {\isasymRightarrow}\ bool{\isachardoublequoteclose}\isanewline
\ \ \isakeyword{where}\isanewline
r{\isacharunderscore}{\kern0pt}trace{\isacharunderscore}{\kern0pt}tt{\isacharcolon}{\kern0pt}\ {\isachardoublequoteopen}HML{\isacharunderscore}{\kern0pt}ready{\isacharunderscore}{\kern0pt}trace\ TT{\isachardoublequoteclose}\ {\isacharbar}{\kern0pt}\isanewline
r{\isacharunderscore}{\kern0pt}trace{\isacharunderscore}{\kern0pt}pos{\isacharcolon}{\kern0pt}\ {\isachardoublequoteopen}HML{\isacharunderscore}{\kern0pt}ready{\isacharunderscore}{\kern0pt}trace\ {\isacharparenleft}{\kern0pt}hml{\isacharunderscore}{\kern0pt}pos\ {\isasymalpha}\ {\isasymphi}{\isacharparenright}{\kern0pt}{\isachardoublequoteclose}\ \isakeyword{if}\ {\isachardoublequoteopen}HML{\isacharunderscore}{\kern0pt}ready{\isacharunderscore}{\kern0pt}trace\ {\isasymphi}{\isachardoublequoteclose}{\isacharbar}{\kern0pt}\isanewline
r{\isacharunderscore}{\kern0pt}trace{\isacharunderscore}{\kern0pt}conj{\isacharcolon}{\kern0pt}\ {\isachardoublequoteopen}HML{\isacharunderscore}{\kern0pt}ready{\isacharunderscore}{\kern0pt}trace\ {\isacharparenleft}{\kern0pt}hml{\isacharunderscore}{\kern0pt}conj\ I\ J\ {\isasymPhi}{\isacharparenright}{\kern0pt}{\isachardoublequoteclose}\ \isanewline
\isakeyword{if}\ {\isachardoublequoteopen}{\isacharparenleft}{\kern0pt}{\isasymexists}x\ {\isasymin}\ {\isacharparenleft}{\kern0pt}{\isasymPhi}\ {\isacharbackquote}{\kern0pt}\ I{\isacharparenright}{\kern0pt}{\isachardot}{\kern0pt}\ HML{\isacharunderscore}{\kern0pt}ready{\isacharunderscore}{\kern0pt}trace\ x\ {\isasymand}\ {\isacharparenleft}{\kern0pt}{\isasymforall}y\ {\isasymin}\ {\isacharparenleft}{\kern0pt}{\isasymPhi}\ {\isacharbackquote}{\kern0pt}\ I{\isacharparenright}{\kern0pt}{\isachardot}{\kern0pt}\ x\ {\isasymnoteq}\ y\ {\isasymlongrightarrow}\ single{\isacharunderscore}{\kern0pt}pos\ y{\isacharparenright}{\kern0pt}{\isacharparenright}{\kern0pt}\isanewline
{\isasymor}\ {\isacharparenleft}{\kern0pt}{\isasymforall}y\ {\isasymin}\ {\isacharparenleft}{\kern0pt}{\isasymPhi}\ {\isacharbackquote}{\kern0pt}\ I{\isacharparenright}{\kern0pt}{\isachardot}{\kern0pt}single{\isacharunderscore}{\kern0pt}pos\ y{\isacharparenright}{\kern0pt}{\isachardoublequoteclose}\isanewline
{\isachardoublequoteopen}{\isacharparenleft}{\kern0pt}{\isasymforall}y\ {\isasymin}\ {\isacharparenleft}{\kern0pt}{\isasymPhi}\ {\isacharbackquote}{\kern0pt}\ J{\isacharparenright}{\kern0pt}{\isachardot}{\kern0pt}\ single{\isacharunderscore}{\kern0pt}pos{\isacharunderscore}{\kern0pt}pos\ y{\isacharparenright}{\kern0pt}{\isachardoublequoteclose}\isanewline
\isanewline
\isacommand{inductive}\isamarkupfalse%
\ HML{\isacharunderscore}{\kern0pt}ready{\isacharunderscore}{\kern0pt}sim\ {\isacharcolon}{\kern0pt}{\isacharcolon}{\kern0pt}\ {\isachardoublequoteopen}{\isacharparenleft}{\kern0pt}{\isacharprime}{\kern0pt}a{\isacharcomma}{\kern0pt}\ {\isacharprime}{\kern0pt}s{\isacharparenright}{\kern0pt}\ hml\ {\isasymRightarrow}\ bool{\isachardoublequoteclose}\isanewline
\ \ \isakeyword{where}\isanewline
{\isachardoublequoteopen}HML{\isacharunderscore}{\kern0pt}ready{\isacharunderscore}{\kern0pt}sim\ TT{\isachardoublequoteclose}\ {\isacharbar}{\kern0pt}\isanewline
{\isachardoublequoteopen}HML{\isacharunderscore}{\kern0pt}ready{\isacharunderscore}{\kern0pt}sim\ {\isacharparenleft}{\kern0pt}hml{\isacharunderscore}{\kern0pt}pos\ {\isasymalpha}\ {\isasymphi}{\isacharparenright}{\kern0pt}{\isachardoublequoteclose}\ \isakeyword{if}\ {\isachardoublequoteopen}HML{\isacharunderscore}{\kern0pt}ready{\isacharunderscore}{\kern0pt}sim\ {\isasymphi}{\isachardoublequoteclose}\ {\isacharbar}{\kern0pt}\isanewline
{\isachardoublequoteopen}HML{\isacharunderscore}{\kern0pt}ready{\isacharunderscore}{\kern0pt}sim\ {\isacharparenleft}{\kern0pt}hml{\isacharunderscore}{\kern0pt}conj\ I\ J\ {\isasymPhi}{\isacharparenright}{\kern0pt}{\isachardoublequoteclose}\ \isakeyword{if}\ \isanewline
{\isachardoublequoteopen}{\isacharparenleft}{\kern0pt}{\isasymforall}x\ {\isasymin}\ {\isacharparenleft}{\kern0pt}{\isasymPhi}\ {\isacharbackquote}{\kern0pt}\ I{\isacharparenright}{\kern0pt}{\isachardot}{\kern0pt}\ HML{\isacharunderscore}{\kern0pt}ready{\isacharunderscore}{\kern0pt}sim\ x{\isacharparenright}{\kern0pt}\ {\isasymand}\ {\isacharparenleft}{\kern0pt}{\isasymforall}y\ {\isasymin}\ {\isacharparenleft}{\kern0pt}{\isasymPhi}\ {\isacharbackquote}{\kern0pt}\ J{\isacharparenright}{\kern0pt}{\isachardot}{\kern0pt}\ single{\isacharunderscore}{\kern0pt}pos{\isacharunderscore}{\kern0pt}pos\ y{\isacharparenright}{\kern0pt}{\isachardoublequoteclose}\isanewline
\isanewline
\isacommand{inductive}\isamarkupfalse%
\ HML{\isacharunderscore}{\kern0pt}{\isadigit{2}}{\isacharunderscore}{\kern0pt}nested{\isacharunderscore}{\kern0pt}sim\ {\isacharcolon}{\kern0pt}{\isacharcolon}{\kern0pt}\ {\isachardoublequoteopen}{\isacharparenleft}{\kern0pt}{\isacharprime}{\kern0pt}a{\isacharcomma}{\kern0pt}\ {\isacharprime}{\kern0pt}s{\isacharparenright}{\kern0pt}\ hml\ {\isasymRightarrow}\ bool{\isachardoublequoteclose}\ \isanewline
\ \ \isakeyword{where}\isanewline
{\isachardoublequoteopen}HML{\isacharunderscore}{\kern0pt}{\isadigit{2}}{\isacharunderscore}{\kern0pt}nested{\isacharunderscore}{\kern0pt}sim\ TT{\isachardoublequoteclose}\ {\isacharbar}{\kern0pt}\isanewline
{\isachardoublequoteopen}HML{\isacharunderscore}{\kern0pt}{\isadigit{2}}{\isacharunderscore}{\kern0pt}nested{\isacharunderscore}{\kern0pt}sim\ {\isacharparenleft}{\kern0pt}hml{\isacharunderscore}{\kern0pt}pos\ {\isasymalpha}\ {\isasymphi}{\isacharparenright}{\kern0pt}{\isachardoublequoteclose}\ \isakeyword{if}\ {\isachardoublequoteopen}HML{\isacharunderscore}{\kern0pt}{\isadigit{2}}{\isacharunderscore}{\kern0pt}nested{\isacharunderscore}{\kern0pt}sim\ {\isasymphi}{\isachardoublequoteclose}\ {\isacharbar}{\kern0pt}\isanewline
{\isachardoublequoteopen}HML{\isacharunderscore}{\kern0pt}{\isadigit{2}}{\isacharunderscore}{\kern0pt}nested{\isacharunderscore}{\kern0pt}sim\ {\isacharparenleft}{\kern0pt}hml{\isacharunderscore}{\kern0pt}conj\ I\ J\ {\isasymPhi}{\isacharparenright}{\kern0pt}{\isachardoublequoteclose}\ \isanewline
\isakeyword{if}\ {\isachardoublequoteopen}{\isacharparenleft}{\kern0pt}{\isasymforall}x\ {\isasymin}\ {\isacharparenleft}{\kern0pt}{\isasymPhi}\ {\isacharbackquote}{\kern0pt}\ I{\isacharparenright}{\kern0pt}{\isachardot}{\kern0pt}\ HML{\isacharunderscore}{\kern0pt}{\isadigit{2}}{\isacharunderscore}{\kern0pt}nested{\isacharunderscore}{\kern0pt}sim\ x{\isacharparenright}{\kern0pt}\ {\isasymand}\ {\isacharparenleft}{\kern0pt}{\isasymforall}y\ {\isasymin}\ {\isacharparenleft}{\kern0pt}{\isasymPhi}\ {\isacharbackquote}{\kern0pt}\ J{\isacharparenright}{\kern0pt}{\isachardot}{\kern0pt}\ HML{\isacharunderscore}{\kern0pt}simulation\ y{\isacharparenright}{\kern0pt}{\isachardoublequoteclose}\isanewline
\ \ \ \ \ \ \ \ \ \ \ \ \ \ \ \ \ \ \ \ \ \ \ \ \ \ \ \ \ \ \ \ \ \ \ \ \ \ \ \ \ \ \ \ \ \ \ \ \ \ \ \ \ \ \ \ \ \ \ \ \ \ \ \ \isanewline
\isacommand{inductive}\isamarkupfalse%
\ HML{\isacharunderscore}{\kern0pt}revivals\ {\isacharcolon}{\kern0pt}{\isacharcolon}{\kern0pt}\ {\isachardoublequoteopen}{\isacharparenleft}{\kern0pt}{\isacharprime}{\kern0pt}a{\isacharcomma}{\kern0pt}\ {\isacharprime}{\kern0pt}s{\isacharparenright}{\kern0pt}\ hml\ {\isasymRightarrow}\ bool{\isachardoublequoteclose}\ \isanewline
\ \ \isakeyword{where}\isanewline
revivals{\isacharunderscore}{\kern0pt}tt{\isacharcolon}{\kern0pt}\ {\isachardoublequoteopen}HML{\isacharunderscore}{\kern0pt}revivals\ TT{\isachardoublequoteclose}\ {\isacharbar}{\kern0pt}\isanewline
revivals{\isacharunderscore}{\kern0pt}pos{\isacharcolon}{\kern0pt}\ {\isachardoublequoteopen}HML{\isacharunderscore}{\kern0pt}revivals\ {\isacharparenleft}{\kern0pt}hml{\isacharunderscore}{\kern0pt}pos\ {\isasymalpha}\ {\isasymphi}{\isacharparenright}{\kern0pt}{\isachardoublequoteclose}\ \isakeyword{if}\ {\isachardoublequoteopen}HML{\isacharunderscore}{\kern0pt}revivals\ {\isasymphi}{\isachardoublequoteclose}\ {\isacharbar}{\kern0pt}\isanewline
revivals{\isacharunderscore}{\kern0pt}conj{\isacharcolon}{\kern0pt}\ {\isachardoublequoteopen}HML{\isacharunderscore}{\kern0pt}revivals\ {\isacharparenleft}{\kern0pt}hml{\isacharunderscore}{\kern0pt}conj\ I\ J\ {\isasymPhi}{\isacharparenright}{\kern0pt}{\isachardoublequoteclose}\ \isakeyword{if}\ {\isachardoublequoteopen}{\isacharparenleft}{\kern0pt}{\isasymforall}x\ {\isasymin}\ {\isacharparenleft}{\kern0pt}{\isasymPhi}\ {\isacharbackquote}{\kern0pt}\ I{\isacharparenright}{\kern0pt}{\isachardot}{\kern0pt}\ {\isasymexists}{\isasymalpha}\ {\isasymchi}{\isachardot}{\kern0pt}\ {\isacharparenleft}{\kern0pt}x\ {\isacharequal}{\kern0pt}\ hml{\isacharunderscore}{\kern0pt}pos\ {\isasymalpha}\ {\isasymchi}{\isacharparenright}{\kern0pt}\ {\isasymand}\ TT{\isacharunderscore}{\kern0pt}like\ {\isasymchi}{\isacharparenright}{\kern0pt}{\isachardoublequoteclose}\isanewline
{\isachardoublequoteopen}{\isacharparenleft}{\kern0pt}{\isasymforall}x\ {\isasymin}\ {\isacharparenleft}{\kern0pt}{\isasymPhi}\ {\isacharbackquote}{\kern0pt}\ J{\isacharparenright}{\kern0pt}{\isachardot}{\kern0pt}\ {\isasymexists}{\isasymalpha}\ {\isasymchi}{\isachardot}{\kern0pt}\ {\isacharparenleft}{\kern0pt}x\ {\isacharequal}{\kern0pt}\ hml{\isacharunderscore}{\kern0pt}pos\ {\isasymalpha}\ {\isasymchi}{\isacharparenright}{\kern0pt}\ {\isasymand}\ TT{\isacharunderscore}{\kern0pt}like\ {\isasymchi}{\isacharparenright}{\kern0pt}{\isachardoublequoteclose}\isanewline
%
\isadelimtheory
\isanewline
%
\endisadelimtheory
%
\isatagtheory
\isacommand{end}\isamarkupfalse%
%
\endisatagtheory
{\isafoldtheory}%
%
\isadelimtheory
%
\endisadelimtheory
%
\end{isabellebody}%
\endinput
%:%file=~/Documents/Isabelle_HOL/HML_list.thy%:%
%:%24=9%:%
%:%36=10%:%
%:%37=11%:%
%:%38=12%:%
%:%39=13%:%
%:%40=14%:%
%:%41=15%:%
%:%42=16%:%
%:%43=17%:%
%:%47=19%:%
%:%48=20%:%
%:%49=21%:%
%:%50=22%:%
%:%51=23%:%
%:%52=24%:%
%:%53=25%:%
%:%57=27%:%
%:%58=28%:%
%:%59=29%:%
%:%60=30%:%
%:%62=33%:%
%:%63=33%:%
%:%64=34%:%
%:%65=35%:%
%:%66=36%:%
%:%68=38%:%
%:%69=39%:%
%:%70=40%:%
%:%71=41%:%
%:%72=42%:%
%:%74=44%:%
%:%75=44%:%
%:%76=45%:%
%:%77=46%:%
%:%78=47%:%
%:%79=48%:%
%:%80=49%:%
%:%81=49%:%
%:%82=50%:%
%:%83=51%:%
%:%84=52%:%
%:%85=53%:%
%:%86=54%:%
%:%87=55%:%
%:%88=55%:%
%:%89=56%:%
%:%90=57%:%
%:%91=58%:%
%:%92=59%:%
%:%93=60%:%
%:%94=61%:%
%:%95=61%:%
%:%96=62%:%
%:%97=63%:%
%:%98=64%:%
%:%99=65%:%
%:%100=66%:%
%:%102=68%:%
%:%103=69%:%
%:%104=70%:%
%:%105=71%:%
%:%106=71%:%
%:%107=72%:%
%:%108=73%:%
%:%109=74%:%
%:%110=75%:%
%:%111=76%:%
%:%112=77%:%
%:%113=78%:%
%:%114=79%:%
%:%115=79%:%
%:%116=80%:%
%:%117=81%:%
%:%118=82%:%
%:%119=83%:%
%:%120=84%:%
%:%121=85%:%
%:%122=86%:%
%:%123=86%:%
%:%124=87%:%
%:%125=88%:%
%:%126=89%:%
%:%127=90%:%
%:%128=91%:%
%:%129=92%:%
%:%130=93%:%
%:%131=94%:%
%:%132=94%:%
%:%133=95%:%
%:%134=96%:%
%:%135=97%:%
%:%136=98%:%
%:%137=99%:%
%:%138=100%:%
%:%139=101%:%
%:%140=102%:%
%:%141=102%:%
%:%142=103%:%
%:%143=104%:%
%:%144=104%:%
%:%145=105%:%
%:%146=106%:%
%:%147=107%:%
%:%148=108%:%
%:%149=109%:%
%:%150=110%:%
%:%151=111%:%
%:%152=111%:%
%:%153=112%:%
%:%154=113%:%
%:%161=114%:%
%:%162=114%:%
%:%163=115%:%
%:%164=115%:%
%:%165=116%:%
%:%166=116%:%
%:%167=117%:%
%:%168=117%:%
%:%169=117%:%
%:%170=117%:%
%:%171=118%:%
%:%172=118%:%
%:%173=118%:%
%:%174=119%:%
%:%175=119%:%
%:%176=120%:%
%:%177=120%:%
%:%178=120%:%
%:%179=120%:%
%:%180=121%:%
%:%186=121%:%
%:%189=122%:%
%:%190=123%:%
%:%191=123%:%
%:%192=124%:%
%:%193=125%:%
%:%194=126%:%
%:%195=126%:%
%:%196=127%:%
%:%197=128%:%
%:%198=129%:%
%:%201=130%:%
%:%205=130%:%
%:%206=130%:%
%:%207=130%:%
%:%212=130%:%
%:%215=131%:%
%:%216=132%:%
%:%217=132%:%
%:%218=133%:%
%:%219=134%:%
%:%222=135%:%
%:%226=135%:%
%:%227=135%:%
%:%228=136%:%
%:%229=136%:%
%:%230=137%:%
%:%231=137%:%
%:%232=138%:%
%:%233=138%:%
%:%238=138%:%
%:%241=139%:%
%:%242=140%:%
%:%243=140%:%
%:%244=141%:%
%:%245=142%:%
%:%248=143%:%
%:%252=143%:%
%:%253=143%:%
%:%254=144%:%
%:%255=144%:%
%:%256=145%:%
%:%257=145%:%
%:%258=146%:%
%:%259=146%:%
%:%268=149%:%
%:%270=150%:%
%:%271=150%:%
%:%272=151%:%
%:%274=153%:%
%:%275=154%:%
%:%277=156%:%
%:%278=156%:%
%:%279=157%:%
%:%280=158%:%
%:%281=159%:%
%:%282=159%:%
%:%285=160%:%
%:%289=160%:%
%:%290=160%:%
%:%291=160%:%
%:%305=162%:%
%:%317=165%:%
%:%318=166%:%
%:%319=167%:%
%:%320=168%:%
%:%322=171%:%
%:%323=171%:%
%:%324=172%:%
%:%326=174%:%
%:%328=175%:%
%:%329=175%:%
%:%332=176%:%
%:%336=176%:%
%:%337=176%:%
%:%342=176%:%
%:%345=177%:%
%:%346=178%:%
%:%347=178%:%
%:%348=179%:%
%:%349=180%:%
%:%352=181%:%
%:%356=181%:%
%:%357=181%:%
%:%358=182%:%
%:%359=182%:%
%:%360=183%:%
%:%361=183%:%
%:%366=183%:%
%:%369=184%:%
%:%370=185%:%
%:%371=186%:%
%:%372=186%:%
%:%375=187%:%
%:%379=187%:%
%:%380=187%:%
%:%389=190%:%
%:%390=191%:%
%:%392=193%:%
%:%393=193%:%
%:%394=194%:%
%:%395=195%:%
%:%396=196%:%
%:%397=196%:%
%:%398=197%:%
%:%405=198%:%
%:%406=198%:%
%:%407=198%:%
%:%416=201%:%
%:%417=202%:%
%:%419=205%:%
%:%420=205%:%
%:%421=206%:%
%:%422=207%:%
%:%423=208%:%
%:%430=209%:%
%:%431=209%:%
%:%432=210%:%
%:%433=210%:%
%:%434=210%:%
%:%435=211%:%
%:%436=211%:%
%:%437=211%:%
%:%438=212%:%
%:%439=212%:%
%:%440=212%:%
%:%441=212%:%
%:%442=213%:%
%:%443=213%:%
%:%444=213%:%
%:%445=214%:%
%:%446=214%:%
%:%447=215%:%
%:%448=215%:%
%:%449=215%:%
%:%450=216%:%
%:%451=216%:%
%:%452=217%:%
%:%453=217%:%
%:%454=217%:%
%:%455=218%:%
%:%456=218%:%
%:%457=219%:%
%:%458=219%:%
%:%459=220%:%
%:%460=220%:%
%:%462=222%:%
%:%463=223%:%
%:%464=223%:%
%:%465=224%:%
%:%466=224%:%
%:%467=225%:%
%:%468=225%:%
%:%469=226%:%
%:%470=226%:%
%:%471=227%:%
%:%472=227%:%
%:%473=227%:%
%:%474=228%:%
%:%475=228%:%
%:%476=229%:%
%:%477=229%:%
%:%478=230%:%
%:%484=230%:%
%:%487=231%:%
%:%488=232%:%
%:%489=232%:%
%:%490=233%:%
%:%491=234%:%
%:%492=235%:%
%:%493=236%:%
%:%494=236%:%
%:%495=237%:%
%:%496=238%:%
%:%497=239%:%
%:%498=240%:%
%:%499=241%:%
%:%500=242%:%
%:%501=242%:%
%:%502=243%:%
%:%504=245%:%
%:%506=247%:%
%:%507=247%:%
%:%508=248%:%
%:%509=249%:%
%:%510=250%:%
%:%511=251%:%
%:%512=252%:%
%:%513=252%:%
%:%514=253%:%
%:%515=254%:%
%:%516=255%:%
%:%517=256%:%
%:%518=257%:%
%:%519=258%:%
%:%520=259%:%
%:%521=259%:%
%:%522=260%:%
%:%523=261%:%
%:%524=262%:%
%:%525=263%:%
%:%526=264%:%
%:%527=265%:%
%:%528=266%:%
%:%529=266%:%
%:%530=267%:%
%:%531=268%:%
%:%532=269%:%
%:%533=269%:%
%:%534=270%:%
%:%535=271%:%
%:%536=272%:%
%:%537=273%:%
%:%538=274%:%
%:%539=275%:%
%:%540=276%:%
%:%541=276%:%
%:%542=277%:%
%:%543=278%:%
%:%544=279%:%
%:%545=280%:%
%:%546=281%:%
%:%547=282%:%
%:%548=283%:%
%:%549=283%:%
%:%550=284%:%
%:%551=285%:%
%:%552=286%:%
%:%553=287%:%
%:%554=288%:%
%:%555=289%:%
%:%556=290%:%
%:%557=290%:%
%:%558=291%:%
%:%559=292%:%
%:%560=293%:%
%:%561=293%:%
%:%562=294%:%
%:%563=295%:%
%:%564=296%:%
%:%565=297%:%
%:%566=298%:%
%:%568=300%:%
%:%569=301%:%
%:%570=302%:%
%:%571=303%:%
%:%572=303%:%
%:%573=304%:%
%:%574=305%:%
%:%575=306%:%
%:%576=307%:%
%:%577=308%:%
%:%578=309%:%
%:%579=310%:%
%:%580=311%:%
%:%581=312%:%
%:%582=312%:%
%:%583=313%:%
%:%584=314%:%
%:%585=315%:%
%:%586=316%:%
%:%587=317%:%
%:%588=318%:%
%:%589=319%:%
%:%590=319%:%
%:%591=320%:%
%:%592=321%:%
%:%593=322%:%
%:%594=323%:%
%:%595=324%:%
%:%596=325%:%
%:%597=326%:%
%:%598=326%:%
%:%599=327%:%
%:%600=328%:%
%:%601=329%:%
%:%602=330%:%
%:%603=331%:%
%:%606=332%:%
%:%611=333%:%

%
\begin{isabellebody}%
\setisabellecontext{Relational{\isacharunderscore}{\kern0pt}Equivalences}%
%
\isadelimtheory
%
\endisadelimtheory
%
\isatagtheory
%
\endisatagtheory
{\isafoldtheory}%
%
\isadelimtheory
\isanewline
%
\endisadelimtheory
\isacommand{context}\isamarkupfalse%
\ lts\isanewline
\isakeyword{begin}%
\begin{isamarkuptext}%
Introduce these definitions later?%
\end{isamarkuptext}\isamarkuptrue%
\isacommand{abbreviation}\isamarkupfalse%
\ traces\ {\isacharcolon}{\kern0pt}{\isacharcolon}{\kern0pt}\ {\isacartoucheopen}{\isacharprime}{\kern0pt}s\ {\isasymRightarrow}\ {\isacharprime}{\kern0pt}a\ list\ set{\isacartoucheclose}\ \isakeyword{where}\isanewline
{\isacartoucheopen}traces\ p\ {\isasymequiv}\ {\isacharbraceleft}{\kern0pt}tr{\isachardot}{\kern0pt}\ {\isasymexists}p{\isacharprime}{\kern0pt}{\isachardot}{\kern0pt}\ p\ {\isasymmapsto}{\isachardollar}{\kern0pt}\ tr\ p{\isacharprime}{\kern0pt}{\isacharbraceright}{\kern0pt}{\isacartoucheclose}\isanewline
\isanewline
\isacommand{abbreviation}\isamarkupfalse%
\ all{\isacharunderscore}{\kern0pt}traces\ {\isacharcolon}{\kern0pt}{\isacharcolon}{\kern0pt}\ {\isachardoublequoteopen}{\isacharprime}{\kern0pt}a\ list\ set{\isachardoublequoteclose}\ \isakeyword{where}\isanewline
{\isachardoublequoteopen}all{\isacharunderscore}{\kern0pt}traces\ {\isasymequiv}{\isacharbraceleft}{\kern0pt}tr{\isachardot}{\kern0pt}\ {\isasymexists}p\ p{\isacharprime}{\kern0pt}{\isachardot}{\kern0pt}\ p\ {\isasymmapsto}{\isachardollar}{\kern0pt}\ tr\ p{\isacharprime}{\kern0pt}{\isacharbraceright}{\kern0pt}{\isachardoublequoteclose}\isanewline
\isanewline
\isacommand{inductive}\isamarkupfalse%
\ paths{\isacharcolon}{\kern0pt}{\isacharcolon}{\kern0pt}\ {\isacartoucheopen}{\isacharprime}{\kern0pt}s\ {\isasymRightarrow}\ {\isacharprime}{\kern0pt}s\ list\ {\isasymRightarrow}\ {\isacharprime}{\kern0pt}s\ {\isasymRightarrow}\ bool{\isacartoucheclose}\ \isakeyword{where}\isanewline
{\isacartoucheopen}paths\ p\ {\isacharbrackleft}{\kern0pt}{\isacharbrackright}{\kern0pt}\ p{\isacartoucheclose}\ {\isacharbar}{\kern0pt}\isanewline
{\isacartoucheopen}paths\ p\ {\isacharparenleft}{\kern0pt}a{\isacharhash}{\kern0pt}as{\isacharparenright}{\kern0pt}\ p{\isacharprime}{\kern0pt}{\isacharprime}{\kern0pt}{\isacartoucheclose}\ \isakeyword{if}\ {\isachardoublequoteopen}{\isasymexists}{\isasymalpha}{\isachardot}{\kern0pt}\ p\ {\isasymmapsto}\ {\isasymalpha}\ a\ {\isasymand}\ {\isacharparenleft}{\kern0pt}paths\ a\ as\ p{\isacharprime}{\kern0pt}{\isacharprime}{\kern0pt}{\isacharparenright}{\kern0pt}{\isachardoublequoteclose}\isanewline
\isanewline
\isacommand{lemma}\isamarkupfalse%
\ path{\isacharunderscore}{\kern0pt}implies{\isacharunderscore}{\kern0pt}seq{\isacharcolon}{\kern0pt}\isanewline
\ \ \isakeyword{assumes}\ A{\isadigit{1}}{\isacharcolon}{\kern0pt}\ {\isachardoublequoteopen}{\isasymexists}xs{\isachardot}{\kern0pt}\ paths\ p\ xs\ p{\isacharprime}{\kern0pt}{\isachardoublequoteclose}\isanewline
\ \ \isakeyword{shows}\ {\isachardoublequoteopen}{\isasymexists}ys{\isachardot}{\kern0pt}\ p\ {\isasymmapsto}{\isachardollar}{\kern0pt}\ ys\ p{\isacharprime}{\kern0pt}{\isachardoublequoteclose}\isanewline
%
\isadelimproof
%
\endisadelimproof
%
\isatagproof
\isacommand{proof}\isamarkupfalse%
{\isacharminus}{\kern0pt}\isanewline
\ \ \isacommand{from}\isamarkupfalse%
\ A{\isadigit{1}}\ \isacommand{obtain}\isamarkupfalse%
\ xs\ \isakeyword{where}\ {\isachardoublequoteopen}paths\ p\ xs\ p{\isacharprime}{\kern0pt}{\isachardoublequoteclose}\ \isacommand{by}\isamarkupfalse%
\ auto\isanewline
\ \ \isacommand{then}\isamarkupfalse%
\ \isacommand{show}\isamarkupfalse%
\ {\isachardoublequoteopen}{\isasymexists}ys{\isachardot}{\kern0pt}\ p\ {\isasymmapsto}{\isachardollar}{\kern0pt}\ ys\ p{\isacharprime}{\kern0pt}{\isachardoublequoteclose}\isanewline
\isacommand{proof}\isamarkupfalse%
\ {\isacharparenleft}{\kern0pt}rule\ local{\isachardot}{\kern0pt}paths{\isachardot}{\kern0pt}induct{\isacharparenright}{\kern0pt}\isanewline
\ \ \isacommand{fix}\isamarkupfalse%
\ q\isanewline
\ \ \isacommand{have}\isamarkupfalse%
\ {\isachardoublequoteopen}q\ {\isasymmapsto}{\isachardollar}{\kern0pt}\ {\isacharbrackleft}{\kern0pt}{\isacharbrackright}{\kern0pt}\ q{\isachardoublequoteclose}\ \isacommand{using}\isamarkupfalse%
\ step{\isacharunderscore}{\kern0pt}sequence{\isachardot}{\kern0pt}intros{\isacharparenleft}{\kern0pt}{\isadigit{1}}{\isacharparenright}{\kern0pt}\isacommand{{\isachardot}{\kern0pt}}\isamarkupfalse%
\isanewline
\ \ \isacommand{then}\isamarkupfalse%
\ \isacommand{show}\isamarkupfalse%
\ {\isachardoublequoteopen}{\isasymexists}ys{\isachardot}{\kern0pt}\ q\ {\isasymmapsto}{\isachardollar}{\kern0pt}\ ys\ q{\isachardoublequoteclose}\ \isacommand{by}\isamarkupfalse%
\ {\isacharparenleft}{\kern0pt}rule\ exI{\isacharparenright}{\kern0pt}\isanewline
\isacommand{next}\isamarkupfalse%
\isanewline
\ \ \isacommand{fix}\isamarkupfalse%
\ p\ a\ as\ p{\isacharprime}{\kern0pt}{\isacharprime}{\kern0pt}\isanewline
\ \ \isacommand{assume}\isamarkupfalse%
\ A{\isadigit{1}}{\isacharcolon}{\kern0pt}\ {\isachardoublequoteopen}{\isasymexists}{\isasymalpha}{\isachardot}{\kern0pt}\ p\ {\isasymmapsto}{\isasymalpha}\ a\ {\isasymand}\ paths\ a\ as\ p{\isacharprime}{\kern0pt}{\isacharprime}{\kern0pt}\ {\isasymand}\ {\isacharparenleft}{\kern0pt}{\isasymexists}ys{\isachardot}{\kern0pt}\ a\ {\isasymmapsto}{\isachardollar}{\kern0pt}\ ys\ p{\isacharprime}{\kern0pt}{\isacharprime}{\kern0pt}{\isacharparenright}{\kern0pt}{\isachardoublequoteclose}\isanewline
\ \ \isacommand{then}\isamarkupfalse%
\ \isacommand{obtain}\isamarkupfalse%
\ ys\ {\isasymalpha}\ \isakeyword{where}\ A{\isadigit{2}}{\isacharcolon}{\kern0pt}\ {\isachardoublequoteopen}a\ {\isasymmapsto}{\isachardollar}{\kern0pt}\ ys\ p{\isacharprime}{\kern0pt}{\isacharprime}{\kern0pt}{\isachardoublequoteclose}\ \isakeyword{and}\ A{\isadigit{3}}{\isacharcolon}{\kern0pt}\ {\isachardoublequoteopen}p\ {\isasymmapsto}{\isasymalpha}\ a{\isachardoublequoteclose}\isanewline
\ \ \ \ \isacommand{by}\isamarkupfalse%
\ blast\isanewline
\ \ \isacommand{then}\isamarkupfalse%
\ \isacommand{have}\isamarkupfalse%
\ {\isachardoublequoteopen}p\ {\isasymmapsto}{\isachardollar}{\kern0pt}\ {\isacharparenleft}{\kern0pt}{\isasymalpha}{\isacharhash}{\kern0pt}ys{\isacharparenright}{\kern0pt}\ p{\isacharprime}{\kern0pt}{\isacharprime}{\kern0pt}{\isachardoublequoteclose}\ \isacommand{using}\isamarkupfalse%
\ step{\isacharunderscore}{\kern0pt}sequence{\isachardot}{\kern0pt}intros{\isacharparenleft}{\kern0pt}{\isadigit{2}}{\isacharparenright}{\kern0pt}\isanewline
\ \ \ \ \isacommand{by}\isamarkupfalse%
\ blast\isanewline
\ \ \isacommand{then}\isamarkupfalse%
\ \isacommand{show}\isamarkupfalse%
\ {\isachardoublequoteopen}{\isasymexists}ys{\isachardot}{\kern0pt}\ p\ {\isasymmapsto}{\isachardollar}{\kern0pt}\ ys\ p{\isacharprime}{\kern0pt}{\isacharprime}{\kern0pt}{\isachardoublequoteclose}\ \isacommand{by}\isamarkupfalse%
\ {\isacharparenleft}{\kern0pt}rule\ exI{\isacharparenright}{\kern0pt}\isanewline
\ \ \isacommand{qed}\isamarkupfalse%
\isanewline
\isacommand{qed}\isamarkupfalse%
%
\endisatagproof
{\isafoldproof}%
%
\isadelimproof
\isanewline
%
\endisadelimproof
\isanewline
\isacommand{lemma}\isamarkupfalse%
\ seq{\isacharunderscore}{\kern0pt}implies{\isacharunderscore}{\kern0pt}path{\isacharcolon}{\kern0pt}\isanewline
\ \ \isakeyword{assumes}\ A{\isadigit{1}}{\isacharcolon}{\kern0pt}\ {\isachardoublequoteopen}{\isasymexists}ys{\isachardot}{\kern0pt}\ p\ {\isasymmapsto}{\isachardollar}{\kern0pt}\ ys\ p{\isacharprime}{\kern0pt}{\isachardoublequoteclose}\isanewline
\ \ \isakeyword{shows}\ {\isachardoublequoteopen}{\isasymexists}xs{\isachardot}{\kern0pt}\ paths\ p\ xs\ p{\isacharprime}{\kern0pt}{\isachardoublequoteclose}\isanewline
%
\isadelimproof
%
\endisadelimproof
%
\isatagproof
\isacommand{proof}\isamarkupfalse%
{\isacharminus}{\kern0pt}\isanewline
\ \ \isacommand{from}\isamarkupfalse%
\ A{\isadigit{1}}\ \isacommand{obtain}\isamarkupfalse%
\ ys\ \isakeyword{where}\ {\isachardoublequoteopen}p\ {\isasymmapsto}{\isachardollar}{\kern0pt}\ ys\ p{\isacharprime}{\kern0pt}{\isachardoublequoteclose}\ \isacommand{by}\isamarkupfalse%
\ auto\isanewline
\ \ \isacommand{then}\isamarkupfalse%
\ \isacommand{show}\isamarkupfalse%
\ {\isachardoublequoteopen}{\isasymexists}xs{\isachardot}{\kern0pt}\ paths\ p\ xs\ p{\isacharprime}{\kern0pt}{\isachardoublequoteclose}\isanewline
\ \ \isacommand{proof}\isamarkupfalse%
{\isacharparenleft}{\kern0pt}rule\ step{\isacharunderscore}{\kern0pt}sequence{\isachardot}{\kern0pt}induct{\isacharparenright}{\kern0pt}\isanewline
\ \ \ \ \isacommand{fix}\isamarkupfalse%
\ p\isanewline
\ \ \ \ \isacommand{have}\isamarkupfalse%
\ {\isachardoublequoteopen}paths\ p\ {\isacharbrackleft}{\kern0pt}{\isacharbrackright}{\kern0pt}\ p{\isachardoublequoteclose}\ \isacommand{using}\isamarkupfalse%
\ paths{\isachardot}{\kern0pt}intros{\isacharparenleft}{\kern0pt}{\isadigit{1}}{\isacharparenright}{\kern0pt}\isacommand{{\isachardot}{\kern0pt}}\isamarkupfalse%
\isanewline
\ \ \ \ \isacommand{then}\isamarkupfalse%
\ \isacommand{show}\isamarkupfalse%
\ {\isachardoublequoteopen}{\isasymexists}xs{\isachardot}{\kern0pt}\ paths\ p\ xs\ p{\isachardoublequoteclose}\ \isacommand{by}\isamarkupfalse%
\ {\isacharparenleft}{\kern0pt}rule\ exI{\isacharparenright}{\kern0pt}\isanewline
\ \ \isacommand{next}\isamarkupfalse%
\isanewline
\ \ \ \ \isacommand{fix}\isamarkupfalse%
\ p\ a\ rt\ p{\isacharprime}{\kern0pt}{\isacharprime}{\kern0pt}\isanewline
\ \ \ \ \isacommand{assume}\isamarkupfalse%
\ {\isachardoublequoteopen}{\isasymexists}p{\isacharprime}{\kern0pt}{\isachardot}{\kern0pt}\ p\ {\isasymmapsto}a\ p{\isacharprime}{\kern0pt}\ {\isasymand}\ p{\isacharprime}{\kern0pt}\ {\isasymmapsto}{\isachardollar}{\kern0pt}\ rt\ p{\isacharprime}{\kern0pt}{\isacharprime}{\kern0pt}\ {\isasymand}\ {\isacharparenleft}{\kern0pt}{\isasymexists}xs{\isachardot}{\kern0pt}\ paths\ p{\isacharprime}{\kern0pt}\ xs\ p{\isacharprime}{\kern0pt}{\isacharprime}{\kern0pt}{\isacharparenright}{\kern0pt}{\isachardoublequoteclose}\isanewline
\ \ \ \ \isacommand{then}\isamarkupfalse%
\ \isacommand{obtain}\isamarkupfalse%
\ p{\isacharprime}{\kern0pt}\ xs\ \isakeyword{where}\ {\isachardoublequoteopen}p\ {\isasymmapsto}a\ p{\isacharprime}{\kern0pt}{\isachardoublequoteclose}\ \isakeyword{and}\ {\isachardoublequoteopen}paths\ p{\isacharprime}{\kern0pt}\ xs\ p{\isacharprime}{\kern0pt}{\isacharprime}{\kern0pt}{\isachardoublequoteclose}\ \isacommand{by}\isamarkupfalse%
\ auto\isanewline
\ \ \ \ \isacommand{then}\isamarkupfalse%
\ \isacommand{have}\isamarkupfalse%
\ {\isachardoublequoteopen}paths\ p\ {\isacharparenleft}{\kern0pt}p{\isacharprime}{\kern0pt}{\isacharhash}{\kern0pt}xs{\isacharparenright}{\kern0pt}\ p{\isacharprime}{\kern0pt}{\isacharprime}{\kern0pt}{\isachardoublequoteclose}\ \isacommand{using}\isamarkupfalse%
\ paths{\isachardot}{\kern0pt}intros{\isacharparenleft}{\kern0pt}{\isadigit{2}}{\isacharparenright}{\kern0pt}\ \isacommand{by}\isamarkupfalse%
\ blast\isanewline
\ \ \ \ \isacommand{then}\isamarkupfalse%
\ \isacommand{show}\isamarkupfalse%
\ {\isachardoublequoteopen}{\isasymexists}xs{\isachardot}{\kern0pt}\ paths\ p\ xs\ p{\isacharprime}{\kern0pt}{\isacharprime}{\kern0pt}{\isachardoublequoteclose}\ \isacommand{by}\isamarkupfalse%
\ {\isacharparenleft}{\kern0pt}rule\ exI{\isacharparenright}{\kern0pt}\isanewline
\ \ \isacommand{qed}\isamarkupfalse%
\isanewline
\isacommand{qed}\isamarkupfalse%
%
\endisatagproof
{\isafoldproof}%
%
\isadelimproof
%
\endisadelimproof
%
\begin{isamarkuptext}%
Trace preorder as inclusion of trace sets%
\end{isamarkuptext}\isamarkuptrue%
\isacommand{definition}\isamarkupfalse%
\ trace{\isacharunderscore}{\kern0pt}preordered\ {\isacharparenleft}{\kern0pt}\isakeyword{infix}\ {\isacartoucheopen}{\isasymlesssim}T{\isacartoucheclose}\ {\isadigit{6}}{\isadigit{0}}{\isacharparenright}{\kern0pt}\isakeyword{where}\isanewline
{\isacartoucheopen}trace{\isacharunderscore}{\kern0pt}preordered\ p\ q\ {\isasymequiv}\ traces\ p\ {\isasymsubseteq}\ traces\ q{\isacartoucheclose}%
\begin{isamarkuptext}%
Trace equivalence as mutual preorder%
\end{isamarkuptext}\isamarkuptrue%
\isacommand{abbreviation}\isamarkupfalse%
\ trace{\isacharunderscore}{\kern0pt}equivalent\ {\isacharparenleft}{\kern0pt}\isakeyword{infix}\ {\isacartoucheopen}{\isasymsimeq}T{\isacartoucheclose}\ {\isadigit{6}}{\isadigit{0}}{\isacharparenright}{\kern0pt}\ \isakeyword{where}\isanewline
{\isacartoucheopen}p\ {\isasymsimeq}T\ q\ {\isasymequiv}\ p\ {\isasymlesssim}T\ q\ {\isasymand}\ q\ {\isasymlesssim}T\ p{\isacartoucheclose}%
\begin{isamarkuptext}%
Trace preorder is transitive%
\end{isamarkuptext}\isamarkuptrue%
\isacommand{lemma}\isamarkupfalse%
\ trace{\isacharunderscore}{\kern0pt}preorder{\isacharunderscore}{\kern0pt}transitive{\isacharcolon}{\kern0pt}\isanewline
\ \ \isakeyword{shows}\ {\isacartoucheopen}transp\ {\isacharparenleft}{\kern0pt}{\isasymlesssim}T{\isacharparenright}{\kern0pt}{\isacartoucheclose}\isanewline
%
\isadelimproof
\ \ %
\endisadelimproof
%
\isatagproof
\isacommand{unfolding}\isamarkupfalse%
\ transp{\isacharunderscore}{\kern0pt}def\ trace{\isacharunderscore}{\kern0pt}preordered{\isacharunderscore}{\kern0pt}def\ \isacommand{by}\isamarkupfalse%
\ blast%
\endisatagproof
{\isafoldproof}%
%
\isadelimproof
\isanewline
%
\endisadelimproof
\isanewline
\isacommand{lemma}\isamarkupfalse%
\ empty{\isacharunderscore}{\kern0pt}trace{\isacharunderscore}{\kern0pt}trivial{\isacharcolon}{\kern0pt}\isanewline
\ \ \isakeyword{fixes}\ p\isanewline
\ \ \isakeyword{shows}\ {\isacartoucheopen}{\isacharbrackleft}{\kern0pt}{\isacharbrackright}{\kern0pt}\ {\isasymin}\ traces\ p{\isacartoucheclose}\isanewline
%
\isadelimproof
\ \ %
\endisadelimproof
%
\isatagproof
\isacommand{using}\isamarkupfalse%
\ step{\isacharunderscore}{\kern0pt}sequence{\isachardot}{\kern0pt}intros\ \isacommand{by}\isamarkupfalse%
\ blast%
\endisatagproof
{\isafoldproof}%
%
\isadelimproof
\isanewline
%
\endisadelimproof
\isanewline
\isacommand{lemma}\isamarkupfalse%
\ {\isacartoucheopen}equivp\ {\isacharparenleft}{\kern0pt}{\isasymsimeq}T{\isacharparenright}{\kern0pt}{\isacartoucheclose}\isanewline
%
\isadelimproof
%
\endisadelimproof
%
\isatagproof
\isacommand{proof}\isamarkupfalse%
\ {\isacharparenleft}{\kern0pt}rule\ equivpI{\isacharparenright}{\kern0pt}\isanewline
\ \ \isacommand{show}\isamarkupfalse%
\ {\isacartoucheopen}reflp\ {\isacharparenleft}{\kern0pt}{\isasymsimeq}T{\isacharparenright}{\kern0pt}{\isacartoucheclose}\isanewline
\ \ \ \ \isacommand{unfolding}\isamarkupfalse%
\ reflp{\isacharunderscore}{\kern0pt}def\ trace{\isacharunderscore}{\kern0pt}preordered{\isacharunderscore}{\kern0pt}def\ \isacommand{by}\isamarkupfalse%
\ blast\isanewline
\ \ \isacommand{show}\isamarkupfalse%
\ {\isacartoucheopen}symp\ {\isacharparenleft}{\kern0pt}{\isasymsimeq}T{\isacharparenright}{\kern0pt}{\isacartoucheclose}\isanewline
\ \ \ \ \isacommand{unfolding}\isamarkupfalse%
\ symp{\isacharunderscore}{\kern0pt}def\ \isacommand{by}\isamarkupfalse%
\ blast\isanewline
\ \ \isacommand{show}\isamarkupfalse%
\ {\isacartoucheopen}transp\ {\isacharparenleft}{\kern0pt}{\isasymsimeq}T{\isacharparenright}{\kern0pt}{\isacartoucheclose}\isanewline
\ \ \ \ \isacommand{unfolding}\isamarkupfalse%
\ transp{\isacharunderscore}{\kern0pt}def\ trace{\isacharunderscore}{\kern0pt}preordered{\isacharunderscore}{\kern0pt}def\ \isacommand{by}\isamarkupfalse%
\ blast\isanewline
\isacommand{qed}\isamarkupfalse%
%
\endisatagproof
{\isafoldproof}%
%
\isadelimproof
%
\endisadelimproof
%
\begin{isamarkuptext}%
Failure Pairs%
\end{isamarkuptext}\isamarkuptrue%
\isacommand{abbreviation}\isamarkupfalse%
\ failure{\isacharunderscore}{\kern0pt}pairs\ {\isacharcolon}{\kern0pt}{\isacharcolon}{\kern0pt}\ {\isacartoucheopen}{\isacharprime}{\kern0pt}s\ {\isasymRightarrow}\ {\isacharparenleft}{\kern0pt}{\isacharprime}{\kern0pt}a\ list\ {\isasymtimes}\ {\isacharprime}{\kern0pt}a\ set{\isacharparenright}{\kern0pt}\ set{\isacartoucheclose}\isanewline
\ \ \isakeyword{where}\isanewline
{\isacartoucheopen}failure{\isacharunderscore}{\kern0pt}pairs\ p\ {\isasymequiv}\ {\isacharbraceleft}{\kern0pt}{\isacharparenleft}{\kern0pt}xs{\isacharcomma}{\kern0pt}\ F{\isacharparenright}{\kern0pt}{\isacharbar}{\kern0pt}xs\ F{\isachardot}{\kern0pt}\ {\isasymexists}p{\isacharprime}{\kern0pt}{\isachardot}{\kern0pt}\ p\ {\isasymmapsto}{\isachardollar}{\kern0pt}\ xs\ p{\isacharprime}{\kern0pt}\ {\isasymand}\ {\isacharparenleft}{\kern0pt}initial{\isacharunderscore}{\kern0pt}actions\ p{\isacharprime}{\kern0pt}\ {\isasyminter}\ F\ {\isacharequal}{\kern0pt}\ {\isacharbraceleft}{\kern0pt}{\isacharbraceright}{\kern0pt}{\isacharparenright}{\kern0pt}{\isacharbraceright}{\kern0pt}{\isacartoucheclose}%
\begin{isamarkuptext}%
Failure preorder and -equivalence%
\end{isamarkuptext}\isamarkuptrue%
\isacommand{definition}\isamarkupfalse%
\ failure{\isacharunderscore}{\kern0pt}preordered\ {\isacharparenleft}{\kern0pt}\isakeyword{infix}\ {\isacartoucheopen}{\isasymlesssim}F{\isacartoucheclose}\ {\isadigit{6}}{\isadigit{0}}{\isacharparenright}{\kern0pt}\ \isakeyword{where}\isanewline
{\isacartoucheopen}p\ {\isasymlesssim}F\ q\ {\isasymequiv}\ failure{\isacharunderscore}{\kern0pt}pairs\ p\ {\isasymsubseteq}\ failure{\isacharunderscore}{\kern0pt}pairs\ q{\isacartoucheclose}\isanewline
\isanewline
\isacommand{abbreviation}\isamarkupfalse%
\ failure{\isacharunderscore}{\kern0pt}equivalent\ {\isacharparenleft}{\kern0pt}\isakeyword{infix}\ {\isacartoucheopen}{\isasymsimeq}F{\isacartoucheclose}\ {\isadigit{6}}{\isadigit{0}}{\isacharparenright}{\kern0pt}\ \isakeyword{where}\isanewline
{\isacartoucheopen}\ p\ {\isasymsimeq}F\ q\ {\isasymequiv}\ p\ {\isasymlesssim}F\ q\ {\isasymand}\ q\ {\isasymlesssim}F\ p{\isacartoucheclose}%
\begin{isamarkuptext}%
Possible future sets%
\end{isamarkuptext}\isamarkuptrue%
\isacommand{abbreviation}\isamarkupfalse%
\ possible{\isacharunderscore}{\kern0pt}future{\isacharunderscore}{\kern0pt}pairs\ {\isacharcolon}{\kern0pt}{\isacharcolon}{\kern0pt}\ {\isacartoucheopen}{\isacharprime}{\kern0pt}s\ {\isasymRightarrow}\ {\isacharparenleft}{\kern0pt}{\isacharprime}{\kern0pt}a\ list\ {\isasymtimes}\ {\isacharprime}{\kern0pt}a\ list\ set{\isacharparenright}{\kern0pt}\ set{\isacartoucheclose}\isanewline
\ \ \isakeyword{where}\isanewline
{\isacartoucheopen}possible{\isacharunderscore}{\kern0pt}future{\isacharunderscore}{\kern0pt}pairs\ p\ {\isasymequiv}\ {\isacharbraceleft}{\kern0pt}{\isacharparenleft}{\kern0pt}xs{\isacharcomma}{\kern0pt}\ X{\isacharparenright}{\kern0pt}{\isacharbar}{\kern0pt}xs\ X{\isachardot}{\kern0pt}\ {\isasymexists}p{\isacharprime}{\kern0pt}{\isachardot}{\kern0pt}\ p\ {\isasymmapsto}{\isachardollar}{\kern0pt}\ xs\ p{\isacharprime}{\kern0pt}\ {\isasymand}\ traces\ p{\isacharprime}{\kern0pt}\ {\isacharequal}{\kern0pt}\ X{\isacharbraceright}{\kern0pt}{\isacartoucheclose}\isanewline
\isanewline
\isacommand{definition}\isamarkupfalse%
\ possible{\isacharunderscore}{\kern0pt}futures{\isacharunderscore}{\kern0pt}preordered\ {\isacharparenleft}{\kern0pt}\isakeyword{infix}\ {\isacartoucheopen}{\isasymlesssim}PF{\isacartoucheclose}\ {\isadigit{6}}{\isadigit{0}}{\isacharparenright}{\kern0pt}\ \isakeyword{where}\isanewline
{\isacartoucheopen}p\ {\isasymlesssim}PF\ q\ {\isasymequiv}\ {\isacharparenleft}{\kern0pt}possible{\isacharunderscore}{\kern0pt}future{\isacharunderscore}{\kern0pt}pairs\ p\ {\isasymsubseteq}\ possible{\isacharunderscore}{\kern0pt}future{\isacharunderscore}{\kern0pt}pairs\ q{\isacharparenright}{\kern0pt}{\isacartoucheclose}\isanewline
\isanewline
\isacommand{definition}\isamarkupfalse%
\ possible{\isacharunderscore}{\kern0pt}futures{\isacharunderscore}{\kern0pt}equivalent\ {\isacharparenleft}{\kern0pt}\isakeyword{infix}\ {\isacartoucheopen}{\isasymsimeq}PF{\isacartoucheclose}\ {\isadigit{6}}{\isadigit{0}}{\isacharparenright}{\kern0pt}\ \isakeyword{where}\isanewline
{\isacartoucheopen}p\ {\isasymsimeq}PF\ q\ {\isasymequiv}\ {\isacharparenleft}{\kern0pt}possible{\isacharunderscore}{\kern0pt}future{\isacharunderscore}{\kern0pt}pairs\ p\ {\isacharequal}{\kern0pt}\ possible{\isacharunderscore}{\kern0pt}future{\isacharunderscore}{\kern0pt}pairs\ q{\isacharparenright}{\kern0pt}{\isacartoucheclose}\isanewline
\isanewline
\isacommand{lemma}\isamarkupfalse%
\ PF{\isacharunderscore}{\kern0pt}trans{\isacharcolon}{\kern0pt}\ {\isachardoublequoteopen}transp\ {\isacharparenleft}{\kern0pt}{\isasymsimeq}PF{\isacharparenright}{\kern0pt}{\isachardoublequoteclose}\isanewline
%
\isadelimproof
\ \ %
\endisadelimproof
%
\isatagproof
\isacommand{unfolding}\isamarkupfalse%
\ possible{\isacharunderscore}{\kern0pt}futures{\isacharunderscore}{\kern0pt}equivalent{\isacharunderscore}{\kern0pt}def\isanewline
\ \ \isacommand{by}\isamarkupfalse%
\ {\isacharparenleft}{\kern0pt}simp\ add{\isacharcolon}{\kern0pt}\ transp{\isacharunderscore}{\kern0pt}def{\isacharparenright}{\kern0pt}%
\endisatagproof
{\isafoldproof}%
%
\isadelimproof
\isanewline
%
\endisadelimproof
\isanewline
\isacommand{lemma}\isamarkupfalse%
\ pf{\isacharunderscore}{\kern0pt}implies{\isacharunderscore}{\kern0pt}trace{\isacharunderscore}{\kern0pt}preord{\isacharcolon}{\kern0pt}\isanewline
\ \ \isakeyword{assumes}\ {\isacartoucheopen}p\ {\isasymlesssim}PF\ q{\isacartoucheclose}\isanewline
\ \ \isakeyword{shows}\ {\isacartoucheopen}p\ {\isasymlesssim}T\ q{\isacartoucheclose}\isanewline
%
\isadelimproof
\ \ %
\endisadelimproof
%
\isatagproof
\isacommand{using}\isamarkupfalse%
\ assms\ \isacommand{unfolding}\isamarkupfalse%
\ trace{\isacharunderscore}{\kern0pt}preordered{\isacharunderscore}{\kern0pt}def\isanewline
\isacommand{proof}\isamarkupfalse%
\ safe\isanewline
\ \ \isacommand{fix}\isamarkupfalse%
\ p{\isacharprime}{\kern0pt}{\isacharprime}{\kern0pt}\ tr\isanewline
\ \ \isacommand{assume}\isamarkupfalse%
\ {\isacartoucheopen}p\ {\isasymlesssim}PF\ q{\isacartoucheclose}\ \isakeyword{and}\ {\isacartoucheopen}p\ {\isasymmapsto}{\isachardollar}{\kern0pt}\ tr\ p{\isacharprime}{\kern0pt}{\isacharprime}{\kern0pt}{\isacartoucheclose}\isanewline
\ \ \isacommand{thus}\isamarkupfalse%
\ {\isacartoucheopen}{\isasymexists}q{\isacharprime}{\kern0pt}{\isachardot}{\kern0pt}\ q\ {\isasymmapsto}{\isachardollar}{\kern0pt}\ tr\ q{\isacharprime}{\kern0pt}{\isacartoucheclose}\isanewline
\ \ \isacommand{proof}\isamarkupfalse%
\ {\isacharparenleft}{\kern0pt}induct\ tr\ arbitrary{\isacharcolon}{\kern0pt}\ p\ q{\isacharparenright}{\kern0pt}\isanewline
\ \ \ \ \isacommand{case}\isamarkupfalse%
\ Nil\isanewline
\ \ \ \ \isacommand{show}\isamarkupfalse%
\ {\isacharquery}{\kern0pt}case\ \isacommand{using}\isamarkupfalse%
\ step{\isacharunderscore}{\kern0pt}sequence{\isachardot}{\kern0pt}intros{\isacharparenleft}{\kern0pt}{\isadigit{1}}{\isacharparenright}{\kern0pt}\ \isacommand{by}\isamarkupfalse%
\ blast\isanewline
\ \ \isacommand{next}\isamarkupfalse%
\isanewline
\ \ \ \ \isacommand{case}\isamarkupfalse%
\ {\isacharparenleft}{\kern0pt}Cons\ a\ tr{\isacharparenright}{\kern0pt}\isanewline
\ \ \ \ \isacommand{from}\isamarkupfalse%
\ Cons\ \isacommand{have}\isamarkupfalse%
\ {\isachardoublequoteopen}{\isasymexists}q{\isacharprime}{\kern0pt}{\isacharprime}{\kern0pt}{\isachardot}{\kern0pt}\ q\ {\isasymmapsto}{\isachardollar}{\kern0pt}\ {\isacharparenleft}{\kern0pt}a{\isacharhash}{\kern0pt}tr{\isacharparenright}{\kern0pt}\ q{\isacharprime}{\kern0pt}{\isacharprime}{\kern0pt}{\isachardoublequoteclose}\ \isacommand{using}\isamarkupfalse%
\ Cons{\isacharparenleft}{\kern0pt}{\isadigit{2}}{\isacharparenright}{\kern0pt}\ \isanewline
\ \ \ \ \ \ \isacommand{unfolding}\isamarkupfalse%
\ possible{\isacharunderscore}{\kern0pt}futures{\isacharunderscore}{\kern0pt}preordered{\isacharunderscore}{\kern0pt}def\ \isanewline
\ \ \ \ \ \ \isacommand{by}\isamarkupfalse%
\ {\isacharparenleft}{\kern0pt}smt\ {\isacharparenleft}{\kern0pt}z{\isadigit{3}}{\isacharparenright}{\kern0pt}\ Collect{\isacharunderscore}{\kern0pt}mono{\isacharunderscore}{\kern0pt}iff\ prod{\isachardot}{\kern0pt}inject{\isacharparenright}{\kern0pt}\isanewline
\ \ \ \ \isacommand{then}\isamarkupfalse%
\ \isacommand{show}\isamarkupfalse%
\ {\isacharquery}{\kern0pt}case\ \isanewline
\ \ \ \ \ \ \isacommand{by}\isamarkupfalse%
\ blast\isanewline
\ \ \isacommand{qed}\isamarkupfalse%
\isanewline
\isacommand{qed}\isamarkupfalse%
%
\endisatagproof
{\isafoldproof}%
%
\isadelimproof
%
\endisadelimproof
%
\begin{isamarkuptext}%
isomorphism%
\end{isamarkuptext}\isamarkuptrue%
\isacommand{definition}\isamarkupfalse%
\ isomorphism\ {\isacharcolon}{\kern0pt}{\isacharcolon}{\kern0pt}\ {\isacartoucheopen}{\isacharparenleft}{\kern0pt}{\isacharprime}{\kern0pt}s\ {\isasymRightarrow}\ {\isacharprime}{\kern0pt}s{\isacharparenright}{\kern0pt}\ {\isasymRightarrow}\ bool{\isacartoucheclose}\ \isakeyword{where}\isanewline
{\isacartoucheopen}isomorphism\ f\ {\isasymequiv}\ bij\ f\ {\isasymand}\ {\isacharparenleft}{\kern0pt}{\isasymforall}p\ a\ p{\isacharprime}{\kern0pt}{\isachardot}{\kern0pt}\ p\ {\isasymmapsto}\ a\ p{\isacharprime}{\kern0pt}\ {\isasymlongleftrightarrow}\ f\ p\ {\isasymmapsto}\ a\ {\isacharparenleft}{\kern0pt}f\ p{\isacharprime}{\kern0pt}{\isacharparenright}{\kern0pt}{\isacharparenright}{\kern0pt}{\isacartoucheclose}\isanewline
\isanewline
\isacommand{definition}\isamarkupfalse%
\ is{\isacharunderscore}{\kern0pt}isomorphic\ {\isacharcolon}{\kern0pt}{\isacharcolon}{\kern0pt}\ {\isacartoucheopen}{\isacharprime}{\kern0pt}s\ {\isasymRightarrow}\ {\isacharprime}{\kern0pt}s\ {\isasymRightarrow}\ bool{\isacartoucheclose}\ {\isacharparenleft}{\kern0pt}\isakeyword{infix}\ {\isacartoucheopen}{\isasymsimeq}ISO{\isacartoucheclose}\ {\isadigit{6}}{\isadigit{0}}{\isacharparenright}{\kern0pt}\ \isakeyword{where}\isanewline
{\isacartoucheopen}p\ {\isasymsimeq}ISO\ q\ {\isasymequiv}\ {\isasymexists}f{\isachardot}{\kern0pt}\ isomorphism\ f\ {\isasymand}\ {\isacharparenleft}{\kern0pt}f\ p{\isacharparenright}{\kern0pt}\ {\isacharequal}{\kern0pt}\ q{\isacartoucheclose}%
\begin{isamarkuptext}%
Two states are simulation preordered if they can be related by
  a simulation relation. (Implied by isometry.)%
\end{isamarkuptext}\isamarkuptrue%
\isacommand{definition}\isamarkupfalse%
\ simulation\isanewline
\ \ \isakeyword{where}\ {\isacartoucheopen}simulation\ R\ {\isasymequiv}\isanewline
\ \ \ \ {\isasymforall}p\ q\ a\ p{\isacharprime}{\kern0pt}{\isachardot}{\kern0pt}\ p\ {\isasymmapsto}\ a\ p{\isacharprime}{\kern0pt}\ {\isasymand}\ R\ p\ q\ {\isasymlongrightarrow}\ {\isacharparenleft}{\kern0pt}{\isasymexists}q{\isacharprime}{\kern0pt}{\isachardot}{\kern0pt}\ q\ {\isasymmapsto}\ a\ q{\isacharprime}{\kern0pt}\ {\isasymand}\ R\ p{\isacharprime}{\kern0pt}\ q{\isacharprime}{\kern0pt}{\isacharparenright}{\kern0pt}{\isacartoucheclose}\isanewline
\isanewline
\isacommand{definition}\isamarkupfalse%
\ simulated{\isacharunderscore}{\kern0pt}by\ {\isacharparenleft}{\kern0pt}\isakeyword{infix}\ {\isacartoucheopen}{\isasymlesssim}S{\isacartoucheclose}\ {\isadigit{6}}{\isadigit{0}}{\isacharparenright}{\kern0pt}\isanewline
\ \ \isakeyword{where}\ {\isacartoucheopen}p\ {\isasymlesssim}S\ q\ {\isasymequiv}\ {\isasymexists}R{\isachardot}{\kern0pt}\ R\ p\ q\ {\isasymand}\ simulation\ R{\isacartoucheclose}%
\begin{isamarkuptext}%
Simulation preorder implies trace preorder%
\end{isamarkuptext}\isamarkuptrue%
\isacommand{lemma}\isamarkupfalse%
\ sim{\isacharunderscore}{\kern0pt}implies{\isacharunderscore}{\kern0pt}trace{\isacharunderscore}{\kern0pt}preord{\isacharcolon}{\kern0pt}\isanewline
\ \ \isakeyword{assumes}\ {\isacartoucheopen}p\ {\isasymlesssim}S\ q{\isacartoucheclose}\isanewline
\ \ \isakeyword{shows}\ {\isacartoucheopen}p\ {\isasymlesssim}T\ q{\isacartoucheclose}\isanewline
%
\isadelimproof
\ \ %
\endisadelimproof
%
\isatagproof
\isacommand{using}\isamarkupfalse%
\ assms\ \isacommand{unfolding}\isamarkupfalse%
\ trace{\isacharunderscore}{\kern0pt}preordered{\isacharunderscore}{\kern0pt}def\isanewline
\isacommand{proof}\isamarkupfalse%
\ safe\isanewline
\ \ \isacommand{fix}\isamarkupfalse%
\ p{\isacharprime}{\kern0pt}{\isacharprime}{\kern0pt}\ tr\isanewline
\ \ \isacommand{assume}\isamarkupfalse%
\ {\isacartoucheopen}p\ {\isasymlesssim}S\ q{\isacartoucheclose}\ \isakeyword{and}\ {\isacartoucheopen}p\ {\isasymmapsto}{\isachardollar}{\kern0pt}\ tr\ p{\isacharprime}{\kern0pt}{\isacharprime}{\kern0pt}{\isacartoucheclose}\isanewline
\ \ \isacommand{thus}\isamarkupfalse%
\ {\isacartoucheopen}{\isasymexists}q{\isacharprime}{\kern0pt}{\isachardot}{\kern0pt}\ q\ {\isasymmapsto}{\isachardollar}{\kern0pt}\ tr\ q{\isacharprime}{\kern0pt}{\isacartoucheclose}\isanewline
\ \ \isacommand{proof}\isamarkupfalse%
\ {\isacharparenleft}{\kern0pt}induct\ tr\ arbitrary{\isacharcolon}{\kern0pt}\ p\ q{\isacharparenright}{\kern0pt}\isanewline
\ \ \ \ \isacommand{case}\isamarkupfalse%
\ Nil\isanewline
\ \ \ \ \isacommand{show}\isamarkupfalse%
\ {\isacharquery}{\kern0pt}case\ \isacommand{using}\isamarkupfalse%
\ step{\isacharunderscore}{\kern0pt}sequence{\isachardot}{\kern0pt}intros{\isacharparenleft}{\kern0pt}{\isadigit{1}}{\isacharparenright}{\kern0pt}\ \isacommand{by}\isamarkupfalse%
\ blast\isanewline
\ \ \isacommand{next}\isamarkupfalse%
\isanewline
\ \ \ \ \isacommand{case}\isamarkupfalse%
\ {\isacharparenleft}{\kern0pt}Cons\ a\ tr{\isacharparenright}{\kern0pt}\isanewline
\ \ \ \ \isacommand{obtain}\isamarkupfalse%
\ p{\isacharprime}{\kern0pt}\ \isakeyword{where}\ {\isacartoucheopen}p\ {\isasymmapsto}\ a\ p{\isacharprime}{\kern0pt}{\isacartoucheclose}\ {\isacartoucheopen}p{\isacharprime}{\kern0pt}\ {\isasymmapsto}{\isachardollar}{\kern0pt}\ tr\ p{\isacharprime}{\kern0pt}{\isacharprime}{\kern0pt}{\isacartoucheclose}\isanewline
\ \ \ \ \ \ \isacommand{using}\isamarkupfalse%
\ Cons{\isachardot}{\kern0pt}prems{\isacharparenleft}{\kern0pt}{\isadigit{2}}{\isacharparenright}{\kern0pt}\ step{\isacharunderscore}{\kern0pt}sequence{\isachardot}{\kern0pt}simps\isanewline
\ \ \ \ \ \ \isacommand{by}\isamarkupfalse%
\ blast\isanewline
\ \ \ \ \isacommand{obtain}\isamarkupfalse%
\ q{\isacharprime}{\kern0pt}\ \isakeyword{where}\ {\isacartoucheopen}q\ {\isasymmapsto}\ a\ q{\isacharprime}{\kern0pt}{\isacartoucheclose}\ {\isacartoucheopen}p{\isacharprime}{\kern0pt}\ {\isasymlesssim}S\ q{\isacharprime}{\kern0pt}{\isacartoucheclose}\isanewline
\ \ \ \ \ \ \isacommand{using}\isamarkupfalse%
\ Cons{\isachardot}{\kern0pt}prems{\isacharparenleft}{\kern0pt}{\isadigit{1}}{\isacharparenright}{\kern0pt}\ {\isacartoucheopen}p\ {\isasymmapsto}\ a\ p{\isacharprime}{\kern0pt}{\isacartoucheclose}\ \isacommand{unfolding}\isamarkupfalse%
\ simulated{\isacharunderscore}{\kern0pt}by{\isacharunderscore}{\kern0pt}def\ simulation{\isacharunderscore}{\kern0pt}def\isanewline
\ \ \ \ \ \ \isacommand{by}\isamarkupfalse%
\ blast\isanewline
\ \ \ \ \isacommand{obtain}\isamarkupfalse%
\ q{\isacharprime}{\kern0pt}{\isacharprime}{\kern0pt}\ \isakeyword{where}\ {\isacartoucheopen}q{\isacharprime}{\kern0pt}\ {\isasymmapsto}{\isachardollar}{\kern0pt}\ tr\ q{\isacharprime}{\kern0pt}{\isacharprime}{\kern0pt}{\isacartoucheclose}\isanewline
\ \ \ \ \ \ \isacommand{using}\isamarkupfalse%
\ Cons{\isachardot}{\kern0pt}hyps\ {\isacartoucheopen}p{\isacharprime}{\kern0pt}\ {\isasymlesssim}S\ q{\isacharprime}{\kern0pt}{\isacartoucheclose}\ {\isacartoucheopen}p{\isacharprime}{\kern0pt}\ {\isasymmapsto}{\isachardollar}{\kern0pt}\ tr\ p{\isacharprime}{\kern0pt}{\isacharprime}{\kern0pt}{\isacartoucheclose}\isanewline
\ \ \ \ \ \ \isacommand{by}\isamarkupfalse%
\ blast\isanewline
\ \ \ \ \isacommand{then}\isamarkupfalse%
\ \isacommand{show}\isamarkupfalse%
\ {\isacharquery}{\kern0pt}case\isanewline
\ \ \ \ \ \ \isacommand{using}\isamarkupfalse%
\ {\isacartoucheopen}q\ {\isasymmapsto}\ a\ q{\isacharprime}{\kern0pt}{\isacartoucheclose}\ step{\isacharunderscore}{\kern0pt}sequence{\isachardot}{\kern0pt}intros{\isacharparenleft}{\kern0pt}{\isadigit{2}}{\isacharparenright}{\kern0pt}\isanewline
\ \ \ \ \ \ \isacommand{by}\isamarkupfalse%
\ blast\isanewline
\ \ \isacommand{qed}\isamarkupfalse%
\isanewline
\isacommand{qed}\isamarkupfalse%
%
\endisatagproof
{\isafoldproof}%
%
\isadelimproof
%
\endisadelimproof
%
\begin{isamarkuptext}%
Two states are bisimilar if they can be related by a symmetric simulation.%
\end{isamarkuptext}\isamarkuptrue%
\isacommand{definition}\isamarkupfalse%
\ bisimilar\ {\isacharparenleft}{\kern0pt}\isakeyword{infix}\ {\isacartoucheopen}{\isasymsimeq}B{\isacartoucheclose}\ {\isadigit{8}}{\isadigit{0}}{\isacharparenright}{\kern0pt}\ \isakeyword{where}\isanewline
\ \ {\isacartoucheopen}p\ {\isasymsimeq}B\ q\ {\isasymequiv}\ {\isasymexists}R{\isachardot}{\kern0pt}\ simulation\ R\ {\isasymand}\ symp\ R\ {\isasymand}\ R\ p\ q{\isacartoucheclose}%
\begin{isamarkuptext}%
Bisimilarity is a simulation.%
\end{isamarkuptext}\isamarkuptrue%
\isacommand{lemma}\isamarkupfalse%
\ bisim{\isacharunderscore}{\kern0pt}sim{\isacharcolon}{\kern0pt}\isanewline
\ \ \isakeyword{shows}\ {\isacartoucheopen}simulation\ {\isacharparenleft}{\kern0pt}{\isasymsimeq}B{\isacharparenright}{\kern0pt}{\isacartoucheclose}\isanewline
%
\isadelimproof
\ \ %
\endisadelimproof
%
\isatagproof
\isacommand{unfolding}\isamarkupfalse%
\ bisimilar{\isacharunderscore}{\kern0pt}def\ simulation{\isacharunderscore}{\kern0pt}def\ \isacommand{by}\isamarkupfalse%
\ blast%
\endisatagproof
{\isafoldproof}%
%
\isadelimproof
\isanewline
%
\endisadelimproof
\isanewline
\isanewline
\isacommand{end}\isamarkupfalse%
\isanewline
%
\isadelimtheory
%
\endisadelimtheory
%
\isatagtheory
\isacommand{end}\isamarkupfalse%
%
\endisatagtheory
{\isafoldtheory}%
%
\isadelimtheory
%
\endisadelimtheory
%
\end{isabellebody}%
\endinput
%:%file=~/Documents/Isabelle_HOL/Relational_Equivalences.thy%:%
%:%15=8%:%
%:%18=9%:%
%:%19=9%:%
%:%20=10%:%
%:%22=11%:%
%:%24=13%:%
%:%25=13%:%
%:%26=14%:%
%:%27=15%:%
%:%28=16%:%
%:%29=16%:%
%:%30=17%:%
%:%31=18%:%
%:%32=19%:%
%:%33=19%:%
%:%34=20%:%
%:%35=21%:%
%:%36=22%:%
%:%37=23%:%
%:%38=23%:%
%:%39=24%:%
%:%40=25%:%
%:%47=26%:%
%:%48=26%:%
%:%49=27%:%
%:%50=27%:%
%:%51=27%:%
%:%52=27%:%
%:%53=28%:%
%:%54=28%:%
%:%55=28%:%
%:%56=29%:%
%:%57=29%:%
%:%58=30%:%
%:%59=30%:%
%:%60=31%:%
%:%61=31%:%
%:%62=31%:%
%:%63=31%:%
%:%64=32%:%
%:%65=32%:%
%:%66=32%:%
%:%67=32%:%
%:%68=33%:%
%:%69=33%:%
%:%70=34%:%
%:%71=34%:%
%:%72=35%:%
%:%73=35%:%
%:%74=36%:%
%:%75=36%:%
%:%76=36%:%
%:%77=37%:%
%:%78=37%:%
%:%79=38%:%
%:%80=38%:%
%:%81=38%:%
%:%82=38%:%
%:%83=39%:%
%:%84=39%:%
%:%85=40%:%
%:%86=40%:%
%:%87=40%:%
%:%88=40%:%
%:%89=41%:%
%:%90=41%:%
%:%91=42%:%
%:%97=42%:%
%:%100=43%:%
%:%101=44%:%
%:%102=44%:%
%:%103=45%:%
%:%104=46%:%
%:%111=47%:%
%:%112=47%:%
%:%113=48%:%
%:%114=48%:%
%:%115=48%:%
%:%116=48%:%
%:%117=49%:%
%:%118=49%:%
%:%119=49%:%
%:%120=50%:%
%:%121=50%:%
%:%122=51%:%
%:%123=51%:%
%:%124=52%:%
%:%125=52%:%
%:%126=52%:%
%:%127=52%:%
%:%128=53%:%
%:%129=53%:%
%:%130=53%:%
%:%131=53%:%
%:%132=54%:%
%:%133=54%:%
%:%134=55%:%
%:%135=55%:%
%:%136=56%:%
%:%137=56%:%
%:%138=57%:%
%:%139=57%:%
%:%140=57%:%
%:%141=57%:%
%:%142=58%:%
%:%143=58%:%
%:%144=58%:%
%:%145=58%:%
%:%146=58%:%
%:%147=59%:%
%:%148=59%:%
%:%149=59%:%
%:%150=59%:%
%:%151=60%:%
%:%152=60%:%
%:%153=61%:%
%:%163=63%:%
%:%165=65%:%
%:%166=65%:%
%:%167=66%:%
%:%169=68%:%
%:%171=70%:%
%:%172=70%:%
%:%173=71%:%
%:%175=73%:%
%:%177=75%:%
%:%178=75%:%
%:%179=76%:%
%:%182=77%:%
%:%186=77%:%
%:%187=77%:%
%:%188=77%:%
%:%193=77%:%
%:%196=78%:%
%:%197=79%:%
%:%198=79%:%
%:%199=80%:%
%:%200=81%:%
%:%203=82%:%
%:%207=82%:%
%:%208=82%:%
%:%209=82%:%
%:%214=82%:%
%:%217=83%:%
%:%218=84%:%
%:%219=84%:%
%:%226=85%:%
%:%227=85%:%
%:%228=86%:%
%:%229=86%:%
%:%230=87%:%
%:%231=87%:%
%:%232=87%:%
%:%233=88%:%
%:%234=88%:%
%:%235=89%:%
%:%236=89%:%
%:%237=89%:%
%:%238=90%:%
%:%239=90%:%
%:%240=91%:%
%:%241=91%:%
%:%242=91%:%
%:%243=92%:%
%:%253=94%:%
%:%255=96%:%
%:%256=96%:%
%:%257=97%:%
%:%258=98%:%
%:%260=100%:%
%:%262=102%:%
%:%263=102%:%
%:%264=103%:%
%:%265=104%:%
%:%266=105%:%
%:%267=105%:%
%:%268=106%:%
%:%270=108%:%
%:%272=110%:%
%:%273=110%:%
%:%274=111%:%
%:%275=112%:%
%:%276=113%:%
%:%277=114%:%
%:%278=114%:%
%:%279=115%:%
%:%280=116%:%
%:%281=117%:%
%:%282=117%:%
%:%283=118%:%
%:%284=119%:%
%:%285=120%:%
%:%286=120%:%
%:%289=121%:%
%:%293=121%:%
%:%294=121%:%
%:%295=122%:%
%:%296=122%:%
%:%301=122%:%
%:%304=123%:%
%:%305=124%:%
%:%306=124%:%
%:%307=125%:%
%:%308=126%:%
%:%311=127%:%
%:%315=127%:%
%:%316=127%:%
%:%317=127%:%
%:%318=128%:%
%:%319=128%:%
%:%320=129%:%
%:%321=129%:%
%:%322=130%:%
%:%323=130%:%
%:%324=131%:%
%:%325=131%:%
%:%326=132%:%
%:%327=132%:%
%:%328=133%:%
%:%329=133%:%
%:%330=134%:%
%:%331=134%:%
%:%332=134%:%
%:%333=134%:%
%:%334=135%:%
%:%335=135%:%
%:%336=136%:%
%:%337=136%:%
%:%338=137%:%
%:%339=137%:%
%:%340=137%:%
%:%341=137%:%
%:%342=138%:%
%:%343=138%:%
%:%344=139%:%
%:%345=139%:%
%:%346=140%:%
%:%347=140%:%
%:%348=140%:%
%:%349=141%:%
%:%350=141%:%
%:%351=142%:%
%:%352=142%:%
%:%353=143%:%
%:%363=146%:%
%:%365=148%:%
%:%366=148%:%
%:%367=149%:%
%:%368=150%:%
%:%369=151%:%
%:%370=151%:%
%:%371=152%:%
%:%373=154%:%
%:%374=155%:%
%:%376=157%:%
%:%377=157%:%
%:%378=158%:%
%:%379=159%:%
%:%380=160%:%
%:%381=161%:%
%:%382=161%:%
%:%383=162%:%
%:%385=164%:%
%:%387=166%:%
%:%388=166%:%
%:%389=167%:%
%:%390=168%:%
%:%393=169%:%
%:%397=169%:%
%:%398=169%:%
%:%399=169%:%
%:%400=170%:%
%:%401=170%:%
%:%402=171%:%
%:%403=171%:%
%:%404=172%:%
%:%405=172%:%
%:%406=173%:%
%:%407=173%:%
%:%408=174%:%
%:%409=174%:%
%:%410=175%:%
%:%411=175%:%
%:%412=176%:%
%:%413=176%:%
%:%414=176%:%
%:%415=176%:%
%:%416=177%:%
%:%417=177%:%
%:%418=178%:%
%:%419=178%:%
%:%420=179%:%
%:%421=179%:%
%:%422=180%:%
%:%423=180%:%
%:%424=181%:%
%:%425=181%:%
%:%426=182%:%
%:%427=182%:%
%:%428=183%:%
%:%429=183%:%
%:%430=183%:%
%:%431=184%:%
%:%432=184%:%
%:%433=185%:%
%:%434=185%:%
%:%435=186%:%
%:%436=186%:%
%:%437=187%:%
%:%438=187%:%
%:%439=188%:%
%:%440=188%:%
%:%441=188%:%
%:%442=189%:%
%:%443=189%:%
%:%444=190%:%
%:%445=190%:%
%:%446=191%:%
%:%447=191%:%
%:%448=192%:%
%:%458=194%:%
%:%460=196%:%
%:%461=196%:%
%:%462=197%:%
%:%464=199%:%
%:%466=201%:%
%:%467=201%:%
%:%468=202%:%
%:%471=203%:%
%:%475=203%:%
%:%476=203%:%
%:%477=203%:%
%:%482=203%:%
%:%485=204%:%
%:%486=205%:%
%:%487=206%:%
%:%488=206%:%
%:%495=207%:%

%
\begin{isabellebody}%
\setisabellecontext{HML{\isacharunderscore}{\kern0pt}definitions}%
%
\isadelimtheory
%
\endisadelimtheory
%
\isatagtheory
\isacommand{theory}\isamarkupfalse%
\ HML{\isacharunderscore}{\kern0pt}definitions\isanewline
\isakeyword{imports}\ HML{\isacharunderscore}{\kern0pt}list\isanewline
\isakeyword{begin}%
\endisatagtheory
{\isafoldtheory}%
%
\isadelimtheory
\isanewline
%
\endisadelimtheory
\isanewline
\isacommand{inductive}\isamarkupfalse%
\ hml{\isacharunderscore}{\kern0pt}trace\ {\isacharcolon}{\kern0pt}{\isacharcolon}{\kern0pt}\ {\isachardoublequoteopen}{\isacharparenleft}{\kern0pt}{\isacharprime}{\kern0pt}a{\isacharcomma}{\kern0pt}\ {\isacharprime}{\kern0pt}s{\isacharparenright}{\kern0pt}hml\ {\isasymRightarrow}\ bool{\isachardoublequoteclose}\ \isakeyword{where}\isanewline
{\isachardoublequoteopen}hml{\isacharunderscore}{\kern0pt}trace\ TT{\isachardoublequoteclose}\ {\isacharbar}{\kern0pt}\isanewline
{\isachardoublequoteopen}hml{\isacharunderscore}{\kern0pt}trace\ {\isacharparenleft}{\kern0pt}hml{\isacharunderscore}{\kern0pt}pos\ {\isasymalpha}\ {\isasymphi}{\isacharparenright}{\kern0pt}{\isachardoublequoteclose}\ \isakeyword{if}\ {\isachardoublequoteopen}hml{\isacharunderscore}{\kern0pt}trace\ {\isasymphi}{\isachardoublequoteclose}\isanewline
\isanewline
\isacommand{inductive}\isamarkupfalse%
\ hml{\isacharunderscore}{\kern0pt}failure\ {\isacharcolon}{\kern0pt}{\isacharcolon}{\kern0pt}\ {\isachardoublequoteopen}{\isacharparenleft}{\kern0pt}{\isacharprime}{\kern0pt}a{\isacharcomma}{\kern0pt}\ {\isacharprime}{\kern0pt}s{\isacharparenright}{\kern0pt}hml\ {\isasymRightarrow}\ bool{\isachardoublequoteclose}\isanewline
\ \ \isakeyword{where}\isanewline
failure{\isacharunderscore}{\kern0pt}tt{\isacharcolon}{\kern0pt}\ {\isachardoublequoteopen}hml{\isacharunderscore}{\kern0pt}failure\ TT{\isachardoublequoteclose}\ {\isacharbar}{\kern0pt}\isanewline
failure{\isacharunderscore}{\kern0pt}pos{\isacharcolon}{\kern0pt}\ {\isachardoublequoteopen}hml{\isacharunderscore}{\kern0pt}failure\ {\isacharparenleft}{\kern0pt}hml{\isacharunderscore}{\kern0pt}pos\ {\isasymalpha}\ {\isasymphi}{\isacharparenright}{\kern0pt}{\isachardoublequoteclose}\ \isakeyword{if}\ {\isachardoublequoteopen}hml{\isacharunderscore}{\kern0pt}failure\ {\isasymphi}{\isachardoublequoteclose}\ {\isacharbar}{\kern0pt}\isanewline
failure{\isacharunderscore}{\kern0pt}conj{\isacharcolon}{\kern0pt}\ {\isachardoublequoteopen}hml{\isacharunderscore}{\kern0pt}failure\ {\isacharparenleft}{\kern0pt}hml{\isacharunderscore}{\kern0pt}conj\ I\ J\ {\isasympsi}s{\isacharparenright}{\kern0pt}{\isachardoublequoteclose}\ \isanewline
\isakeyword{if}\ {\isachardoublequoteopen}I\ {\isacharequal}{\kern0pt}\ {\isacharbraceleft}{\kern0pt}{\isacharbraceright}{\kern0pt}{\isachardoublequoteclose}\ {\isachardoublequoteopen}{\isacharparenleft}{\kern0pt}{\isasymforall}j\ {\isasymin}\ J{\isachardot}{\kern0pt}\ {\isacharparenleft}{\kern0pt}{\isasymexists}{\isasymalpha}{\isachardot}{\kern0pt}\ {\isacharparenleft}{\kern0pt}{\isacharparenleft}{\kern0pt}{\isasympsi}s\ j{\isacharparenright}{\kern0pt}\ {\isacharequal}{\kern0pt}\ hml{\isacharunderscore}{\kern0pt}pos\ {\isasymalpha}\ TT{\isacharparenright}{\kern0pt}{\isacharparenright}{\kern0pt}\ {\isasymor}\ {\isasympsi}s\ j\ {\isacharequal}{\kern0pt}\ TT{\isacharparenright}{\kern0pt}{\isachardoublequoteclose}\ \isanewline
\isanewline
\isacommand{inductive}\isamarkupfalse%
\ hml{\isacharunderscore}{\kern0pt}readiness\ {\isacharcolon}{\kern0pt}{\isacharcolon}{\kern0pt}\ {\isachardoublequoteopen}{\isacharparenleft}{\kern0pt}{\isacharprime}{\kern0pt}a{\isacharcomma}{\kern0pt}\ {\isacharprime}{\kern0pt}s{\isacharparenright}{\kern0pt}hml\ {\isasymRightarrow}\ bool{\isachardoublequoteclose}\isanewline
\ \ \isakeyword{where}\isanewline
read{\isacharunderscore}{\kern0pt}tt{\isacharcolon}{\kern0pt}\ {\isachardoublequoteopen}hml{\isacharunderscore}{\kern0pt}readiness\ TT{\isachardoublequoteclose}\ {\isacharbar}{\kern0pt}\isanewline
read{\isacharunderscore}{\kern0pt}pos{\isacharcolon}{\kern0pt}\ {\isachardoublequoteopen}hml{\isacharunderscore}{\kern0pt}readiness\ {\isacharparenleft}{\kern0pt}hml{\isacharunderscore}{\kern0pt}pos\ {\isasymalpha}\ {\isasymphi}{\isacharparenright}{\kern0pt}{\isachardoublequoteclose}\ \isakeyword{if}\ {\isachardoublequoteopen}hml{\isacharunderscore}{\kern0pt}readiness\ {\isasymphi}{\isachardoublequoteclose}{\isacharbar}{\kern0pt}\isanewline
read{\isacharunderscore}{\kern0pt}conj{\isacharcolon}{\kern0pt}\ {\isachardoublequoteopen}hml{\isacharunderscore}{\kern0pt}readiness\ {\isacharparenleft}{\kern0pt}hml{\isacharunderscore}{\kern0pt}conj\ I\ J\ {\isasymPhi}{\isacharparenright}{\kern0pt}{\isachardoublequoteclose}\ \isanewline
\isakeyword{if}\ {\isachardoublequoteopen}{\isasymforall}x\ {\isasymin}\ {\isacharparenleft}{\kern0pt}{\isasymPhi}\ {\isacharbackquote}{\kern0pt}\ {\isacharparenleft}{\kern0pt}I\ {\isasymunion}\ J{\isacharparenright}{\kern0pt}{\isacharparenright}{\kern0pt}{\isachardot}{\kern0pt}\ {\isacharparenleft}{\kern0pt}{\isasymexists}{\isasymalpha}{\isachardot}{\kern0pt}\ x\ {\isacharequal}{\kern0pt}\ {\isacharparenleft}{\kern0pt}hml{\isacharunderscore}{\kern0pt}pos\ {\isasymalpha}\ TT{\isacharcolon}{\kern0pt}{\isacharcolon}{\kern0pt}{\isacharparenleft}{\kern0pt}{\isacharprime}{\kern0pt}a{\isacharcomma}{\kern0pt}\ {\isacharprime}{\kern0pt}s{\isacharparenright}{\kern0pt}hml{\isacharparenright}{\kern0pt}{\isacharparenright}{\kern0pt}\ {\isasymor}\ x\ {\isacharequal}{\kern0pt}\ TT{\isachardoublequoteclose}\isanewline
\isanewline
\isacommand{inductive}\isamarkupfalse%
\ hml{\isacharunderscore}{\kern0pt}impossible{\isacharunderscore}{\kern0pt}futures\ {\isacharcolon}{\kern0pt}{\isacharcolon}{\kern0pt}\ \ {\isachardoublequoteopen}{\isacharparenleft}{\kern0pt}{\isacharprime}{\kern0pt}a{\isacharcomma}{\kern0pt}\ {\isacharprime}{\kern0pt}s{\isacharparenright}{\kern0pt}hml\ {\isasymRightarrow}\ bool{\isachardoublequoteclose}\isanewline
\ \ \isakeyword{where}\isanewline
\ \ if{\isacharunderscore}{\kern0pt}tt{\isacharcolon}{\kern0pt}\ {\isachardoublequoteopen}hml{\isacharunderscore}{\kern0pt}impossible{\isacharunderscore}{\kern0pt}futures\ TT{\isachardoublequoteclose}\ {\isacharbar}{\kern0pt}\isanewline
\ \ if{\isacharunderscore}{\kern0pt}pos{\isacharcolon}{\kern0pt}\ {\isachardoublequoteopen}hml{\isacharunderscore}{\kern0pt}impossible{\isacharunderscore}{\kern0pt}futures\ {\isacharparenleft}{\kern0pt}hml{\isacharunderscore}{\kern0pt}pos\ {\isasymalpha}\ {\isasymphi}{\isacharparenright}{\kern0pt}{\isachardoublequoteclose}\ \isakeyword{if}\ {\isachardoublequoteopen}hml{\isacharunderscore}{\kern0pt}impossible{\isacharunderscore}{\kern0pt}futures\ {\isasymphi}{\isachardoublequoteclose}\ {\isacharbar}{\kern0pt}\isanewline
if{\isacharunderscore}{\kern0pt}conj{\isacharcolon}{\kern0pt}\ {\isachardoublequoteopen}hml{\isacharunderscore}{\kern0pt}impossible{\isacharunderscore}{\kern0pt}futures\ {\isacharparenleft}{\kern0pt}hml{\isacharunderscore}{\kern0pt}conj\ I\ J\ {\isasymPhi}{\isacharparenright}{\kern0pt}{\isachardoublequoteclose}\isanewline
\isakeyword{if}\ {\isachardoublequoteopen}I\ {\isacharequal}{\kern0pt}\ {\isacharbraceleft}{\kern0pt}{\isacharbraceright}{\kern0pt}{\isachardoublequoteclose}\ {\isachardoublequoteopen}{\isasymforall}x\ {\isasymin}\ {\isacharparenleft}{\kern0pt}{\isasymPhi}\ {\isacharbackquote}{\kern0pt}\ J{\isacharparenright}{\kern0pt}{\isachardot}{\kern0pt}\ {\isacharparenleft}{\kern0pt}hml{\isacharunderscore}{\kern0pt}trace\ x{\isacharparenright}{\kern0pt}{\isachardoublequoteclose}\isanewline
\isanewline
\isacommand{inductive}\isamarkupfalse%
\ hml{\isacharunderscore}{\kern0pt}possible{\isacharunderscore}{\kern0pt}futures\ {\isacharcolon}{\kern0pt}{\isacharcolon}{\kern0pt}\ {\isachardoublequoteopen}{\isacharparenleft}{\kern0pt}{\isacharprime}{\kern0pt}a{\isacharcomma}{\kern0pt}\ {\isacharprime}{\kern0pt}s{\isacharparenright}{\kern0pt}hml\ {\isasymRightarrow}\ bool{\isachardoublequoteclose}\isanewline
\ \ \isakeyword{where}\isanewline
pf{\isacharunderscore}{\kern0pt}tt{\isacharcolon}{\kern0pt}\ {\isachardoublequoteopen}hml{\isacharunderscore}{\kern0pt}possible{\isacharunderscore}{\kern0pt}futures\ TT{\isachardoublequoteclose}\ {\isacharbar}{\kern0pt}\isanewline
pf{\isacharunderscore}{\kern0pt}pos{\isacharcolon}{\kern0pt}\ {\isachardoublequoteopen}hml{\isacharunderscore}{\kern0pt}possible{\isacharunderscore}{\kern0pt}futures\ {\isacharparenleft}{\kern0pt}hml{\isacharunderscore}{\kern0pt}pos\ {\isasymalpha}\ {\isasymphi}{\isacharparenright}{\kern0pt}{\isachardoublequoteclose}\ \isakeyword{if}\ {\isachardoublequoteopen}hml{\isacharunderscore}{\kern0pt}possible{\isacharunderscore}{\kern0pt}futures\ {\isasymphi}{\isachardoublequoteclose}\ {\isacharbar}{\kern0pt}\isanewline
pf{\isacharunderscore}{\kern0pt}conj{\isacharcolon}{\kern0pt}\ {\isachardoublequoteopen}hml{\isacharunderscore}{\kern0pt}possible{\isacharunderscore}{\kern0pt}futures\ {\isacharparenleft}{\kern0pt}hml{\isacharunderscore}{\kern0pt}conj\ I\ J\ {\isasymPhi}{\isacharparenright}{\kern0pt}{\isachardoublequoteclose}\ \isanewline
\isakeyword{if}\ {\isachardoublequoteopen}{\isasymforall}x\ {\isasymin}\ {\isacharparenleft}{\kern0pt}{\isasymPhi}\ {\isacharbackquote}{\kern0pt}\ {\isacharparenleft}{\kern0pt}I{\isasymunion}\ J{\isacharparenright}{\kern0pt}{\isacharparenright}{\kern0pt}{\isachardot}{\kern0pt}\ {\isacharparenleft}{\kern0pt}hml{\isacharunderscore}{\kern0pt}trace\ x{\isacharparenright}{\kern0pt}{\isachardoublequoteclose}\isanewline
\isanewline
\isacommand{definition}\isamarkupfalse%
\ hml{\isacharunderscore}{\kern0pt}possible{\isacharunderscore}{\kern0pt}futures{\isacharunderscore}{\kern0pt}formulas\ \isakeyword{where}\isanewline
{\isachardoublequoteopen}hml{\isacharunderscore}{\kern0pt}possible{\isacharunderscore}{\kern0pt}futures{\isacharunderscore}{\kern0pt}formulas\ {\isasymequiv}\ {\isacharbraceleft}{\kern0pt}{\isasymphi}{\isachardot}{\kern0pt}\ hml{\isacharunderscore}{\kern0pt}possible{\isacharunderscore}{\kern0pt}futures\ {\isasymphi}{\isacharbraceright}{\kern0pt}{\isachardoublequoteclose}\isanewline
\isanewline
\isacommand{inductive}\isamarkupfalse%
\ hml{\isacharunderscore}{\kern0pt}failure{\isacharunderscore}{\kern0pt}trace\ {\isacharcolon}{\kern0pt}{\isacharcolon}{\kern0pt}\ {\isachardoublequoteopen}{\isacharparenleft}{\kern0pt}{\isacharprime}{\kern0pt}a{\isacharcomma}{\kern0pt}\ {\isacharprime}{\kern0pt}s{\isacharparenright}{\kern0pt}hml\ {\isasymRightarrow}\ bool{\isachardoublequoteclose}\ \isakeyword{where}\isanewline
{\isachardoublequoteopen}hml{\isacharunderscore}{\kern0pt}failure{\isacharunderscore}{\kern0pt}trace\ TT{\isachardoublequoteclose}\ {\isacharbar}{\kern0pt}\isanewline
{\isachardoublequoteopen}hml{\isacharunderscore}{\kern0pt}failure{\isacharunderscore}{\kern0pt}trace\ {\isacharparenleft}{\kern0pt}hml{\isacharunderscore}{\kern0pt}pos\ {\isasymalpha}\ {\isasymphi}{\isacharparenright}{\kern0pt}{\isachardoublequoteclose}\ \isakeyword{if}\ {\isachardoublequoteopen}hml{\isacharunderscore}{\kern0pt}failure{\isacharunderscore}{\kern0pt}trace\ {\isasymphi}{\isachardoublequoteclose}\ {\isacharbar}{\kern0pt}\isanewline
{\isachardoublequoteopen}hml{\isacharunderscore}{\kern0pt}failure{\isacharunderscore}{\kern0pt}trace\ {\isacharparenleft}{\kern0pt}hml{\isacharunderscore}{\kern0pt}conj\ I\ J\ {\isasymPhi}{\isacharparenright}{\kern0pt}{\isachardoublequoteclose}\ \isanewline
\ \ \isakeyword{if}\ {\isachardoublequoteopen}{\isacharparenleft}{\kern0pt}{\isasymPhi}\ {\isacharbackquote}{\kern0pt}\ I{\isacharparenright}{\kern0pt}\ {\isacharequal}{\kern0pt}\ {\isacharbraceleft}{\kern0pt}{\isacharbraceright}{\kern0pt}\ {\isasymor}\ {\isacharparenleft}{\kern0pt}{\isasymexists}i\ {\isasymin}\ {\isasymPhi}\ {\isacharbackquote}{\kern0pt}\ I{\isachardot}{\kern0pt}\ {\isasymPhi}\ {\isacharbackquote}{\kern0pt}\ I\ {\isacharequal}{\kern0pt}\ {\isacharbraceleft}{\kern0pt}i{\isacharbraceright}{\kern0pt}\ {\isasymand}\ hml{\isacharunderscore}{\kern0pt}failure{\isacharunderscore}{\kern0pt}trace\ i{\isacharparenright}{\kern0pt}{\isachardoublequoteclose}\isanewline
\ \ \ \ \ {\isachardoublequoteopen}{\isasymforall}j\ {\isasymin}\ {\isasymPhi}\ {\isacharbackquote}{\kern0pt}\ J{\isachardot}{\kern0pt}\ {\isasymexists}{\isasymalpha}{\isachardot}{\kern0pt}\ j\ {\isacharequal}{\kern0pt}\ {\isacharparenleft}{\kern0pt}hml{\isacharunderscore}{\kern0pt}pos\ {\isasymalpha}\ TT{\isacharparenright}{\kern0pt}\ {\isasymor}\ j\ {\isacharequal}{\kern0pt}\ TT{\isachardoublequoteclose}\ \isanewline
\isanewline
\isacommand{inductive}\isamarkupfalse%
\ hml{\isacharunderscore}{\kern0pt}ready{\isacharunderscore}{\kern0pt}trace\ {\isacharcolon}{\kern0pt}{\isacharcolon}{\kern0pt}\ {\isachardoublequoteopen}{\isacharparenleft}{\kern0pt}{\isacharprime}{\kern0pt}a{\isacharcomma}{\kern0pt}\ {\isacharprime}{\kern0pt}s{\isacharparenright}{\kern0pt}hml\ {\isasymRightarrow}\ bool{\isachardoublequoteclose}\isanewline
\ \ \isakeyword{where}\isanewline
r{\isacharunderscore}{\kern0pt}trace{\isacharunderscore}{\kern0pt}tt{\isacharcolon}{\kern0pt}\ {\isachardoublequoteopen}hml{\isacharunderscore}{\kern0pt}ready{\isacharunderscore}{\kern0pt}trace\ TT{\isachardoublequoteclose}\ {\isacharbar}{\kern0pt}\isanewline
r{\isacharunderscore}{\kern0pt}trace{\isacharunderscore}{\kern0pt}pos{\isacharcolon}{\kern0pt}\ {\isachardoublequoteopen}hml{\isacharunderscore}{\kern0pt}ready{\isacharunderscore}{\kern0pt}trace\ {\isacharparenleft}{\kern0pt}hml{\isacharunderscore}{\kern0pt}pos\ {\isasymalpha}\ {\isasymphi}{\isacharparenright}{\kern0pt}{\isachardoublequoteclose}\ \isakeyword{if}\ {\isachardoublequoteopen}hml{\isacharunderscore}{\kern0pt}ready{\isacharunderscore}{\kern0pt}trace\ {\isasymphi}{\isachardoublequoteclose}{\isacharbar}{\kern0pt}\isanewline
r{\isacharunderscore}{\kern0pt}trace{\isacharunderscore}{\kern0pt}conj{\isacharcolon}{\kern0pt}\ {\isachardoublequoteopen}hml{\isacharunderscore}{\kern0pt}ready{\isacharunderscore}{\kern0pt}trace\ {\isacharparenleft}{\kern0pt}hml{\isacharunderscore}{\kern0pt}conj\ I\ J\ {\isasymPhi}{\isacharparenright}{\kern0pt}{\isachardoublequoteclose}\ \isanewline
\isakeyword{if}\ {\isachardoublequoteopen}{\isacharparenleft}{\kern0pt}{\isasymexists}x\ {\isasymin}\ {\isacharparenleft}{\kern0pt}{\isasymPhi}\ {\isacharbackquote}{\kern0pt}\ I{\isacharparenright}{\kern0pt}{\isachardot}{\kern0pt}\ hml{\isacharunderscore}{\kern0pt}ready{\isacharunderscore}{\kern0pt}trace\ x\ {\isasymand}\ {\isacharparenleft}{\kern0pt}{\isasymforall}y\ {\isasymin}\ {\isacharparenleft}{\kern0pt}{\isasymPhi}\ {\isacharbackquote}{\kern0pt}\ I{\isacharparenright}{\kern0pt}{\isachardot}{\kern0pt}\ x\ {\isasymnoteq}\ y\ {\isasymlongrightarrow}\ {\isacharparenleft}{\kern0pt}{\isasymexists}{\isasymalpha}{\isachardot}{\kern0pt}\ y\ {\isacharequal}{\kern0pt}\ {\isacharparenleft}{\kern0pt}hml{\isacharunderscore}{\kern0pt}pos\ {\isasymalpha}\ TT{\isacharparenright}{\kern0pt}{\isacharparenright}{\kern0pt}{\isacharparenright}{\kern0pt}{\isacharparenright}{\kern0pt}\isanewline
{\isasymor}\ {\isacharparenleft}{\kern0pt}{\isasymforall}y\ {\isasymin}\ {\isacharparenleft}{\kern0pt}{\isasymPhi}\ {\isacharbackquote}{\kern0pt}\ I{\isacharparenright}{\kern0pt}{\isachardot}{\kern0pt}{\isacharparenleft}{\kern0pt}{\isasymexists}{\isasymalpha}{\isachardot}{\kern0pt}\ y\ {\isacharequal}{\kern0pt}\ {\isacharparenleft}{\kern0pt}hml{\isacharunderscore}{\kern0pt}pos\ {\isasymalpha}\ TT{\isacharparenright}{\kern0pt}{\isacharparenright}{\kern0pt}{\isacharparenright}{\kern0pt}{\isachardoublequoteclose}\isanewline
{\isachardoublequoteopen}{\isacharparenleft}{\kern0pt}{\isasymforall}y\ {\isasymin}\ {\isacharparenleft}{\kern0pt}{\isasymPhi}\ {\isacharbackquote}{\kern0pt}\ J{\isacharparenright}{\kern0pt}{\isachardot}{\kern0pt}\ {\isacharparenleft}{\kern0pt}{\isasymexists}{\isasymalpha}{\isachardot}{\kern0pt}\ y\ {\isacharequal}{\kern0pt}\ {\isacharparenleft}{\kern0pt}hml{\isacharunderscore}{\kern0pt}pos\ {\isasymalpha}\ TT{\isacharparenright}{\kern0pt}{\isacharparenright}{\kern0pt}{\isacharparenright}{\kern0pt}{\isachardoublequoteclose}\isanewline
\isanewline
\isacommand{inductive}\isamarkupfalse%
\ hml{\isacharunderscore}{\kern0pt}ready{\isacharunderscore}{\kern0pt}sim\ {\isacharcolon}{\kern0pt}{\isacharcolon}{\kern0pt}\ {\isachardoublequoteopen}{\isacharparenleft}{\kern0pt}{\isacharprime}{\kern0pt}a{\isacharcomma}{\kern0pt}\ {\isacharprime}{\kern0pt}s{\isacharparenright}{\kern0pt}\ hml\ {\isasymRightarrow}\ bool{\isachardoublequoteclose}\isanewline
\ \ \isakeyword{where}\isanewline
{\isachardoublequoteopen}hml{\isacharunderscore}{\kern0pt}ready{\isacharunderscore}{\kern0pt}sim\ TT{\isachardoublequoteclose}\ {\isacharbar}{\kern0pt}\isanewline
{\isachardoublequoteopen}hml{\isacharunderscore}{\kern0pt}ready{\isacharunderscore}{\kern0pt}sim\ {\isacharparenleft}{\kern0pt}hml{\isacharunderscore}{\kern0pt}pos\ {\isasymalpha}\ {\isasymphi}{\isacharparenright}{\kern0pt}{\isachardoublequoteclose}\ \isakeyword{if}\ {\isachardoublequoteopen}hml{\isacharunderscore}{\kern0pt}ready{\isacharunderscore}{\kern0pt}sim\ {\isasymphi}{\isachardoublequoteclose}\ {\isacharbar}{\kern0pt}\isanewline
{\isachardoublequoteopen}hml{\isacharunderscore}{\kern0pt}ready{\isacharunderscore}{\kern0pt}sim\ {\isacharparenleft}{\kern0pt}hml{\isacharunderscore}{\kern0pt}conj\ I\ J\ {\isasymPhi}{\isacharparenright}{\kern0pt}{\isachardoublequoteclose}\ \isakeyword{if}\ \isanewline
{\isachardoublequoteopen}{\isacharparenleft}{\kern0pt}{\isasymforall}x\ {\isasymin}\ {\isacharparenleft}{\kern0pt}{\isasymPhi}\ {\isacharbackquote}{\kern0pt}\ I{\isacharparenright}{\kern0pt}{\isachardot}{\kern0pt}\ hml{\isacharunderscore}{\kern0pt}ready{\isacharunderscore}{\kern0pt}sim\ x{\isacharparenright}{\kern0pt}\ {\isasymand}\ {\isacharparenleft}{\kern0pt}{\isasymforall}y\ {\isasymin}\ {\isacharparenleft}{\kern0pt}{\isasymPhi}\ {\isacharbackquote}{\kern0pt}\ J{\isacharparenright}{\kern0pt}{\isachardot}{\kern0pt}\ {\isacharparenleft}{\kern0pt}{\isasymexists}{\isasymalpha}{\isachardot}{\kern0pt}\ y\ {\isacharequal}{\kern0pt}\ {\isacharparenleft}{\kern0pt}hml{\isacharunderscore}{\kern0pt}pos\ {\isasymalpha}\ TT{\isacharparenright}{\kern0pt}{\isacharparenright}{\kern0pt}{\isacharparenright}{\kern0pt}{\isachardoublequoteclose}\isanewline
\isanewline
\isacommand{inductive}\isamarkupfalse%
\ hml{\isacharunderscore}{\kern0pt}{\isadigit{2}}{\isacharunderscore}{\kern0pt}nested{\isacharunderscore}{\kern0pt}sim\ {\isacharcolon}{\kern0pt}{\isacharcolon}{\kern0pt}\ {\isachardoublequoteopen}{\isacharparenleft}{\kern0pt}{\isacharprime}{\kern0pt}a{\isacharcomma}{\kern0pt}\ {\isacharprime}{\kern0pt}s{\isacharparenright}{\kern0pt}\ hml\ {\isasymRightarrow}\ bool{\isachardoublequoteclose}\ \isanewline
\ \ \isakeyword{where}\isanewline
{\isachardoublequoteopen}hml{\isacharunderscore}{\kern0pt}{\isadigit{2}}{\isacharunderscore}{\kern0pt}nested{\isacharunderscore}{\kern0pt}sim\ TT{\isachardoublequoteclose}\ {\isacharbar}{\kern0pt}\isanewline
{\isachardoublequoteopen}hml{\isacharunderscore}{\kern0pt}{\isadigit{2}}{\isacharunderscore}{\kern0pt}nested{\isacharunderscore}{\kern0pt}sim\ {\isacharparenleft}{\kern0pt}hml{\isacharunderscore}{\kern0pt}pos\ {\isasymalpha}\ {\isasymphi}{\isacharparenright}{\kern0pt}{\isachardoublequoteclose}\ \isakeyword{if}\ {\isachardoublequoteopen}hml{\isacharunderscore}{\kern0pt}{\isadigit{2}}{\isacharunderscore}{\kern0pt}nested{\isacharunderscore}{\kern0pt}sim\ {\isasymphi}{\isachardoublequoteclose}\ {\isacharbar}{\kern0pt}\isanewline
{\isachardoublequoteopen}hml{\isacharunderscore}{\kern0pt}{\isadigit{2}}{\isacharunderscore}{\kern0pt}nested{\isacharunderscore}{\kern0pt}sim\ {\isacharparenleft}{\kern0pt}hml{\isacharunderscore}{\kern0pt}conj\ I\ J\ {\isasymPhi}{\isacharparenright}{\kern0pt}{\isachardoublequoteclose}\ \isanewline
\isakeyword{if}\ {\isachardoublequoteopen}{\isacharparenleft}{\kern0pt}{\isasymforall}x\ {\isasymin}\ {\isacharparenleft}{\kern0pt}{\isasymPhi}\ {\isacharbackquote}{\kern0pt}\ I{\isacharparenright}{\kern0pt}{\isachardot}{\kern0pt}\ hml{\isacharunderscore}{\kern0pt}{\isadigit{2}}{\isacharunderscore}{\kern0pt}nested{\isacharunderscore}{\kern0pt}sim\ x{\isacharparenright}{\kern0pt}\ {\isasymand}\ {\isacharparenleft}{\kern0pt}{\isasymforall}y\ {\isasymin}\ {\isacharparenleft}{\kern0pt}{\isasymPhi}\ {\isacharbackquote}{\kern0pt}\ J{\isacharparenright}{\kern0pt}{\isachardot}{\kern0pt}\ HML{\isacharunderscore}{\kern0pt}simulation\ y{\isacharparenright}{\kern0pt}{\isachardoublequoteclose}\isanewline
\isanewline
\isacommand{context}\isamarkupfalse%
\ lts\ \isakeyword{begin}\ \isanewline
\isanewline
\isacommand{lemma}\isamarkupfalse%
\ alt{\isacharunderscore}{\kern0pt}trace{\isacharunderscore}{\kern0pt}def{\isacharunderscore}{\kern0pt}implies{\isacharunderscore}{\kern0pt}trace{\isacharunderscore}{\kern0pt}def{\isacharcolon}{\kern0pt}\isanewline
\ \ \isakeyword{fixes}\ {\isasymphi}\ {\isacharcolon}{\kern0pt}{\isacharcolon}{\kern0pt}\ {\isachardoublequoteopen}{\isacharparenleft}{\kern0pt}{\isacharprime}{\kern0pt}a{\isacharcomma}{\kern0pt}\ {\isacharprime}{\kern0pt}s{\isacharparenright}{\kern0pt}\ hml{\isachardoublequoteclose}\isanewline
\ \ \isakeyword{assumes}\ {\isachardoublequoteopen}hml{\isacharunderscore}{\kern0pt}trace\ {\isasymphi}{\isachardoublequoteclose}\isanewline
\ \ \isakeyword{shows}\ {\isachardoublequoteopen}{\isasymexists}{\isasympsi}{\isachardot}{\kern0pt}\ HML{\isacharunderscore}{\kern0pt}trace\ {\isasympsi}\ {\isasymand}\ {\isacharparenleft}{\kern0pt}{\isasymforall}s{\isachardot}{\kern0pt}\ {\isacharparenleft}{\kern0pt}s\ {\isasymTurnstile}\ {\isasymphi}{\isacharparenright}{\kern0pt}\ {\isasymlongleftrightarrow}\ {\isacharparenleft}{\kern0pt}s\ {\isasymTurnstile}\ {\isasympsi}{\isacharparenright}{\kern0pt}{\isacharparenright}{\kern0pt}{\isachardoublequoteclose}\isanewline
%
\isadelimproof
\ \ %
\endisadelimproof
%
\isatagproof
\isacommand{using}\isamarkupfalse%
\ assms\ \ \isanewline
\isacommand{proof}\isamarkupfalse%
{\isacharparenleft}{\kern0pt}induction\ {\isasymphi}\ rule{\isacharcolon}{\kern0pt}\ hml{\isacharunderscore}{\kern0pt}trace{\isachardot}{\kern0pt}induct{\isacharparenright}{\kern0pt}\isanewline
\ \ \isacommand{case}\isamarkupfalse%
\ {\isadigit{1}}\isanewline
\ \ \isacommand{then}\isamarkupfalse%
\ \isacommand{show}\isamarkupfalse%
\ {\isacharquery}{\kern0pt}case\ \isacommand{using}\isamarkupfalse%
\ trace{\isacharunderscore}{\kern0pt}tt\ \isacommand{by}\isamarkupfalse%
\ blast\isanewline
\isacommand{next}\isamarkupfalse%
\isanewline
\ \ \isacommand{case}\isamarkupfalse%
\ {\isacharparenleft}{\kern0pt}{\isadigit{2}}\ {\isasymphi}\ {\isasymalpha}{\isacharparenright}{\kern0pt}\isanewline
\ \ \isacommand{then}\isamarkupfalse%
\ \isacommand{obtain}\isamarkupfalse%
\ {\isasympsi}\ \isakeyword{where}\ {\isachardoublequoteopen}HML{\isacharunderscore}{\kern0pt}trace\ {\isasympsi}{\isachardoublequoteclose}\ \isakeyword{and}\ IH{\isacharcolon}{\kern0pt}\ {\isachardoublequoteopen}{\isacharparenleft}{\kern0pt}{\isasymforall}s{\isachardot}{\kern0pt}\ {\isacharparenleft}{\kern0pt}s\ {\isasymTurnstile}\ {\isasymphi}{\isacharparenright}{\kern0pt}\ {\isacharequal}{\kern0pt}\ {\isacharparenleft}{\kern0pt}s\ {\isasymTurnstile}\ {\isasympsi}{\isacharparenright}{\kern0pt}{\isacharparenright}{\kern0pt}{\isachardoublequoteclose}\ \isacommand{by}\isamarkupfalse%
\ blast\isanewline
\ \ \isacommand{hence}\isamarkupfalse%
\ {\isachardoublequoteopen}HML{\isacharunderscore}{\kern0pt}trace\ {\isacharparenleft}{\kern0pt}hml{\isacharunderscore}{\kern0pt}pos\ {\isasymalpha}\ {\isasympsi}{\isacharparenright}{\kern0pt}{\isachardoublequoteclose}\ \isanewline
\ \ \ \ \isacommand{by}\isamarkupfalse%
\ {\isacharparenleft}{\kern0pt}simp\ add{\isacharcolon}{\kern0pt}\ trace{\isacharunderscore}{\kern0pt}pos{\isacharparenright}{\kern0pt}\isanewline
\ \ \isacommand{have}\isamarkupfalse%
\ {\isachardoublequoteopen}{\isacharparenleft}{\kern0pt}{\isasymforall}s{\isachardot}{\kern0pt}\ {\isacharparenleft}{\kern0pt}s\ {\isasymTurnstile}\ hml{\isacharunderscore}{\kern0pt}pos\ {\isasymalpha}\ {\isasymphi}{\isacharparenright}{\kern0pt}\ {\isacharequal}{\kern0pt}\ {\isacharparenleft}{\kern0pt}s\ {\isasymTurnstile}\ {\isacharparenleft}{\kern0pt}hml{\isacharunderscore}{\kern0pt}pos\ {\isasymalpha}\ {\isasympsi}{\isacharparenright}{\kern0pt}{\isacharparenright}{\kern0pt}{\isacharparenright}{\kern0pt}{\isachardoublequoteclose}\ \isacommand{using}\isamarkupfalse%
\ IH\ \isanewline
\ \ \ \ \isacommand{by}\isamarkupfalse%
\ simp\isanewline
\ \ \isacommand{then}\isamarkupfalse%
\ \isacommand{show}\isamarkupfalse%
\ {\isacharquery}{\kern0pt}case\ \isanewline
\ \ \ \ \isacommand{using}\isamarkupfalse%
\ {\isacartoucheopen}HML{\isacharunderscore}{\kern0pt}trace\ {\isacharparenleft}{\kern0pt}hml{\isacharunderscore}{\kern0pt}pos\ {\isasymalpha}\ {\isasympsi}{\isacharparenright}{\kern0pt}{\isacartoucheclose}\ \isacommand{by}\isamarkupfalse%
\ blast\isanewline
\isacommand{qed}\isamarkupfalse%
%
\endisatagproof
{\isafoldproof}%
%
\isadelimproof
\isanewline
%
\endisadelimproof
\isanewline
\isacommand{lemma}\isamarkupfalse%
\ trace{\isacharunderscore}{\kern0pt}def{\isacharunderscore}{\kern0pt}implies{\isacharunderscore}{\kern0pt}alt{\isacharunderscore}{\kern0pt}trace{\isacharunderscore}{\kern0pt}def{\isacharcolon}{\kern0pt}\isanewline
\ \ \isakeyword{fixes}\ {\isasymphi}\ {\isacharcolon}{\kern0pt}{\isacharcolon}{\kern0pt}\ {\isachardoublequoteopen}{\isacharparenleft}{\kern0pt}{\isacharprime}{\kern0pt}a{\isacharcomma}{\kern0pt}\ {\isacharprime}{\kern0pt}s{\isacharparenright}{\kern0pt}\ hml{\isachardoublequoteclose}\isanewline
\ \ \isakeyword{assumes}\ {\isachardoublequoteopen}HML{\isacharunderscore}{\kern0pt}trace\ {\isasymphi}{\isachardoublequoteclose}\isanewline
\ \ \isakeyword{shows}\ {\isachardoublequoteopen}{\isasymexists}{\isasympsi}{\isachardot}{\kern0pt}\ hml{\isacharunderscore}{\kern0pt}trace\ {\isasympsi}\ {\isasymand}\ {\isacharparenleft}{\kern0pt}{\isasymforall}s{\isachardot}{\kern0pt}\ {\isacharparenleft}{\kern0pt}s\ {\isasymTurnstile}\ {\isasymphi}{\isacharparenright}{\kern0pt}\ {\isasymlongleftrightarrow}\ {\isacharparenleft}{\kern0pt}s\ {\isasymTurnstile}\ {\isasympsi}{\isacharparenright}{\kern0pt}{\isacharparenright}{\kern0pt}{\isachardoublequoteclose}\isanewline
%
\isadelimproof
\ \ %
\endisadelimproof
%
\isatagproof
\isacommand{using}\isamarkupfalse%
\ assms\ \isacommand{proof}\isamarkupfalse%
{\isacharparenleft}{\kern0pt}induct{\isacharparenright}{\kern0pt}\isanewline
\ \ \isacommand{case}\isamarkupfalse%
\ trace{\isacharunderscore}{\kern0pt}tt\isanewline
\ \ \isacommand{then}\isamarkupfalse%
\ \isacommand{show}\isamarkupfalse%
\ {\isacharquery}{\kern0pt}case\ \isanewline
\ \ \ \ \isacommand{using}\isamarkupfalse%
\ hml{\isacharunderscore}{\kern0pt}trace{\isachardot}{\kern0pt}intros{\isacharparenleft}{\kern0pt}{\isadigit{1}}{\isacharparenright}{\kern0pt}\ \isacommand{by}\isamarkupfalse%
\ blast\isanewline
\isacommand{next}\isamarkupfalse%
\isanewline
\ \ \isacommand{case}\isamarkupfalse%
\ {\isacharparenleft}{\kern0pt}trace{\isacharunderscore}{\kern0pt}conj\ {\isasympsi}s{\isacharparenright}{\kern0pt}\isanewline
\ \ \isacommand{have}\isamarkupfalse%
\ {\isachardoublequoteopen}hml{\isacharunderscore}{\kern0pt}trace\ TT{\isachardoublequoteclose}\ \isanewline
\ \ \ \ \isacommand{using}\isamarkupfalse%
\ hml{\isacharunderscore}{\kern0pt}trace{\isachardot}{\kern0pt}intros{\isacharparenleft}{\kern0pt}{\isadigit{1}}{\isacharparenright}{\kern0pt}\ \isacommand{by}\isamarkupfalse%
\ blast\isanewline
\ \ \isacommand{have}\isamarkupfalse%
\ {\isachardoublequoteopen}{\isacharparenleft}{\kern0pt}{\isasymforall}s{\isachardot}{\kern0pt}\ {\isacharparenleft}{\kern0pt}s\ {\isasymTurnstile}\ hml{\isacharunderscore}{\kern0pt}conj\ {\isacharbraceleft}{\kern0pt}{\isacharbraceright}{\kern0pt}\ {\isacharbraceleft}{\kern0pt}{\isacharbraceright}{\kern0pt}\ {\isasympsi}s{\isacharparenright}{\kern0pt}\ {\isacharequal}{\kern0pt}\ {\isacharparenleft}{\kern0pt}s\ {\isasymTurnstile}\ TT{\isacharparenright}{\kern0pt}{\isacharparenright}{\kern0pt}{\isachardoublequoteclose}\ \isanewline
\ \ \ \ \isacommand{by}\isamarkupfalse%
\ simp\isanewline
\ \ \isacommand{then}\isamarkupfalse%
\ \isacommand{show}\isamarkupfalse%
\ {\isacharquery}{\kern0pt}case\ \isacommand{using}\isamarkupfalse%
\ {\isacartoucheopen}hml{\isacharunderscore}{\kern0pt}trace\ TT{\isacartoucheclose}\ \isacommand{by}\isamarkupfalse%
\ blast\isanewline
\isacommand{next}\isamarkupfalse%
\isanewline
\ \ \isacommand{case}\isamarkupfalse%
\ {\isacharparenleft}{\kern0pt}trace{\isacharunderscore}{\kern0pt}pos\ {\isasymphi}\ {\isasymalpha}{\isacharparenright}{\kern0pt}\isanewline
\ \ \isacommand{then}\isamarkupfalse%
\ \isacommand{obtain}\isamarkupfalse%
\ {\isasympsi}\ \isakeyword{where}\ IH{\isacharcolon}{\kern0pt}\ {\isachardoublequoteopen}hml{\isacharunderscore}{\kern0pt}trace\ {\isasympsi}{\isachardoublequoteclose}\ {\isachardoublequoteopen}{\isacharparenleft}{\kern0pt}{\isasymforall}s{\isachardot}{\kern0pt}\ {\isacharparenleft}{\kern0pt}s\ {\isasymTurnstile}\ {\isasymphi}{\isacharparenright}{\kern0pt}\ {\isacharequal}{\kern0pt}\ {\isacharparenleft}{\kern0pt}s\ {\isasymTurnstile}\ {\isasympsi}{\isacharparenright}{\kern0pt}{\isacharparenright}{\kern0pt}{\isachardoublequoteclose}\ \isacommand{by}\isamarkupfalse%
\ blast\isanewline
\ \ \isacommand{hence}\isamarkupfalse%
\ {\isachardoublequoteopen}{\isacharparenleft}{\kern0pt}{\isasymforall}s{\isachardot}{\kern0pt}\ {\isacharparenleft}{\kern0pt}s\ {\isasymTurnstile}\ hml{\isacharunderscore}{\kern0pt}pos\ {\isasymalpha}\ {\isasymphi}{\isacharparenright}{\kern0pt}\ {\isacharequal}{\kern0pt}\ {\isacharparenleft}{\kern0pt}s\ {\isasymTurnstile}\ {\isacharparenleft}{\kern0pt}hml{\isacharunderscore}{\kern0pt}pos\ {\isasymalpha}\ {\isasympsi}{\isacharparenright}{\kern0pt}{\isacharparenright}{\kern0pt}{\isacharparenright}{\kern0pt}{\isachardoublequoteclose}\ \isacommand{using}\isamarkupfalse%
\ hml{\isacharunderscore}{\kern0pt}sem{\isacharunderscore}{\kern0pt}pos\ \isacommand{by}\isamarkupfalse%
\ simp\isanewline
\ \ \isacommand{from}\isamarkupfalse%
\ IH\ \isacommand{have}\isamarkupfalse%
\ {\isachardoublequoteopen}hml{\isacharunderscore}{\kern0pt}trace\ {\isacharparenleft}{\kern0pt}hml{\isacharunderscore}{\kern0pt}pos\ {\isasymalpha}\ {\isasympsi}{\isacharparenright}{\kern0pt}{\isachardoublequoteclose}\ \isanewline
\ \ \ \ \isacommand{using}\isamarkupfalse%
\ hml{\isacharunderscore}{\kern0pt}trace{\isachardot}{\kern0pt}simps\ \isacommand{by}\isamarkupfalse%
\ blast\isanewline
\ \ \isacommand{then}\isamarkupfalse%
\ \isacommand{show}\isamarkupfalse%
\ {\isacharquery}{\kern0pt}case\ \isacommand{using}\isamarkupfalse%
\ {\isacartoucheopen}{\isacharparenleft}{\kern0pt}{\isasymforall}s{\isachardot}{\kern0pt}\ {\isacharparenleft}{\kern0pt}s\ {\isasymTurnstile}\ hml{\isacharunderscore}{\kern0pt}pos\ {\isasymalpha}\ {\isasymphi}{\isacharparenright}{\kern0pt}\ {\isacharequal}{\kern0pt}\ {\isacharparenleft}{\kern0pt}s\ {\isasymTurnstile}\ {\isacharparenleft}{\kern0pt}hml{\isacharunderscore}{\kern0pt}pos\ {\isasymalpha}\ {\isasympsi}{\isacharparenright}{\kern0pt}{\isacharparenright}{\kern0pt}{\isacharparenright}{\kern0pt}{\isacartoucheclose}\ \isanewline
\ \ \ \ \isacommand{by}\isamarkupfalse%
\ blast\isanewline
\isacommand{qed}\isamarkupfalse%
%
\endisatagproof
{\isafoldproof}%
%
\isadelimproof
\isanewline
%
\endisadelimproof
\isanewline
\isacommand{lemma}\isamarkupfalse%
\ trace{\isacharunderscore}{\kern0pt}definitions{\isacharunderscore}{\kern0pt}equivalent{\isacharcolon}{\kern0pt}\ \isanewline
\ \ {\isachardoublequoteopen}{\isasymforall}{\isasymphi}{\isachardot}{\kern0pt}\ {\isacharparenleft}{\kern0pt}HML{\isacharunderscore}{\kern0pt}trace\ {\isasymphi}\ {\isasymlongrightarrow}\ {\isacharparenleft}{\kern0pt}{\isasymexists}{\isasympsi}{\isachardot}{\kern0pt}\ hml{\isacharunderscore}{\kern0pt}trace\ {\isasympsi}\ {\isasymand}\ {\isacharparenleft}{\kern0pt}s\ {\isasymTurnstile}\ {\isasympsi}\ {\isasymlongleftrightarrow}\ s\ {\isasymTurnstile}\ {\isasymphi}{\isacharparenright}{\kern0pt}{\isacharparenright}{\kern0pt}{\isacharparenright}{\kern0pt}{\isachardoublequoteclose}\isanewline
\ \ {\isachardoublequoteopen}{\isasymforall}{\isasymphi}{\isachardot}{\kern0pt}\ {\isacharparenleft}{\kern0pt}hml{\isacharunderscore}{\kern0pt}trace\ {\isasymphi}\ {\isasymlongrightarrow}\ {\isacharparenleft}{\kern0pt}{\isasymexists}{\isasympsi}{\isachardot}{\kern0pt}\ HML{\isacharunderscore}{\kern0pt}trace\ {\isasympsi}\ {\isasymand}\ {\isacharparenleft}{\kern0pt}s\ {\isasymTurnstile}\ {\isasympsi}\ {\isasymlongleftrightarrow}\ s\ {\isasymTurnstile}\ {\isasymphi}{\isacharparenright}{\kern0pt}{\isacharparenright}{\kern0pt}{\isacharparenright}{\kern0pt}{\isachardoublequoteclose}\isanewline
%
\isadelimproof
\ \ %
\endisadelimproof
%
\isatagproof
\isacommand{using}\isamarkupfalse%
\ trace{\isacharunderscore}{\kern0pt}def{\isacharunderscore}{\kern0pt}implies{\isacharunderscore}{\kern0pt}alt{\isacharunderscore}{\kern0pt}trace{\isacharunderscore}{\kern0pt}def\ alt{\isacharunderscore}{\kern0pt}trace{\isacharunderscore}{\kern0pt}def{\isacharunderscore}{\kern0pt}implies{\isacharunderscore}{\kern0pt}trace{\isacharunderscore}{\kern0pt}def\ \isacommand{by}\isamarkupfalse%
\ blast{\isacharplus}{\kern0pt}%
\endisatagproof
{\isafoldproof}%
%
\isadelimproof
\isanewline
%
\endisadelimproof
\isanewline
\isacommand{lemma}\isamarkupfalse%
\ alt{\isacharunderscore}{\kern0pt}failure{\isacharunderscore}{\kern0pt}def{\isacharunderscore}{\kern0pt}implies{\isacharunderscore}{\kern0pt}failure{\isacharunderscore}{\kern0pt}def{\isacharcolon}{\kern0pt}\isanewline
\ \ \isakeyword{fixes}\ {\isasymphi}\ {\isacharcolon}{\kern0pt}{\isacharcolon}{\kern0pt}\ {\isachardoublequoteopen}{\isacharparenleft}{\kern0pt}{\isacharprime}{\kern0pt}a{\isacharcomma}{\kern0pt}\ {\isacharprime}{\kern0pt}s{\isacharparenright}{\kern0pt}\ hml{\isachardoublequoteclose}\isanewline
\ \ \isakeyword{assumes}\ {\isachardoublequoteopen}hml{\isacharunderscore}{\kern0pt}failure\ {\isasymphi}{\isachardoublequoteclose}\isanewline
\ \ \isakeyword{shows}\ {\isachardoublequoteopen}{\isasymexists}{\isasympsi}{\isachardot}{\kern0pt}\ HML{\isacharunderscore}{\kern0pt}failure\ {\isasympsi}\ {\isasymand}\ {\isacharparenleft}{\kern0pt}{\isasymforall}s{\isachardot}{\kern0pt}\ {\isacharparenleft}{\kern0pt}s\ {\isasymTurnstile}\ {\isasymphi}{\isacharparenright}{\kern0pt}\ {\isasymlongleftrightarrow}\ {\isacharparenleft}{\kern0pt}s\ {\isasymTurnstile}\ {\isasympsi}{\isacharparenright}{\kern0pt}{\isacharparenright}{\kern0pt}{\isachardoublequoteclose}\isanewline
%
\isadelimproof
\ \ %
\endisadelimproof
%
\isatagproof
\isacommand{using}\isamarkupfalse%
\ assms\ \isacommand{proof}\isamarkupfalse%
\ induct\isanewline
\ \ \isacommand{case}\isamarkupfalse%
\ failure{\isacharunderscore}{\kern0pt}tt\isanewline
\ \ \isacommand{then}\isamarkupfalse%
\ \isacommand{show}\isamarkupfalse%
\ {\isacharquery}{\kern0pt}case\ \isanewline
\ \ \ \ \isacommand{using}\isamarkupfalse%
\ HML{\isacharunderscore}{\kern0pt}failure{\isachardot}{\kern0pt}failure{\isacharunderscore}{\kern0pt}tt\ \isacommand{by}\isamarkupfalse%
\ blast\isanewline
\isacommand{next}\isamarkupfalse%
\isanewline
\ \ \isacommand{case}\isamarkupfalse%
\ {\isacharparenleft}{\kern0pt}failure{\isacharunderscore}{\kern0pt}pos\ {\isasymphi}\ {\isasymalpha}{\isacharparenright}{\kern0pt}\isanewline
\ \ \isacommand{then}\isamarkupfalse%
\ \isacommand{show}\isamarkupfalse%
\ {\isacharquery}{\kern0pt}case\ \isanewline
\ \ \ \ \isacommand{using}\isamarkupfalse%
\ HML{\isacharunderscore}{\kern0pt}failure{\isachardot}{\kern0pt}failure{\isacharunderscore}{\kern0pt}pos\ \isacommand{by}\isamarkupfalse%
\ fastforce\isanewline
\isacommand{next}\isamarkupfalse%
\isanewline
\ \ \isacommand{case}\isamarkupfalse%
\ {\isacharparenleft}{\kern0pt}failure{\isacharunderscore}{\kern0pt}conj\ I\ J\ {\isasympsi}s{\isacharparenright}{\kern0pt}\isanewline
\ \ \isacommand{have}\isamarkupfalse%
\ {\isachardoublequoteopen}HML{\isacharunderscore}{\kern0pt}failure\ {\isacharparenleft}{\kern0pt}hml{\isacharunderscore}{\kern0pt}conj\ I\ J\ {\isasympsi}s{\isacharparenright}{\kern0pt}{\isachardoublequoteclose}\isanewline
\ \ \ \ \isacommand{by}\isamarkupfalse%
\ {\isacharparenleft}{\kern0pt}metis\ HML{\isacharunderscore}{\kern0pt}failure{\isachardot}{\kern0pt}failure{\isacharunderscore}{\kern0pt}conj\ TT{\isacharunderscore}{\kern0pt}like{\isachardot}{\kern0pt}intros{\isacharparenleft}{\kern0pt}{\isadigit{1}}{\isacharparenright}{\kern0pt}\ empty{\isacharunderscore}{\kern0pt}iff\ failure{\isacharunderscore}{\kern0pt}conj{\isachardot}{\kern0pt}hyps{\isacharparenleft}{\kern0pt}{\isadigit{1}}{\isacharparenright}{\kern0pt}\ failure{\isacharunderscore}{\kern0pt}conj{\isachardot}{\kern0pt}hyps{\isacharparenleft}{\kern0pt}{\isadigit{2}}{\isacharparenright}{\kern0pt}{\isacharparenright}{\kern0pt}\isanewline
\ \ \isacommand{then}\isamarkupfalse%
\ \isacommand{show}\isamarkupfalse%
\ {\isacharquery}{\kern0pt}case\ \isanewline
\ \ \ \ \isacommand{by}\isamarkupfalse%
\ blast\isanewline
\isacommand{qed}\isamarkupfalse%
%
\endisatagproof
{\isafoldproof}%
%
\isadelimproof
\isanewline
%
\endisadelimproof
\isanewline
\isacommand{lemma}\isamarkupfalse%
\ failure{\isacharunderscore}{\kern0pt}def{\isacharunderscore}{\kern0pt}implies{\isacharunderscore}{\kern0pt}alt{\isacharunderscore}{\kern0pt}failure{\isacharunderscore}{\kern0pt}def{\isacharcolon}{\kern0pt}\isanewline
\ \ \isakeyword{fixes}\ {\isasymphi}\ {\isacharcolon}{\kern0pt}{\isacharcolon}{\kern0pt}\ {\isachardoublequoteopen}{\isacharparenleft}{\kern0pt}{\isacharprime}{\kern0pt}a{\isacharcomma}{\kern0pt}\ {\isacharprime}{\kern0pt}s{\isacharparenright}{\kern0pt}\ hml{\isachardoublequoteclose}\isanewline
\ \ \isakeyword{assumes}\ {\isachardoublequoteopen}HML{\isacharunderscore}{\kern0pt}failure\ {\isasymphi}{\isachardoublequoteclose}\isanewline
\ \ \isakeyword{shows}\ {\isachardoublequoteopen}{\isasymexists}{\isasympsi}{\isachardot}{\kern0pt}\ hml{\isacharunderscore}{\kern0pt}failure\ {\isasympsi}\ {\isasymand}\ {\isacharparenleft}{\kern0pt}{\isasymforall}s{\isachardot}{\kern0pt}\ {\isacharparenleft}{\kern0pt}s\ {\isasymTurnstile}\ {\isasymphi}{\isacharparenright}{\kern0pt}\ {\isasymlongleftrightarrow}\ {\isacharparenleft}{\kern0pt}s\ {\isasymTurnstile}\ {\isasympsi}{\isacharparenright}{\kern0pt}{\isacharparenright}{\kern0pt}{\isachardoublequoteclose}\isanewline
%
\isadelimproof
\ \ %
\endisadelimproof
%
\isatagproof
\isacommand{using}\isamarkupfalse%
\ assms\ \isacommand{proof}\isamarkupfalse%
{\isacharparenleft}{\kern0pt}induct{\isacharparenright}{\kern0pt}\isanewline
\ \ \isacommand{case}\isamarkupfalse%
\ failure{\isacharunderscore}{\kern0pt}tt\isanewline
\ \ \isacommand{then}\isamarkupfalse%
\ \isacommand{show}\isamarkupfalse%
\ {\isacharquery}{\kern0pt}case\ \isanewline
\ \ \ \ \isacommand{using}\isamarkupfalse%
\ hml{\isacharunderscore}{\kern0pt}failure{\isachardot}{\kern0pt}failure{\isacharunderscore}{\kern0pt}tt\ \isacommand{by}\isamarkupfalse%
\ blast\isanewline
\isacommand{next}\isamarkupfalse%
\isanewline
\ \ \isacommand{case}\isamarkupfalse%
\ {\isacharparenleft}{\kern0pt}failure{\isacharunderscore}{\kern0pt}pos\ {\isasymphi}\ {\isasymalpha}{\isacharparenright}{\kern0pt}\isanewline
\ \ \isacommand{then}\isamarkupfalse%
\ \isacommand{obtain}\isamarkupfalse%
\ {\isasympsi}\ \isakeyword{where}\ {\isachardoublequoteopen}hml{\isacharunderscore}{\kern0pt}failure\ {\isasympsi}{\isachardoublequoteclose}\ {\isachardoublequoteopen}{\isacharparenleft}{\kern0pt}{\isasymforall}s{\isachardot}{\kern0pt}\ {\isacharparenleft}{\kern0pt}s\ {\isasymTurnstile}\ {\isasymphi}{\isacharparenright}{\kern0pt}\ {\isacharequal}{\kern0pt}\ {\isacharparenleft}{\kern0pt}s\ {\isasymTurnstile}\ {\isasympsi}{\isacharparenright}{\kern0pt}{\isacharparenright}{\kern0pt}{\isachardoublequoteclose}\ \isacommand{by}\isamarkupfalse%
\ blast\isanewline
\ \ \isacommand{hence}\isamarkupfalse%
\ {\isachardoublequoteopen}hml{\isacharunderscore}{\kern0pt}failure\ {\isacharparenleft}{\kern0pt}hml{\isacharunderscore}{\kern0pt}pos\ {\isasymalpha}\ {\isasympsi}{\isacharparenright}{\kern0pt}\ {\isasymand}\ {\isacharparenleft}{\kern0pt}{\isasymforall}s{\isachardot}{\kern0pt}\ {\isacharparenleft}{\kern0pt}s\ {\isasymTurnstile}\ hml{\isacharunderscore}{\kern0pt}pos\ {\isasymalpha}\ {\isasymphi}{\isacharparenright}{\kern0pt}\ {\isacharequal}{\kern0pt}\ {\isacharparenleft}{\kern0pt}s\ {\isasymTurnstile}\ {\isacharparenleft}{\kern0pt}hml{\isacharunderscore}{\kern0pt}pos\ {\isasymalpha}\ {\isasympsi}{\isacharparenright}{\kern0pt}{\isacharparenright}{\kern0pt}{\isacharparenright}{\kern0pt}{\isachardoublequoteclose}\ \isanewline
\ \ \ \ \isacommand{by}\isamarkupfalse%
\ {\isacharparenleft}{\kern0pt}simp\ add{\isacharcolon}{\kern0pt}\ hml{\isacharunderscore}{\kern0pt}failure{\isachardot}{\kern0pt}failure{\isacharunderscore}{\kern0pt}pos{\isacharparenright}{\kern0pt}\isanewline
\ \ \isacommand{then}\isamarkupfalse%
\ \isacommand{show}\isamarkupfalse%
\ {\isacharquery}{\kern0pt}case\ \isacommand{by}\isamarkupfalse%
\ blast\isanewline
\isacommand{next}\isamarkupfalse%
\isanewline
\ \ \isacommand{case}\isamarkupfalse%
\ {\isacharparenleft}{\kern0pt}failure{\isacharunderscore}{\kern0pt}conj\ I\ {\isasympsi}s\ J{\isacharparenright}{\kern0pt}\isanewline
\ \ \isacommand{then}\isamarkupfalse%
\ \isacommand{show}\isamarkupfalse%
\ {\isacharquery}{\kern0pt}case\ \isacommand{proof}\isamarkupfalse%
{\isacharparenleft}{\kern0pt}cases\ {\isachardoublequoteopen}{\isasymnot}{\isacharparenleft}{\kern0pt}{\isasymexists}j\ {\isasymin}\ J{\isachardot}{\kern0pt}\ TT{\isacharunderscore}{\kern0pt}like\ {\isacharparenleft}{\kern0pt}{\isasympsi}s\ j{\isacharparenright}{\kern0pt}{\isacharparenright}{\kern0pt}\ {\isasymand}\ {\isasympsi}s\ {\isacharbackquote}{\kern0pt}\ I\ {\isasyminter}\ {\isasympsi}s\ {\isacharbackquote}{\kern0pt}\ J\ {\isacharequal}{\kern0pt}\ {\isacharbraceleft}{\kern0pt}{\isacharbraceright}{\kern0pt}{\isachardoublequoteclose}{\isacharparenright}{\kern0pt}\isanewline
\ \ \ \ \isacommand{case}\isamarkupfalse%
\ True\isanewline
\ \ \ \ \isacommand{hence}\isamarkupfalse%
\ {\isachardoublequoteopen}{\isasymforall}j\ {\isasymin}\ J{\isachardot}{\kern0pt}\ {\isacharparenleft}{\kern0pt}{\isasymexists}{\isasymalpha}\ {\isasymchi}{\isachardot}{\kern0pt}\ {\isasympsi}s\ j\ {\isacharequal}{\kern0pt}\ hml{\isacharunderscore}{\kern0pt}pos\ {\isasymalpha}\ {\isasymchi}\ {\isasymand}\ TT{\isacharunderscore}{\kern0pt}like\ {\isasymchi}{\isacharparenright}{\kern0pt}{\isachardoublequoteclose}\ \isanewline
\ \ \ \ \ \ \isacommand{using}\isamarkupfalse%
\ local{\isachardot}{\kern0pt}failure{\isacharunderscore}{\kern0pt}conj\ \isacommand{by}\isamarkupfalse%
\ blast\isanewline
\ \ \ \ \isacommand{define}\isamarkupfalse%
\ {\isasymPsi}\ \isakeyword{where}\ {\isachardoublequoteopen}{\isasymPsi}\ {\isasymequiv}\ {\isacharparenleft}{\kern0pt}{\isasymlambda}i{\isachardot}{\kern0pt}\ {\isacharparenleft}{\kern0pt}if\ i\ {\isasymin}\ J\isanewline
\ \ \ \ \ \ \ \ \ \ \ \ \ \ \ \ \ \ \ \ \ \ \ \ \ \ then\ {\isacharparenleft}{\kern0pt}hml{\isacharunderscore}{\kern0pt}pos\ {\isacharparenleft}{\kern0pt}SOME\ {\isasymalpha}{\isachardot}{\kern0pt}\ {\isasymexists}{\isasymchi}{\isachardot}{\kern0pt}\ {\isasympsi}s\ i\ {\isacharequal}{\kern0pt}\ hml{\isacharunderscore}{\kern0pt}pos\ {\isasymalpha}\ {\isasymchi}\ {\isasymand}\ TT{\isacharunderscore}{\kern0pt}like\ {\isasymchi}{\isacharparenright}{\kern0pt}\ TT{\isacharparenright}{\kern0pt}{\isacharcolon}{\kern0pt}{\isacharcolon}{\kern0pt}{\isacharparenleft}{\kern0pt}{\isacharprime}{\kern0pt}a{\isacharcomma}{\kern0pt}\ {\isacharprime}{\kern0pt}s{\isacharparenright}{\kern0pt}hml\ \isanewline
\ \ \ \ \ \ \ \ \ \ \ \ \ \ \ \ \ \ \ \ \ \ \ \ \ \ else\ undefined{\isacharparenright}{\kern0pt}{\isacharparenright}{\kern0pt}{\isachardoublequoteclose}\isanewline
\ \ \ \ \isacommand{hence}\isamarkupfalse%
\ {\isachardoublequoteopen}{\isasymforall}{\isasympsi}\ {\isasymin}\ {\isasymPsi}\ {\isacharbackquote}{\kern0pt}\ J{\isachardot}{\kern0pt}\ {\isasymexists}{\isasymalpha}{\isachardot}{\kern0pt}\ {\isasympsi}\ {\isacharequal}{\kern0pt}\ hml{\isacharunderscore}{\kern0pt}pos\ {\isasymalpha}\ TT{\isachardoublequoteclose}\ \isanewline
\ \ \ \ \ \ \isacommand{by}\isamarkupfalse%
\ force\isanewline
\ \ \ \ \isacommand{hence}\isamarkupfalse%
\ {\isachardoublequoteopen}{\isasymforall}j\ {\isasymin}\ J{\isachardot}{\kern0pt}\ {\isasymexists}{\isasymalpha}\ {\isasymchi}{\isachardot}{\kern0pt}\ {\isasympsi}s\ j\ {\isacharequal}{\kern0pt}\ hml{\isacharunderscore}{\kern0pt}pos\ {\isasymalpha}\ {\isasymchi}\ {\isasymand}\ TT{\isacharunderscore}{\kern0pt}like\ {\isasymchi}\ {\isasymand}\ {\isasymPsi}\ j\ {\isacharequal}{\kern0pt}\ hml{\isacharunderscore}{\kern0pt}pos\ {\isasymalpha}\ TT{\isachardoublequoteclose}\ \isanewline
\ \ \ \ \ \ \isacommand{using}\isamarkupfalse%
\ {\isasymPsi}{\isacharunderscore}{\kern0pt}def\ {\isacartoucheopen}{\isasymforall}j{\isasymin}J{\isachardot}{\kern0pt}\ {\isasymexists}{\isasymalpha}\ {\isasymchi}{\isachardot}{\kern0pt}\ {\isasympsi}s\ j\ {\isacharequal}{\kern0pt}\ hml{\isacharunderscore}{\kern0pt}pos\ {\isasymalpha}\ {\isasymchi}\ {\isasymand}\ TT{\isacharunderscore}{\kern0pt}like\ {\isasymchi}{\isacartoucheclose}\ \isacommand{by}\isamarkupfalse%
\ fastforce\isanewline
\ \ \ \ \isacommand{hence}\isamarkupfalse%
\ {\isachardoublequoteopen}{\isacharparenleft}{\kern0pt}{\isasymforall}s{\isachardot}{\kern0pt}\ {\isasymforall}j\ {\isasymin}\ J{\isachardot}{\kern0pt}\ {\isacharparenleft}{\kern0pt}{\isasymnot}{\isacharparenleft}{\kern0pt}s\ {\isasymTurnstile}\ {\isacharparenleft}{\kern0pt}{\isasymPsi}\ j{\isacharparenright}{\kern0pt}{\isacharparenright}{\kern0pt}\ {\isacharequal}{\kern0pt}\ {\isacharparenleft}{\kern0pt}{\isasymnot}{\isacharparenleft}{\kern0pt}s\ {\isasymTurnstile}\ {\isacharparenleft}{\kern0pt}{\isasympsi}s\ j{\isacharparenright}{\kern0pt}{\isacharparenright}{\kern0pt}{\isacharparenright}{\kern0pt}{\isacharparenright}{\kern0pt}{\isacharparenright}{\kern0pt}{\isachardoublequoteclose}\ \isanewline
\ \ \ \ \ \ \isacommand{using}\isamarkupfalse%
\ True\ HML{\isacharunderscore}{\kern0pt}true{\isacharunderscore}{\kern0pt}TT{\isacharunderscore}{\kern0pt}like\ HML{\isacharunderscore}{\kern0pt}true{\isacharunderscore}{\kern0pt}def\ \isacommand{by}\isamarkupfalse%
\ auto\isanewline
\ \ \ \ \isacommand{have}\isamarkupfalse%
\ {\isachardoublequoteopen}{\isasymforall}s{\isachardot}{\kern0pt}\ {\isasymforall}i\ {\isasymin}\ I{\isachardot}{\kern0pt}\ s\ {\isasymTurnstile}\ {\isasympsi}s\ i{\isachardoublequoteclose}\ \isanewline
\ \ \ \ \ \ \isacommand{using}\isamarkupfalse%
\ HML{\isacharunderscore}{\kern0pt}true{\isacharunderscore}{\kern0pt}TT{\isacharunderscore}{\kern0pt}like\ HML{\isacharunderscore}{\kern0pt}true{\isacharunderscore}{\kern0pt}def\ local{\isachardot}{\kern0pt}failure{\isacharunderscore}{\kern0pt}conj\ \isacommand{by}\isamarkupfalse%
\ blast\isanewline
\ \ \ \ \isacommand{hence}\isamarkupfalse%
\ {\isachardoublequoteopen}{\isacharparenleft}{\kern0pt}{\isasymforall}s{\isachardot}{\kern0pt}\ {\isacharparenleft}{\kern0pt}s\ {\isasymTurnstile}\ hml{\isacharunderscore}{\kern0pt}conj\ I\ J\ {\isasympsi}s{\isacharparenright}{\kern0pt}\ {\isacharequal}{\kern0pt}\ {\isacharparenleft}{\kern0pt}s\ {\isasymTurnstile}\ {\isacharparenleft}{\kern0pt}hml{\isacharunderscore}{\kern0pt}conj\ {\isacharbraceleft}{\kern0pt}{\isacharbraceright}{\kern0pt}\ J\ {\isasymPsi}{\isacharparenright}{\kern0pt}{\isacharparenright}{\kern0pt}{\isacharparenright}{\kern0pt}{\isachardoublequoteclose}\isanewline
\ \ \ \ \ \ \isacommand{using}\isamarkupfalse%
\ {\isacartoucheopen}{\isacharparenleft}{\kern0pt}{\isasymforall}s{\isachardot}{\kern0pt}\ {\isasymforall}j\ {\isasymin}\ J{\isachardot}{\kern0pt}\ {\isacharparenleft}{\kern0pt}{\isasymnot}{\isacharparenleft}{\kern0pt}s\ {\isasymTurnstile}\ {\isacharparenleft}{\kern0pt}{\isasymPsi}\ j{\isacharparenright}{\kern0pt}{\isacharparenright}{\kern0pt}\ {\isacharequal}{\kern0pt}\ {\isacharparenleft}{\kern0pt}{\isasymnot}{\isacharparenleft}{\kern0pt}s\ {\isasymTurnstile}\ {\isacharparenleft}{\kern0pt}{\isasympsi}s\ j{\isacharparenright}{\kern0pt}{\isacharparenright}{\kern0pt}{\isacharparenright}{\kern0pt}{\isacharparenright}{\kern0pt}{\isacharparenright}{\kern0pt}{\isacartoucheclose}\isanewline
\ \ \ \ \ \ \isacommand{by}\isamarkupfalse%
\ simp\isanewline
\ \ \ \ \isacommand{have}\isamarkupfalse%
\ {\isachardoublequoteopen}hml{\isacharunderscore}{\kern0pt}failure\ {\isacharparenleft}{\kern0pt}hml{\isacharunderscore}{\kern0pt}conj\ {\isacharbraceleft}{\kern0pt}{\isacharbraceright}{\kern0pt}\ J\ {\isasymPsi}{\isacharparenright}{\kern0pt}{\isachardoublequoteclose}\ \isanewline
\ \ \ \ \ \ \isacommand{using}\isamarkupfalse%
\ {\isasymPsi}{\isacharunderscore}{\kern0pt}def\ hml{\isacharunderscore}{\kern0pt}failure{\isachardot}{\kern0pt}failure{\isacharunderscore}{\kern0pt}conj\isanewline
\ \ \ \ \ \ \isacommand{by}\isamarkupfalse%
\ {\isacharparenleft}{\kern0pt}metis\ {\isacharparenleft}{\kern0pt}no{\isacharunderscore}{\kern0pt}types{\isacharcomma}{\kern0pt}\ lifting{\isacharparenright}{\kern0pt}{\isacharparenright}{\kern0pt}\isanewline
\ \ \ \ \isacommand{then}\isamarkupfalse%
\ \isacommand{show}\isamarkupfalse%
\ {\isacharquery}{\kern0pt}thesis\ \isanewline
\ \ \ \ \ \ \isacommand{using}\isamarkupfalse%
\ {\isacartoucheopen}{\isasymforall}s{\isachardot}{\kern0pt}\ {\isacharparenleft}{\kern0pt}s\ {\isasymTurnstile}\ hml{\isacharunderscore}{\kern0pt}conj\ I\ J\ {\isasympsi}s{\isacharparenright}{\kern0pt}\ {\isacharequal}{\kern0pt}\ {\isacharparenleft}{\kern0pt}s\ {\isasymTurnstile}\ hml{\isacharunderscore}{\kern0pt}conj\ {\isacharbraceleft}{\kern0pt}{\isacharbraceright}{\kern0pt}\ J\ {\isasymPsi}{\isacharparenright}{\kern0pt}{\isacartoucheclose}\ \isacommand{by}\isamarkupfalse%
\ blast\isanewline
\ \ \isacommand{next}\isamarkupfalse%
\isanewline
\ \ \ \ \isacommand{case}\isamarkupfalse%
\ False\isanewline
\ \ \ \ \isacommand{hence}\isamarkupfalse%
\ {\isachardoublequoteopen}{\isasymforall}s{\isachardot}{\kern0pt}\ {\isasymnot}{\isacharparenleft}{\kern0pt}s\ {\isasymTurnstile}\ hml{\isacharunderscore}{\kern0pt}conj\ I\ J\ {\isasympsi}s{\isacharparenright}{\kern0pt}{\isachardoublequoteclose}\ \isanewline
\ \ \ \ \ \ \isacommand{using}\isamarkupfalse%
\ HML{\isacharunderscore}{\kern0pt}true{\isacharunderscore}{\kern0pt}TT{\isacharunderscore}{\kern0pt}like\ HML{\isacharunderscore}{\kern0pt}true{\isacharunderscore}{\kern0pt}def\ \isacommand{by}\isamarkupfalse%
\ fastforce\isanewline
\ \ \ \ \isacommand{from}\isamarkupfalse%
\ False\ \isacommand{obtain}\isamarkupfalse%
\ {\isasymphi}\ i{\isacharunderscore}{\kern0pt}{\isasymphi}\ \isakeyword{where}\ {\isachardoublequoteopen}{\isasymphi}\ {\isasymin}\ {\isasympsi}s\ {\isacharbackquote}{\kern0pt}\ I\ {\isasyminter}\ {\isasympsi}s\ {\isacharbackquote}{\kern0pt}\ J\ {\isasymor}\ {\isacharparenleft}{\kern0pt}{\isasymphi}\ {\isasymin}\ {\isasympsi}s\ {\isacharbackquote}{\kern0pt}\ J\ {\isasymand}\ TT{\isacharunderscore}{\kern0pt}like\ {\isasymphi}{\isacharparenright}{\kern0pt}{\isachardoublequoteclose}\ {\isachardoublequoteopen}{\isasympsi}s\ i{\isacharunderscore}{\kern0pt}{\isasymphi}\ {\isacharequal}{\kern0pt}\ {\isasymphi}{\isachardoublequoteclose}\isanewline
\ \ \ \ \ \ \isacommand{by}\isamarkupfalse%
\ blast\isanewline
\ \ \ \ \isacommand{define}\isamarkupfalse%
\ {\isasymPsi}\ \isakeyword{where}\ {\isachardoublequoteopen}{\isasymPsi}\ {\isasymequiv}\ {\isacharparenleft}{\kern0pt}{\isasymlambda}i{\isachardot}{\kern0pt}\ {\isacharparenleft}{\kern0pt}if\ i\ {\isacharequal}{\kern0pt}\ i{\isacharunderscore}{\kern0pt}{\isasymphi}\ then\ TT{\isacharcolon}{\kern0pt}{\isacharcolon}{\kern0pt}{\isacharparenleft}{\kern0pt}{\isacharprime}{\kern0pt}a{\isacharcomma}{\kern0pt}\ {\isacharprime}{\kern0pt}s{\isacharparenright}{\kern0pt}hml\ else\ undefined{\isacharparenright}{\kern0pt}{\isacharparenright}{\kern0pt}{\isachardoublequoteclose}\isanewline
\ \ \ \ \isacommand{hence}\isamarkupfalse%
\ {\isachardoublequoteopen}{\isasymforall}s{\isachardot}{\kern0pt}\ {\isasymnot}{\isacharparenleft}{\kern0pt}s\ {\isasymTurnstile}\ {\isacharparenleft}{\kern0pt}hml{\isacharunderscore}{\kern0pt}conj\ {\isacharbraceleft}{\kern0pt}{\isacharbraceright}{\kern0pt}\ {\isacharbraceleft}{\kern0pt}i{\isacharunderscore}{\kern0pt}{\isasymphi}{\isacharbraceright}{\kern0pt}\ {\isasymPsi}{\isacharparenright}{\kern0pt}{\isacharparenright}{\kern0pt}{\isachardoublequoteclose}\ \isanewline
\ \ \ \ \ \ \isacommand{by}\isamarkupfalse%
\ auto\isanewline
\ \ \ \ \isacommand{have}\isamarkupfalse%
\ {\isachardoublequoteopen}hml{\isacharunderscore}{\kern0pt}failure\ {\isacharparenleft}{\kern0pt}hml{\isacharunderscore}{\kern0pt}conj\ {\isacharbraceleft}{\kern0pt}{\isacharbraceright}{\kern0pt}\ {\isacharbraceleft}{\kern0pt}i{\isacharunderscore}{\kern0pt}{\isasymphi}{\isacharbraceright}{\kern0pt}\ {\isasymPsi}{\isacharparenright}{\kern0pt}{\isachardoublequoteclose}\ \isanewline
\ \ \ \ \ \ \isacommand{by}\isamarkupfalse%
\ {\isacharparenleft}{\kern0pt}simp\ add{\isacharcolon}{\kern0pt}\ {\isasymPsi}{\isacharunderscore}{\kern0pt}def\ hml{\isacharunderscore}{\kern0pt}failure{\isachardot}{\kern0pt}failure{\isacharunderscore}{\kern0pt}conj{\isacharparenright}{\kern0pt}\isanewline
\ \ \ \ \isacommand{then}\isamarkupfalse%
\ \isacommand{show}\isamarkupfalse%
\ {\isacharquery}{\kern0pt}thesis\ \isanewline
\ \ \ \ \ \ \isacommand{using}\isamarkupfalse%
\ {\isacartoucheopen}{\isasymforall}s{\isachardot}{\kern0pt}\ {\isasymnot}\ s\ {\isasymTurnstile}\ hml{\isacharunderscore}{\kern0pt}conj\ I\ J\ {\isasympsi}s{\isacartoucheclose}\ {\isacartoucheopen}{\isasymforall}s{\isachardot}{\kern0pt}\ {\isasymnot}\ s\ {\isasymTurnstile}\ hml{\isacharunderscore}{\kern0pt}conj\ {\isacharbraceleft}{\kern0pt}{\isacharbraceright}{\kern0pt}\ {\isacharbraceleft}{\kern0pt}i{\isacharunderscore}{\kern0pt}{\isasymphi}{\isacharbraceright}{\kern0pt}\ {\isasymPsi}{\isacartoucheclose}\ \isacommand{by}\isamarkupfalse%
\ blast\isanewline
\ \ \isacommand{qed}\isamarkupfalse%
\isanewline
\isacommand{qed}\isamarkupfalse%
%
\endisatagproof
{\isafoldproof}%
%
\isadelimproof
\isanewline
%
\endisadelimproof
\isanewline
\isacommand{lemma}\isamarkupfalse%
\ failure{\isacharunderscore}{\kern0pt}definitions{\isacharunderscore}{\kern0pt}equivalent{\isacharcolon}{\kern0pt}\ \isanewline
\ \ {\isachardoublequoteopen}{\isasymforall}{\isasymphi}{\isachardot}{\kern0pt}\ {\isacharparenleft}{\kern0pt}HML{\isacharunderscore}{\kern0pt}failure\ {\isasymphi}\ {\isasymlongrightarrow}\ {\isacharparenleft}{\kern0pt}{\isasymexists}{\isasympsi}{\isachardot}{\kern0pt}\ hml{\isacharunderscore}{\kern0pt}failure\ {\isasympsi}\ {\isasymand}\ {\isacharparenleft}{\kern0pt}s\ {\isasymTurnstile}\ {\isasympsi}\ {\isasymlongleftrightarrow}\ s\ {\isasymTurnstile}\ {\isasymphi}{\isacharparenright}{\kern0pt}{\isacharparenright}{\kern0pt}{\isacharparenright}{\kern0pt}{\isachardoublequoteclose}\isanewline
\ \ {\isachardoublequoteopen}{\isasymforall}{\isasymphi}{\isachardot}{\kern0pt}\ {\isacharparenleft}{\kern0pt}hml{\isacharunderscore}{\kern0pt}failure\ {\isasymphi}\ {\isasymlongrightarrow}\ {\isacharparenleft}{\kern0pt}{\isasymexists}{\isasympsi}{\isachardot}{\kern0pt}\ HML{\isacharunderscore}{\kern0pt}failure\ {\isasympsi}\ {\isasymand}\ {\isacharparenleft}{\kern0pt}s\ {\isasymTurnstile}\ {\isasympsi}\ {\isasymlongleftrightarrow}\ s\ {\isasymTurnstile}\ {\isasymphi}{\isacharparenright}{\kern0pt}{\isacharparenright}{\kern0pt}{\isacharparenright}{\kern0pt}{\isachardoublequoteclose}\isanewline
%
\isadelimproof
\ \ %
\endisadelimproof
%
\isatagproof
\isacommand{using}\isamarkupfalse%
\ failure{\isacharunderscore}{\kern0pt}def{\isacharunderscore}{\kern0pt}implies{\isacharunderscore}{\kern0pt}alt{\isacharunderscore}{\kern0pt}failure{\isacharunderscore}{\kern0pt}def\ alt{\isacharunderscore}{\kern0pt}failure{\isacharunderscore}{\kern0pt}def{\isacharunderscore}{\kern0pt}implies{\isacharunderscore}{\kern0pt}failure{\isacharunderscore}{\kern0pt}def\ \isacommand{by}\isamarkupfalse%
\ blast{\isacharplus}{\kern0pt}%
\endisatagproof
{\isafoldproof}%
%
\isadelimproof
\isanewline
%
\endisadelimproof
\isanewline
\isacommand{lemma}\isamarkupfalse%
\ alt{\isacharunderscore}{\kern0pt}readiness{\isacharunderscore}{\kern0pt}def{\isacharunderscore}{\kern0pt}implies{\isacharunderscore}{\kern0pt}readiness{\isacharunderscore}{\kern0pt}def{\isacharcolon}{\kern0pt}\isanewline
\ \ \isakeyword{fixes}\ {\isasymphi}\ {\isacharcolon}{\kern0pt}{\isacharcolon}{\kern0pt}\ {\isachardoublequoteopen}{\isacharparenleft}{\kern0pt}{\isacharprime}{\kern0pt}a{\isacharcomma}{\kern0pt}\ {\isacharprime}{\kern0pt}s{\isacharparenright}{\kern0pt}\ hml{\isachardoublequoteclose}\isanewline
\ \ \isakeyword{assumes}\ {\isachardoublequoteopen}hml{\isacharunderscore}{\kern0pt}readiness\ {\isasymphi}{\isachardoublequoteclose}\isanewline
\ \ \isakeyword{shows}\ {\isachardoublequoteopen}{\isasymexists}{\isasympsi}{\isachardot}{\kern0pt}\ HML{\isacharunderscore}{\kern0pt}readiness\ {\isasympsi}\ {\isasymand}\ {\isacharparenleft}{\kern0pt}{\isasymforall}s{\isachardot}{\kern0pt}\ {\isacharparenleft}{\kern0pt}s\ {\isasymTurnstile}\ {\isasymphi}{\isacharparenright}{\kern0pt}\ {\isasymlongleftrightarrow}\ {\isacharparenleft}{\kern0pt}s\ {\isasymTurnstile}\ {\isasympsi}{\isacharparenright}{\kern0pt}{\isacharparenright}{\kern0pt}{\isachardoublequoteclose}\isanewline
%
\isadelimproof
\ \ %
\endisadelimproof
%
\isatagproof
\isacommand{using}\isamarkupfalse%
\ assms\ \isacommand{proof}\isamarkupfalse%
\ induct\isanewline
\ \ \isacommand{case}\isamarkupfalse%
\ read{\isacharunderscore}{\kern0pt}tt\isanewline
\ \ \isacommand{then}\isamarkupfalse%
\ \isacommand{show}\isamarkupfalse%
\ {\isacharquery}{\kern0pt}case\ \isanewline
\ \ \ \ \isacommand{using}\isamarkupfalse%
\ HML{\isacharunderscore}{\kern0pt}readiness{\isachardot}{\kern0pt}read{\isacharunderscore}{\kern0pt}tt\ \isacommand{by}\isamarkupfalse%
\ blast\isanewline
\isacommand{next}\isamarkupfalse%
\isanewline
\ \ \isacommand{case}\isamarkupfalse%
\ {\isacharparenleft}{\kern0pt}read{\isacharunderscore}{\kern0pt}pos\ {\isasymphi}\ {\isasymalpha}{\isacharparenright}{\kern0pt}\isanewline
\ \ \isacommand{then}\isamarkupfalse%
\ \isacommand{show}\isamarkupfalse%
\ {\isacharquery}{\kern0pt}case\ \isanewline
\ \ \ \ \isacommand{using}\isamarkupfalse%
\ HML{\isacharunderscore}{\kern0pt}readiness{\isachardot}{\kern0pt}read{\isacharunderscore}{\kern0pt}pos\ \isacommand{by}\isamarkupfalse%
\ fastforce\isanewline
\isacommand{next}\isamarkupfalse%
\isanewline
\ \ \isacommand{case}\isamarkupfalse%
\ {\isacharparenleft}{\kern0pt}read{\isacharunderscore}{\kern0pt}conj\ {\isasymPhi}\ I\ J{\isacharparenright}{\kern0pt}\isanewline
\ \ \isacommand{hence}\isamarkupfalse%
\ {\isachardoublequoteopen}HML{\isacharunderscore}{\kern0pt}readiness\ {\isacharparenleft}{\kern0pt}hml{\isacharunderscore}{\kern0pt}conj\ I\ J\ {\isasymPhi}{\isacharparenright}{\kern0pt}{\isachardoublequoteclose}\ \isanewline
\ \ \ \ \isacommand{by}\isamarkupfalse%
\ {\isacharparenleft}{\kern0pt}metis\ HML{\isacharunderscore}{\kern0pt}readiness{\isachardot}{\kern0pt}read{\isacharunderscore}{\kern0pt}conj\ TT{\isacharunderscore}{\kern0pt}like{\isachardot}{\kern0pt}simps{\isacharparenright}{\kern0pt}\isanewline
\ \ \isacommand{then}\isamarkupfalse%
\ \isacommand{show}\isamarkupfalse%
\ {\isacharquery}{\kern0pt}case\ \isacommand{by}\isamarkupfalse%
\ blast\isanewline
\isacommand{qed}\isamarkupfalse%
%
\endisatagproof
{\isafoldproof}%
%
\isadelimproof
\isanewline
%
\endisadelimproof
\isanewline
\isacommand{lemma}\isamarkupfalse%
\ readiness{\isacharunderscore}{\kern0pt}def{\isacharunderscore}{\kern0pt}implies{\isacharunderscore}{\kern0pt}alt{\isacharunderscore}{\kern0pt}readiness{\isacharunderscore}{\kern0pt}def{\isacharcolon}{\kern0pt}\isanewline
\ \ \isakeyword{fixes}\ {\isasymphi}\ {\isacharcolon}{\kern0pt}{\isacharcolon}{\kern0pt}\ {\isachardoublequoteopen}{\isacharparenleft}{\kern0pt}{\isacharprime}{\kern0pt}a{\isacharcomma}{\kern0pt}\ {\isacharprime}{\kern0pt}s{\isacharparenright}{\kern0pt}\ hml{\isachardoublequoteclose}\isanewline
\ \ \isakeyword{assumes}\ {\isachardoublequoteopen}HML{\isacharunderscore}{\kern0pt}readiness\ {\isasymphi}{\isachardoublequoteclose}\isanewline
\ \ \isakeyword{shows}\ {\isachardoublequoteopen}{\isasymexists}{\isasympsi}{\isachardot}{\kern0pt}\ hml{\isacharunderscore}{\kern0pt}readiness\ {\isasympsi}\ {\isasymand}\ {\isacharparenleft}{\kern0pt}{\isasymforall}s{\isachardot}{\kern0pt}\ {\isacharparenleft}{\kern0pt}s\ {\isasymTurnstile}\ {\isasymphi}{\isacharparenright}{\kern0pt}\ {\isasymlongleftrightarrow}\ {\isacharparenleft}{\kern0pt}s\ {\isasymTurnstile}\ {\isasympsi}{\isacharparenright}{\kern0pt}{\isacharparenright}{\kern0pt}{\isachardoublequoteclose}\isanewline
%
\isadelimproof
\ \ %
\endisadelimproof
%
\isatagproof
\isacommand{using}\isamarkupfalse%
\ assms\ \isacommand{proof}\isamarkupfalse%
{\isacharparenleft}{\kern0pt}induct{\isacharparenright}{\kern0pt}\isanewline
\ \ \isacommand{case}\isamarkupfalse%
\ read{\isacharunderscore}{\kern0pt}tt\isanewline
\ \ \isacommand{then}\isamarkupfalse%
\ \isacommand{show}\isamarkupfalse%
\ {\isacharquery}{\kern0pt}case\ \isanewline
\ \ \ \ \isacommand{using}\isamarkupfalse%
\ hml{\isacharunderscore}{\kern0pt}readiness{\isachardot}{\kern0pt}read{\isacharunderscore}{\kern0pt}tt\ \isacommand{by}\isamarkupfalse%
\ blast\isanewline
\isacommand{next}\isamarkupfalse%
\isanewline
\ \ \isacommand{case}\isamarkupfalse%
\ {\isacharparenleft}{\kern0pt}read{\isacharunderscore}{\kern0pt}pos\ {\isasymphi}\ {\isasymalpha}{\isacharparenright}{\kern0pt}\isanewline
\ \ \isacommand{then}\isamarkupfalse%
\ \isacommand{obtain}\isamarkupfalse%
\ {\isasympsi}\ \isakeyword{where}\ {\isachardoublequoteopen}hml{\isacharunderscore}{\kern0pt}readiness\ {\isasympsi}{\isachardoublequoteclose}\ {\isachardoublequoteopen}{\isacharparenleft}{\kern0pt}{\isasymforall}s{\isachardot}{\kern0pt}\ {\isacharparenleft}{\kern0pt}s\ {\isasymTurnstile}\ {\isasymphi}{\isacharparenright}{\kern0pt}\ {\isacharequal}{\kern0pt}\ {\isacharparenleft}{\kern0pt}s\ {\isasymTurnstile}\ {\isasympsi}{\isacharparenright}{\kern0pt}{\isacharparenright}{\kern0pt}{\isachardoublequoteclose}\ \isacommand{by}\isamarkupfalse%
\ blast\isanewline
\ \ \isacommand{hence}\isamarkupfalse%
\ {\isachardoublequoteopen}hml{\isacharunderscore}{\kern0pt}readiness\ {\isacharparenleft}{\kern0pt}hml{\isacharunderscore}{\kern0pt}pos\ {\isasymalpha}\ {\isasympsi}{\isacharparenright}{\kern0pt}\ {\isasymand}\ {\isacharparenleft}{\kern0pt}{\isasymforall}s{\isachardot}{\kern0pt}\ {\isacharparenleft}{\kern0pt}s\ {\isasymTurnstile}\ hml{\isacharunderscore}{\kern0pt}pos\ {\isasymalpha}\ {\isasymphi}{\isacharparenright}{\kern0pt}\ {\isacharequal}{\kern0pt}\ {\isacharparenleft}{\kern0pt}s\ {\isasymTurnstile}\ {\isacharparenleft}{\kern0pt}hml{\isacharunderscore}{\kern0pt}pos\ {\isasymalpha}\ {\isasympsi}{\isacharparenright}{\kern0pt}{\isacharparenright}{\kern0pt}{\isacharparenright}{\kern0pt}{\isachardoublequoteclose}\isanewline
\ \ \ \ \isacommand{by}\isamarkupfalse%
\ {\isacharparenleft}{\kern0pt}simp\ add{\isacharcolon}{\kern0pt}\ hml{\isacharunderscore}{\kern0pt}readiness{\isachardot}{\kern0pt}read{\isacharunderscore}{\kern0pt}pos{\isacharparenright}{\kern0pt}\isanewline
\ \ \isacommand{then}\isamarkupfalse%
\ \isacommand{show}\isamarkupfalse%
\ {\isacharquery}{\kern0pt}case\ \isacommand{by}\isamarkupfalse%
\ blast\isanewline
\isacommand{next}\isamarkupfalse%
\isanewline
\ \ \isacommand{case}\isamarkupfalse%
\ {\isacharparenleft}{\kern0pt}read{\isacharunderscore}{\kern0pt}conj\ {\isasymPhi}\ I\ J{\isacharparenright}{\kern0pt}\isanewline
\ \ \isacommand{then}\isamarkupfalse%
\ \isacommand{consider}\isamarkupfalse%
\ {\isachardoublequoteopen}{\isasymPhi}\ {\isacharbackquote}{\kern0pt}\ I\ {\isasyminter}\ {\isasymPhi}\ {\isacharbackquote}{\kern0pt}\ J\ {\isacharequal}{\kern0pt}\ {\isacharbraceleft}{\kern0pt}{\isacharbraceright}{\kern0pt}\ {\isasymand}\ {\isacharparenleft}{\kern0pt}{\isasymforall}x\ {\isasymin}\ {\isacharparenleft}{\kern0pt}{\isasymPhi}\ {\isacharbackquote}{\kern0pt}\ J{\isacharparenright}{\kern0pt}{\isachardot}{\kern0pt}\ {\isacharparenleft}{\kern0pt}{\isasymexists}{\isasymalpha}\ {\isasymchi}{\isachardot}{\kern0pt}\ x\ {\isacharequal}{\kern0pt}\ hml{\isacharunderscore}{\kern0pt}pos\ {\isasymalpha}\ {\isasymchi}\ {\isasymand}\ TT{\isacharunderscore}{\kern0pt}like\ {\isasymchi}{\isacharparenright}{\kern0pt}{\isacharparenright}{\kern0pt}{\isachardoublequoteclose}\isanewline
\ \ \ \ {\isacharbar}{\kern0pt}\ {\isachardoublequoteopen}{\isasymPhi}\ {\isacharbackquote}{\kern0pt}\ I\ {\isasyminter}\ {\isasymPhi}\ {\isacharbackquote}{\kern0pt}\ J\ {\isasymnoteq}\ {\isacharbraceleft}{\kern0pt}{\isacharbraceright}{\kern0pt}\ {\isasymor}\ {\isacharparenleft}{\kern0pt}{\isasymexists}x\ {\isasymin}{\isasymPhi}{\isacharbackquote}{\kern0pt}\ J{\isachardot}{\kern0pt}\ {\isacharparenleft}{\kern0pt}TT{\isacharunderscore}{\kern0pt}like\ x{\isacharparenright}{\kern0pt}{\isacharparenright}{\kern0pt}{\isachardoublequoteclose}\ \isanewline
\ \ \ \ \isacommand{by}\isamarkupfalse%
\ blast\isanewline
\ \ \isacommand{then}\isamarkupfalse%
\ \isacommand{show}\isamarkupfalse%
\ {\isacharquery}{\kern0pt}case\ \isacommand{proof}\isamarkupfalse%
{\isacharparenleft}{\kern0pt}cases{\isacharparenright}{\kern0pt}\isanewline
\ \ \ \ \isacommand{case}\isamarkupfalse%
\ {\isadigit{1}}\isanewline
\ \ \ \ \isacommand{hence}\isamarkupfalse%
\ {\isachardoublequoteopen}{\isasymforall}j\ {\isasymin}\ J{\isachardot}{\kern0pt}\ {\isacharparenleft}{\kern0pt}{\isasymexists}{\isasymalpha}\ {\isasymchi}{\isachardot}{\kern0pt}\ {\isasymPhi}\ j\ {\isacharequal}{\kern0pt}\ hml{\isacharunderscore}{\kern0pt}pos\ {\isasymalpha}\ {\isasymchi}\ {\isasymand}\ TT{\isacharunderscore}{\kern0pt}like\ {\isasymchi}{\isacharparenright}{\kern0pt}{\isachardoublequoteclose}\ \isanewline
\ \ \ \ \ \ \isacommand{by}\isamarkupfalse%
\ blast\isanewline
\ \ \ \ \isacommand{define}\isamarkupfalse%
\ {\isasymPsi}\ \isakeyword{where}\ {\isachardoublequoteopen}{\isasymPsi}\ {\isasymequiv}\ {\isacharparenleft}{\kern0pt}{\isasymlambda}i{\isachardot}{\kern0pt}\ {\isacharparenleft}{\kern0pt}if\ {\isacharparenleft}{\kern0pt}{\isasymexists}{\isasymalpha}\ {\isasymchi}{\isachardot}{\kern0pt}\ {\isasymPhi}\ i\ {\isacharequal}{\kern0pt}\ hml{\isacharunderscore}{\kern0pt}pos\ {\isasymalpha}\ {\isasymchi}\ {\isasymand}\ TT{\isacharunderscore}{\kern0pt}like\ {\isasymchi}{\isacharparenright}{\kern0pt}\isanewline
\ \ \ \ \ \ \ \ \ \ \ \ \ \ \ \ \ \ \ \ \ \ \ \ \ \ then\ {\isacharparenleft}{\kern0pt}hml{\isacharunderscore}{\kern0pt}pos\ {\isacharparenleft}{\kern0pt}SOME\ {\isasymalpha}{\isachardot}{\kern0pt}\ {\isasymexists}{\isasymchi}{\isachardot}{\kern0pt}\ {\isasymPhi}\ i\ {\isacharequal}{\kern0pt}\ hml{\isacharunderscore}{\kern0pt}pos\ {\isasymalpha}\ {\isasymchi}\ {\isasymand}\ TT{\isacharunderscore}{\kern0pt}like\ {\isasymchi}{\isacharparenright}{\kern0pt}\ TT{\isacharparenright}{\kern0pt}{\isacharcolon}{\kern0pt}{\isacharcolon}{\kern0pt}{\isacharparenleft}{\kern0pt}{\isacharprime}{\kern0pt}a{\isacharcomma}{\kern0pt}\ {\isacharprime}{\kern0pt}s{\isacharparenright}{\kern0pt}hml\ \isanewline
\ \ \ \ \ \ \ \ \ \ \ \ \ \ \ \ \ \ \ \ \ \ \ \ \ \ else\ undefined{\isacharparenright}{\kern0pt}{\isacharparenright}{\kern0pt}{\isachardoublequoteclose}\isanewline
\ \ \ \ \isacommand{hence}\isamarkupfalse%
\ {\isachardoublequoteopen}{\isasymforall}{\isasympsi}\ {\isasymin}\ {\isasymPsi}\ {\isacharbackquote}{\kern0pt}\ J{\isachardot}{\kern0pt}\ {\isasymexists}{\isasymalpha}{\isachardot}{\kern0pt}\ {\isasympsi}\ {\isacharequal}{\kern0pt}\ hml{\isacharunderscore}{\kern0pt}pos\ {\isasymalpha}\ TT{\isachardoublequoteclose}\isanewline
\ \ \ \ \ \ \isacommand{by}\isamarkupfalse%
\ {\isacharparenleft}{\kern0pt}simp\ add{\isacharcolon}{\kern0pt}\ {\isacartoucheopen}{\isasymforall}j{\isasymin}J{\isachardot}{\kern0pt}\ {\isasymexists}{\isasymalpha}\ {\isasymchi}{\isachardot}{\kern0pt}\ {\isasymPhi}\ j\ {\isacharequal}{\kern0pt}\ hml{\isacharunderscore}{\kern0pt}pos\ {\isasymalpha}\ {\isasymchi}\ {\isasymand}\ TT{\isacharunderscore}{\kern0pt}like\ {\isasymchi}{\isacartoucheclose}{\isacharparenright}{\kern0pt}\isanewline
\ \ \ \ \isacommand{define}\isamarkupfalse%
\ I{\isacharprime}{\kern0pt}\ \isakeyword{where}\ {\isachardoublequoteopen}I{\isacharprime}{\kern0pt}\ {\isasymequiv}\ {\isacharbraceleft}{\kern0pt}i{\isachardot}{\kern0pt}\ i\ {\isasymin}\ I\ {\isasymand}\ {\isacharparenleft}{\kern0pt}{\isacharparenleft}{\kern0pt}{\isasymexists}{\isasymalpha}\ {\isasymchi}{\isachardot}{\kern0pt}\ {\isasymPhi}\ i\ {\isacharequal}{\kern0pt}\ hml{\isacharunderscore}{\kern0pt}pos\ {\isasymalpha}\ {\isasymchi}\ {\isasymand}\ TT{\isacharunderscore}{\kern0pt}like\ {\isasymchi}{\isacharparenright}{\kern0pt}{\isacharparenright}{\kern0pt}{\isacharbraceright}{\kern0pt}{\isachardoublequoteclose}\isanewline
\ \ \ \ \isacommand{have}\isamarkupfalse%
\ {\isachardoublequoteopen}{\isasymforall}{\isasympsi}\ {\isasymin}\ {\isasymPsi}\ {\isacharbackquote}{\kern0pt}\ I{\isacharprime}{\kern0pt}{\isachardot}{\kern0pt}\ {\isasymexists}{\isasymalpha}{\isachardot}{\kern0pt}\ {\isasympsi}\ {\isacharequal}{\kern0pt}\ hml{\isacharunderscore}{\kern0pt}pos\ {\isasymalpha}\ TT{\isachardoublequoteclose}\isanewline
\ \ \ \ \ \ \isacommand{unfolding}\isamarkupfalse%
\ I{\isacharprime}{\kern0pt}{\isacharunderscore}{\kern0pt}def\ {\isasymPsi}{\isacharunderscore}{\kern0pt}def\isanewline
\ \ \ \ \ \ \isacommand{by}\isamarkupfalse%
\ force\isanewline
\ \ \ \ \isacommand{hence}\isamarkupfalse%
\ {\isachardoublequoteopen}{\isasymforall}j\ {\isasymin}\ {\isacharparenleft}{\kern0pt}J\ {\isasymunion}\ I{\isacharprime}{\kern0pt}{\isacharparenright}{\kern0pt}{\isachardot}{\kern0pt}\ {\isasymexists}{\isasymalpha}\ {\isasymchi}{\isachardot}{\kern0pt}\ {\isasymPhi}\ j\ {\isacharequal}{\kern0pt}\ hml{\isacharunderscore}{\kern0pt}pos\ {\isasymalpha}\ {\isasymchi}\ {\isasymand}\ TT{\isacharunderscore}{\kern0pt}like\ {\isasymchi}\ {\isasymand}\ {\isasymPsi}\ j\ {\isacharequal}{\kern0pt}\ hml{\isacharunderscore}{\kern0pt}pos\ {\isasymalpha}\ TT{\isachardoublequoteclose}\ \isanewline
\ \ \ \ \ \ \isacommand{using}\isamarkupfalse%
\ {\isasymPsi}{\isacharunderscore}{\kern0pt}def\ {\isacartoucheopen}{\isasymforall}j\ {\isasymin}\ J{\isachardot}{\kern0pt}\ {\isacharparenleft}{\kern0pt}{\isasymexists}{\isasymalpha}\ {\isasymchi}{\isachardot}{\kern0pt}\ {\isasymPhi}\ j\ {\isacharequal}{\kern0pt}\ hml{\isacharunderscore}{\kern0pt}pos\ {\isasymalpha}\ {\isasymchi}\ {\isasymand}\ TT{\isacharunderscore}{\kern0pt}like\ {\isasymchi}{\isacharparenright}{\kern0pt}{\isacartoucheclose}\ \isanewline
\ \ \ \ \ \ \isacommand{unfolding}\isamarkupfalse%
\ {\isasymPsi}{\isacharunderscore}{\kern0pt}def\ I{\isacharprime}{\kern0pt}{\isacharunderscore}{\kern0pt}def\isanewline
\ \ \ \ \ \ \isacommand{by}\isamarkupfalse%
\ force\isanewline
\ \ \ \ \isacommand{hence}\isamarkupfalse%
\ {\isachardoublequoteopen}{\isacharparenleft}{\kern0pt}{\isasymforall}s{\isachardot}{\kern0pt}\ {\isasymforall}j\ {\isasymin}\ J\ {\isasymunion}\ I{\isacharprime}{\kern0pt}{\isachardot}{\kern0pt}\ {\isacharparenleft}{\kern0pt}{\isasymnot}{\isacharparenleft}{\kern0pt}s\ {\isasymTurnstile}\ {\isacharparenleft}{\kern0pt}{\isasymPsi}\ j{\isacharparenright}{\kern0pt}{\isacharparenright}{\kern0pt}\ {\isacharequal}{\kern0pt}\ {\isacharparenleft}{\kern0pt}{\isasymnot}{\isacharparenleft}{\kern0pt}s\ {\isasymTurnstile}\ {\isacharparenleft}{\kern0pt}{\isasymPhi}\ j{\isacharparenright}{\kern0pt}{\isacharparenright}{\kern0pt}{\isacharparenright}{\kern0pt}{\isacharparenright}{\kern0pt}{\isacharparenright}{\kern0pt}{\isachardoublequoteclose}\ \isanewline
\ \ \ \ \ \ \isacommand{using}\isamarkupfalse%
\ HML{\isacharunderscore}{\kern0pt}true{\isacharunderscore}{\kern0pt}TT{\isacharunderscore}{\kern0pt}like\ HML{\isacharunderscore}{\kern0pt}true{\isacharunderscore}{\kern0pt}def\ \isanewline
\ \ \ \ \ \ \isacommand{by}\isamarkupfalse%
\ {\isacharparenleft}{\kern0pt}metis\ hml{\isacharunderscore}{\kern0pt}sem{\isacharunderscore}{\kern0pt}pos\ hml{\isacharunderscore}{\kern0pt}sem{\isacharunderscore}{\kern0pt}tt{\isacharparenright}{\kern0pt}\isanewline
\ \ \ \ \isacommand{have}\isamarkupfalse%
\ {\isachardoublequoteopen}{\isasymforall}x\ {\isasymin}\ {\isacharparenleft}{\kern0pt}I\ {\isacharminus}{\kern0pt}\ I{\isacharprime}{\kern0pt}{\isacharparenright}{\kern0pt}{\isachardot}{\kern0pt}\ TT{\isacharunderscore}{\kern0pt}like\ {\isacharparenleft}{\kern0pt}{\isasymPhi}\ x{\isacharparenright}{\kern0pt}{\isachardoublequoteclose}\isanewline
\ \ \ \ \ \ \isacommand{using}\isamarkupfalse%
\ read{\isacharunderscore}{\kern0pt}conj\ {\isadigit{1}}\isanewline
\ \ \ \ \ \ \isacommand{unfolding}\isamarkupfalse%
\ I{\isacharprime}{\kern0pt}{\isacharunderscore}{\kern0pt}def\isanewline
\ \ \ \ \ \ \isacommand{by}\isamarkupfalse%
\ auto\isanewline
\ \ \ \ \isacommand{hence}\isamarkupfalse%
\ {\isachardoublequoteopen}{\isacharparenleft}{\kern0pt}{\isasymforall}s{\isachardot}{\kern0pt}\ {\isacharparenleft}{\kern0pt}s\ {\isasymTurnstile}\ hml{\isacharunderscore}{\kern0pt}conj\ I\ J\ {\isasymPhi}{\isacharparenright}{\kern0pt}\ {\isacharequal}{\kern0pt}\ {\isacharparenleft}{\kern0pt}s\ {\isasymTurnstile}\ {\isacharparenleft}{\kern0pt}hml{\isacharunderscore}{\kern0pt}conj\ I{\isacharprime}{\kern0pt}\ J\ {\isasymPhi}{\isacharparenright}{\kern0pt}{\isacharparenright}{\kern0pt}{\isacharparenright}{\kern0pt}{\isachardoublequoteclose}\isanewline
\ \ \ \ \ \ \isacommand{using}\isamarkupfalse%
\ HML{\isacharunderscore}{\kern0pt}true{\isacharunderscore}{\kern0pt}TT{\isacharunderscore}{\kern0pt}like\ read{\isacharunderscore}{\kern0pt}conj\ {\isadigit{1}}\isanewline
\ \ \ \ \ \ \isacommand{unfolding}\isamarkupfalse%
\ I{\isacharprime}{\kern0pt}{\isacharunderscore}{\kern0pt}def\ HML{\isacharunderscore}{\kern0pt}true{\isacharunderscore}{\kern0pt}def\ \isanewline
\ \ \ \ \ \ \isacommand{by}\isamarkupfalse%
\ {\isacharparenleft}{\kern0pt}smt\ {\isacharparenleft}{\kern0pt}verit{\isacharcomma}{\kern0pt}\ del{\isacharunderscore}{\kern0pt}insts{\isacharparenright}{\kern0pt}\ Diff{\isacharunderscore}{\kern0pt}iff\ hml{\isacharunderscore}{\kern0pt}sem{\isacharunderscore}{\kern0pt}conj\ mem{\isacharunderscore}{\kern0pt}Collect{\isacharunderscore}{\kern0pt}eq{\isacharparenright}{\kern0pt}\isanewline
\ \ \ \ \isacommand{hence}\isamarkupfalse%
\ {\isachardoublequoteopen}{\isacharparenleft}{\kern0pt}{\isasymforall}s{\isachardot}{\kern0pt}\ {\isacharparenleft}{\kern0pt}s\ {\isasymTurnstile}\ hml{\isacharunderscore}{\kern0pt}conj\ I\ J\ {\isasymPhi}{\isacharparenright}{\kern0pt}\ {\isacharequal}{\kern0pt}\ {\isacharparenleft}{\kern0pt}s\ {\isasymTurnstile}\ {\isacharparenleft}{\kern0pt}hml{\isacharunderscore}{\kern0pt}conj\ I{\isacharprime}{\kern0pt}\ J\ {\isasymPsi}{\isacharparenright}{\kern0pt}{\isacharparenright}{\kern0pt}{\isacharparenright}{\kern0pt}{\isachardoublequoteclose}\isanewline
\ \ \ \ \ \ \isacommand{using}\isamarkupfalse%
\ {\isacartoucheopen}{\isacharparenleft}{\kern0pt}{\isasymforall}s{\isachardot}{\kern0pt}\ {\isasymforall}j\ {\isasymin}\ J\ {\isasymunion}\ I{\isacharprime}{\kern0pt}{\isachardot}{\kern0pt}\ {\isacharparenleft}{\kern0pt}{\isasymnot}{\isacharparenleft}{\kern0pt}s\ {\isasymTurnstile}\ {\isacharparenleft}{\kern0pt}{\isasymPsi}\ j{\isacharparenright}{\kern0pt}{\isacharparenright}{\kern0pt}\ {\isacharequal}{\kern0pt}\ {\isacharparenleft}{\kern0pt}{\isasymnot}{\isacharparenleft}{\kern0pt}s\ {\isasymTurnstile}\ {\isacharparenleft}{\kern0pt}{\isasymPhi}\ j{\isacharparenright}{\kern0pt}{\isacharparenright}{\kern0pt}{\isacharparenright}{\kern0pt}{\isacharparenright}{\kern0pt}{\isacharparenright}{\kern0pt}{\isacartoucheclose}\isanewline
\ \ \ \ \ \ \isacommand{by}\isamarkupfalse%
\ simp\isanewline
\ \ \ \ \isacommand{have}\isamarkupfalse%
\ {\isachardoublequoteopen}hml{\isacharunderscore}{\kern0pt}readiness\ {\isacharparenleft}{\kern0pt}hml{\isacharunderscore}{\kern0pt}conj\ I{\isacharprime}{\kern0pt}\ J\ {\isasymPsi}{\isacharparenright}{\kern0pt}{\isachardoublequoteclose}\ \isanewline
\ \ \ \ \ \ \isacommand{using}\isamarkupfalse%
\ {\isasymPsi}{\isacharunderscore}{\kern0pt}def\ I{\isacharprime}{\kern0pt}{\isacharunderscore}{\kern0pt}def\isanewline
\ \ \ \ \ \ \isacommand{using}\isamarkupfalse%
\ hml{\isacharunderscore}{\kern0pt}readiness{\isachardot}{\kern0pt}simps\ \isanewline
\ \ \ \ \ \ \isacommand{by}\isamarkupfalse%
\ {\isacharparenleft}{\kern0pt}smt\ {\isacharparenleft}{\kern0pt}verit{\isacharcomma}{\kern0pt}\ best{\isacharparenright}{\kern0pt}\ Un{\isacharunderscore}{\kern0pt}iff\ {\isacartoucheopen}{\isasymforall}{\isasympsi}{\isasymin}{\isasymPsi}\ {\isacharbackquote}{\kern0pt}\ I{\isacharprime}{\kern0pt}{\isachardot}{\kern0pt}\ {\isasymexists}{\isasymalpha}{\isachardot}{\kern0pt}\ {\isasympsi}\ {\isacharequal}{\kern0pt}\ hml{\isacharunderscore}{\kern0pt}pos\ {\isasymalpha}\ TT{\isacartoucheclose}\ {\isacartoucheopen}{\isasymforall}{\isasympsi}{\isasymin}{\isasymPsi}\ {\isacharbackquote}{\kern0pt}\ J{\isachardot}{\kern0pt}\ {\isasymexists}{\isasymalpha}{\isachardot}{\kern0pt}\ {\isasympsi}\ {\isacharequal}{\kern0pt}\ hml{\isacharunderscore}{\kern0pt}pos\ {\isasymalpha}\ TT{\isacartoucheclose}\ image{\isacharunderscore}{\kern0pt}Un{\isacharparenright}{\kern0pt}\isanewline
\ \ \ \ \isacommand{then}\isamarkupfalse%
\ \isacommand{show}\isamarkupfalse%
\ {\isacharquery}{\kern0pt}thesis\ \isanewline
\ \ \ \ \ \ \isacommand{using}\isamarkupfalse%
\ {\isacartoucheopen}{\isasymforall}s{\isachardot}{\kern0pt}\ {\isacharparenleft}{\kern0pt}s\ {\isasymTurnstile}\ hml{\isacharunderscore}{\kern0pt}conj\ I\ J\ {\isasymPhi}{\isacharparenright}{\kern0pt}\ {\isacharequal}{\kern0pt}\ {\isacharparenleft}{\kern0pt}s\ {\isasymTurnstile}\ {\isacharparenleft}{\kern0pt}hml{\isacharunderscore}{\kern0pt}conj\ I{\isacharprime}{\kern0pt}\ J\ {\isasymPsi}{\isacharparenright}{\kern0pt}{\isacharparenright}{\kern0pt}{\isacartoucheclose}\ \isacommand{by}\isamarkupfalse%
\ blast\isanewline
\ \ \isacommand{next}\isamarkupfalse%
\isanewline
\ \ \ \ \isacommand{case}\isamarkupfalse%
\ {\isadigit{2}}\isanewline
\ \ \ \ \isacommand{hence}\isamarkupfalse%
\ {\isachardoublequoteopen}{\isasymforall}s{\isachardot}{\kern0pt}\ {\isasymnot}s\ {\isasymTurnstile}\ {\isacharparenleft}{\kern0pt}hml{\isacharunderscore}{\kern0pt}conj\ I\ J\ {\isasymPhi}{\isacharparenright}{\kern0pt}{\isachardoublequoteclose}\ \isanewline
\ \ \ \ \ \ \isacommand{using}\isamarkupfalse%
\ HML{\isacharunderscore}{\kern0pt}true{\isacharunderscore}{\kern0pt}TT{\isacharunderscore}{\kern0pt}like\ HML{\isacharunderscore}{\kern0pt}true{\isacharunderscore}{\kern0pt}def\ \isacommand{by}\isamarkupfalse%
\ fastforce\ \isanewline
\ \ \ \ \isacommand{obtain}\isamarkupfalse%
\ {\isasymphi}\ i{\isacharunderscore}{\kern0pt}{\isasymphi}\ \isakeyword{where}\ {\isachardoublequoteopen}{\isasymphi}\ {\isasymin}\ {\isasymPhi}\ {\isacharbackquote}{\kern0pt}\ I\ {\isasyminter}\ {\isasymPhi}\ {\isacharbackquote}{\kern0pt}\ J\ {\isasymor}\ {\isacharparenleft}{\kern0pt}{\isasymphi}\ {\isasymin}\ {\isasymPhi}\ {\isacharbackquote}{\kern0pt}\ J\ {\isasymand}\ TT{\isacharunderscore}{\kern0pt}like\ {\isasymphi}{\isacharparenright}{\kern0pt}{\isachardoublequoteclose}\ {\isachardoublequoteopen}{\isasymPhi}\ i{\isacharunderscore}{\kern0pt}{\isasymphi}\ {\isacharequal}{\kern0pt}\ {\isasymphi}{\isachardoublequoteclose}\isanewline
\ \ \ \ \ \ \isacommand{using}\isamarkupfalse%
\ {\isadigit{2}}\ \isacommand{by}\isamarkupfalse%
\ blast\isanewline
\ \ \ \ \isacommand{define}\isamarkupfalse%
\ {\isasymPsi}\ \isakeyword{where}\ {\isachardoublequoteopen}{\isasymPsi}\ {\isasymequiv}\ {\isacharparenleft}{\kern0pt}{\isasymlambda}i{\isachardot}{\kern0pt}\ {\isacharparenleft}{\kern0pt}if\ i\ {\isacharequal}{\kern0pt}\ i{\isacharunderscore}{\kern0pt}{\isasymphi}\ then\ TT{\isacharcolon}{\kern0pt}{\isacharcolon}{\kern0pt}{\isacharparenleft}{\kern0pt}{\isacharprime}{\kern0pt}a{\isacharcomma}{\kern0pt}\ {\isacharprime}{\kern0pt}s{\isacharparenright}{\kern0pt}hml\ else\ undefined{\isacharparenright}{\kern0pt}{\isacharparenright}{\kern0pt}{\isachardoublequoteclose}\isanewline
\ \ \ \ \isacommand{have}\isamarkupfalse%
\ {\isachardoublequoteopen}hml{\isacharunderscore}{\kern0pt}readiness\ {\isacharparenleft}{\kern0pt}hml{\isacharunderscore}{\kern0pt}conj\ {\isacharbraceleft}{\kern0pt}{\isacharbraceright}{\kern0pt}\ {\isacharbraceleft}{\kern0pt}i{\isacharunderscore}{\kern0pt}{\isasymphi}{\isacharbraceright}{\kern0pt}\ {\isasymPsi}{\isacharparenright}{\kern0pt}{\isachardoublequoteclose}\ \isanewline
\ \ \ \ \ \ \isacommand{by}\isamarkupfalse%
\ {\isacharparenleft}{\kern0pt}simp\ add{\isacharcolon}{\kern0pt}\ {\isasymPsi}{\isacharunderscore}{\kern0pt}def\ hml{\isacharunderscore}{\kern0pt}readiness{\isachardot}{\kern0pt}read{\isacharunderscore}{\kern0pt}conj{\isacharparenright}{\kern0pt}\isanewline
\ \ \ \ \isacommand{have}\isamarkupfalse%
\ {\isachardoublequoteopen}{\isasymforall}s{\isachardot}{\kern0pt}\ {\isasymnot}s\ {\isasymTurnstile}\ {\isacharparenleft}{\kern0pt}hml{\isacharunderscore}{\kern0pt}conj\ {\isacharbraceleft}{\kern0pt}{\isacharbraceright}{\kern0pt}\ {\isacharbraceleft}{\kern0pt}i{\isacharunderscore}{\kern0pt}{\isasymphi}{\isacharbraceright}{\kern0pt}\ {\isasymPsi}{\isacharparenright}{\kern0pt}{\isachardoublequoteclose}\ \isanewline
\ \ \ \ \ \ \isacommand{by}\isamarkupfalse%
\ {\isacharparenleft}{\kern0pt}simp\ add{\isacharcolon}{\kern0pt}\ {\isasymPsi}{\isacharunderscore}{\kern0pt}def{\isacharparenright}{\kern0pt}\isanewline
\ \ \ \ \isacommand{then}\isamarkupfalse%
\ \isacommand{show}\isamarkupfalse%
\ {\isacharquery}{\kern0pt}thesis\ \isanewline
\ \ \ \ \ \ \isacommand{using}\isamarkupfalse%
\ {\isacartoucheopen}{\isasymforall}s{\isachardot}{\kern0pt}\ {\isasymnot}\ s\ {\isasymTurnstile}\ hml{\isacharunderscore}{\kern0pt}conj\ I\ J\ {\isasymPhi}{\isacartoucheclose}\ {\isacartoucheopen}hml{\isacharunderscore}{\kern0pt}readiness\ {\isacharparenleft}{\kern0pt}hml{\isacharunderscore}{\kern0pt}conj\ {\isacharbraceleft}{\kern0pt}{\isacharbraceright}{\kern0pt}\ {\isacharbraceleft}{\kern0pt}i{\isacharunderscore}{\kern0pt}{\isasymphi}{\isacharbraceright}{\kern0pt}\ {\isasymPsi}{\isacharparenright}{\kern0pt}{\isacartoucheclose}\ \isacommand{by}\isamarkupfalse%
\ blast\isanewline
\ \ \isacommand{qed}\isamarkupfalse%
\isanewline
\isacommand{qed}\isamarkupfalse%
%
\endisatagproof
{\isafoldproof}%
%
\isadelimproof
\isanewline
%
\endisadelimproof
\isanewline
\isacommand{lemma}\isamarkupfalse%
\ readiness{\isacharunderscore}{\kern0pt}definitions{\isacharunderscore}{\kern0pt}equivalent{\isacharcolon}{\kern0pt}\ \isanewline
\ \ {\isachardoublequoteopen}{\isasymforall}{\isasymphi}{\isachardot}{\kern0pt}\ {\isacharparenleft}{\kern0pt}HML{\isacharunderscore}{\kern0pt}readiness\ {\isasymphi}\ {\isasymlongrightarrow}\ {\isacharparenleft}{\kern0pt}{\isasymexists}{\isasympsi}{\isachardot}{\kern0pt}\ hml{\isacharunderscore}{\kern0pt}readiness\ {\isasympsi}\ {\isasymand}\ {\isacharparenleft}{\kern0pt}s\ {\isasymTurnstile}\ {\isasympsi}\ {\isasymlongleftrightarrow}\ s\ {\isasymTurnstile}\ {\isasymphi}{\isacharparenright}{\kern0pt}{\isacharparenright}{\kern0pt}{\isacharparenright}{\kern0pt}{\isachardoublequoteclose}\isanewline
\ \ {\isachardoublequoteopen}{\isasymforall}{\isasymphi}{\isachardot}{\kern0pt}\ {\isacharparenleft}{\kern0pt}hml{\isacharunderscore}{\kern0pt}readiness\ {\isasymphi}\ {\isasymlongrightarrow}\ {\isacharparenleft}{\kern0pt}{\isasymexists}{\isasympsi}{\isachardot}{\kern0pt}\ HML{\isacharunderscore}{\kern0pt}readiness\ {\isasympsi}\ {\isasymand}\ {\isacharparenleft}{\kern0pt}s\ {\isasymTurnstile}\ {\isasympsi}\ {\isasymlongleftrightarrow}\ s\ {\isasymTurnstile}\ {\isasymphi}{\isacharparenright}{\kern0pt}{\isacharparenright}{\kern0pt}{\isacharparenright}{\kern0pt}{\isachardoublequoteclose}\isanewline
%
\isadelimproof
\ \ %
\endisadelimproof
%
\isatagproof
\isacommand{using}\isamarkupfalse%
\ readiness{\isacharunderscore}{\kern0pt}def{\isacharunderscore}{\kern0pt}implies{\isacharunderscore}{\kern0pt}alt{\isacharunderscore}{\kern0pt}readiness{\isacharunderscore}{\kern0pt}def\ alt{\isacharunderscore}{\kern0pt}readiness{\isacharunderscore}{\kern0pt}def{\isacharunderscore}{\kern0pt}implies{\isacharunderscore}{\kern0pt}readiness{\isacharunderscore}{\kern0pt}def\ \isacommand{by}\isamarkupfalse%
\ blast{\isacharplus}{\kern0pt}%
\endisatagproof
{\isafoldproof}%
%
\isadelimproof
\isanewline
%
\endisadelimproof
\isanewline
\isacommand{lemma}\isamarkupfalse%
\ alt{\isacharunderscore}{\kern0pt}impossible{\isacharunderscore}{\kern0pt}futures{\isacharunderscore}{\kern0pt}def{\isacharunderscore}{\kern0pt}implies{\isacharunderscore}{\kern0pt}impossible{\isacharunderscore}{\kern0pt}futures{\isacharunderscore}{\kern0pt}def{\isacharcolon}{\kern0pt}\isanewline
\ \ \isakeyword{fixes}\ {\isasymphi}\ {\isacharcolon}{\kern0pt}{\isacharcolon}{\kern0pt}\ {\isachardoublequoteopen}{\isacharparenleft}{\kern0pt}{\isacharprime}{\kern0pt}a{\isacharcomma}{\kern0pt}\ {\isacharprime}{\kern0pt}s{\isacharparenright}{\kern0pt}\ hml{\isachardoublequoteclose}\isanewline
\ \ \isakeyword{assumes}\ {\isachardoublequoteopen}hml{\isacharunderscore}{\kern0pt}impossible{\isacharunderscore}{\kern0pt}futures\ {\isasymphi}{\isachardoublequoteclose}\isanewline
\ \ \isakeyword{shows}\ {\isachardoublequoteopen}{\isasymexists}{\isasympsi}{\isachardot}{\kern0pt}\ HML{\isacharunderscore}{\kern0pt}impossible{\isacharunderscore}{\kern0pt}futures\ {\isasympsi}\ {\isasymand}\ {\isacharparenleft}{\kern0pt}{\isasymforall}s{\isachardot}{\kern0pt}\ {\isacharparenleft}{\kern0pt}s\ {\isasymTurnstile}\ {\isasymphi}{\isacharparenright}{\kern0pt}\ {\isasymlongleftrightarrow}\ {\isacharparenleft}{\kern0pt}s\ {\isasymTurnstile}\ {\isasympsi}{\isacharparenright}{\kern0pt}{\isacharparenright}{\kern0pt}{\isachardoublequoteclose}\isanewline
%
\isadelimproof
\ \ %
\endisadelimproof
%
\isatagproof
\isacommand{using}\isamarkupfalse%
\ assms\ \isacommand{proof}\isamarkupfalse%
\ induct\isanewline
\ \ \isacommand{case}\isamarkupfalse%
\ if{\isacharunderscore}{\kern0pt}tt\isanewline
\ \ \isacommand{then}\isamarkupfalse%
\ \isacommand{show}\isamarkupfalse%
\ {\isacharquery}{\kern0pt}case\ \isanewline
\ \ \ \ \isacommand{using}\isamarkupfalse%
\ HML{\isacharunderscore}{\kern0pt}impossible{\isacharunderscore}{\kern0pt}futures{\isachardot}{\kern0pt}if{\isacharunderscore}{\kern0pt}tt\ \isacommand{by}\isamarkupfalse%
\ blast\isanewline
\isacommand{next}\isamarkupfalse%
\isanewline
\ \ \isacommand{case}\isamarkupfalse%
\ {\isacharparenleft}{\kern0pt}if{\isacharunderscore}{\kern0pt}pos\ {\isasymphi}\ {\isasymalpha}{\isacharparenright}{\kern0pt}\isanewline
\ \ \isacommand{then}\isamarkupfalse%
\ \isacommand{show}\isamarkupfalse%
\ {\isacharquery}{\kern0pt}case\ \isanewline
\ \ \ \ \isacommand{using}\isamarkupfalse%
\ HML{\isacharunderscore}{\kern0pt}impossible{\isacharunderscore}{\kern0pt}futures{\isachardot}{\kern0pt}if{\isacharunderscore}{\kern0pt}pos\ \isacommand{by}\isamarkupfalse%
\ fastforce\isanewline
\isacommand{next}\isamarkupfalse%
\isanewline
\ \ \isacommand{case}\isamarkupfalse%
\ {\isacharparenleft}{\kern0pt}if{\isacharunderscore}{\kern0pt}conj\ I\ {\isasymPhi}\ J{\isacharparenright}{\kern0pt}\isanewline
\ \ \isacommand{then}\isamarkupfalse%
\ \isacommand{consider}\isamarkupfalse%
\ {\isachardoublequoteopen}{\isasymPhi}\ {\isacharbackquote}{\kern0pt}\ I\ {\isasyminter}\ {\isasymPhi}\ {\isacharbackquote}{\kern0pt}\ J\ {\isasymnoteq}\ {\isacharbraceleft}{\kern0pt}{\isacharbraceright}{\kern0pt}\ {\isasymor}\ {\isacharparenleft}{\kern0pt}{\isasymexists}x{\isasymin}{\isasymPhi}\ {\isacharbackquote}{\kern0pt}\ J{\isachardot}{\kern0pt}\ x\ {\isacharequal}{\kern0pt}\ TT{\isacharparenright}{\kern0pt}{\isachardoublequoteclose}\isanewline
\ \ \ \ {\isacharbar}{\kern0pt}\ {\isachardoublequoteopen}{\isasymPhi}\ {\isacharbackquote}{\kern0pt}\ I\ {\isasyminter}\ {\isasymPhi}\ {\isacharbackquote}{\kern0pt}\ J\ {\isacharequal}{\kern0pt}\ {\isacharbraceleft}{\kern0pt}{\isacharbraceright}{\kern0pt}\ {\isasymand}\ {\isacharparenleft}{\kern0pt}{\isasymforall}x{\isasymin}{\isasymPhi}{\isacharbackquote}{\kern0pt}J{\isachardot}{\kern0pt}\ x\ {\isasymnoteq}\ TT{\isacharparenright}{\kern0pt}\ {\isasymand}\ {\isacharparenleft}{\kern0pt}{\isasymexists}x{\isachardot}{\kern0pt}\ x\ {\isasymin}\ {\isasymPhi}{\isacharbackquote}{\kern0pt}J{\isacharparenright}{\kern0pt}{\isachardoublequoteclose}\isanewline
\ \ \ \ {\isacharbar}{\kern0pt}\ {\isachardoublequoteopen}{\isasymPhi}\ {\isacharbackquote}{\kern0pt}\ J\ {\isacharequal}{\kern0pt}\ {\isacharbraceleft}{\kern0pt}{\isacharbraceright}{\kern0pt}{\isachardoublequoteclose}\isanewline
\ \ \ \ \isacommand{by}\isamarkupfalse%
\ blast\isanewline
\ \ \isacommand{then}\isamarkupfalse%
\ \isacommand{show}\isamarkupfalse%
\ {\isacharquery}{\kern0pt}case\ \isacommand{proof}\isamarkupfalse%
{\isacharparenleft}{\kern0pt}cases{\isacharparenright}{\kern0pt}\isanewline
\ \ \ \ \isacommand{case}\isamarkupfalse%
\ {\isadigit{1}}\isanewline
\ \ \ \ \isacommand{hence}\isamarkupfalse%
\ {\isachardoublequoteopen}{\isasymforall}s{\isachardot}{\kern0pt}\ {\isasymnot}s\ {\isasymTurnstile}\ {\isacharparenleft}{\kern0pt}hml{\isacharunderscore}{\kern0pt}conj\ I\ J\ {\isasymPhi}{\isacharparenright}{\kern0pt}{\isachardoublequoteclose}\ \isanewline
\ \ \ \ \ \ \isacommand{using}\isamarkupfalse%
\ HML{\isacharunderscore}{\kern0pt}true{\isacharunderscore}{\kern0pt}TT{\isacharunderscore}{\kern0pt}like\ HML{\isacharunderscore}{\kern0pt}true{\isacharunderscore}{\kern0pt}def\ \isacommand{by}\isamarkupfalse%
\ fastforce\isanewline
\ \ \ \ \isacommand{obtain}\isamarkupfalse%
\ {\isasymphi}\ i{\isacharunderscore}{\kern0pt}{\isasymphi}\ \isakeyword{where}\ {\isachardoublequoteopen}{\isasymphi}\ {\isasymin}\ {\isasymPhi}\ {\isacharbackquote}{\kern0pt}\ I\ {\isasyminter}\ {\isasymPhi}\ {\isacharbackquote}{\kern0pt}\ J\ {\isasymor}\ {\isacharparenleft}{\kern0pt}{\isasymphi}\ {\isasymin}\ {\isasymPhi}\ {\isacharbackquote}{\kern0pt}\ J\ {\isasymand}\ {\isasymphi}\ {\isacharequal}{\kern0pt}\ TT{\isacharparenright}{\kern0pt}{\isachardoublequoteclose}\ {\isachardoublequoteopen}{\isasymPhi}\ i{\isacharunderscore}{\kern0pt}{\isasymphi}\ {\isacharequal}{\kern0pt}\ {\isasymphi}{\isachardoublequoteclose}\isanewline
\ \ \ \ \ \ \isacommand{using}\isamarkupfalse%
\ {\isadigit{1}}\ \isacommand{by}\isamarkupfalse%
\ blast\isanewline
\ \ \ \ \isacommand{define}\isamarkupfalse%
\ {\isasymPsi}\ \isakeyword{where}\ {\isachardoublequoteopen}{\isasymPsi}\ {\isasymequiv}\ {\isacharparenleft}{\kern0pt}{\isasymlambda}i{\isachardot}{\kern0pt}\ {\isacharparenleft}{\kern0pt}if\ i\ {\isacharequal}{\kern0pt}\ i{\isacharunderscore}{\kern0pt}{\isasymphi}\ then\ TT{\isacharcolon}{\kern0pt}{\isacharcolon}{\kern0pt}{\isacharparenleft}{\kern0pt}{\isacharprime}{\kern0pt}a{\isacharcomma}{\kern0pt}\ {\isacharprime}{\kern0pt}s{\isacharparenright}{\kern0pt}hml\ else\ undefined{\isacharparenright}{\kern0pt}{\isacharparenright}{\kern0pt}{\isachardoublequoteclose}\isanewline
\ \ \ \ \isacommand{have}\isamarkupfalse%
\ {\isachardoublequoteopen}HML{\isacharunderscore}{\kern0pt}impossible{\isacharunderscore}{\kern0pt}futures\ {\isacharparenleft}{\kern0pt}hml{\isacharunderscore}{\kern0pt}conj\ {\isacharbraceleft}{\kern0pt}{\isacharbraceright}{\kern0pt}\ {\isacharbraceleft}{\kern0pt}i{\isacharunderscore}{\kern0pt}{\isasymphi}{\isacharbraceright}{\kern0pt}\ {\isasymPsi}{\isacharparenright}{\kern0pt}{\isachardoublequoteclose}\ \isanewline
\ \ \ \ \ \ \isacommand{using}\isamarkupfalse%
\ {\isasymPsi}{\isacharunderscore}{\kern0pt}def\ HML{\isacharunderscore}{\kern0pt}impossible{\isacharunderscore}{\kern0pt}futures{\isachardot}{\kern0pt}simps\ trace{\isacharunderscore}{\kern0pt}tt\ \isacommand{by}\isamarkupfalse%
\ fastforce\isanewline
\ \ \ \isacommand{have}\isamarkupfalse%
\ {\isachardoublequoteopen}{\isasymforall}s{\isachardot}{\kern0pt}\ {\isasymnot}s\ {\isasymTurnstile}\ {\isacharparenleft}{\kern0pt}hml{\isacharunderscore}{\kern0pt}conj\ {\isacharbraceleft}{\kern0pt}{\isacharbraceright}{\kern0pt}\ {\isacharbraceleft}{\kern0pt}i{\isacharunderscore}{\kern0pt}{\isasymphi}{\isacharbraceright}{\kern0pt}\ {\isasymPsi}{\isacharparenright}{\kern0pt}{\isachardoublequoteclose}\isanewline
\ \ \ \ \ \ \isacommand{by}\isamarkupfalse%
\ {\isacharparenleft}{\kern0pt}simp\ add{\isacharcolon}{\kern0pt}\ {\isasymPsi}{\isacharunderscore}{\kern0pt}def{\isacharparenright}{\kern0pt}\isanewline
\ \ \ \ \isacommand{then}\isamarkupfalse%
\ \isacommand{show}\isamarkupfalse%
\ {\isacharquery}{\kern0pt}thesis\ \isanewline
\ \ \ \ \ \ \isacommand{using}\isamarkupfalse%
\ {\isacartoucheopen}{\isasymforall}s{\isachardot}{\kern0pt}\ {\isasymnot}\ s\ {\isasymTurnstile}\ hml{\isacharunderscore}{\kern0pt}conj\ I\ J\ {\isasymPhi}{\isacartoucheclose}\ {\isacartoucheopen}HML{\isacharunderscore}{\kern0pt}impossible{\isacharunderscore}{\kern0pt}futures\ {\isacharparenleft}{\kern0pt}hml{\isacharunderscore}{\kern0pt}conj\ {\isacharbraceleft}{\kern0pt}{\isacharbraceright}{\kern0pt}\ {\isacharbraceleft}{\kern0pt}i{\isacharunderscore}{\kern0pt}{\isasymphi}{\isacharbraceright}{\kern0pt}\ {\isasymPsi}{\isacharparenright}{\kern0pt}{\isacartoucheclose}\ \isacommand{by}\isamarkupfalse%
\ blast\isanewline
\ \ \isacommand{next}\isamarkupfalse%
\isanewline
\ \ \ \ \isacommand{case}\isamarkupfalse%
\ {\isadigit{2}}\isanewline
\ \ \ \ \isacommand{hence}\isamarkupfalse%
\ {\isachardoublequoteopen}{\isasymforall}x\ {\isasymin}\ {\isasymPhi}{\isacharbackquote}{\kern0pt}J{\isachardot}{\kern0pt}\ {\isasymexists}{\isasymalpha}\ {\isasymphi}{\isachardot}{\kern0pt}\ x\ {\isacharequal}{\kern0pt}\ {\isacharparenleft}{\kern0pt}hml{\isacharunderscore}{\kern0pt}pos\ {\isasymalpha}\ {\isasymphi}{\isacharparenright}{\kern0pt}\ {\isasymand}\ hml{\isacharunderscore}{\kern0pt}trace\ {\isasymphi}{\isachardoublequoteclose}\isanewline
\ \ \ \ \ \ \isacommand{using}\isamarkupfalse%
\ if{\isacharunderscore}{\kern0pt}conj\ \isanewline
\ \ \ \ \ \ \isacommand{by}\isamarkupfalse%
\ {\isacharparenleft}{\kern0pt}meson\ hml{\isacharunderscore}{\kern0pt}trace{\isachardot}{\kern0pt}cases{\isacharparenright}{\kern0pt}\isanewline
\ \ \ \ \isacommand{hence}\isamarkupfalse%
\ {\isachardoublequoteopen}{\isasymforall}x\ {\isasymin}\ {\isasymPhi}{\isacharbackquote}{\kern0pt}J{\isachardot}{\kern0pt}\ {\isasymexists}{\isasymalpha}\ {\isasymphi}{\isachardot}{\kern0pt}\ x\ {\isacharequal}{\kern0pt}\ {\isacharparenleft}{\kern0pt}hml{\isacharunderscore}{\kern0pt}pos\ {\isasymalpha}\ {\isasymphi}{\isacharparenright}{\kern0pt}\ {\isasymand}\ hml{\isacharunderscore}{\kern0pt}trace\ {\isasymphi}\ {\isasymand}\ {\isacharparenleft}{\kern0pt}{\isasymexists}{\isasympsi}{\isachardot}{\kern0pt}\ HML{\isacharunderscore}{\kern0pt}trace\ {\isasympsi}\ {\isasymand}\ {\isacharparenleft}{\kern0pt}{\isasymforall}s{\isachardot}{\kern0pt}\ s\ {\isasymTurnstile}\ {\isasymphi}\ {\isasymlongleftrightarrow}\ s\ {\isasymTurnstile}\ {\isasympsi}{\isacharparenright}{\kern0pt}{\isacharparenright}{\kern0pt}{\isachardoublequoteclose}\isanewline
\ \ \ \ \ \ \isacommand{using}\isamarkupfalse%
\ alt{\isacharunderscore}{\kern0pt}trace{\isacharunderscore}{\kern0pt}def{\isacharunderscore}{\kern0pt}implies{\isacharunderscore}{\kern0pt}trace{\isacharunderscore}{\kern0pt}def\ \isacommand{by}\isamarkupfalse%
\ meson\isanewline
\ \ \ \ \isacommand{hence}\isamarkupfalse%
\ SOME{\isacharunderscore}{\kern0pt}ex{\isacharcolon}{\kern0pt}\ {\isachardoublequoteopen}{\isasymforall}j\ {\isasymin}\ J{\isachardot}{\kern0pt}\ {\isasymexists}{\isasymalpha}{\isachardot}{\kern0pt}\ {\isacharparenleft}{\kern0pt}{\isasymexists}{\isasymphi}{\isachardot}{\kern0pt}\ {\isasymPhi}\ j\ {\isacharequal}{\kern0pt}\ hml{\isacharunderscore}{\kern0pt}pos\ {\isasymalpha}\ {\isasymphi}{\isacharparenright}{\kern0pt}\ {\isasymand}\ {\isacharparenleft}{\kern0pt}{\isasymexists}{\isasympsi}{\isachardot}{\kern0pt}\ {\isasymexists}{\isasymphi}{\isachardot}{\kern0pt}\ {\isasymPhi}\ j\ {\isacharequal}{\kern0pt}\ hml{\isacharunderscore}{\kern0pt}pos\ {\isasymalpha}\ {\isasymphi}\ {\isasymand}\ HML{\isacharunderscore}{\kern0pt}trace\ {\isasympsi}\ {\isasymand}\ {\isacharparenleft}{\kern0pt}{\isasymforall}s{\isachardot}{\kern0pt}\ s\ {\isasymTurnstile}\ {\isasymphi}\ {\isasymlongleftrightarrow}\ s\ {\isasymTurnstile}\ {\isasympsi}{\isacharparenright}{\kern0pt}{\isacharparenright}{\kern0pt}{\isachardoublequoteclose}\isanewline
\ \ \ \ \ \ \isacommand{by}\isamarkupfalse%
\ {\isacharparenleft}{\kern0pt}meson\ imageI{\isacharparenright}{\kern0pt}\isanewline
\ \ \ \ \isacommand{hence}\isamarkupfalse%
\ SOME{\isacharunderscore}{\kern0pt}all{\isacharcolon}{\kern0pt}\ {\isachardoublequoteopen}{\isasymforall}j\ {\isasymin}\ J{\isachardot}{\kern0pt}\ {\isasymforall}{\isasymalpha}\ {\isasymphi}{\isachardot}{\kern0pt}\ {\isasymPhi}\ j\ {\isacharequal}{\kern0pt}\ hml{\isacharunderscore}{\kern0pt}pos\ {\isasymalpha}\ {\isasymphi}\ {\isasymlongrightarrow}\ {\isacharparenleft}{\kern0pt}{\isasymexists}{\isasympsi}{\isachardot}{\kern0pt}\ HML{\isacharunderscore}{\kern0pt}trace\ {\isasympsi}\ {\isasymand}\ {\isacharparenleft}{\kern0pt}{\isasymforall}s{\isachardot}{\kern0pt}\ s\ {\isasymTurnstile}\ {\isasymphi}\ {\isasymlongleftrightarrow}\ s\ {\isasymTurnstile}\ {\isasympsi}{\isacharparenright}{\kern0pt}{\isacharparenright}{\kern0pt}{\isachardoublequoteclose}\isanewline
\ \ \ \ \ \ \isacommand{by}\isamarkupfalse%
\ fastforce\isanewline
\ \ \ \ \isacommand{define}\isamarkupfalse%
\ {\isasymPsi}\ \isakeyword{where}\ {\isachardoublequoteopen}{\isasymPsi}\ {\isasymequiv}\ {\isacharparenleft}{\kern0pt}{\isasymlambda}i{\isachardot}{\kern0pt}\ {\isacharparenleft}{\kern0pt}if\ i\ {\isasymin}\ J\isanewline
\ \ \ \ then\ {\isacharparenleft}{\kern0pt}hml{\isacharunderscore}{\kern0pt}pos\ {\isacharparenleft}{\kern0pt}SOME\ {\isasymalpha}{\isachardot}{\kern0pt}\ {\isasymexists}{\isasymphi}{\isachardot}{\kern0pt}\ {\isasymPhi}\ i\ {\isacharequal}{\kern0pt}\ hml{\isacharunderscore}{\kern0pt}pos\ {\isasymalpha}\ {\isasymphi}{\isacharparenright}{\kern0pt}\ \isanewline
\ \ \ \ \ \ {\isacharparenleft}{\kern0pt}SOME\ {\isasympsi}{\isachardot}{\kern0pt}\ {\isasymexists}{\isasymalpha}\ {\isasymphi}{\isachardot}{\kern0pt}\ {\isasymPhi}\ i\ {\isacharequal}{\kern0pt}\ hml{\isacharunderscore}{\kern0pt}pos\ {\isasymalpha}\ {\isasymphi}\ {\isasymand}\ HML{\isacharunderscore}{\kern0pt}trace\ {\isasympsi}\ {\isasymand}\ {\isacharparenleft}{\kern0pt}{\isasymforall}s{\isachardot}{\kern0pt}\ s\ {\isasymTurnstile}\ {\isasymphi}\ {\isasymlongleftrightarrow}\ s\ {\isasymTurnstile}\ {\isasympsi}{\isacharparenright}{\kern0pt}{\isacharparenright}{\kern0pt}{\isacharparenright}{\kern0pt}\ \isanewline
\ \ \ \ else\ undefined{\isacharparenright}{\kern0pt}{\isacharparenright}{\kern0pt}{\isachardoublequoteclose}\isanewline
\ \ \ \ \isacommand{hence}\isamarkupfalse%
\ {\isachardoublequoteopen}{\isasymforall}j\ {\isasymin}\ J{\isachardot}{\kern0pt}\ {\isasymexists}{\isasymalpha}\ {\isasympsi}{\isachardot}{\kern0pt}\ {\isasymPsi}\ j\ {\isacharequal}{\kern0pt}\ hml{\isacharunderscore}{\kern0pt}pos\ {\isasymalpha}\ {\isasympsi}{\isachardoublequoteclose}\isanewline
\ \ \ \ \ \ \isacommand{using}\isamarkupfalse%
\ SOME{\isacharunderscore}{\kern0pt}ex\ \isanewline
\ \ \ \ \ \ \isacommand{by}\isamarkupfalse%
\ simp\isanewline
\ \ \ \ \isacommand{have}\isamarkupfalse%
\ {\isachardoublequoteopen}{\isasymforall}j\ {\isasymin}\ J{\isachardot}{\kern0pt}\ {\isasymexists}{\isasymalpha}\ {\isasymphi}{\isachardot}{\kern0pt}\ {\isasymPhi}\ j\ {\isacharequal}{\kern0pt}\ hml{\isacharunderscore}{\kern0pt}pos\ {\isasymalpha}\ {\isasymphi}\ {\isasymand}\ {\isacharparenleft}{\kern0pt}{\isasymexists}{\isasympsi}{\isachardot}{\kern0pt}\ HML{\isacharunderscore}{\kern0pt}trace\ {\isasympsi}\ {\isasymand}\ {\isacharparenleft}{\kern0pt}{\isasymforall}s{\isachardot}{\kern0pt}\ s\ {\isasymTurnstile}\ {\isasymphi}\ {\isasymlongleftrightarrow}\ s\ {\isasymTurnstile}\ {\isasympsi}{\isacharparenright}{\kern0pt}{\isacharparenright}{\kern0pt}{\isachardoublequoteclose}\isanewline
\ \ \ \ \ \ \isacommand{using}\isamarkupfalse%
\ SOME{\isacharunderscore}{\kern0pt}ex\ \isacommand{by}\isamarkupfalse%
\ blast\isanewline
\ \ \ \ \isacommand{have}\isamarkupfalse%
\ {\isachardoublequoteopen}{\isasymforall}j\ {\isasymin}\ J{\isachardot}{\kern0pt}\ {\isasymexists}{\isasymalpha}\ {\isasymphi}{\isachardot}{\kern0pt}\ {\isasymPhi}\ j\ {\isacharequal}{\kern0pt}\ hml{\isacharunderscore}{\kern0pt}pos\ {\isasymalpha}\ {\isasymphi}\ {\isasymand}\ {\isacharparenleft}{\kern0pt}{\isasymexists}{\isasympsi}{\isachardot}{\kern0pt}\ {\isasymexists}{\isasymalpha}\ {\isasymphi}{\isachardot}{\kern0pt}\ {\isasymPhi}\ j\ {\isacharequal}{\kern0pt}\ hml{\isacharunderscore}{\kern0pt}pos\ {\isasymalpha}\ {\isasymphi}\ {\isasymand}\ HML{\isacharunderscore}{\kern0pt}trace\ {\isasympsi}\ {\isasymand}\ {\isacharparenleft}{\kern0pt}{\isasymforall}s{\isachardot}{\kern0pt}\ s\ {\isasymTurnstile}\ {\isasymphi}\ {\isasymlongleftrightarrow}\ s\ {\isasymTurnstile}\ {\isasympsi}{\isacharparenright}{\kern0pt}{\isacharparenright}{\kern0pt}{\isachardoublequoteclose}\isanewline
\ \ \ \ \ \ \isacommand{using}\isamarkupfalse%
\ SOME{\isacharunderscore}{\kern0pt}ex\ \isacommand{by}\isamarkupfalse%
\ blast\isanewline
\ \ \ \ \isacommand{have}\isamarkupfalse%
\ {\isachardoublequoteopen}{\isasymforall}j\ {\isasymin}\ J{\isachardot}{\kern0pt}\ {\isasymforall}{\isasymalpha}\ {\isasymphi}\ {\isasympsi}{\isachardot}{\kern0pt}\ {\isasymPhi}\ j\ {\isacharequal}{\kern0pt}\ hml{\isacharunderscore}{\kern0pt}pos\ {\isasymalpha}\ {\isasymphi}\ {\isasymand}\ {\isasymPhi}\ j\ {\isacharequal}{\kern0pt}\ {\isasympsi}\ {\isasymlongrightarrow}\ {\isasympsi}\ {\isacharequal}{\kern0pt}\ hml{\isacharunderscore}{\kern0pt}pos\ {\isasymalpha}\ {\isasymphi}{\isachardoublequoteclose}\ \isanewline
\ \ \ \ \ \ \isacommand{by}\isamarkupfalse%
\ blast\isanewline
\ \ \ \ \isacommand{from}\isamarkupfalse%
\ SOME{\isacharunderscore}{\kern0pt}all\ \isacommand{have}\isamarkupfalse%
\ {\isachardoublequoteopen}{\isasymforall}j\ {\isasymin}\ J{\isachardot}{\kern0pt}\ {\isasymexists}{\isasymalpha}\ {\isasympsi}{\isachardot}{\kern0pt}\ {\isasymPsi}\ j\ {\isacharequal}{\kern0pt}\ hml{\isacharunderscore}{\kern0pt}pos\ {\isasymalpha}\ {\isasympsi}\ {\isasymand}\ HML{\isacharunderscore}{\kern0pt}trace\ {\isasympsi}{\isachardoublequoteclose}\isanewline
\ \ \ \ \ \ \isacommand{using}\isamarkupfalse%
\ SOME{\isacharunderscore}{\kern0pt}all\ {\isasymPsi}{\isacharunderscore}{\kern0pt}def\ SOME{\isacharunderscore}{\kern0pt}ex\ someI{\isacharunderscore}{\kern0pt}ex\ \isanewline
\ \ \ \ \ \ \isacommand{by}\isamarkupfalse%
\ {\isacharparenleft}{\kern0pt}smt\ {\isacharparenleft}{\kern0pt}verit{\isacharcomma}{\kern0pt}\ best{\isacharparenright}{\kern0pt}{\isacharparenright}{\kern0pt}\ \isanewline
\ \ \ \ \isacommand{hence}\isamarkupfalse%
\ {\isachardoublequoteopen}{\isasymforall}j\ {\isasymin}\ J{\isachardot}{\kern0pt}\ {\isasymexists}{\isasymalpha}\ {\isasympsi}\ {\isasymphi}{\isachardot}{\kern0pt}\ {\isasymPsi}\ j\ {\isacharequal}{\kern0pt}\ hml{\isacharunderscore}{\kern0pt}pos\ {\isasymalpha}\ {\isasympsi}\ {\isasymand}\ HML{\isacharunderscore}{\kern0pt}trace\ {\isasympsi}\ {\isasymand}\ {\isasymPhi}\ j\ {\isacharequal}{\kern0pt}\ hml{\isacharunderscore}{\kern0pt}pos\ {\isasymalpha}\ {\isasymphi}{\isachardoublequoteclose}\isanewline
\ \ \ \ \ \ \isacommand{using}\isamarkupfalse%
\ SOME{\isacharunderscore}{\kern0pt}ex\ {\isasymPsi}{\isacharunderscore}{\kern0pt}def\ \isacommand{by}\isamarkupfalse%
\ fastforce\ \isanewline
\ \ \ \ \isacommand{have}\isamarkupfalse%
\ {\isachardoublequoteopen}{\isasymforall}j\ {\isasymin}\ J{\isachardot}{\kern0pt}\ {\isasymexists}{\isasymalpha}\ {\isasymphi}{\isachardot}{\kern0pt}\ {\isasymPhi}\ j\ {\isacharequal}{\kern0pt}\ hml{\isacharunderscore}{\kern0pt}pos\ {\isasymalpha}\ {\isasymphi}\ {\isasymand}\ {\isacharparenleft}{\kern0pt}{\isasymforall}{\isasympsi}{\isachardot}{\kern0pt}\ {\isasympsi}\ {\isasymnoteq}\ {\isacharparenleft}{\kern0pt}hml{\isacharunderscore}{\kern0pt}pos\ {\isasymalpha}\ {\isasymphi}{\isacharparenright}{\kern0pt}\ {\isasymlongrightarrow}\ {\isasymPhi}\ j\ {\isasymnoteq}\ {\isasympsi}{\isacharparenright}{\kern0pt}{\isachardoublequoteclose}\ \isanewline
\ \ \ \ \ \ \isacommand{using}\isamarkupfalse%
\ SOME{\isacharunderscore}{\kern0pt}ex\ \isacommand{by}\isamarkupfalse%
\ blast\isanewline
\isanewline
\ \ \ \ \isacommand{have}\isamarkupfalse%
\ {\isachardoublequoteopen}{\isasymAnd}j{\isachardot}{\kern0pt}\ j\ {\isasymin}\ J\ {\isasymLongrightarrow}\ {\isasymexists}{\isasymalpha}\ {\isasympsi}\ {\isasymphi}{\isachardot}{\kern0pt}\ {\isasymPsi}\ j\ {\isacharequal}{\kern0pt}\ hml{\isacharunderscore}{\kern0pt}pos\ {\isasymalpha}\ {\isasympsi}\ {\isasymand}\ HML{\isacharunderscore}{\kern0pt}trace\ {\isasympsi}\ {\isasymand}\ {\isasymPhi}\ j\ {\isacharequal}{\kern0pt}\ hml{\isacharunderscore}{\kern0pt}pos\ {\isasymalpha}\ {\isasymphi}\ {\isasymand}\ {\isacharparenleft}{\kern0pt}{\isasymforall}s{\isachardot}{\kern0pt}\ s\ {\isasymTurnstile}\ {\isasymphi}\ {\isasymlongleftrightarrow}\ s\ {\isasymTurnstile}\ {\isasympsi}{\isacharparenright}{\kern0pt}{\isachardoublequoteclose}\isanewline
\ \ \ \ \isacommand{proof}\isamarkupfalse%
\ {\isacharminus}{\kern0pt}\isanewline
\ \ \ \ \ \ \isacommand{{\isacharbraceleft}{\kern0pt}}\isamarkupfalse%
\isanewline
\ \ \ \ \ \ \ \ \isacommand{fix}\isamarkupfalse%
\ j\ \isacommand{assume}\isamarkupfalse%
\ j{\isacharunderscore}{\kern0pt}in{\isacharunderscore}{\kern0pt}J{\isacharcolon}{\kern0pt}\ {\isachardoublequoteopen}j\ {\isasymin}\ J{\isachardoublequoteclose}\isanewline
\ \ \ \ \ \ \ \ \isacommand{then}\isamarkupfalse%
\ \isacommand{show}\isamarkupfalse%
\ {\isachardoublequoteopen}{\isasymexists}{\isasymalpha}\ {\isasympsi}\ {\isasymphi}{\isachardot}{\kern0pt}\ {\isasymPsi}\ j\ {\isacharequal}{\kern0pt}\ hml{\isacharunderscore}{\kern0pt}pos\ {\isasymalpha}\ {\isasympsi}\ {\isasymand}\ HML{\isacharunderscore}{\kern0pt}trace\ {\isasympsi}\ {\isasymand}\ {\isasymPhi}\ j\ {\isacharequal}{\kern0pt}\ hml{\isacharunderscore}{\kern0pt}pos\ {\isasymalpha}\ {\isasymphi}\ {\isasymand}\ {\isacharparenleft}{\kern0pt}{\isasymforall}s{\isachardot}{\kern0pt}\ {\isacharparenleft}{\kern0pt}s\ {\isasymTurnstile}\ {\isasymphi}{\isacharparenright}{\kern0pt}\ {\isacharequal}{\kern0pt}\ {\isacharparenleft}{\kern0pt}s\ {\isasymTurnstile}\ {\isasympsi}{\isacharparenright}{\kern0pt}{\isacharparenright}{\kern0pt}{\isachardoublequoteclose}\isanewline
\ \ \ \ \ \ \ \ \isacommand{proof}\isamarkupfalse%
{\isacharminus}{\kern0pt}\isanewline
\ \ \ \ \ \ \ \ \isacommand{have}\isamarkupfalse%
\ {\isachardoublequoteopen}{\isasymPsi}\ j\ {\isacharequal}{\kern0pt}\ {\isacharparenleft}{\kern0pt}if\ j\ {\isasymin}\ J\ \isanewline
\ \ \ \ \ \ \ \ \ \ \ \ \ \ \ \ \ \ \ \ then\ {\isacharparenleft}{\kern0pt}hml{\isacharunderscore}{\kern0pt}pos\ {\isacharparenleft}{\kern0pt}SOME\ {\isasymalpha}{\isachardot}{\kern0pt}\ {\isasymexists}{\isasymphi}{\isachardot}{\kern0pt}\ {\isasymPhi}\ j\ {\isacharequal}{\kern0pt}\ hml{\isacharunderscore}{\kern0pt}pos\ {\isasymalpha}\ {\isasymphi}{\isacharparenright}{\kern0pt}\ \isanewline
\ \ \ \ \ \ \ \ \ \ \ \ \ \ \ \ \ \ \ \ \ \ \ \ \ \ \ \ \ \ {\isacharparenleft}{\kern0pt}SOME\ {\isasympsi}{\isachardot}{\kern0pt}\ {\isasymexists}{\isasymalpha}\ {\isasymphi}{\isachardot}{\kern0pt}\ {\isasymPhi}\ j\ {\isacharequal}{\kern0pt}\ hml{\isacharunderscore}{\kern0pt}pos\ {\isasymalpha}\ {\isasymphi}\ {\isasymand}\ HML{\isacharunderscore}{\kern0pt}trace\ {\isasympsi}\ {\isasymand}\ {\isacharparenleft}{\kern0pt}{\isasymforall}s{\isachardot}{\kern0pt}\ s\ {\isasymTurnstile}\ {\isasymphi}\ {\isasymlongleftrightarrow}\ s\ {\isasymTurnstile}\ {\isasympsi}{\isacharparenright}{\kern0pt}{\isacharparenright}{\kern0pt}{\isacharparenright}{\kern0pt}\ \isanewline
\ \ \ \ \ \ \ \ \ \ \ \ \ \ \ \ \ \ \ \ else\ undefined{\isacharparenright}{\kern0pt}{\isachardoublequoteclose}\isanewline
\ \ \ \ \ \ \ \ \ \ \isacommand{by}\isamarkupfalse%
\ {\isacharparenleft}{\kern0pt}simp\ add{\isacharcolon}{\kern0pt}\ {\isasymPsi}{\isacharunderscore}{\kern0pt}def\ j{\isacharunderscore}{\kern0pt}in{\isacharunderscore}{\kern0pt}J{\isacharparenright}{\kern0pt}\isanewline
\ \ \ \ \ \ \ \ \isacommand{also}\isamarkupfalse%
\ \isacommand{have}\isamarkupfalse%
\ {\isachardoublequoteopen}{\isachardot}{\kern0pt}{\isachardot}{\kern0pt}{\isachardot}{\kern0pt}\ {\isacharequal}{\kern0pt}\ {\isacharparenleft}{\kern0pt}hml{\isacharunderscore}{\kern0pt}pos\ {\isacharparenleft}{\kern0pt}SOME\ {\isasymalpha}{\isachardot}{\kern0pt}\ {\isasymexists}{\isasymphi}{\isachardot}{\kern0pt}\ {\isasymPhi}\ j\ {\isacharequal}{\kern0pt}\ hml{\isacharunderscore}{\kern0pt}pos\ {\isasymalpha}\ {\isasymphi}{\isacharparenright}{\kern0pt}\ \isanewline
\ \ \ \ \ \ \ \ \ \ \ \ \ \ \ \ \ \ \ \ \ \ \ \ \ \ \ \ \ \ {\isacharparenleft}{\kern0pt}SOME\ {\isasympsi}{\isachardot}{\kern0pt}\ {\isasymexists}{\isasymalpha}\ {\isasymphi}{\isachardot}{\kern0pt}\ {\isasymPhi}\ j\ {\isacharequal}{\kern0pt}\ hml{\isacharunderscore}{\kern0pt}pos\ {\isasymalpha}\ {\isasymphi}\ {\isasymand}\ HML{\isacharunderscore}{\kern0pt}trace\ {\isasympsi}\ {\isasymand}\ {\isacharparenleft}{\kern0pt}{\isasymforall}s{\isachardot}{\kern0pt}\ s\ {\isasymTurnstile}\ {\isasymphi}\ {\isasymlongleftrightarrow}\ s\ {\isasymTurnstile}\ {\isasympsi}{\isacharparenright}{\kern0pt}{\isacharparenright}{\kern0pt}{\isacharparenright}{\kern0pt}{\isachardoublequoteclose}\isanewline
\ \ \ \ \ \ \ \ \ \ \isacommand{using}\isamarkupfalse%
\ j{\isacharunderscore}{\kern0pt}in{\isacharunderscore}{\kern0pt}J\ \isacommand{by}\isamarkupfalse%
\ simp\isanewline
\ \ \ \ \ \ \ \ \isacommand{finally}\isamarkupfalse%
\ \isacommand{obtain}\isamarkupfalse%
\ {\isasymalpha}\ {\isasymphi}\ {\isasympsi}\ \isakeyword{where}\ \isanewline
\ \ \ \ \ \ \ \ \ \ psi{\isacharunderscore}{\kern0pt}def{\isacharcolon}{\kern0pt}\ {\isachardoublequoteopen}{\isasymPsi}\ j\ {\isacharequal}{\kern0pt}\ hml{\isacharunderscore}{\kern0pt}pos\ {\isasymalpha}\ {\isasympsi}{\isachardoublequoteclose}\ \isakeyword{and}\ \isanewline
\ \ \ \ \ \ \ \ \ \ trace{\isacharunderscore}{\kern0pt}psi{\isacharcolon}{\kern0pt}\ {\isachardoublequoteopen}HML{\isacharunderscore}{\kern0pt}trace\ {\isasympsi}{\isachardoublequoteclose}\ \isakeyword{and}\ \isanewline
\ \ \ \ \ \ \ \ \ \ phi{\isacharunderscore}{\kern0pt}alpha{\isacharunderscore}{\kern0pt}def{\isacharcolon}{\kern0pt}\ {\isachardoublequoteopen}{\isasymPhi}\ j\ {\isacharequal}{\kern0pt}\ hml{\isacharunderscore}{\kern0pt}pos\ {\isasymalpha}\ {\isasymphi}{\isachardoublequoteclose}\ \isakeyword{and}\isanewline
\ \ \ \ \ \ \ \ \ \ phi{\isacharunderscore}{\kern0pt}psi{\isacharunderscore}{\kern0pt}equivalence{\isacharcolon}{\kern0pt}\ {\isachardoublequoteopen}{\isasymforall}s{\isachardot}{\kern0pt}\ s\ {\isasymTurnstile}\ {\isasymphi}\ {\isasymlongleftrightarrow}\ s\ {\isasymTurnstile}\ {\isasympsi}{\isachardoublequoteclose}\ \isanewline
\ \ \ \ \ \ \ \ \ \ \isacommand{using}\isamarkupfalse%
\ SOME{\isacharunderscore}{\kern0pt}all\ {\isacartoucheopen}{\isasymforall}j{\isasymin}J{\isachardot}{\kern0pt}\ {\isasymexists}{\isasymalpha}\ {\isasympsi}\ {\isasymphi}{\isachardot}{\kern0pt}\ {\isasymPsi}\ j\ {\isacharequal}{\kern0pt}\ hml{\isacharunderscore}{\kern0pt}pos\ {\isasymalpha}\ {\isasympsi}\ {\isasymand}\ HML{\isacharunderscore}{\kern0pt}trace\ {\isasympsi}\ {\isasymand}\ {\isasymPhi}\ j\ {\isacharequal}{\kern0pt}\ hml{\isacharunderscore}{\kern0pt}pos\ {\isasymalpha}\ {\isasymphi}{\isacartoucheclose}\ hml{\isachardot}{\kern0pt}inject{\isacharparenleft}{\kern0pt}{\isadigit{1}}{\isacharparenright}{\kern0pt}\ j{\isacharunderscore}{\kern0pt}in{\isacharunderscore}{\kern0pt}J\ someI{\isacharunderscore}{\kern0pt}ex\isanewline
\ \ \ \ \ \ \ \ \ \ \isacommand{by}\isamarkupfalse%
\ {\isacharparenleft}{\kern0pt}smt\ {\isacharparenleft}{\kern0pt}verit{\isacharcomma}{\kern0pt}\ ccfv{\isacharunderscore}{\kern0pt}threshold{\isacharparenright}{\kern0pt}{\isacharparenright}{\kern0pt}\isanewline
\ \ \ \ \ \ \ \ \isacommand{have}\isamarkupfalse%
\ {\isachardoublequoteopen}{\isasymPsi}\ j\ {\isacharequal}{\kern0pt}\ hml{\isacharunderscore}{\kern0pt}pos\ {\isasymalpha}\ {\isasympsi}\ {\isasymand}\ HML{\isacharunderscore}{\kern0pt}trace\ {\isasympsi}\ {\isasymand}\ {\isasymPhi}\ j\ {\isacharequal}{\kern0pt}\ hml{\isacharunderscore}{\kern0pt}pos\ {\isasymalpha}\ {\isasymphi}\ {\isasymand}\ {\isacharparenleft}{\kern0pt}{\isasymforall}s{\isachardot}{\kern0pt}\ s\ {\isasymTurnstile}\ {\isasymphi}\ {\isasymlongleftrightarrow}\ s\ {\isasymTurnstile}\ {\isasympsi}{\isacharparenright}{\kern0pt}{\isachardoublequoteclose}\isanewline
\ \ \ \ \ \ \ \ \ \ \isacommand{using}\isamarkupfalse%
\ phi{\isacharunderscore}{\kern0pt}alpha{\isacharunderscore}{\kern0pt}def\ phi{\isacharunderscore}{\kern0pt}psi{\isacharunderscore}{\kern0pt}equivalence\ psi{\isacharunderscore}{\kern0pt}def\ trace{\isacharunderscore}{\kern0pt}psi\ \isacommand{by}\isamarkupfalse%
\ blast\isanewline
\ \ \ \ \ \ \ \ \isacommand{then}\isamarkupfalse%
\ \isacommand{show}\isamarkupfalse%
\ {\isachardoublequoteopen}{\isasymexists}{\isasymalpha}\ {\isasympsi}\ {\isasymphi}{\isachardot}{\kern0pt}\ {\isasymPsi}\ j\ {\isacharequal}{\kern0pt}\ hml{\isacharunderscore}{\kern0pt}pos\ {\isasymalpha}\ {\isasympsi}\ {\isasymand}\ HML{\isacharunderscore}{\kern0pt}trace\ {\isasympsi}\ {\isasymand}\ {\isasymPhi}\ j\ {\isacharequal}{\kern0pt}\ hml{\isacharunderscore}{\kern0pt}pos\ {\isasymalpha}\ {\isasymphi}\ {\isasymand}\ {\isacharparenleft}{\kern0pt}{\isasymforall}s{\isachardot}{\kern0pt}\ {\isacharparenleft}{\kern0pt}s\ {\isasymTurnstile}\ {\isasymphi}{\isacharparenright}{\kern0pt}\ {\isacharequal}{\kern0pt}\ {\isacharparenleft}{\kern0pt}s\ {\isasymTurnstile}\ {\isasympsi}{\isacharparenright}{\kern0pt}{\isacharparenright}{\kern0pt}{\isachardoublequoteclose}\isanewline
\ \ \ \ \ \ \ \ \ \ \isacommand{by}\isamarkupfalse%
\ blast\isanewline
\ \ \ \ \ \ \isacommand{qed}\isamarkupfalse%
\isanewline
\ \ \ \ \isacommand{{\isacharbraceright}{\kern0pt}}\isamarkupfalse%
\isanewline
\ \ \isacommand{qed}\isamarkupfalse%
\isanewline
\ \ \ \ \isacommand{hence}\isamarkupfalse%
\ {\isachardoublequoteopen}{\isasymforall}j\ {\isasymin}\ J{\isachardot}{\kern0pt}\ {\isacharparenleft}{\kern0pt}{\isasymexists}{\isasymalpha}\ {\isasympsi}\ {\isasymphi}{\isachardot}{\kern0pt}\ {\isasymPsi}\ j\ {\isacharequal}{\kern0pt}\ hml{\isacharunderscore}{\kern0pt}pos\ {\isasymalpha}\ {\isasympsi}\ {\isasymand}\ {\isasymPhi}\ j\ {\isacharequal}{\kern0pt}\ hml{\isacharunderscore}{\kern0pt}pos\ {\isasymalpha}\ {\isasymphi}\ {\isasymand}\ HML{\isacharunderscore}{\kern0pt}trace\ {\isasympsi}\ {\isasymand}\ {\isacharparenleft}{\kern0pt}{\isasymforall}s{\isachardot}{\kern0pt}\ s\ {\isasymTurnstile}\ {\isasymphi}\ {\isasymlongleftrightarrow}\ s\ {\isasymTurnstile}\ {\isasympsi}{\isacharparenright}{\kern0pt}{\isacharparenright}{\kern0pt}{\isachardoublequoteclose}\ \isanewline
\ \ \ \ \ \ \isacommand{using}\isamarkupfalse%
\ SOME{\isacharunderscore}{\kern0pt}all\ {\isasymPsi}{\isacharunderscore}{\kern0pt}def\ SOME{\isacharunderscore}{\kern0pt}ex\ someI{\isacharunderscore}{\kern0pt}ex\ \isanewline
\ \ \ \ \ \ \isacommand{by}\isamarkupfalse%
\ auto\isanewline
\ \ \ \ \isacommand{hence}\isamarkupfalse%
\ {\isachardoublequoteopen}{\isasymforall}j\ {\isasymin}\ J{\isachardot}{\kern0pt}\ {\isasymforall}s{\isachardot}{\kern0pt}\ s\ {\isasymTurnstile}\ {\isasymPsi}\ j\ {\isasymlongleftrightarrow}\ s\ {\isasymTurnstile}\ {\isasymPhi}\ j{\isachardoublequoteclose}\isanewline
\ \ \ \ \ \ \isacommand{using}\isamarkupfalse%
\ SOME{\isacharunderscore}{\kern0pt}ex\ {\isasymPsi}{\isacharunderscore}{\kern0pt}def\ \isacommand{by}\isamarkupfalse%
\ fastforce\isanewline
\ \ \ \ \isacommand{hence}\isamarkupfalse%
\ {\isachardoublequoteopen}{\isasymforall}s{\isachardot}{\kern0pt}\ s\ {\isasymTurnstile}\ {\isacharparenleft}{\kern0pt}hml{\isacharunderscore}{\kern0pt}conj\ I\ J\ {\isasymPhi}{\isacharparenright}{\kern0pt}\ {\isasymlongleftrightarrow}\ s\ {\isasymTurnstile}\ {\isacharparenleft}{\kern0pt}hml{\isacharunderscore}{\kern0pt}conj\ {\isacharbraceleft}{\kern0pt}{\isacharbraceright}{\kern0pt}\ J\ {\isasymPsi}{\isacharparenright}{\kern0pt}{\isachardoublequoteclose}\isanewline
\ \ \ \ \ \ \isacommand{by}\isamarkupfalse%
\ {\isacharparenleft}{\kern0pt}simp\ add{\isacharcolon}{\kern0pt}\ if{\isacharunderscore}{\kern0pt}conj{\isachardot}{\kern0pt}hyps{\isacharparenleft}{\kern0pt}{\isadigit{1}}{\isacharparenright}{\kern0pt}{\isacharparenright}{\kern0pt}\isanewline
\ \ \ \ \isacommand{have}\isamarkupfalse%
\ {\isachardoublequoteopen}{\isasymforall}j\ {\isasymin}\ J{\isachardot}{\kern0pt}\ HML{\isacharunderscore}{\kern0pt}trace\ {\isacharparenleft}{\kern0pt}{\isasymPsi}\ j{\isacharparenright}{\kern0pt}{\isachardoublequoteclose}\ \isanewline
\ \ \ \ \ \ \isacommand{using}\isamarkupfalse%
\ {\isacartoucheopen}{\isasymforall}j{\isasymin}J{\isachardot}{\kern0pt}\ {\isasymexists}{\isasymalpha}\ {\isasympsi}\ {\isasymphi}{\isachardot}{\kern0pt}\ {\isasymPsi}\ j\ {\isacharequal}{\kern0pt}\ hml{\isacharunderscore}{\kern0pt}pos\ {\isasymalpha}\ {\isasympsi}\ {\isasymand}\ {\isasymPhi}\ j\ {\isacharequal}{\kern0pt}\ hml{\isacharunderscore}{\kern0pt}pos\ {\isasymalpha}\ {\isasymphi}\ {\isasymand}\ HML{\isacharunderscore}{\kern0pt}trace\ {\isasympsi}\ {\isasymand}\ {\isacharparenleft}{\kern0pt}{\isasymforall}s{\isachardot}{\kern0pt}\ {\isacharparenleft}{\kern0pt}s\ {\isasymTurnstile}\ {\isasymphi}{\isacharparenright}{\kern0pt}\ {\isacharequal}{\kern0pt}\ {\isacharparenleft}{\kern0pt}s\ {\isasymTurnstile}\ {\isasympsi}{\isacharparenright}{\kern0pt}{\isacharparenright}{\kern0pt}{\isacartoucheclose}\ trace{\isacharunderscore}{\kern0pt}pos\ \isacommand{by}\isamarkupfalse%
\ fastforce\isanewline
\ \ \ \ \isacommand{hence}\isamarkupfalse%
\ {\isachardoublequoteopen}HML{\isacharunderscore}{\kern0pt}impossible{\isacharunderscore}{\kern0pt}futures\ {\isacharparenleft}{\kern0pt}hml{\isacharunderscore}{\kern0pt}conj\ {\isacharbraceleft}{\kern0pt}{\isacharbraceright}{\kern0pt}\ J\ {\isasymPsi}{\isacharparenright}{\kern0pt}{\isachardoublequoteclose}\isanewline
\ \ \ \ \ \ \isacommand{by}\isamarkupfalse%
\ {\isacharparenleft}{\kern0pt}simp\ add{\isacharcolon}{\kern0pt}\ HML{\isacharunderscore}{\kern0pt}impossible{\isacharunderscore}{\kern0pt}futures{\isachardot}{\kern0pt}if{\isacharunderscore}{\kern0pt}conj{\isacharparenright}{\kern0pt}\isanewline
\ \ \ \ \isacommand{then}\isamarkupfalse%
\ \isacommand{show}\isamarkupfalse%
\ {\isacharquery}{\kern0pt}thesis\ \isanewline
\ \ \ \ \ \ \isacommand{using}\isamarkupfalse%
\ {\isacartoucheopen}{\isasymforall}s{\isachardot}{\kern0pt}\ {\isacharparenleft}{\kern0pt}s\ {\isasymTurnstile}\ hml{\isacharunderscore}{\kern0pt}conj\ I\ J\ {\isasymPhi}{\isacharparenright}{\kern0pt}\ {\isacharequal}{\kern0pt}\ {\isacharparenleft}{\kern0pt}s\ {\isasymTurnstile}\ hml{\isacharunderscore}{\kern0pt}conj\ {\isacharbraceleft}{\kern0pt}{\isacharbraceright}{\kern0pt}\ J\ {\isasymPsi}{\isacharparenright}{\kern0pt}{\isacartoucheclose}\ \isacommand{by}\isamarkupfalse%
\ blast\isanewline
\ \ \isacommand{next}\isamarkupfalse%
\isanewline
\ \ \ \ \isacommand{case}\isamarkupfalse%
\ {\isadigit{3}}\isanewline
\ \ \ \ \isacommand{hence}\isamarkupfalse%
\ {\isachardoublequoteopen}{\isasymforall}s{\isachardot}{\kern0pt}\ s\ {\isasymTurnstile}\ {\isacharparenleft}{\kern0pt}hml{\isacharunderscore}{\kern0pt}conj\ I\ J\ {\isasymPhi}{\isacharparenright}{\kern0pt}{\isachardoublequoteclose}\ {\isachardoublequoteopen}HML{\isacharunderscore}{\kern0pt}impossible{\isacharunderscore}{\kern0pt}futures\ TT{\isachardoublequoteclose}\ {\isachardoublequoteopen}{\isasymforall}s{\isachardot}{\kern0pt}\ s\ {\isasymTurnstile}\ TT{\isachardoublequoteclose}\ \isanewline
\ \ \ \ \ \ \isacommand{by}\isamarkupfalse%
\ {\isacharparenleft}{\kern0pt}simp\ add{\isacharcolon}{\kern0pt}\ if{\isacharunderscore}{\kern0pt}conj{\isachardot}{\kern0pt}hyps{\isacharparenleft}{\kern0pt}{\isadigit{1}}{\isacharparenright}{\kern0pt}\ HML{\isacharunderscore}{\kern0pt}impossible{\isacharunderscore}{\kern0pt}futures{\isachardot}{\kern0pt}if{\isacharunderscore}{\kern0pt}tt{\isacharparenright}{\kern0pt}{\isacharplus}{\kern0pt}\isanewline
\ \ \ \ \isacommand{then}\isamarkupfalse%
\ \isacommand{show}\isamarkupfalse%
\ {\isacharquery}{\kern0pt}thesis\ \isacommand{by}\isamarkupfalse%
\ blast\isanewline
\ \ \isacommand{qed}\isamarkupfalse%
\ \isanewline
\isacommand{qed}\isamarkupfalse%
%
\endisatagproof
{\isafoldproof}%
%
\isadelimproof
\isanewline
%
\endisadelimproof
\isanewline
\isacommand{lemma}\isamarkupfalse%
\ impossible{\isacharunderscore}{\kern0pt}futures{\isacharunderscore}{\kern0pt}def{\isacharunderscore}{\kern0pt}implies{\isacharunderscore}{\kern0pt}alt{\isacharunderscore}{\kern0pt}impossible{\isacharunderscore}{\kern0pt}futures{\isacharunderscore}{\kern0pt}def{\isacharcolon}{\kern0pt}\isanewline
\ \ \isakeyword{fixes}\ {\isasymphi}\ {\isacharcolon}{\kern0pt}{\isacharcolon}{\kern0pt}\ {\isachardoublequoteopen}{\isacharparenleft}{\kern0pt}{\isacharprime}{\kern0pt}a{\isacharcomma}{\kern0pt}\ {\isacharprime}{\kern0pt}s{\isacharparenright}{\kern0pt}\ hml{\isachardoublequoteclose}\isanewline
\ \ \isakeyword{assumes}\ {\isachardoublequoteopen}HML{\isacharunderscore}{\kern0pt}impossible{\isacharunderscore}{\kern0pt}futures\ {\isasymphi}{\isachardoublequoteclose}\isanewline
\ \ \isakeyword{shows}\ {\isachardoublequoteopen}{\isasymexists}{\isasympsi}{\isachardot}{\kern0pt}\ hml{\isacharunderscore}{\kern0pt}impossible{\isacharunderscore}{\kern0pt}futures\ {\isasympsi}\ {\isasymand}\ {\isacharparenleft}{\kern0pt}{\isasymforall}s{\isachardot}{\kern0pt}\ {\isacharparenleft}{\kern0pt}s\ {\isasymTurnstile}\ {\isasymphi}{\isacharparenright}{\kern0pt}\ {\isasymlongleftrightarrow}\ {\isacharparenleft}{\kern0pt}s\ {\isasymTurnstile}\ {\isasympsi}{\isacharparenright}{\kern0pt}{\isacharparenright}{\kern0pt}{\isachardoublequoteclose}\isanewline
%
\isadelimproof
\ \ %
\endisadelimproof
%
\isatagproof
\isacommand{using}\isamarkupfalse%
\ assms\ \isacommand{proof}\isamarkupfalse%
\ induct\isanewline
\ \ \isacommand{case}\isamarkupfalse%
\ if{\isacharunderscore}{\kern0pt}tt\isanewline
\ \ \isacommand{then}\isamarkupfalse%
\ \isacommand{show}\isamarkupfalse%
\ {\isacharquery}{\kern0pt}case\ \isanewline
\ \ \ \ \isacommand{using}\isamarkupfalse%
\ hml{\isacharunderscore}{\kern0pt}impossible{\isacharunderscore}{\kern0pt}futures{\isachardot}{\kern0pt}if{\isacharunderscore}{\kern0pt}tt\ \isacommand{by}\isamarkupfalse%
\ blast\isanewline
\isacommand{next}\isamarkupfalse%
\isanewline
\ \ \isacommand{case}\isamarkupfalse%
\ {\isacharparenleft}{\kern0pt}if{\isacharunderscore}{\kern0pt}pos\ {\isasymphi}\ {\isasymalpha}{\isacharparenright}{\kern0pt}\isanewline
\ \ \isacommand{then}\isamarkupfalse%
\ \isacommand{show}\isamarkupfalse%
\ {\isacharquery}{\kern0pt}case\ \isanewline
\ \ \ \ \isacommand{using}\isamarkupfalse%
\ hml{\isacharunderscore}{\kern0pt}impossible{\isacharunderscore}{\kern0pt}futures{\isachardot}{\kern0pt}if{\isacharunderscore}{\kern0pt}pos\ \isacommand{by}\isamarkupfalse%
\ force\isanewline
\isacommand{next}\isamarkupfalse%
\isanewline
\ \ \isacommand{case}\isamarkupfalse%
\ {\isacharparenleft}{\kern0pt}if{\isacharunderscore}{\kern0pt}conj\ {\isasymPhi}\ I\ J{\isacharparenright}{\kern0pt}\isanewline
\ \ \isacommand{hence}\isamarkupfalse%
\ {\isachardoublequoteopen}{\isasymforall}x\ {\isasymin}\ {\isasymPhi}{\isacharbackquote}{\kern0pt}J{\isachardot}{\kern0pt}\ {\isacharparenleft}{\kern0pt}{\isasymexists}{\isasympsi}{\isachardot}{\kern0pt}\ hml{\isacharunderscore}{\kern0pt}trace\ {\isasympsi}\ {\isasymand}\ {\isacharparenleft}{\kern0pt}{\isasymforall}s{\isachardot}{\kern0pt}\ s\ {\isasymTurnstile}\ x\ {\isasymlongleftrightarrow}\ s\ {\isasymTurnstile}\ {\isasympsi}{\isacharparenright}{\kern0pt}{\isacharparenright}{\kern0pt}{\isachardoublequoteclose}\isanewline
\ \ \ \ \isacommand{using}\isamarkupfalse%
\ trace{\isacharunderscore}{\kern0pt}def{\isacharunderscore}{\kern0pt}implies{\isacharunderscore}{\kern0pt}alt{\isacharunderscore}{\kern0pt}trace{\isacharunderscore}{\kern0pt}def\ \isacommand{by}\isamarkupfalse%
\ blast\ \isanewline
\ \ \isacommand{hence}\isamarkupfalse%
\ {\isachardoublequoteopen}{\isasymforall}j\ {\isasymin}\ J{\isachardot}{\kern0pt}\ {\isacharparenleft}{\kern0pt}{\isasymexists}{\isasympsi}{\isachardot}{\kern0pt}\ hml{\isacharunderscore}{\kern0pt}trace\ {\isasympsi}\ {\isasymand}\ {\isacharparenleft}{\kern0pt}{\isasymforall}s{\isachardot}{\kern0pt}\ s\ {\isasymTurnstile}\ {\isasymPhi}\ j\ {\isasymlongleftrightarrow}\ s\ {\isasymTurnstile}\ {\isasympsi}{\isacharparenright}{\kern0pt}{\isacharparenright}{\kern0pt}{\isachardoublequoteclose}\isanewline
\ \ \ \ \isacommand{by}\isamarkupfalse%
\ blast\isanewline
\ \ \isacommand{hence}\isamarkupfalse%
\ {\isachardoublequoteopen}{\isasymAnd}j{\isachardot}{\kern0pt}\ j\ {\isasymin}\ J\ {\isasymLongrightarrow}\ {\isasymexists}{\isasympsi}{\isachardot}{\kern0pt}\ hml{\isacharunderscore}{\kern0pt}trace\ {\isasympsi}\ {\isasymand}\ {\isacharparenleft}{\kern0pt}{\isasymforall}s{\isachardot}{\kern0pt}\ s\ {\isasymTurnstile}\ {\isasymPhi}\ j\ {\isasymlongleftrightarrow}\ s\ {\isasymTurnstile}\ {\isasympsi}{\isacharparenright}{\kern0pt}{\isachardoublequoteclose}\ \isacommand{by}\isamarkupfalse%
\ blast\isanewline
\ \ \isacommand{define}\isamarkupfalse%
\ {\isasymPsi}\ \isakeyword{where}\ {\isachardoublequoteopen}{\isasymPsi}\ {\isasymequiv}\ {\isacharparenleft}{\kern0pt}{\isasymlambda}i{\isachardot}{\kern0pt}\ {\isacharparenleft}{\kern0pt}if\ i\ {\isasymin}\ J\ then\ {\isacharparenleft}{\kern0pt}SOME\ {\isasympsi}{\isachardot}{\kern0pt}\ hml{\isacharunderscore}{\kern0pt}trace\ {\isasympsi}\ {\isasymand}\ {\isacharparenleft}{\kern0pt}{\isasymforall}s{\isachardot}{\kern0pt}\ s\ {\isasymTurnstile}\ {\isasymPhi}\ i\ {\isasymlongleftrightarrow}\ s\ {\isasymTurnstile}\ {\isasympsi}{\isacharparenright}{\kern0pt}{\isacharparenright}{\kern0pt}\ \isanewline
\ \ \ \ \ \ \ \ \ \ \ \ \ \ \ \ \ \ \ \ \ \ \ \ \ \ \ \ \ \ else\ undefined{\isacharparenright}{\kern0pt}{\isacharparenright}{\kern0pt}{\isachardoublequoteclose}\isanewline
\ \ \isacommand{have}\isamarkupfalse%
\ {\isachardoublequoteopen}{\isasymAnd}j{\isachardot}{\kern0pt}\ j\ {\isasymin}\ J\ {\isasymLongrightarrow}\ hml{\isacharunderscore}{\kern0pt}trace\ {\isacharparenleft}{\kern0pt}{\isasymPsi}\ j{\isacharparenright}{\kern0pt}\ {\isasymand}\ {\isacharparenleft}{\kern0pt}{\isasymforall}s{\isachardot}{\kern0pt}\ s\ {\isasymTurnstile}\ {\isasymPhi}\ j\ {\isasymlongleftrightarrow}\ s\ {\isasymTurnstile}\ {\isasymPsi}\ j{\isacharparenright}{\kern0pt}{\isachardoublequoteclose}\isanewline
\ \ \ \ \isacommand{unfolding}\isamarkupfalse%
\ {\isasymPsi}{\isacharunderscore}{\kern0pt}def\ \isacommand{using}\isamarkupfalse%
\ {\isacartoucheopen}{\isasymAnd}j{\isachardot}{\kern0pt}\ j\ {\isasymin}\ J\ {\isasymLongrightarrow}\ {\isasymexists}{\isasympsi}{\isachardot}{\kern0pt}\ hml{\isacharunderscore}{\kern0pt}trace\ {\isasympsi}\ {\isasymand}\ {\isacharparenleft}{\kern0pt}{\isasymforall}s{\isachardot}{\kern0pt}\ s\ {\isasymTurnstile}\ {\isasymPhi}\ j\ {\isasymlongleftrightarrow}\ s\ {\isasymTurnstile}\ {\isasympsi}{\isacharparenright}{\kern0pt}{\isacartoucheclose}\isanewline
\ \ \ \ \isacommand{by}\isamarkupfalse%
\ {\isacharparenleft}{\kern0pt}smt\ {\isacharparenleft}{\kern0pt}verit{\isacharcomma}{\kern0pt}\ ccfv{\isacharunderscore}{\kern0pt}SIG{\isacharparenright}{\kern0pt}\ someI{\isacharparenright}{\kern0pt}\isanewline
\ \ \isacommand{hence}\isamarkupfalse%
\ {\isachardoublequoteopen}{\isasymforall}j\ {\isasymin}\ J{\isachardot}{\kern0pt}\ hml{\isacharunderscore}{\kern0pt}trace\ {\isacharparenleft}{\kern0pt}{\isasymPsi}\ j{\isacharparenright}{\kern0pt}\ {\isasymand}\ {\isacharparenleft}{\kern0pt}{\isasymforall}s{\isachardot}{\kern0pt}\ s\ {\isasymTurnstile}\ {\isasymPhi}\ j\ {\isasymlongleftrightarrow}\ s\ {\isasymTurnstile}\ {\isasymPsi}\ j{\isacharparenright}{\kern0pt}{\isachardoublequoteclose}\ \isanewline
\ \ \ \ \isacommand{by}\isamarkupfalse%
\ blast\isanewline
\ \ \isacommand{hence}\isamarkupfalse%
\ {\isachardoublequoteopen}hml{\isacharunderscore}{\kern0pt}impossible{\isacharunderscore}{\kern0pt}futures\ {\isacharparenleft}{\kern0pt}hml{\isacharunderscore}{\kern0pt}conj\ {\isacharbraceleft}{\kern0pt}{\isacharbraceright}{\kern0pt}\ J\ {\isasymPsi}{\isacharparenright}{\kern0pt}{\isachardoublequoteclose}\ \isanewline
\ \ \ \ \isacommand{using}\isamarkupfalse%
\ hml{\isacharunderscore}{\kern0pt}impossible{\isacharunderscore}{\kern0pt}futures{\isachardot}{\kern0pt}simps\ \isacommand{by}\isamarkupfalse%
\ fastforce\isanewline
\ \ \isacommand{have}\isamarkupfalse%
\ {\isachardoublequoteopen}{\isasymforall}s{\isachardot}{\kern0pt}\ s\ {\isasymTurnstile}\ {\isacharparenleft}{\kern0pt}hml{\isacharunderscore}{\kern0pt}conj\ I\ J\ {\isasymPhi}{\isacharparenright}{\kern0pt}\ {\isasymlongleftrightarrow}\ s\ {\isasymTurnstile}\ {\isacharparenleft}{\kern0pt}hml{\isacharunderscore}{\kern0pt}conj\ {\isacharbraceleft}{\kern0pt}{\isacharbraceright}{\kern0pt}\ J\ {\isasymPsi}{\isacharparenright}{\kern0pt}{\isachardoublequoteclose}\ \isanewline
\ \ \ \ \isacommand{using}\isamarkupfalse%
\ HML{\isacharunderscore}{\kern0pt}true{\isacharunderscore}{\kern0pt}TT{\isacharunderscore}{\kern0pt}like\ HML{\isacharunderscore}{\kern0pt}true{\isacharunderscore}{\kern0pt}def\ {\isacartoucheopen}{\isasymforall}j\ {\isasymin}\ J{\isachardot}{\kern0pt}\ hml{\isacharunderscore}{\kern0pt}trace\ {\isacharparenleft}{\kern0pt}{\isasymPsi}\ j{\isacharparenright}{\kern0pt}\ {\isasymand}\ {\isacharparenleft}{\kern0pt}{\isasymforall}s{\isachardot}{\kern0pt}\ s\ {\isasymTurnstile}\ {\isasymPhi}\ j\ {\isasymlongleftrightarrow}\ s\ {\isasymTurnstile}\ {\isasymPsi}\ j{\isacharparenright}{\kern0pt}{\isacartoucheclose}\ if{\isacharunderscore}{\kern0pt}conj{\isachardot}{\kern0pt}hyps{\isacharparenleft}{\kern0pt}{\isadigit{1}}{\isacharparenright}{\kern0pt}\ \isacommand{by}\isamarkupfalse%
\ auto\isanewline
\ \ \isacommand{then}\isamarkupfalse%
\ \isacommand{show}\isamarkupfalse%
\ {\isacharquery}{\kern0pt}case\ \isanewline
\ \ \ \ \isacommand{using}\isamarkupfalse%
\ {\isacartoucheopen}hml{\isacharunderscore}{\kern0pt}impossible{\isacharunderscore}{\kern0pt}futures\ {\isacharparenleft}{\kern0pt}hml{\isacharunderscore}{\kern0pt}conj\ {\isacharbraceleft}{\kern0pt}{\isacharbraceright}{\kern0pt}\ J\ {\isasymPsi}{\isacharparenright}{\kern0pt}{\isacartoucheclose}\ \isacommand{by}\isamarkupfalse%
\ blast\isanewline
\isacommand{qed}\isamarkupfalse%
%
\endisatagproof
{\isafoldproof}%
%
\isadelimproof
\isanewline
%
\endisadelimproof
\isanewline
\isacommand{lemma}\isamarkupfalse%
\ alt{\isacharunderscore}{\kern0pt}failure{\isacharunderscore}{\kern0pt}trace{\isacharunderscore}{\kern0pt}def{\isacharunderscore}{\kern0pt}implies{\isacharunderscore}{\kern0pt}failure{\isacharunderscore}{\kern0pt}trace{\isacharunderscore}{\kern0pt}def{\isacharcolon}{\kern0pt}\isanewline
\ \ \isakeyword{fixes}\ {\isasymphi}\ {\isacharcolon}{\kern0pt}{\isacharcolon}{\kern0pt}\ {\isachardoublequoteopen}{\isacharparenleft}{\kern0pt}{\isacharprime}{\kern0pt}a{\isacharcomma}{\kern0pt}\ {\isacharprime}{\kern0pt}s{\isacharparenright}{\kern0pt}\ hml{\isachardoublequoteclose}\isanewline
\ \ \isakeyword{assumes}\ {\isachardoublequoteopen}hml{\isacharunderscore}{\kern0pt}failure{\isacharunderscore}{\kern0pt}trace\ {\isasymphi}{\isachardoublequoteclose}\isanewline
\ \ \isakeyword{shows}\ {\isachardoublequoteopen}{\isasymexists}{\isasympsi}{\isachardot}{\kern0pt}\ HML{\isacharunderscore}{\kern0pt}failure{\isacharunderscore}{\kern0pt}trace\ {\isasympsi}\ {\isasymand}\ {\isacharparenleft}{\kern0pt}{\isasymforall}s{\isachardot}{\kern0pt}\ {\isacharparenleft}{\kern0pt}s\ {\isasymTurnstile}\ {\isasymphi}{\isacharparenright}{\kern0pt}\ {\isasymlongleftrightarrow}\ {\isacharparenleft}{\kern0pt}s\ {\isasymTurnstile}\ {\isasympsi}{\isacharparenright}{\kern0pt}{\isacharparenright}{\kern0pt}{\isachardoublequoteclose}\isanewline
%
\isadelimproof
\ \ %
\endisadelimproof
%
\isatagproof
\isacommand{using}\isamarkupfalse%
\ assms\ \isacommand{proof}\isamarkupfalse%
\ induct\isanewline
\ \ \isacommand{case}\isamarkupfalse%
\ {\isadigit{1}}\isanewline
\ \ \isacommand{then}\isamarkupfalse%
\ \isacommand{show}\isamarkupfalse%
\ {\isacharquery}{\kern0pt}case\isanewline
\ \ \ \ \isacommand{using}\isamarkupfalse%
\ f{\isacharunderscore}{\kern0pt}trace{\isacharunderscore}{\kern0pt}tt\ \isacommand{by}\isamarkupfalse%
\ blast\isanewline
\isacommand{next}\isamarkupfalse%
\isanewline
\ \ \isacommand{case}\isamarkupfalse%
\ {\isacharparenleft}{\kern0pt}{\isadigit{2}}\ {\isasymphi}\ {\isasymalpha}{\isacharparenright}{\kern0pt}\isanewline
\ \ \isacommand{then}\isamarkupfalse%
\ \isacommand{obtain}\isamarkupfalse%
\ {\isasympsi}\ \isakeyword{where}\ {\isachardoublequoteopen}HML{\isacharunderscore}{\kern0pt}failure{\isacharunderscore}{\kern0pt}trace\ {\isasympsi}{\isachardoublequoteclose}\ {\isachardoublequoteopen}{\isacharparenleft}{\kern0pt}{\isasymforall}s{\isachardot}{\kern0pt}\ {\isacharparenleft}{\kern0pt}s\ {\isasymTurnstile}\ {\isasymphi}{\isacharparenright}{\kern0pt}\ {\isacharequal}{\kern0pt}\ {\isacharparenleft}{\kern0pt}s\ {\isasymTurnstile}\ {\isasympsi}{\isacharparenright}{\kern0pt}{\isacharparenright}{\kern0pt}{\isachardoublequoteclose}\ \isacommand{by}\isamarkupfalse%
\ blast\isanewline
\ \ \isacommand{have}\isamarkupfalse%
\ {\isachardoublequoteopen}HML{\isacharunderscore}{\kern0pt}failure{\isacharunderscore}{\kern0pt}trace\ {\isacharparenleft}{\kern0pt}hml{\isacharunderscore}{\kern0pt}pos\ {\isasymalpha}\ {\isasympsi}{\isacharparenright}{\kern0pt}{\isachardoublequoteclose}\ \isanewline
\ \ \ \ \isacommand{by}\isamarkupfalse%
\ {\isacharparenleft}{\kern0pt}simp\ add{\isacharcolon}{\kern0pt}\ {\isacartoucheopen}HML{\isacharunderscore}{\kern0pt}failure{\isacharunderscore}{\kern0pt}trace\ {\isasympsi}{\isacartoucheclose}\ f{\isacharunderscore}{\kern0pt}trace{\isacharunderscore}{\kern0pt}pos{\isacharparenright}{\kern0pt}\isanewline
\ \ \isacommand{have}\isamarkupfalse%
\ {\isachardoublequoteopen}{\isacharparenleft}{\kern0pt}{\isasymforall}s{\isachardot}{\kern0pt}\ {\isacharparenleft}{\kern0pt}s\ {\isasymTurnstile}\ hml{\isacharunderscore}{\kern0pt}pos\ {\isasymalpha}\ {\isasymphi}{\isacharparenright}{\kern0pt}\ {\isacharequal}{\kern0pt}\ {\isacharparenleft}{\kern0pt}s\ {\isasymTurnstile}\ {\isacharparenleft}{\kern0pt}hml{\isacharunderscore}{\kern0pt}pos\ {\isasymalpha}\ {\isasympsi}{\isacharparenright}{\kern0pt}{\isacharparenright}{\kern0pt}{\isacharparenright}{\kern0pt}{\isachardoublequoteclose}\ \isanewline
\ \ \ \ \isacommand{by}\isamarkupfalse%
\ {\isacharparenleft}{\kern0pt}simp\ add{\isacharcolon}{\kern0pt}\ {\isacartoucheopen}{\isasymforall}s{\isachardot}{\kern0pt}\ {\isacharparenleft}{\kern0pt}s\ {\isasymTurnstile}\ {\isasymphi}{\isacharparenright}{\kern0pt}\ {\isacharequal}{\kern0pt}\ {\isacharparenleft}{\kern0pt}s\ {\isasymTurnstile}\ {\isasympsi}{\isacharparenright}{\kern0pt}{\isacartoucheclose}{\isacharparenright}{\kern0pt}\isanewline
\ \ \isacommand{then}\isamarkupfalse%
\ \isacommand{show}\isamarkupfalse%
\ {\isacharquery}{\kern0pt}case\ \isanewline
\ \ \ \ \isacommand{using}\isamarkupfalse%
\ {\isacartoucheopen}HML{\isacharunderscore}{\kern0pt}failure{\isacharunderscore}{\kern0pt}trace\ {\isacharparenleft}{\kern0pt}hml{\isacharunderscore}{\kern0pt}pos\ {\isasymalpha}\ {\isasympsi}{\isacharparenright}{\kern0pt}{\isacartoucheclose}\ \isacommand{by}\isamarkupfalse%
\ blast\isanewline
\isacommand{next}\isamarkupfalse%
\isanewline
\ \ \isacommand{case}\isamarkupfalse%
\ {\isacharparenleft}{\kern0pt}{\isadigit{3}}\ {\isasymPhi}\ I\ J{\isacharparenright}{\kern0pt}\isanewline
\ \ \isacommand{hence}\isamarkupfalse%
\ neg{\isacharunderscore}{\kern0pt}case{\isacharcolon}{\kern0pt}\ {\isachardoublequoteopen}{\isasymforall}j{\isasymin}{\isasymPhi}\ {\isacharbackquote}{\kern0pt}\ J{\isachardot}{\kern0pt}\ stacked{\isacharunderscore}{\kern0pt}pos{\isacharunderscore}{\kern0pt}conj{\isacharunderscore}{\kern0pt}pos\ j{\isachardoublequoteclose}\ \isanewline
\ \ \ \ \isacommand{using}\isamarkupfalse%
\ stacked{\isacharunderscore}{\kern0pt}pos{\isacharunderscore}{\kern0pt}conj{\isacharunderscore}{\kern0pt}pos{\isachardot}{\kern0pt}simps\ nested{\isacharunderscore}{\kern0pt}empty{\isacharunderscore}{\kern0pt}pos{\isacharunderscore}{\kern0pt}conj{\isachardot}{\kern0pt}intros{\isacharparenleft}{\kern0pt}{\isadigit{1}}{\isacharparenright}{\kern0pt}\ \isacommand{by}\isamarkupfalse%
\ auto\isanewline
\ \ \isacommand{consider}\isamarkupfalse%
\ {\isachardoublequoteopen}{\isasymPhi}\ {\isacharbackquote}{\kern0pt}\ I\ {\isacharequal}{\kern0pt}\ {\isacharbraceleft}{\kern0pt}{\isacharbraceright}{\kern0pt}{\isachardoublequoteclose}\isanewline
{\isacharbar}{\kern0pt}\ {\isachardoublequoteopen}{\isacharparenleft}{\kern0pt}{\isasymexists}i{\isasymin}{\isasymPhi}\ {\isacharbackquote}{\kern0pt}\ I{\isachardot}{\kern0pt}\isanewline
\ \ \ \ \ \ \ \ {\isasymPhi}\ {\isacharbackquote}{\kern0pt}\ I\ {\isacharequal}{\kern0pt}\ {\isacharbraceleft}{\kern0pt}i{\isacharbraceright}{\kern0pt}\ {\isasymand}\ hml{\isacharunderscore}{\kern0pt}failure{\isacharunderscore}{\kern0pt}trace\ i\ {\isasymand}\ {\isacharparenleft}{\kern0pt}{\isasymexists}{\isasympsi}{\isachardot}{\kern0pt}\ HML{\isacharunderscore}{\kern0pt}failure{\isacharunderscore}{\kern0pt}trace\ {\isasympsi}\ {\isasymand}\ {\isacharparenleft}{\kern0pt}{\isasymforall}s{\isachardot}{\kern0pt}\ {\isacharparenleft}{\kern0pt}s\ {\isasymTurnstile}\ i{\isacharparenright}{\kern0pt}\ {\isacharequal}{\kern0pt}\ {\isacharparenleft}{\kern0pt}s\ {\isasymTurnstile}\ {\isasympsi}{\isacharparenright}{\kern0pt}{\isacharparenright}{\kern0pt}{\isacharparenright}{\kern0pt}{\isacharparenright}{\kern0pt}\isanewline
{\isasymand}\ {\isacharparenleft}{\kern0pt}{\isasymforall}j{\isasymin}{\isasymPhi}\ {\isacharbackquote}{\kern0pt}\ J{\isachardot}{\kern0pt}\ {\isasymexists}{\isasymalpha}{\isachardot}{\kern0pt}\ j\ {\isacharequal}{\kern0pt}\ hml{\isacharunderscore}{\kern0pt}pos\ {\isasymalpha}\ TT\ {\isasymor}\ j\ {\isacharequal}{\kern0pt}\ TT{\isacharparenright}{\kern0pt}\ {\isasymand}\ I\ {\isasyminter}\ J\ {\isacharequal}{\kern0pt}\ {\isacharbraceleft}{\kern0pt}{\isacharbraceright}{\kern0pt}{\isachardoublequoteclose}\isanewline
{\isacharbar}{\kern0pt}\ {\isachardoublequoteopen}{\isacharparenleft}{\kern0pt}{\isasymexists}i{\isasymin}{\isasymPhi}\ {\isacharbackquote}{\kern0pt}\ I{\isachardot}{\kern0pt}\isanewline
\ \ \ \ \ \ \ \ {\isasymPhi}\ {\isacharbackquote}{\kern0pt}\ I\ {\isacharequal}{\kern0pt}\ {\isacharbraceleft}{\kern0pt}i{\isacharbraceright}{\kern0pt}\ {\isasymand}\ hml{\isacharunderscore}{\kern0pt}failure{\isacharunderscore}{\kern0pt}trace\ i\ {\isasymand}\ {\isacharparenleft}{\kern0pt}{\isasymexists}{\isasympsi}{\isachardot}{\kern0pt}\ HML{\isacharunderscore}{\kern0pt}failure{\isacharunderscore}{\kern0pt}trace\ {\isasympsi}\ {\isasymand}\ {\isacharparenleft}{\kern0pt}{\isasymforall}s{\isachardot}{\kern0pt}\ {\isacharparenleft}{\kern0pt}s\ {\isasymTurnstile}\ i{\isacharparenright}{\kern0pt}\ {\isacharequal}{\kern0pt}\ {\isacharparenleft}{\kern0pt}s\ {\isasymTurnstile}\ {\isasympsi}{\isacharparenright}{\kern0pt}{\isacharparenright}{\kern0pt}{\isacharparenright}{\kern0pt}{\isacharparenright}{\kern0pt}\isanewline
{\isasymand}\ {\isacharparenleft}{\kern0pt}{\isasymforall}j{\isasymin}{\isasymPhi}\ {\isacharbackquote}{\kern0pt}\ J{\isachardot}{\kern0pt}\ {\isasymexists}{\isasymalpha}{\isachardot}{\kern0pt}\ j\ {\isacharequal}{\kern0pt}\ hml{\isacharunderscore}{\kern0pt}pos\ {\isasymalpha}\ TT\ {\isasymor}\ j\ {\isacharequal}{\kern0pt}\ TT{\isacharparenright}{\kern0pt}\ {\isasymand}\ I\ {\isasyminter}\ J\ {\isasymnoteq}\ {\isacharbraceleft}{\kern0pt}{\isacharbraceright}{\kern0pt}{\isachardoublequoteclose}\isanewline
\ \ \ \ \isacommand{using}\isamarkupfalse%
\ {\isadigit{3}}\ \isacommand{by}\isamarkupfalse%
\ linarith\isanewline
\isacommand{then}\isamarkupfalse%
\ \isacommand{show}\isamarkupfalse%
\ {\isacharquery}{\kern0pt}case\ \isacommand{proof}\isamarkupfalse%
{\isacharparenleft}{\kern0pt}cases{\isacharparenright}{\kern0pt}\isanewline
\ \ \isacommand{case}\isamarkupfalse%
\ {\isadigit{1}}\isanewline
\ \ \isacommand{hence}\isamarkupfalse%
\ {\isachardoublequoteopen}HML{\isacharunderscore}{\kern0pt}failure{\isacharunderscore}{\kern0pt}trace\ {\isacharparenleft}{\kern0pt}hml{\isacharunderscore}{\kern0pt}conj\ I\ J\ {\isasymPhi}{\isacharparenright}{\kern0pt}\ {\isasymand}\ {\isacharparenleft}{\kern0pt}{\isasymforall}s{\isachardot}{\kern0pt}\ {\isacharparenleft}{\kern0pt}s\ {\isasymTurnstile}\ hml{\isacharunderscore}{\kern0pt}conj\ I\ J\ {\isasymPhi}{\isacharparenright}{\kern0pt}\ {\isacharequal}{\kern0pt}\ {\isacharparenleft}{\kern0pt}s\ {\isasymTurnstile}\ {\isacharparenleft}{\kern0pt}hml{\isacharunderscore}{\kern0pt}conj\ I\ J\ {\isasymPhi}{\isacharparenright}{\kern0pt}{\isacharparenright}{\kern0pt}{\isacharparenright}{\kern0pt}{\isachardoublequoteclose}\isanewline
\ \ \ \ \isacommand{using}\isamarkupfalse%
\ neg{\isacharunderscore}{\kern0pt}case\ \isanewline
\ \ \ \ \isacommand{by}\isamarkupfalse%
\ {\isacharparenleft}{\kern0pt}simp\ add{\isacharcolon}{\kern0pt}\ f{\isacharunderscore}{\kern0pt}trace{\isacharunderscore}{\kern0pt}conj{\isacharparenright}{\kern0pt}\isanewline
\ \ \isacommand{then}\isamarkupfalse%
\ \isacommand{show}\isamarkupfalse%
\ {\isacharquery}{\kern0pt}thesis\ \isacommand{by}\isamarkupfalse%
\ blast\isanewline
\isacommand{next}\isamarkupfalse%
\isanewline
\ \ \isacommand{case}\isamarkupfalse%
\ {\isadigit{2}}\isanewline
\ \ \isacommand{then}\isamarkupfalse%
\ \isacommand{obtain}\isamarkupfalse%
\ i\ {\isasympsi}\ \isakeyword{where}\ IH{\isacharcolon}{\kern0pt}\ {\isachardoublequoteopen}i{\isasymin}{\isasymPhi}\ {\isacharbackquote}{\kern0pt}\ I{\isachardoublequoteclose}\ {\isachardoublequoteopen}{\isasymPhi}\ {\isacharbackquote}{\kern0pt}\ I\ {\isacharequal}{\kern0pt}\ {\isacharbraceleft}{\kern0pt}i{\isacharbraceright}{\kern0pt}{\isachardoublequoteclose}\ {\isachardoublequoteopen}hml{\isacharunderscore}{\kern0pt}failure{\isacharunderscore}{\kern0pt}trace\ i{\isachardoublequoteclose}\ {\isachardoublequoteopen}HML{\isacharunderscore}{\kern0pt}failure{\isacharunderscore}{\kern0pt}trace\ {\isasympsi}{\isachardoublequoteclose}\ {\isachardoublequoteopen}{\isacharparenleft}{\kern0pt}{\isasymforall}s{\isachardot}{\kern0pt}\ {\isacharparenleft}{\kern0pt}s\ {\isasymTurnstile}\ i{\isacharparenright}{\kern0pt}\ {\isacharequal}{\kern0pt}\ {\isacharparenleft}{\kern0pt}s\ {\isasymTurnstile}\ {\isasympsi}{\isacharparenright}{\kern0pt}{\isacharparenright}{\kern0pt}{\isachardoublequoteclose}\isanewline
\ \ \ \ \isacommand{by}\isamarkupfalse%
\ auto\isanewline
\ \ \isacommand{define}\isamarkupfalse%
\ {\isasymPsi}\ \isakeyword{where}\ \ {\isasymPsi}{\isacharunderscore}{\kern0pt}def{\isacharcolon}{\kern0pt}\ {\isachardoublequoteopen}{\isasymPsi}\ {\isacharequal}{\kern0pt}\ {\isacharparenleft}{\kern0pt}{\isasymlambda}x{\isachardot}{\kern0pt}\ if\ x\ {\isasymin}\ I\ then\ {\isasympsi}\ else\ {\isacharparenleft}{\kern0pt}if\ x\ {\isasymin}\ J\ then\ {\isasymPhi}\ x\ else\ undefined{\isacharparenright}{\kern0pt}{\isacharparenright}{\kern0pt}{\isachardoublequoteclose}\isanewline
\ \ \isacommand{have}\isamarkupfalse%
\ {\isachardoublequoteopen}{\isasymPsi}\ {\isacharbackquote}{\kern0pt}\ I\ {\isacharequal}{\kern0pt}\ {\isacharbraceleft}{\kern0pt}{\isasympsi}{\isacharbraceright}{\kern0pt}{\isachardoublequoteclose}\ \isacommand{unfolding}\isamarkupfalse%
\ {\isasymPsi}{\isacharunderscore}{\kern0pt}def\ \isacommand{using}\isamarkupfalse%
\ {\isacartoucheopen}{\isasymPhi}\ {\isacharbackquote}{\kern0pt}\ I\ {\isacharequal}{\kern0pt}\ {\isacharbraceleft}{\kern0pt}i{\isacharbraceright}{\kern0pt}{\isacartoucheclose}\ \isacommand{by}\isamarkupfalse%
\ auto\isanewline
\ \ \isacommand{hence}\isamarkupfalse%
\ pos{\isacharcolon}{\kern0pt}\ {\isachardoublequoteopen}{\isacharparenleft}{\kern0pt}{\isasymexists}{\isasympsi}\ {\isasymin}\ {\isacharparenleft}{\kern0pt}{\isasymPsi}\ {\isacharbackquote}{\kern0pt}\ I{\isacharparenright}{\kern0pt}{\isachardot}{\kern0pt}\ {\isacharparenleft}{\kern0pt}HML{\isacharunderscore}{\kern0pt}failure{\isacharunderscore}{\kern0pt}trace\ {\isasympsi}{\isacharparenright}{\kern0pt}\ {\isasymand}\ {\isacharparenleft}{\kern0pt}{\isasymforall}y\ {\isasymin}\ {\isacharparenleft}{\kern0pt}{\isasymPsi}\ {\isacharbackquote}{\kern0pt}\ I{\isacharparenright}{\kern0pt}{\isachardot}{\kern0pt}\ {\isasympsi}\ {\isasymnoteq}\ y\ {\isasymlongrightarrow}\ nested{\isacharunderscore}{\kern0pt}empty{\isacharunderscore}{\kern0pt}conj\ y{\isacharparenright}{\kern0pt}{\isacharparenright}{\kern0pt}{\isachardoublequoteclose}\isanewline
\ \ \ \ \isacommand{by}\isamarkupfalse%
\ {\isacharparenleft}{\kern0pt}simp\ add{\isacharcolon}{\kern0pt}\ {\isacartoucheopen}HML{\isacharunderscore}{\kern0pt}failure{\isacharunderscore}{\kern0pt}trace\ {\isasympsi}{\isacartoucheclose}{\isacharparenright}{\kern0pt}\isanewline
\ \ \isacommand{have}\isamarkupfalse%
\ {\isachardoublequoteopen}{\isasymforall}{\isasympsi}\ {\isasymin}\ {\isasymPsi}\ {\isacharbackquote}{\kern0pt}\ J{\isachardot}{\kern0pt}\ stacked{\isacharunderscore}{\kern0pt}pos{\isacharunderscore}{\kern0pt}conj{\isacharunderscore}{\kern0pt}pos\ {\isasympsi}{\isachardoublequoteclose}\isanewline
\ \ \ \ \isacommand{unfolding}\isamarkupfalse%
\ {\isasymPsi}{\isacharunderscore}{\kern0pt}def\isanewline
\ \ \ \ \isacommand{using}\isamarkupfalse%
\ neg{\isacharunderscore}{\kern0pt}case\ {\isadigit{2}}\isanewline
\ \ \ \ \isacommand{by}\isamarkupfalse%
\ auto\isanewline
\ \ \isacommand{hence}\isamarkupfalse%
\ {\isachardoublequoteopen}HML{\isacharunderscore}{\kern0pt}failure{\isacharunderscore}{\kern0pt}trace\ {\isacharparenleft}{\kern0pt}hml{\isacharunderscore}{\kern0pt}conj\ I\ J\ {\isasymPsi}{\isacharparenright}{\kern0pt}{\isachardoublequoteclose}\ \isacommand{using}\isamarkupfalse%
\ pos\ \isanewline
\ \ \ \ \isacommand{by}\isamarkupfalse%
\ {\isacharparenleft}{\kern0pt}simp\ add{\isacharcolon}{\kern0pt}\ f{\isacharunderscore}{\kern0pt}trace{\isacharunderscore}{\kern0pt}conj{\isacharparenright}{\kern0pt}\isanewline
\ \ \isacommand{from}\isamarkupfalse%
\ {\isasymPsi}{\isacharunderscore}{\kern0pt}def\ \isacommand{have}\isamarkupfalse%
\ {\isachardoublequoteopen}{\isacharparenleft}{\kern0pt}{\isasymforall}s{\isachardot}{\kern0pt}\ {\isasymforall}j\ {\isasymin}\ J{\isachardot}{\kern0pt}\ {\isacharparenleft}{\kern0pt}{\isasymnot}{\isacharparenleft}{\kern0pt}s\ {\isasymTurnstile}\ {\isacharparenleft}{\kern0pt}{\isasymPsi}\ j{\isacharparenright}{\kern0pt}{\isacharparenright}{\kern0pt}\ {\isacharequal}{\kern0pt}\ {\isacharparenleft}{\kern0pt}{\isasymnot}{\isacharparenleft}{\kern0pt}s\ {\isasymTurnstile}\ {\isacharparenleft}{\kern0pt}{\isasymPhi}\ j{\isacharparenright}{\kern0pt}{\isacharparenright}{\kern0pt}{\isacharparenright}{\kern0pt}{\isacharparenright}{\kern0pt}{\isacharparenright}{\kern0pt}{\isachardoublequoteclose}\ \isacommand{using}\isamarkupfalse%
\ IH\ \isanewline
\ \ \ \ \isacommand{by}\isamarkupfalse%
\ auto\isanewline
\ \ \isacommand{from}\isamarkupfalse%
\ {\isasymPsi}{\isacharunderscore}{\kern0pt}def\ \isacommand{have}\isamarkupfalse%
\ {\isachardoublequoteopen}{\isacharparenleft}{\kern0pt}{\isasymforall}s{\isachardot}{\kern0pt}\ {\isasymforall}i\ {\isasymin}\ I{\isachardot}{\kern0pt}\ {\isacharparenleft}{\kern0pt}{\isasymnot}{\isacharparenleft}{\kern0pt}s\ {\isasymTurnstile}\ {\isacharparenleft}{\kern0pt}{\isasymPsi}\ i{\isacharparenright}{\kern0pt}{\isacharparenright}{\kern0pt}\ {\isacharequal}{\kern0pt}\ {\isacharparenleft}{\kern0pt}{\isasymnot}{\isacharparenleft}{\kern0pt}s\ {\isasymTurnstile}\ {\isacharparenleft}{\kern0pt}{\isasymPhi}\ i{\isacharparenright}{\kern0pt}{\isacharparenright}{\kern0pt}{\isacharparenright}{\kern0pt}{\isacharparenright}{\kern0pt}{\isacharparenright}{\kern0pt}{\isachardoublequoteclose}\ \isacommand{using}\isamarkupfalse%
\ IH\ \isanewline
\ \ \ \ \isacommand{by}\isamarkupfalse%
\ {\isacharparenleft}{\kern0pt}metis\ emptyE\ imageI\ insertE{\isacharparenright}{\kern0pt}\isanewline
\ \ \isacommand{have}\isamarkupfalse%
\ {\isachardoublequoteopen}{\isacharparenleft}{\kern0pt}{\isasymforall}s{\isachardot}{\kern0pt}\ {\isacharparenleft}{\kern0pt}s\ {\isasymTurnstile}\ hml{\isacharunderscore}{\kern0pt}conj\ I\ J\ {\isasymPhi}{\isacharparenright}{\kern0pt}\ {\isacharequal}{\kern0pt}\ {\isacharparenleft}{\kern0pt}s\ {\isasymTurnstile}\ {\isacharparenleft}{\kern0pt}hml{\isacharunderscore}{\kern0pt}conj\ I\ J\ {\isasymPsi}{\isacharparenright}{\kern0pt}{\isacharparenright}{\kern0pt}{\isacharparenright}{\kern0pt}{\isachardoublequoteclose}\ \isacommand{using}\isamarkupfalse%
\ IH\ hml{\isacharunderscore}{\kern0pt}sem{\isacharunderscore}{\kern0pt}conj\ {\isasymPsi}{\isacharunderscore}{\kern0pt}def\ \isanewline
\ \ \ \ \isacommand{using}\isamarkupfalse%
\ {\isacartoucheopen}{\isasymforall}s{\isachardot}{\kern0pt}\ {\isasymforall}i{\isasymin}I{\isachardot}{\kern0pt}\ {\isacharparenleft}{\kern0pt}s\ {\isasymTurnstile}\ {\isasymPsi}\ i{\isacharparenright}{\kern0pt}\ {\isasymnoteq}\ {\isacharparenleft}{\kern0pt}{\isasymnot}\ s\ {\isasymTurnstile}\ {\isasymPhi}\ i{\isacharparenright}{\kern0pt}{\isacartoucheclose}\ \isacommand{by}\isamarkupfalse%
\ auto\isanewline
\ \ \isacommand{then}\isamarkupfalse%
\ \isacommand{show}\isamarkupfalse%
\ {\isacharquery}{\kern0pt}thesis\ \isacommand{using}\isamarkupfalse%
\ {\isacartoucheopen}HML{\isacharunderscore}{\kern0pt}failure{\isacharunderscore}{\kern0pt}trace\ {\isacharparenleft}{\kern0pt}hml{\isacharunderscore}{\kern0pt}conj\ I\ J\ {\isasymPsi}{\isacharparenright}{\kern0pt}{\isacartoucheclose}\ \isacommand{by}\isamarkupfalse%
\ blast\isanewline
\isacommand{next}\isamarkupfalse%
\isanewline
\ \ \isacommand{case}\isamarkupfalse%
\ {\isadigit{3}}\isanewline
\ \ \isacommand{then}\isamarkupfalse%
\ \isacommand{obtain}\isamarkupfalse%
\ i\ {\isasympsi}\ \isakeyword{where}\ IH{\isacharcolon}{\kern0pt}\ {\isachardoublequoteopen}i{\isasymin}{\isasymPhi}\ {\isacharbackquote}{\kern0pt}\ I{\isachardoublequoteclose}\ {\isachardoublequoteopen}{\isasymPhi}\ {\isacharbackquote}{\kern0pt}\ I\ {\isacharequal}{\kern0pt}\ {\isacharbraceleft}{\kern0pt}i{\isacharbraceright}{\kern0pt}{\isachardoublequoteclose}\ {\isachardoublequoteopen}hml{\isacharunderscore}{\kern0pt}failure{\isacharunderscore}{\kern0pt}trace\ i{\isachardoublequoteclose}\ {\isachardoublequoteopen}HML{\isacharunderscore}{\kern0pt}failure{\isacharunderscore}{\kern0pt}trace\ {\isasympsi}{\isachardoublequoteclose}\ {\isachardoublequoteopen}{\isacharparenleft}{\kern0pt}{\isasymforall}s{\isachardot}{\kern0pt}\ {\isacharparenleft}{\kern0pt}s\ {\isasymTurnstile}\ i{\isacharparenright}{\kern0pt}\ {\isacharequal}{\kern0pt}\ {\isacharparenleft}{\kern0pt}s\ {\isasymTurnstile}\ {\isasympsi}{\isacharparenright}{\kern0pt}{\isacharparenright}{\kern0pt}{\isachardoublequoteclose}\isanewline
\ \ \ \ \isacommand{by}\isamarkupfalse%
\ blast\isanewline
\ \ \isacommand{then}\isamarkupfalse%
\ \isacommand{obtain}\isamarkupfalse%
\ j\ \isakeyword{where}\ {\isachardoublequoteopen}j\ {\isasymin}\ I\ {\isasyminter}\ J{\isachardoublequoteclose}\ \isanewline
\ \ \ \ \isacommand{using}\isamarkupfalse%
\ {\isachardoublequoteopen}{\isadigit{3}}{\isachardoublequoteclose}\ \isacommand{by}\isamarkupfalse%
\ auto\ \isanewline
\ \ \isacommand{from}\isamarkupfalse%
\ {\isadigit{3}}\ \isacommand{have}\isamarkupfalse%
\ {\isachardoublequoteopen}{\isacharparenleft}{\kern0pt}{\isasymforall}s{\isachardot}{\kern0pt}\ {\isasymnot}{\isacharparenleft}{\kern0pt}s\ {\isasymTurnstile}\ hml{\isacharunderscore}{\kern0pt}conj\ I\ J\ {\isasymPhi}{\isacharparenright}{\kern0pt}{\isacharparenright}{\kern0pt}{\isachardoublequoteclose}\isanewline
\ \ \ \ \isacommand{using}\isamarkupfalse%
\ index{\isacharunderscore}{\kern0pt}sets{\isacharunderscore}{\kern0pt}conj{\isacharunderscore}{\kern0pt}disjunct\ \isanewline
\ \ \ \ \isacommand{by}\isamarkupfalse%
\ presburger\isanewline
\ \ \isacommand{define}\isamarkupfalse%
\ {\isasymPsi}\ \isakeyword{where}\ {\isachardoublequoteopen}{\isasymPsi}\ {\isacharequal}{\kern0pt}\ {\isacharparenleft}{\kern0pt}{\isasymlambda}x{\isachardot}{\kern0pt}\ if\ x\ {\isasymin}\ {\isacharparenleft}{\kern0pt}I\ {\isasyminter}\ J{\isacharparenright}{\kern0pt}\ then\ TT{\isacharcolon}{\kern0pt}{\isacharcolon}{\kern0pt}{\isacharparenleft}{\kern0pt}{\isacharprime}{\kern0pt}a{\isacharcomma}{\kern0pt}\ {\isacharprime}{\kern0pt}s{\isacharparenright}{\kern0pt}\ hml\ else\ undefined{\isacharparenright}{\kern0pt}{\isachardoublequoteclose}\isanewline
\ \ \isacommand{with}\isamarkupfalse%
\ {\isacartoucheopen}j\ {\isasymin}\ I\ {\isasyminter}\ J{\isacartoucheclose}\ \isacommand{have}\isamarkupfalse%
\ {\isachardoublequoteopen}{\isasymPsi}\ {\isacharbackquote}{\kern0pt}\ {\isacharparenleft}{\kern0pt}I\ {\isasyminter}\ J{\isacharparenright}{\kern0pt}\ {\isacharequal}{\kern0pt}\ {\isacharbraceleft}{\kern0pt}TT{\isacharbraceright}{\kern0pt}{\isachardoublequoteclose}\ \isanewline
\ \ \ \ \isacommand{by}\isamarkupfalse%
\ auto\isanewline
\ \ \isacommand{have}\isamarkupfalse%
\ {\isachardoublequoteopen}stacked{\isacharunderscore}{\kern0pt}pos{\isacharunderscore}{\kern0pt}conj{\isacharunderscore}{\kern0pt}pos\ TT{\isachardoublequoteclose}\ \isanewline
\ \ \ \ \isacommand{using}\isamarkupfalse%
\ stacked{\isacharunderscore}{\kern0pt}pos{\isacharunderscore}{\kern0pt}conj{\isacharunderscore}{\kern0pt}pos{\isachardot}{\kern0pt}intros{\isacharparenleft}{\kern0pt}{\isadigit{1}}{\isacharparenright}{\kern0pt}\ \isacommand{by}\isamarkupfalse%
\ blast\isanewline
\ \ \isacommand{hence}\isamarkupfalse%
\ {\isachardoublequoteopen}{\isacharparenleft}{\kern0pt}{\isasymforall}y\ {\isasymin}\ {\isacharparenleft}{\kern0pt}{\isasymPsi}\ {\isacharbackquote}{\kern0pt}\ {\isacharparenleft}{\kern0pt}I\ {\isasyminter}\ J{\isacharparenright}{\kern0pt}{\isacharparenright}{\kern0pt}{\isachardot}{\kern0pt}\ stacked{\isacharunderscore}{\kern0pt}pos{\isacharunderscore}{\kern0pt}conj{\isacharunderscore}{\kern0pt}pos\ y{\isacharparenright}{\kern0pt}{\isachardoublequoteclose}\ \isacommand{using}\isamarkupfalse%
\ {\isasymPsi}{\isacharunderscore}{\kern0pt}def\ {\isacartoucheopen}j\ {\isasymin}\ I\ {\isasyminter}\ J{\isacartoucheclose}\ \isanewline
\ \ \ \ \isacommand{using}\isamarkupfalse%
\ {\isacartoucheopen}{\isasymPsi}\ {\isacharbackquote}{\kern0pt}\ {\isacharparenleft}{\kern0pt}I\ {\isasyminter}\ J{\isacharparenright}{\kern0pt}\ {\isacharequal}{\kern0pt}\ {\isacharbraceleft}{\kern0pt}TT{\isacharbraceright}{\kern0pt}{\isacartoucheclose}\ \isacommand{by}\isamarkupfalse%
\ fastforce\isanewline
\ \ \isacommand{have}\isamarkupfalse%
\ {\isachardoublequoteopen}{\isacharparenleft}{\kern0pt}{\isasymforall}y\ {\isasymin}\ {\isacharparenleft}{\kern0pt}{\isasymPsi}\ {\isacharbackquote}{\kern0pt}\ {\isacharbraceleft}{\kern0pt}{\isacharbraceright}{\kern0pt}{\isacharparenright}{\kern0pt}{\isachardot}{\kern0pt}\ nested{\isacharunderscore}{\kern0pt}empty{\isacharunderscore}{\kern0pt}conj\ y{\isacharparenright}{\kern0pt}\ {\isasymand}\ {\isacharparenleft}{\kern0pt}{\isasymforall}y\ {\isasymin}\ {\isacharparenleft}{\kern0pt}{\isasymPsi}\ {\isacharbackquote}{\kern0pt}\ {\isacharparenleft}{\kern0pt}I\ {\isasyminter}\ J{\isacharparenright}{\kern0pt}{\isacharparenright}{\kern0pt}{\isachardot}{\kern0pt}\ stacked{\isacharunderscore}{\kern0pt}pos{\isacharunderscore}{\kern0pt}conj{\isacharunderscore}{\kern0pt}pos\ y{\isacharparenright}{\kern0pt}{\isachardoublequoteclose}\ \isanewline
\ \ \ \ \isacommand{using}\isamarkupfalse%
\ neg{\isacharunderscore}{\kern0pt}case\ \isanewline
\ \ \ \ \isacommand{using}\isamarkupfalse%
\ {\isacartoucheopen}{\isasymforall}y{\isasymin}{\isasymPsi}\ {\isacharbackquote}{\kern0pt}\ {\isacharparenleft}{\kern0pt}I\ {\isasyminter}\ J{\isacharparenright}{\kern0pt}{\isachardot}{\kern0pt}\ stacked{\isacharunderscore}{\kern0pt}pos{\isacharunderscore}{\kern0pt}conj{\isacharunderscore}{\kern0pt}pos\ y{\isacartoucheclose}\ \isacommand{by}\isamarkupfalse%
\ blast\isanewline
\ \ \isacommand{hence}\isamarkupfalse%
\ f{\isacharunderscore}{\kern0pt}trace{\isacharcolon}{\kern0pt}\ {\isachardoublequoteopen}{\isacharparenleft}{\kern0pt}{\isacharparenleft}{\kern0pt}{\isasymexists}{\isasympsi}{\isasymin}{\isacharparenleft}{\kern0pt}{\isasymPsi}\ {\isacharbackquote}{\kern0pt}\ {\isacharparenleft}{\kern0pt}{\isacharbraceleft}{\kern0pt}{\isacharbraceright}{\kern0pt}{\isacharcolon}{\kern0pt}{\isacharcolon}{\kern0pt}{\isacharprime}{\kern0pt}s\ set{\isacharparenright}{\kern0pt}{\isacharparenright}{\kern0pt}{\isachardot}{\kern0pt}\ HML{\isacharunderscore}{\kern0pt}failure{\isacharunderscore}{\kern0pt}trace\ {\isasympsi}\ {\isasymand}\ {\isacharparenleft}{\kern0pt}{\isasymforall}y{\isasymin}{\isacharparenleft}{\kern0pt}{\isasymPsi}\ {\isacharbackquote}{\kern0pt}\ {\isacharparenleft}{\kern0pt}{\isacharbraceleft}{\kern0pt}{\isacharbraceright}{\kern0pt}{\isacharcolon}{\kern0pt}{\isacharcolon}{\kern0pt}{\isacharprime}{\kern0pt}s\ set{\isacharparenright}{\kern0pt}{\isacharparenright}{\kern0pt}{\isachardot}{\kern0pt}\ {\isasympsi}\ {\isasymnoteq}\ y\ {\isasymlongrightarrow}\ nested{\isacharunderscore}{\kern0pt}empty{\isacharunderscore}{\kern0pt}conj\ y{\isacharparenright}{\kern0pt}{\isacharparenright}{\kern0pt}\ {\isasymor}\isanewline
\ {\isacharparenleft}{\kern0pt}{\isasymforall}y{\isasymin}{\isacharparenleft}{\kern0pt}{\isasymPsi}\ {\isacharbackquote}{\kern0pt}\ {\isacharparenleft}{\kern0pt}{\isacharbraceleft}{\kern0pt}{\isacharbraceright}{\kern0pt}{\isacharcolon}{\kern0pt}{\isacharcolon}{\kern0pt}{\isacharprime}{\kern0pt}s\ set{\isacharparenright}{\kern0pt}{\isacharparenright}{\kern0pt}{\isachardot}{\kern0pt}\ nested{\isacharunderscore}{\kern0pt}empty{\isacharunderscore}{\kern0pt}conj\ y{\isacharparenright}{\kern0pt}{\isacharparenright}{\kern0pt}\ {\isasymand}\isanewline
{\isacharparenleft}{\kern0pt}{\isasymforall}y{\isasymin}{\isacharparenleft}{\kern0pt}{\isasymPsi}\ {\isacharbackquote}{\kern0pt}\ {\isacharparenleft}{\kern0pt}I\ {\isasyminter}\ J{\isacharparenright}{\kern0pt}{\isacharparenright}{\kern0pt}{\isachardot}{\kern0pt}\ stacked{\isacharunderscore}{\kern0pt}pos{\isacharunderscore}{\kern0pt}conj{\isacharunderscore}{\kern0pt}pos\ y{\isacharparenright}{\kern0pt}{\isachardoublequoteclose}\ \isanewline
\ \ \ \ \isacommand{by}\isamarkupfalse%
\ fastforce\ \isanewline
\ \ \isacommand{define}\isamarkupfalse%
\ {\isasympsi}\ \isakeyword{where}\ {\isachardoublequoteopen}{\isasympsi}\ {\isasymequiv}\ {\isacharparenleft}{\kern0pt}hml{\isacharunderscore}{\kern0pt}conj\ {\isacharparenleft}{\kern0pt}{\isacharbraceleft}{\kern0pt}{\isacharbraceright}{\kern0pt}{\isacharcolon}{\kern0pt}{\isacharcolon}{\kern0pt}{\isacharprime}{\kern0pt}s\ set{\isacharparenright}{\kern0pt}\ {\isacharparenleft}{\kern0pt}I\ {\isasyminter}\ J{\isacharparenright}{\kern0pt}\ {\isasymPsi}{\isacharparenright}{\kern0pt}{\isachardoublequoteclose}\isanewline
\ \ \isacommand{have}\isamarkupfalse%
\ {\isachardoublequoteopen}HML{\isacharunderscore}{\kern0pt}failure{\isacharunderscore}{\kern0pt}trace\ {\isasympsi}{\isachardoublequoteclose}\ \isacommand{unfolding}\isamarkupfalse%
\ {\isasympsi}{\isacharunderscore}{\kern0pt}def\ \isacommand{using}\isamarkupfalse%
\ f{\isacharunderscore}{\kern0pt}trace{\isacharunderscore}{\kern0pt}conj\ f{\isacharunderscore}{\kern0pt}trace\ \isanewline
\ \ \ \ \isacommand{by}\isamarkupfalse%
\ fastforce\isanewline
\ \ \isacommand{have}\isamarkupfalse%
\ {\isachardoublequoteopen}{\isasymforall}s{\isachardot}{\kern0pt}\ {\isasymnot}\ s\ {\isasymTurnstile}\ {\isasympsi}{\isachardoublequoteclose}\ \isanewline
\ \ \ \ \isacommand{using}\isamarkupfalse%
\ {\isasymPsi}{\isacharunderscore}{\kern0pt}def\ {\isacartoucheopen}j\ {\isasymin}\ I\ {\isasyminter}\ J{\isacartoucheclose}\ {\isasympsi}{\isacharunderscore}{\kern0pt}def\ \isacommand{by}\isamarkupfalse%
\ auto\isanewline
\ \ \isacommand{then}\isamarkupfalse%
\ \isacommand{show}\isamarkupfalse%
\ {\isacharquery}{\kern0pt}thesis\ \isacommand{using}\isamarkupfalse%
\ {\isacartoucheopen}HML{\isacharunderscore}{\kern0pt}failure{\isacharunderscore}{\kern0pt}trace\ {\isasympsi}{\isacartoucheclose}\ \isanewline
\ \ \ \ \isacommand{using}\isamarkupfalse%
\ {\isacartoucheopen}{\isasymforall}s{\isachardot}{\kern0pt}\ {\isasymnot}\ s\ {\isasymTurnstile}\ hml{\isacharunderscore}{\kern0pt}conj\ I\ J\ {\isasymPhi}{\isacartoucheclose}\ \isacommand{by}\isamarkupfalse%
\ blast\isanewline
\ \ \isacommand{qed}\isamarkupfalse%
\isanewline
\isacommand{qed}\isamarkupfalse%
%
\endisatagproof
{\isafoldproof}%
%
\isadelimproof
\isanewline
%
\endisadelimproof
\isanewline
\isacommand{lemma}\isamarkupfalse%
\ stacked{\isacharunderscore}{\kern0pt}pos{\isacharunderscore}{\kern0pt}rewriting{\isacharcolon}{\kern0pt}\isanewline
\ \ \isakeyword{assumes}\ {\isachardoublequoteopen}stacked{\isacharunderscore}{\kern0pt}pos{\isacharunderscore}{\kern0pt}conj{\isacharunderscore}{\kern0pt}pos\ {\isasymphi}{\isachardoublequoteclose}\ {\isachardoublequoteopen}{\isasymnot}HML{\isacharunderscore}{\kern0pt}true\ {\isasymphi}{\isachardoublequoteclose}\isanewline
\ \ \isakeyword{shows}\ {\isachardoublequoteopen}{\isasymexists}{\isasymalpha}{\isachardot}{\kern0pt}\ {\isacharparenleft}{\kern0pt}{\isasymforall}s{\isachardot}{\kern0pt}\ {\isacharparenleft}{\kern0pt}s\ {\isasymTurnstile}\ {\isasymphi}{\isacharparenright}{\kern0pt}\ {\isasymlongleftrightarrow}\ {\isacharparenleft}{\kern0pt}s\ {\isasymTurnstile}\ {\isacharparenleft}{\kern0pt}hml{\isacharunderscore}{\kern0pt}pos\ {\isasymalpha}\ TT{\isacharparenright}{\kern0pt}{\isacharparenright}{\kern0pt}{\isacharparenright}{\kern0pt}{\isachardoublequoteclose}\isanewline
%
\isadelimproof
\ \ %
\endisadelimproof
%
\isatagproof
\isacommand{using}\isamarkupfalse%
\ assms\ \isacommand{proof}\isamarkupfalse%
{\isacharparenleft}{\kern0pt}induction\ {\isasymphi}{\isacharparenright}{\kern0pt}\isanewline
\ \ \isacommand{case}\isamarkupfalse%
\ {\isadigit{1}}\isanewline
\ \ \isacommand{then}\isamarkupfalse%
\ \isacommand{show}\isamarkupfalse%
\ {\isacharquery}{\kern0pt}case\ \isanewline
\ \ \ \ \isacommand{unfolding}\isamarkupfalse%
\ HML{\isacharunderscore}{\kern0pt}true{\isacharunderscore}{\kern0pt}def\isanewline
\ \ \ \ \isacommand{using}\isamarkupfalse%
\ TT{\isacharunderscore}{\kern0pt}like{\isachardot}{\kern0pt}intros{\isacharparenleft}{\kern0pt}{\isadigit{1}}{\isacharparenright}{\kern0pt}\ HML{\isacharunderscore}{\kern0pt}true{\isacharunderscore}{\kern0pt}TT{\isacharunderscore}{\kern0pt}like\ \isacommand{by}\isamarkupfalse%
\ simp\isanewline
\isacommand{next}\isamarkupfalse%
\isanewline
\ \ \isacommand{case}\isamarkupfalse%
\ {\isacharparenleft}{\kern0pt}{\isadigit{2}}\ {\isasympsi}\ uu{\isacharparenright}{\kern0pt}\isanewline
\ \ \isacommand{then}\isamarkupfalse%
\ \isacommand{show}\isamarkupfalse%
\ {\isacharquery}{\kern0pt}case\ \isanewline
\ \ \ \ \isacommand{using}\isamarkupfalse%
\ HML{\isacharunderscore}{\kern0pt}true{\isacharunderscore}{\kern0pt}def\ HML{\isacharunderscore}{\kern0pt}true{\isacharunderscore}{\kern0pt}nested{\isacharunderscore}{\kern0pt}empty{\isacharunderscore}{\kern0pt}pos{\isacharunderscore}{\kern0pt}conj\ \isacommand{by}\isamarkupfalse%
\ auto\isanewline
\isacommand{next}\isamarkupfalse%
\isanewline
\ \ \isacommand{case}\isamarkupfalse%
\ {\isacharparenleft}{\kern0pt}{\isadigit{3}}\ {\isasymPhi}\ I\ J{\isacharparenright}{\kern0pt}\isanewline
\ \ \isacommand{have}\isamarkupfalse%
\ {\isachardoublequoteopen}{\isacharparenleft}{\kern0pt}{\isasymforall}{\isasympsi}{\isasymin}{\isasymPhi}\ {\isacharbackquote}{\kern0pt}\ I{\isachardot}{\kern0pt}\ nested{\isacharunderscore}{\kern0pt}empty{\isacharunderscore}{\kern0pt}pos{\isacharunderscore}{\kern0pt}conj\ {\isasympsi}\ {\isasymlongrightarrow}\ HML{\isacharunderscore}{\kern0pt}true\ {\isasympsi}{\isacharparenright}{\kern0pt}{\isachardoublequoteclose}\ \isanewline
\ \ \ \ \isacommand{using}\isamarkupfalse%
\ lts{\isachardot}{\kern0pt}HML{\isacharunderscore}{\kern0pt}true{\isacharunderscore}{\kern0pt}nested{\isacharunderscore}{\kern0pt}empty{\isacharunderscore}{\kern0pt}pos{\isacharunderscore}{\kern0pt}conj\ \isacommand{by}\isamarkupfalse%
\ blast\isanewline
\ \ \isacommand{have}\isamarkupfalse%
\ {\isachardoublequoteopen}{\isacharparenleft}{\kern0pt}{\isacharparenleft}{\kern0pt}{\isasymforall}{\isasympsi}{\isasymin}{\isasymPhi}\ {\isacharbackquote}{\kern0pt}\ I{\isachardot}{\kern0pt}\ nested{\isacharunderscore}{\kern0pt}empty{\isacharunderscore}{\kern0pt}pos{\isacharunderscore}{\kern0pt}conj\ {\isasympsi}{\isacharparenright}{\kern0pt}\ {\isasymand}\ {\isasymPhi}\ {\isacharbackquote}{\kern0pt}\ J\ {\isacharequal}{\kern0pt}\ {\isacharbraceleft}{\kern0pt}{\isacharbraceright}{\kern0pt}{\isacharparenright}{\kern0pt}\ {\isasymlongrightarrow}\ HML{\isacharunderscore}{\kern0pt}true\ {\isacharparenleft}{\kern0pt}hml{\isacharunderscore}{\kern0pt}conj\ I\ J\ {\isasymPhi}{\isacharparenright}{\kern0pt}{\isachardoublequoteclose}\isanewline
\ \ \ \ \isacommand{by}\isamarkupfalse%
\ {\isacharparenleft}{\kern0pt}simp\ add{\isacharcolon}{\kern0pt}\ lts{\isachardot}{\kern0pt}HML{\isacharunderscore}{\kern0pt}true{\isacharunderscore}{\kern0pt}nested{\isacharunderscore}{\kern0pt}empty{\isacharunderscore}{\kern0pt}pos{\isacharunderscore}{\kern0pt}conj\ nested{\isacharunderscore}{\kern0pt}empty{\isacharunderscore}{\kern0pt}pos{\isacharunderscore}{\kern0pt}conj{\isachardot}{\kern0pt}intros{\isacharparenleft}{\kern0pt}{\isadigit{2}}{\isacharparenright}{\kern0pt}{\isacharparenright}{\kern0pt}\isanewline
\ \ \isacommand{with}\isamarkupfalse%
\ {\isadigit{3}}\ \isacommand{obtain}\isamarkupfalse%
\ {\isasymphi}\ \isakeyword{where}\ {\isachardoublequoteopen}{\isasymphi}{\isasymin}{\isasymPhi}\ {\isacharbackquote}{\kern0pt}\ I{\isachardoublequoteclose}\isanewline
\ \ \ \ \ \ \ \ {\isachardoublequoteopen}stacked{\isacharunderscore}{\kern0pt}pos{\isacharunderscore}{\kern0pt}conj{\isacharunderscore}{\kern0pt}pos\ {\isasymphi}{\isachardoublequoteclose}\ {\isachardoublequoteopen}{\isacharparenleft}{\kern0pt}{\isasymnot}\ HML{\isacharunderscore}{\kern0pt}true\ {\isasymphi}\ {\isasymlongrightarrow}\ {\isacharparenleft}{\kern0pt}{\isasymexists}{\isasymalpha}{\isachardot}{\kern0pt}\ {\isasymforall}s{\isachardot}{\kern0pt}\ {\isacharparenleft}{\kern0pt}s\ {\isasymTurnstile}\ {\isasymphi}{\isacharparenright}{\kern0pt}\ {\isacharequal}{\kern0pt}\ {\isacharparenleft}{\kern0pt}s\ {\isasymTurnstile}\ hml{\isacharunderscore}{\kern0pt}pos\ {\isasymalpha}\ TT{\isacharparenright}{\kern0pt}{\isacharparenright}{\kern0pt}{\isacharparenright}{\kern0pt}{\isachardoublequoteclose}\isanewline
\ \ \ \ \ \ \ \ {\isachardoublequoteopen}{\isacharparenleft}{\kern0pt}{\isasymforall}{\isasympsi}{\isasymin}{\isasymPhi}\ {\isacharbackquote}{\kern0pt}\ I{\isachardot}{\kern0pt}\ {\isasympsi}\ {\isasymnoteq}\ {\isasymphi}\ {\isasymlongrightarrow}\ nested{\isacharunderscore}{\kern0pt}empty{\isacharunderscore}{\kern0pt}pos{\isacharunderscore}{\kern0pt}conj\ {\isasympsi}{\isacharparenright}{\kern0pt}{\isachardoublequoteclose}\isanewline
\ \ \ \ \isacommand{by}\isamarkupfalse%
\ meson\isanewline
\ \ \isacommand{then}\isamarkupfalse%
\ \isacommand{have}\isamarkupfalse%
\ {\isachardoublequoteopen}{\isasymnot}\ HML{\isacharunderscore}{\kern0pt}true\ {\isasymphi}{\isachardoublequoteclose}\ \isacommand{using}\isamarkupfalse%
\ {\isadigit{3}}{\isacharparenleft}{\kern0pt}{\isadigit{3}}{\isacharparenright}{\kern0pt}\ {\isacartoucheopen}{\isacharparenleft}{\kern0pt}{\isasymforall}{\isasympsi}{\isasymin}{\isasymPhi}\ {\isacharbackquote}{\kern0pt}\ I{\isachardot}{\kern0pt}\ nested{\isacharunderscore}{\kern0pt}empty{\isacharunderscore}{\kern0pt}pos{\isacharunderscore}{\kern0pt}conj\ {\isasympsi}\ {\isasymlongrightarrow}\ HML{\isacharunderscore}{\kern0pt}true\ {\isasympsi}{\isacharparenright}{\kern0pt}{\isacartoucheclose}\isanewline
\ \ \ \ \isacommand{by}\isamarkupfalse%
\ {\isacharparenleft}{\kern0pt}smt\ {\isacharparenleft}{\kern0pt}verit{\isacharcomma}{\kern0pt}\ ccfv{\isacharunderscore}{\kern0pt}threshold{\isacharparenright}{\kern0pt}\ {\isachardoublequoteopen}{\isadigit{3}}{\isachardot}{\kern0pt}hyps{\isachardoublequoteclose}\ HML{\isacharunderscore}{\kern0pt}true{\isacharunderscore}{\kern0pt}def\ ball{\isacharunderscore}{\kern0pt}imageD\ empty{\isacharunderscore}{\kern0pt}iff\ empty{\isacharunderscore}{\kern0pt}is{\isacharunderscore}{\kern0pt}image\ hml{\isacharunderscore}{\kern0pt}sem{\isacharunderscore}{\kern0pt}conj{\isacharparenright}{\kern0pt}\isanewline
\ \ \isacommand{then}\isamarkupfalse%
\ \isacommand{obtain}\isamarkupfalse%
\ {\isasymalpha}\ \isakeyword{where}\ {\isachardoublequoteopen}{\isasymforall}s{\isachardot}{\kern0pt}\ {\isacharparenleft}{\kern0pt}s\ {\isasymTurnstile}\ {\isasymphi}{\isacharparenright}{\kern0pt}\ {\isacharequal}{\kern0pt}\ {\isacharparenleft}{\kern0pt}s\ {\isasymTurnstile}\ hml{\isacharunderscore}{\kern0pt}pos\ {\isasymalpha}\ TT{\isacharparenright}{\kern0pt}{\isachardoublequoteclose}\ \isanewline
\ \ \ \ \isacommand{using}\isamarkupfalse%
\ {\isacartoucheopen}{\isasymnot}\ HML{\isacharunderscore}{\kern0pt}true\ {\isasymphi}\ {\isasymlongrightarrow}\ {\isacharparenleft}{\kern0pt}{\isasymexists}{\isasymalpha}{\isachardot}{\kern0pt}\ {\isasymforall}s{\isachardot}{\kern0pt}\ {\isacharparenleft}{\kern0pt}s\ {\isasymTurnstile}\ {\isasymphi}{\isacharparenright}{\kern0pt}\ {\isacharequal}{\kern0pt}\ {\isacharparenleft}{\kern0pt}s\ {\isasymTurnstile}\ hml{\isacharunderscore}{\kern0pt}pos\ {\isasymalpha}\ TT{\isacharparenright}{\kern0pt}{\isacharparenright}{\kern0pt}{\isacartoucheclose}\ \isacommand{by}\isamarkupfalse%
\ blast\isanewline
\ \ \isacommand{have}\isamarkupfalse%
\ {\isachardoublequoteopen}{\isasymforall}s{\isachardot}{\kern0pt}\ {\isacharparenleft}{\kern0pt}s\ {\isasymTurnstile}\ hml{\isacharunderscore}{\kern0pt}conj\ I\ J\ {\isasymPhi}{\isacharparenright}{\kern0pt}\ {\isacharequal}{\kern0pt}\ {\isacharparenleft}{\kern0pt}s\ {\isasymTurnstile}\ hml{\isacharunderscore}{\kern0pt}pos\ {\isasymalpha}\ TT{\isacharparenright}{\kern0pt}{\isachardoublequoteclose}\ \isanewline
\ \ \ \ \isacommand{using}\isamarkupfalse%
\ {\isachardoublequoteopen}{\isadigit{3}}{\isachardot}{\kern0pt}hyps{\isachardoublequoteclose}\ {\isachardoublequoteopen}{\isadigit{3}}{\isachardot}{\kern0pt}prems{\isachardoublequoteclose}\ HML{\isacharunderscore}{\kern0pt}true{\isacharunderscore}{\kern0pt}def\ {\isacartoucheopen}{\isasymforall}{\isasympsi}{\isasymin}{\isasymPhi}\ {\isacharbackquote}{\kern0pt}\ I{\isachardot}{\kern0pt}\ {\isasympsi}\ {\isasymnoteq}\ {\isasymphi}\ {\isasymlongrightarrow}\ nested{\isacharunderscore}{\kern0pt}empty{\isacharunderscore}{\kern0pt}pos{\isacharunderscore}{\kern0pt}conj\ {\isasympsi}{\isacartoucheclose}\ {\isacartoucheopen}{\isasymforall}{\isasympsi}{\isasymin}{\isasymPhi}\ {\isacharbackquote}{\kern0pt}\ I{\isachardot}{\kern0pt}\ nested{\isacharunderscore}{\kern0pt}empty{\isacharunderscore}{\kern0pt}pos{\isacharunderscore}{\kern0pt}conj\ {\isasympsi}\ {\isasymlongrightarrow}\ HML{\isacharunderscore}{\kern0pt}true\ {\isasympsi}{\isacartoucheclose}\ {\isacartoucheopen}{\isasymforall}s{\isachardot}{\kern0pt}\ {\isacharparenleft}{\kern0pt}s\ {\isasymTurnstile}\ {\isasymphi}{\isacharparenright}{\kern0pt}\ {\isacharequal}{\kern0pt}\ {\isacharparenleft}{\kern0pt}s\ {\isasymTurnstile}\ hml{\isacharunderscore}{\kern0pt}pos\ {\isasymalpha}\ TT{\isacharparenright}{\kern0pt}{\isacartoucheclose}\ \isacommand{by}\isamarkupfalse%
\ fastforce\isanewline
\ \ \isacommand{then}\isamarkupfalse%
\ \isacommand{show}\isamarkupfalse%
\ {\isacharquery}{\kern0pt}case\ \isacommand{by}\isamarkupfalse%
\ blast\isanewline
\isacommand{qed}\isamarkupfalse%
%
\endisatagproof
{\isafoldproof}%
%
\isadelimproof
\isanewline
%
\endisadelimproof
\isanewline
\isacommand{lemma}\isamarkupfalse%
\ nested{\isacharunderscore}{\kern0pt}empty{\isacharunderscore}{\kern0pt}conj{\isacharunderscore}{\kern0pt}TT{\isacharunderscore}{\kern0pt}or{\isacharunderscore}{\kern0pt}FF{\isacharcolon}{\kern0pt}\isanewline
\ \ \isakeyword{assumes}\ {\isachardoublequoteopen}nested{\isacharunderscore}{\kern0pt}empty{\isacharunderscore}{\kern0pt}conj\ {\isasymphi}{\isachardoublequoteclose}\isanewline
\ \ \isakeyword{shows}\ {\isachardoublequoteopen}{\isacharparenleft}{\kern0pt}{\isasymforall}s{\isachardot}{\kern0pt}\ {\isacharparenleft}{\kern0pt}s\ {\isasymTurnstile}\ {\isasymphi}{\isacharparenright}{\kern0pt}{\isacharparenright}{\kern0pt}\ {\isasymor}\ {\isacharparenleft}{\kern0pt}{\isasymforall}s{\isachardot}{\kern0pt}\ {\isasymnot}{\isacharparenleft}{\kern0pt}s\ {\isasymTurnstile}\ {\isasymphi}{\isacharparenright}{\kern0pt}{\isacharparenright}{\kern0pt}{\isachardoublequoteclose}\isanewline
%
\isadelimproof
\ \ %
\endisadelimproof
%
\isatagproof
\isacommand{using}\isamarkupfalse%
\ assms\ HML{\isacharunderscore}{\kern0pt}true{\isacharunderscore}{\kern0pt}nested{\isacharunderscore}{\kern0pt}empty{\isacharunderscore}{\kern0pt}pos{\isacharunderscore}{\kern0pt}conj\isanewline
\ \ \isacommand{unfolding}\isamarkupfalse%
\ HML{\isacharunderscore}{\kern0pt}true{\isacharunderscore}{\kern0pt}def\isanewline
\ \ \isacommand{by}\isamarkupfalse%
{\isacharparenleft}{\kern0pt}induction{\isacharcomma}{\kern0pt}\ simp{\isacharcomma}{\kern0pt}\ fastforce{\isacharparenright}{\kern0pt}%
\endisatagproof
{\isafoldproof}%
%
\isadelimproof
\isanewline
%
\endisadelimproof
\isanewline
\isacommand{lemma}\isamarkupfalse%
\ failure{\isacharunderscore}{\kern0pt}trace{\isacharunderscore}{\kern0pt}def{\isacharunderscore}{\kern0pt}implies{\isacharunderscore}{\kern0pt}alt{\isacharunderscore}{\kern0pt}failure{\isacharunderscore}{\kern0pt}trace{\isacharunderscore}{\kern0pt}def{\isacharcolon}{\kern0pt}\isanewline
\ \ \isakeyword{fixes}\ {\isasymphi}\ {\isacharcolon}{\kern0pt}{\isacharcolon}{\kern0pt}\ {\isachardoublequoteopen}{\isacharparenleft}{\kern0pt}{\isacharprime}{\kern0pt}a{\isacharcomma}{\kern0pt}\ {\isacharprime}{\kern0pt}s{\isacharparenright}{\kern0pt}\ hml{\isachardoublequoteclose}\isanewline
\ \ \isakeyword{assumes}\ {\isachardoublequoteopen}HML{\isacharunderscore}{\kern0pt}failure{\isacharunderscore}{\kern0pt}trace\ {\isasymphi}{\isachardoublequoteclose}\isanewline
\ \ \isakeyword{shows}\ {\isachardoublequoteopen}{\isasymexists}{\isasympsi}{\isachardot}{\kern0pt}\ hml{\isacharunderscore}{\kern0pt}failure{\isacharunderscore}{\kern0pt}trace\ {\isasympsi}\ {\isasymand}\ {\isacharparenleft}{\kern0pt}{\isasymforall}s{\isachardot}{\kern0pt}\ {\isacharparenleft}{\kern0pt}s\ {\isasymTurnstile}\ {\isasymphi}{\isacharparenright}{\kern0pt}\ {\isasymlongleftrightarrow}\ {\isacharparenleft}{\kern0pt}s\ {\isasymTurnstile}\ {\isasympsi}{\isacharparenright}{\kern0pt}{\isacharparenright}{\kern0pt}{\isachardoublequoteclose}\isanewline
%
\isadelimproof
\ \ %
\endisadelimproof
%
\isatagproof
\isacommand{using}\isamarkupfalse%
\ assms\ \isacommand{proof}\isamarkupfalse%
{\isacharparenleft}{\kern0pt}induct{\isacharparenright}{\kern0pt}\isanewline
\ \ \isacommand{case}\isamarkupfalse%
\ f{\isacharunderscore}{\kern0pt}trace{\isacharunderscore}{\kern0pt}tt\isanewline
\ \ \isacommand{then}\isamarkupfalse%
\ \isacommand{show}\isamarkupfalse%
\ {\isacharquery}{\kern0pt}case\ \isanewline
\ \ \ \ \isacommand{using}\isamarkupfalse%
\ hml{\isacharunderscore}{\kern0pt}failure{\isacharunderscore}{\kern0pt}trace{\isachardot}{\kern0pt}intros{\isacharparenleft}{\kern0pt}{\isadigit{1}}{\isacharparenright}{\kern0pt}\ \isacommand{by}\isamarkupfalse%
\ blast\isanewline
\isacommand{next}\isamarkupfalse%
\isanewline
\ \ \isacommand{case}\isamarkupfalse%
\ {\isacharparenleft}{\kern0pt}f{\isacharunderscore}{\kern0pt}trace{\isacharunderscore}{\kern0pt}pos\ {\isasymphi}\ {\isasymalpha}{\isacharparenright}{\kern0pt}\isanewline
\ \ \isacommand{then}\isamarkupfalse%
\ \isacommand{obtain}\isamarkupfalse%
\ {\isasympsi}\ \isakeyword{where}\ {\isachardoublequoteopen}hml{\isacharunderscore}{\kern0pt}failure{\isacharunderscore}{\kern0pt}trace\ {\isasympsi}{\isachardoublequoteclose}\ {\isachardoublequoteopen}{\isacharparenleft}{\kern0pt}{\isasymforall}s{\isachardot}{\kern0pt}\ {\isacharparenleft}{\kern0pt}s\ {\isasymTurnstile}\ {\isasymphi}{\isacharparenright}{\kern0pt}\ {\isacharequal}{\kern0pt}\ {\isacharparenleft}{\kern0pt}s\ {\isasymTurnstile}\ {\isasympsi}{\isacharparenright}{\kern0pt}{\isacharparenright}{\kern0pt}{\isachardoublequoteclose}\ \isacommand{by}\isamarkupfalse%
\ blast\isanewline
\ \ \isacommand{have}\isamarkupfalse%
\ {\isachardoublequoteopen}hml{\isacharunderscore}{\kern0pt}failure{\isacharunderscore}{\kern0pt}trace\ {\isacharparenleft}{\kern0pt}hml{\isacharunderscore}{\kern0pt}pos\ {\isasymalpha}\ {\isasympsi}{\isacharparenright}{\kern0pt}{\isachardoublequoteclose}\ \isanewline
\ \ \ \ \isacommand{using}\isamarkupfalse%
\ {\isacartoucheopen}hml{\isacharunderscore}{\kern0pt}failure{\isacharunderscore}{\kern0pt}trace\ {\isasympsi}{\isacartoucheclose}\ hml{\isacharunderscore}{\kern0pt}failure{\isacharunderscore}{\kern0pt}trace{\isachardot}{\kern0pt}simps\ \isacommand{by}\isamarkupfalse%
\ blast\isanewline
\ \ \isacommand{have}\isamarkupfalse%
\ {\isachardoublequoteopen}{\isacharparenleft}{\kern0pt}{\isasymforall}s{\isachardot}{\kern0pt}\ {\isacharparenleft}{\kern0pt}s\ {\isasymTurnstile}\ hml{\isacharunderscore}{\kern0pt}pos\ {\isasymalpha}\ {\isasymphi}{\isacharparenright}{\kern0pt}\ {\isacharequal}{\kern0pt}\ {\isacharparenleft}{\kern0pt}s\ {\isasymTurnstile}\ {\isacharparenleft}{\kern0pt}hml{\isacharunderscore}{\kern0pt}pos\ {\isasymalpha}\ {\isasympsi}{\isacharparenright}{\kern0pt}{\isacharparenright}{\kern0pt}{\isacharparenright}{\kern0pt}{\isachardoublequoteclose}\ \isanewline
\ \ \ \ \isacommand{by}\isamarkupfalse%
\ {\isacharparenleft}{\kern0pt}simp\ add{\isacharcolon}{\kern0pt}\ {\isacartoucheopen}{\isasymforall}s{\isachardot}{\kern0pt}\ {\isacharparenleft}{\kern0pt}s\ {\isasymTurnstile}\ {\isasymphi}{\isacharparenright}{\kern0pt}\ {\isacharequal}{\kern0pt}\ {\isacharparenleft}{\kern0pt}s\ {\isasymTurnstile}\ {\isasympsi}{\isacharparenright}{\kern0pt}{\isacartoucheclose}{\isacharparenright}{\kern0pt}\isanewline
\ \ \isacommand{then}\isamarkupfalse%
\ \isacommand{show}\isamarkupfalse%
\ {\isacharquery}{\kern0pt}case\ \isanewline
\ \ \ \ \isacommand{using}\isamarkupfalse%
\ {\isacartoucheopen}hml{\isacharunderscore}{\kern0pt}failure{\isacharunderscore}{\kern0pt}trace\ {\isacharparenleft}{\kern0pt}hml{\isacharunderscore}{\kern0pt}pos\ {\isasymalpha}\ {\isasympsi}{\isacharparenright}{\kern0pt}{\isacartoucheclose}\ \isacommand{by}\isamarkupfalse%
\ blast\isanewline
\isacommand{next}\isamarkupfalse%
\isanewline
\ \ \isacommand{case}\isamarkupfalse%
\ {\isacharparenleft}{\kern0pt}f{\isacharunderscore}{\kern0pt}trace{\isacharunderscore}{\kern0pt}conj\ {\isasymPhi}\ I\ J{\isacharparenright}{\kern0pt}\isanewline
\ \ \isacommand{hence}\isamarkupfalse%
\ neg{\isacharunderscore}{\kern0pt}case{\isacharcolon}{\kern0pt}\ {\isachardoublequoteopen}{\isasymforall}j{\isasymin}{\isasymPhi}\ {\isacharbackquote}{\kern0pt}\ J{\isachardot}{\kern0pt}\ stacked{\isacharunderscore}{\kern0pt}pos{\isacharunderscore}{\kern0pt}conj{\isacharunderscore}{\kern0pt}pos\ j{\isachardoublequoteclose}\ \isanewline
\ \ \ \ \isacommand{using}\isamarkupfalse%
\ stacked{\isacharunderscore}{\kern0pt}pos{\isacharunderscore}{\kern0pt}conj{\isacharunderscore}{\kern0pt}pos{\isachardot}{\kern0pt}simps\ nested{\isacharunderscore}{\kern0pt}empty{\isacharunderscore}{\kern0pt}pos{\isacharunderscore}{\kern0pt}conj{\isachardot}{\kern0pt}intros{\isacharparenleft}{\kern0pt}{\isadigit{1}}{\isacharparenright}{\kern0pt}\ \isacommand{by}\isamarkupfalse%
\ auto\isanewline
\ \ \isacommand{from}\isamarkupfalse%
\ f{\isacharunderscore}{\kern0pt}trace{\isacharunderscore}{\kern0pt}conj\ \isacommand{have}\isamarkupfalse%
\ IH{\isacharcolon}{\kern0pt}\ {\isachardoublequoteopen}{\isacharparenleft}{\kern0pt}{\isacharparenleft}{\kern0pt}{\isasymexists}{\isasympsi}{\isasymin}{\isasymPhi}\ {\isacharbackquote}{\kern0pt}\ I{\isachardot}{\kern0pt}\isanewline
\ \ \ \ \ \ \ \ \ \ \ \ {\isacharparenleft}{\kern0pt}HML{\isacharunderscore}{\kern0pt}failure{\isacharunderscore}{\kern0pt}trace\ {\isasympsi}\ {\isasymand}\ {\isacharparenleft}{\kern0pt}{\isasymexists}{\isasympsi}{\isacharprime}{\kern0pt}{\isachardot}{\kern0pt}\ hml{\isacharunderscore}{\kern0pt}failure{\isacharunderscore}{\kern0pt}trace\ {\isasympsi}{\isacharprime}{\kern0pt}\ {\isasymand}\ {\isacharparenleft}{\kern0pt}{\isasymforall}s{\isachardot}{\kern0pt}\ {\isacharparenleft}{\kern0pt}s\ {\isasymTurnstile}\ {\isasympsi}{\isacharparenright}{\kern0pt}\ {\isacharequal}{\kern0pt}\ {\isacharparenleft}{\kern0pt}s\ {\isasymTurnstile}\ {\isasympsi}{\isacharprime}{\kern0pt}{\isacharparenright}{\kern0pt}{\isacharparenright}{\kern0pt}{\isacharparenright}{\kern0pt}{\isacharparenright}{\kern0pt}\ {\isasymand}\isanewline
\ \ \ \ \ \ \ \ \ \ \ \ {\isacharparenleft}{\kern0pt}{\isasymforall}y{\isasymin}{\isasymPhi}\ {\isacharbackquote}{\kern0pt}\ I{\isachardot}{\kern0pt}\ {\isasympsi}\ {\isasymnoteq}\ y\ {\isasymlongrightarrow}\ nested{\isacharunderscore}{\kern0pt}empty{\isacharunderscore}{\kern0pt}conj\ y{\isacharparenright}{\kern0pt}{\isacharparenright}{\kern0pt}\ {\isasymor}\isanewline
\ \ \ \ \ \ \ \ {\isacharparenleft}{\kern0pt}{\isasymforall}y{\isasymin}{\isasymPhi}\ {\isacharbackquote}{\kern0pt}\ I{\isachardot}{\kern0pt}\ nested{\isacharunderscore}{\kern0pt}empty{\isacharunderscore}{\kern0pt}conj\ y{\isacharparenright}{\kern0pt}{\isacharparenright}{\kern0pt}{\isachardoublequoteclose}\ \isanewline
\ \ \ \ \isacommand{by}\isamarkupfalse%
\ blast\isanewline
\ \ \isacommand{from}\isamarkupfalse%
\ IH\ \isacommand{show}\isamarkupfalse%
\ {\isacharquery}{\kern0pt}case\ \isacommand{proof}\isamarkupfalse%
{\isacharparenleft}{\kern0pt}rule\ disjE{\isacharparenright}{\kern0pt}\isanewline
\ \ \ \ \isacommand{assume}\isamarkupfalse%
\ {\isachardoublequoteopen}{\isasymexists}{\isasympsi}{\isasymin}{\isasymPhi}\ {\isacharbackquote}{\kern0pt}\ I{\isachardot}{\kern0pt}\isanewline
\ \ \ \ \ \ \ {\isacharparenleft}{\kern0pt}HML{\isacharunderscore}{\kern0pt}failure{\isacharunderscore}{\kern0pt}trace\ {\isasympsi}\ {\isasymand}\ {\isacharparenleft}{\kern0pt}{\isasymexists}{\isasympsi}{\isacharprime}{\kern0pt}{\isachardot}{\kern0pt}\ hml{\isacharunderscore}{\kern0pt}failure{\isacharunderscore}{\kern0pt}trace\ {\isasympsi}{\isacharprime}{\kern0pt}\ {\isasymand}\ {\isacharparenleft}{\kern0pt}{\isasymforall}s{\isachardot}{\kern0pt}\ {\isacharparenleft}{\kern0pt}s\ {\isasymTurnstile}\ {\isasympsi}{\isacharparenright}{\kern0pt}\ {\isacharequal}{\kern0pt}\ {\isacharparenleft}{\kern0pt}s\ {\isasymTurnstile}\ {\isasympsi}{\isacharprime}{\kern0pt}{\isacharparenright}{\kern0pt}{\isacharparenright}{\kern0pt}{\isacharparenright}{\kern0pt}{\isacharparenright}{\kern0pt}\ {\isasymand}\isanewline
\ \ \ \ \ \ \ {\isacharparenleft}{\kern0pt}{\isasymforall}y{\isasymin}{\isasymPhi}\ {\isacharbackquote}{\kern0pt}\ I{\isachardot}{\kern0pt}\ {\isasympsi}\ {\isasymnoteq}\ y\ {\isasymlongrightarrow}\ nested{\isacharunderscore}{\kern0pt}empty{\isacharunderscore}{\kern0pt}conj\ y{\isacharparenright}{\kern0pt}{\isachardoublequoteclose}\isanewline
\ \ \ \ \isacommand{then}\isamarkupfalse%
\ \isacommand{obtain}\isamarkupfalse%
\ {\isasymphi}\ {\isasympsi}\ \isakeyword{where}\ {\isachardoublequoteopen}{\isasymphi}{\isasymin}{\isasymPhi}\ {\isacharbackquote}{\kern0pt}\ I{\isachardoublequoteclose}\ {\isachardoublequoteopen}HML{\isacharunderscore}{\kern0pt}failure{\isacharunderscore}{\kern0pt}trace\ {\isasymphi}{\isachardoublequoteclose}\ {\isachardoublequoteopen}hml{\isacharunderscore}{\kern0pt}failure{\isacharunderscore}{\kern0pt}trace\ {\isasympsi}{\isachardoublequoteclose}\ \isanewline
\ \ \ \ \ \ \ \ \ \ \ \ \ \ \ \ \ \ \ \ \ \ \ \ \ \ {\isachardoublequoteopen}{\isacharparenleft}{\kern0pt}{\isasymforall}s{\isachardot}{\kern0pt}\ {\isacharparenleft}{\kern0pt}s\ {\isasymTurnstile}\ {\isasymphi}{\isacharparenright}{\kern0pt}\ {\isacharequal}{\kern0pt}\ {\isacharparenleft}{\kern0pt}s\ {\isasymTurnstile}\ {\isasympsi}{\isacharparenright}{\kern0pt}{\isacharparenright}{\kern0pt}{\isachardoublequoteclose}\ {\isachardoublequoteopen}{\isacharparenleft}{\kern0pt}{\isasymforall}y{\isasymin}{\isasymPhi}\ {\isacharbackquote}{\kern0pt}\ I{\isachardot}{\kern0pt}\ {\isasymphi}\ {\isasymnoteq}\ y\ {\isasymlongrightarrow}\ nested{\isacharunderscore}{\kern0pt}empty{\isacharunderscore}{\kern0pt}conj\ y{\isacharparenright}{\kern0pt}{\isachardoublequoteclose}\isanewline
\ \ \ \ \ \ \isacommand{by}\isamarkupfalse%
\ meson\isanewline
\ \ \ \ \isacommand{then}\isamarkupfalse%
\ \isacommand{obtain}\isamarkupfalse%
\ i{\isacharunderscore}{\kern0pt}{\isasymphi}\ \isakeyword{where}\ {\isachardoublequoteopen}{\isasymPhi}\ i{\isacharunderscore}{\kern0pt}{\isasymphi}\ {\isacharequal}{\kern0pt}\ {\isasymphi}{\isachardoublequoteclose}\ \isanewline
\ \ \ \ \ \ \isacommand{by}\isamarkupfalse%
\ blast\isanewline
\ \ \ \ \isacommand{consider}\isamarkupfalse%
\ {\isachardoublequoteopen}{\isasymexists}y\ {\isasymin}\ {\isasymPhi}{\isacharbackquote}{\kern0pt}I{\isachardot}{\kern0pt}\ {\isasymphi}\ {\isasymnoteq}\ y\ {\isasymand}\ {\isacharparenleft}{\kern0pt}{\isasymforall}s{\isachardot}{\kern0pt}\ {\isasymnot}{\isacharparenleft}{\kern0pt}s\ {\isasymTurnstile}\ y{\isacharparenright}{\kern0pt}{\isacharparenright}{\kern0pt}{\isachardoublequoteclose}\ {\isacharbar}{\kern0pt}\ {\isachardoublequoteopen}{\isacharparenleft}{\kern0pt}{\isasymforall}y{\isasymin}{\isasymPhi}\ {\isacharbackquote}{\kern0pt}\ I{\isachardot}{\kern0pt}\ {\isasymphi}\ {\isasymnoteq}\ y\ {\isasymlongrightarrow}\ HML{\isacharunderscore}{\kern0pt}true\ y{\isacharparenright}{\kern0pt}{\isachardoublequoteclose}\isanewline
\ \ \ \ \ \ \isacommand{unfolding}\isamarkupfalse%
\ HML{\isacharunderscore}{\kern0pt}true{\isacharunderscore}{\kern0pt}def\isanewline
\ \ \ \ \ \ \isacommand{using}\isamarkupfalse%
\ nested{\isacharunderscore}{\kern0pt}empty{\isacharunderscore}{\kern0pt}conj{\isacharunderscore}{\kern0pt}TT{\isacharunderscore}{\kern0pt}or{\isacharunderscore}{\kern0pt}FF\isanewline
\ \ \ \ \ \ \isacommand{using}\isamarkupfalse%
\ {\isacartoucheopen}{\isasymforall}y{\isasymin}{\isasymPhi}\ {\isacharbackquote}{\kern0pt}\ I{\isachardot}{\kern0pt}\ {\isasymphi}\ {\isasymnoteq}\ y\ {\isasymlongrightarrow}\ nested{\isacharunderscore}{\kern0pt}empty{\isacharunderscore}{\kern0pt}conj\ y{\isacartoucheclose}\ \isacommand{by}\isamarkupfalse%
\ blast\isanewline
\ \ \ \ \isacommand{then}\isamarkupfalse%
\ \isacommand{show}\isamarkupfalse%
\ {\isachardoublequoteopen}{\isasymexists}{\isasympsi}{\isachardot}{\kern0pt}\ hml{\isacharunderscore}{\kern0pt}failure{\isacharunderscore}{\kern0pt}trace\ {\isasympsi}\ {\isasymand}\ {\isacharparenleft}{\kern0pt}{\isasymforall}s{\isachardot}{\kern0pt}\ {\isacharparenleft}{\kern0pt}s\ {\isasymTurnstile}\ hml{\isacharunderscore}{\kern0pt}conj\ I\ J\ {\isasymPhi}{\isacharparenright}{\kern0pt}\ {\isacharequal}{\kern0pt}\ {\isacharparenleft}{\kern0pt}s\ {\isasymTurnstile}\ {\isasympsi}{\isacharparenright}{\kern0pt}{\isacharparenright}{\kern0pt}{\isachardoublequoteclose}\isanewline
\ \ \ \ \isacommand{proof}\isamarkupfalse%
{\isacharparenleft}{\kern0pt}cases{\isacharparenright}{\kern0pt}\isanewline
\ \ \ \ \ \ \isacommand{case}\isamarkupfalse%
\ {\isadigit{1}}\isanewline
\ \ \ \ \ \ \isacommand{hence}\isamarkupfalse%
\ {\isachardoublequoteopen}{\isasymforall}s{\isachardot}{\kern0pt}\ {\isasymnot}s\ {\isasymTurnstile}\ {\isacharparenleft}{\kern0pt}hml{\isacharunderscore}{\kern0pt}conj\ I\ J\ {\isasymPhi}{\isacharparenright}{\kern0pt}{\isachardoublequoteclose}\isanewline
\ \ \ \ \ \ \ \ \isacommand{using}\isamarkupfalse%
\ hml{\isacharunderscore}{\kern0pt}sem{\isacharunderscore}{\kern0pt}conj\ \isacommand{by}\isamarkupfalse%
\ blast\isanewline
\ \ \ \ \ \ \isacommand{obtain}\isamarkupfalse%
\ y\ \isakeyword{where}\ {\isachardoublequoteopen}y\ {\isasymin}\ {\isasymPhi}{\isacharbackquote}{\kern0pt}I{\isachardoublequoteclose}\ {\isachardoublequoteopen}{\isasymphi}\ {\isasymnoteq}\ y\ {\isasymand}\ {\isacharparenleft}{\kern0pt}{\isasymforall}s{\isachardot}{\kern0pt}\ {\isasymnot}\ s\ {\isasymTurnstile}\ y{\isacharparenright}{\kern0pt}{\isachardoublequoteclose}\ \isanewline
\ \ \ \ \ \ \ \ \isacommand{using}\isamarkupfalse%
\ {\isachardoublequoteopen}{\isadigit{1}}{\isachardoublequoteclose}\ \isacommand{by}\isamarkupfalse%
\ blast\isanewline
\ \ \ \ \ \ \isacommand{define}\isamarkupfalse%
\ {\isasymPsi}\ \isakeyword{where}\ {\isasymPsi}{\isacharunderscore}{\kern0pt}def{\isacharcolon}{\kern0pt}\ {\isachardoublequoteopen}{\isasymPsi}\ {\isacharequal}{\kern0pt}\ {\isacharparenleft}{\kern0pt}{\isasymlambda}i{\isachardot}{\kern0pt}\ {\isacharparenleft}{\kern0pt}if\ i\ {\isasymin}\ I\ then\ {\isacharparenleft}{\kern0pt}TT{\isacharcolon}{\kern0pt}{\isacharcolon}{\kern0pt}{\isacharparenleft}{\kern0pt}{\isacharprime}{\kern0pt}a{\isacharcomma}{\kern0pt}\ {\isacharprime}{\kern0pt}s{\isacharparenright}{\kern0pt}hml{\isacharparenright}{\kern0pt}\ else\ undefined{\isacharparenright}{\kern0pt}{\isacharparenright}{\kern0pt}{\isachardoublequoteclose}\isanewline
\ \ \ \ \ \ \isacommand{hence}\isamarkupfalse%
\ {\isachardoublequoteopen}{\isasymforall}s{\isachardot}{\kern0pt}\ {\isasymnot}s\ {\isasymTurnstile}\ {\isacharparenleft}{\kern0pt}hml{\isacharunderscore}{\kern0pt}conj\ {\isacharbraceleft}{\kern0pt}{\isacharbraceright}{\kern0pt}\ I\ {\isasymPsi}{\isacharparenright}{\kern0pt}{\isachardoublequoteclose}\ \isanewline
\ \ \ \ \ \ \ \ \isacommand{using}\isamarkupfalse%
\ {\isacartoucheopen}y\ {\isasymin}\ {\isasymPhi}\ {\isacharbackquote}{\kern0pt}\ I{\isacartoucheclose}\ \isacommand{by}\isamarkupfalse%
\ auto\isanewline
\ \ \ \ \ \ \isacommand{have}\isamarkupfalse%
\ {\isachardoublequoteopen}{\isasymPsi}\ {\isacharbackquote}{\kern0pt}\ {\isacharbraceleft}{\kern0pt}{\isacharbraceright}{\kern0pt}\ {\isacharequal}{\kern0pt}\ {\isacharbraceleft}{\kern0pt}{\isacharbraceright}{\kern0pt}{\isachardoublequoteclose}\ {\isachardoublequoteopen}{\isasymforall}j\ {\isasymin}\ {\isasymPsi}\ {\isacharbackquote}{\kern0pt}\ I{\isachardot}{\kern0pt}\ j\ {\isacharequal}{\kern0pt}\ TT{\isachardoublequoteclose}\ \isanewline
\ \ \ \ \ \ \ \ \ \isacommand{apply}\isamarkupfalse%
\ blast\isanewline
\ \ \ \ \ \ \ \ \isacommand{unfolding}\isamarkupfalse%
\ {\isasymPsi}{\isacharunderscore}{\kern0pt}def\ \isanewline
\ \ \ \ \ \ \ \ \isacommand{using}\isamarkupfalse%
\ {\isacartoucheopen}y\ {\isasymin}\ {\isasymPhi}{\isacharbackquote}{\kern0pt}I{\isacartoucheclose}\ \isanewline
\ \ \ \ \ \ \ \ \isacommand{by}\isamarkupfalse%
\ simp\isanewline
\ \ \ \ \ \ \isacommand{hence}\isamarkupfalse%
\ {\isachardoublequoteopen}hml{\isacharunderscore}{\kern0pt}failure{\isacharunderscore}{\kern0pt}trace\ {\isacharparenleft}{\kern0pt}hml{\isacharunderscore}{\kern0pt}conj\ {\isacharbraceleft}{\kern0pt}{\isacharbraceright}{\kern0pt}\ I\ {\isasymPsi}{\isacharparenright}{\kern0pt}{\isachardoublequoteclose}\ \isanewline
\ \ \ \ \ \ \ \ \isacommand{by}\isamarkupfalse%
\ {\isacharparenleft}{\kern0pt}meson\ hml{\isacharunderscore}{\kern0pt}failure{\isacharunderscore}{\kern0pt}trace{\isachardot}{\kern0pt}intros{\isacharparenleft}{\kern0pt}{\isadigit{3}}{\isacharparenright}{\kern0pt}{\isacharparenright}{\kern0pt}\isanewline
\ \ \ \ \ \ \isacommand{then}\isamarkupfalse%
\ \isacommand{show}\isamarkupfalse%
\ {\isacharquery}{\kern0pt}thesis\ \isacommand{using}\isamarkupfalse%
\ {\isacartoucheopen}{\isasymforall}s{\isachardot}{\kern0pt}\ {\isasymnot}s\ {\isasymTurnstile}\ {\isacharparenleft}{\kern0pt}hml{\isacharunderscore}{\kern0pt}conj\ I\ J\ {\isasymPhi}{\isacharparenright}{\kern0pt}{\isacartoucheclose}\ \isanewline
\ \ \ \ \ \ \ \ \isacommand{using}\isamarkupfalse%
\ {\isacartoucheopen}{\isasymforall}s{\isachardot}{\kern0pt}\ {\isasymnot}\ s\ {\isasymTurnstile}\ hml{\isacharunderscore}{\kern0pt}conj\ {\isacharbraceleft}{\kern0pt}{\isacharbraceright}{\kern0pt}\ I\ {\isasymPsi}{\isacartoucheclose}\ \isacommand{by}\isamarkupfalse%
\ blast\isanewline
\ \ \ \ \isacommand{next}\isamarkupfalse%
\isanewline
\ \ \ \ \ \ \isacommand{case}\isamarkupfalse%
\ {\isadigit{2}}\isanewline
\ \ \ \ \ \ \isacommand{consider}\isamarkupfalse%
\ {\isachardoublequoteopen}{\isasymforall}y\ {\isasymin}\ {\isasymPhi}{\isacharbackquote}{\kern0pt}J{\isachardot}{\kern0pt}\ {\isasymexists}{\isasymalpha}{\isachardot}{\kern0pt}\ {\isacharparenleft}{\kern0pt}{\isasymforall}s{\isachardot}{\kern0pt}\ {\isacharparenleft}{\kern0pt}s\ {\isasymTurnstile}\ y{\isacharparenright}{\kern0pt}\ {\isasymlongleftrightarrow}\ {\isacharparenleft}{\kern0pt}s\ {\isasymTurnstile}\ {\isacharparenleft}{\kern0pt}hml{\isacharunderscore}{\kern0pt}pos\ {\isasymalpha}\ TT{\isacharparenright}{\kern0pt}{\isacharparenright}{\kern0pt}{\isacharparenright}{\kern0pt}{\isachardoublequoteclose}\ {\isacharbar}{\kern0pt}\ {\isachardoublequoteopen}{\isacharparenleft}{\kern0pt}{\isasymexists}y{\isasymin}{\isasymPhi}\ {\isacharbackquote}{\kern0pt}\ J{\isachardot}{\kern0pt}\ HML{\isacharunderscore}{\kern0pt}true\ y{\isacharparenright}{\kern0pt}{\isachardoublequoteclose}\isanewline
\ \ \ \ \ \ \ \ \isacommand{unfolding}\isamarkupfalse%
\ HML{\isacharunderscore}{\kern0pt}true{\isacharunderscore}{\kern0pt}def\isanewline
\ \ \ \ \ \ \ \ \isacommand{using}\isamarkupfalse%
\ stacked{\isacharunderscore}{\kern0pt}pos{\isacharunderscore}{\kern0pt}rewriting\ neg{\isacharunderscore}{\kern0pt}case\isanewline
\ \ \ \ \ \ \ \ \isacommand{using}\isamarkupfalse%
\ that{\isacharparenleft}{\kern0pt}{\isadigit{2}}{\isacharparenright}{\kern0pt}\ \isacommand{by}\isamarkupfalse%
\ blast\isanewline
\ \ \ \ \ \ \isacommand{then}\isamarkupfalse%
\ \isacommand{show}\isamarkupfalse%
\ {\isacharquery}{\kern0pt}thesis\ \isacommand{proof}\isamarkupfalse%
{\isacharparenleft}{\kern0pt}cases{\isacharparenright}{\kern0pt}\isanewline
\ \ \ \ \ \ \ \ \isacommand{case}\isamarkupfalse%
\ {\isadigit{1}}\isanewline
\ \ \ \ \ \ \ \ \isacommand{show}\isamarkupfalse%
\ {\isacharquery}{\kern0pt}thesis\ \isacommand{proof}\isamarkupfalse%
{\isacharparenleft}{\kern0pt}cases\ {\isachardoublequoteopen}{\isasymPhi}\ {\isacharbackquote}{\kern0pt}\ I\ {\isasyminter}\ {\isasymPhi}\ {\isacharbackquote}{\kern0pt}\ J\ {\isacharequal}{\kern0pt}\ {\isacharbraceleft}{\kern0pt}{\isacharbraceright}{\kern0pt}{\isachardoublequoteclose}{\isacharparenright}{\kern0pt}\isanewline
\ \ \ \ \ \ \ \ \ \ \isacommand{case}\isamarkupfalse%
\ True\isanewline
\ \ \ \ \ \ \ \ \ \ \isacommand{hence}\isamarkupfalse%
\ {\isachardoublequoteopen}I\ {\isasyminter}\ J\ {\isacharequal}{\kern0pt}\ {\isacharbraceleft}{\kern0pt}{\isacharbraceright}{\kern0pt}{\isachardoublequoteclose}\isanewline
\ \ \ \ \ \ \ \ \ \ \ \ \isacommand{by}\isamarkupfalse%
\ blast\isanewline
\ \ \ \ \ \ \ \ \ \ \isacommand{from}\isamarkupfalse%
\ {\isadigit{2}}\ \isacommand{have}\isamarkupfalse%
\ {\isachardoublequoteopen}{\isasymforall}y{\isasymin}{\isasymPhi}\ {\isacharbackquote}{\kern0pt}\ I{\isachardot}{\kern0pt}\ {\isasymphi}\ {\isasymnoteq}\ y\ {\isasymlongrightarrow}\ {\isacharparenleft}{\kern0pt}{\isasymforall}s{\isachardot}{\kern0pt}\ s\ {\isasymTurnstile}\ y{\isacharparenright}{\kern0pt}{\isachardoublequoteclose}\isanewline
\ \ \ \ \ \ \ \ \ \ \ \ \isacommand{unfolding}\isamarkupfalse%
\ HML{\isacharunderscore}{\kern0pt}true{\isacharunderscore}{\kern0pt}def\ \isanewline
\ \ \ \ \ \ \ \ \ \ \ \ \isacommand{by}\isamarkupfalse%
\ blast\isanewline
\ \ \ \ \ \ \ \ \ \ \isacommand{hence}\isamarkupfalse%
\ {\isachardoublequoteopen}{\isasymforall}s{\isachardot}{\kern0pt}\ {\isacharparenleft}{\kern0pt}{\isasymforall}i\ {\isasymin}\ I{\isachardot}{\kern0pt}\ s\ {\isasymTurnstile}\ {\isacharparenleft}{\kern0pt}{\isasymPhi}\ i{\isacharparenright}{\kern0pt}{\isacharparenright}{\kern0pt}\ {\isasymlongleftrightarrow}\ s\ {\isasymTurnstile}\ {\isasymphi}{\isachardoublequoteclose}\isanewline
\ \ \ \ \ \ \ \ \ \ \ \ \isacommand{using}\isamarkupfalse%
\ {\isacartoucheopen}{\isasymphi}\ {\isasymin}\ {\isasymPhi}\ {\isacharbackquote}{\kern0pt}\ I{\isacartoucheclose}\ \isacommand{by}\isamarkupfalse%
\ auto\isanewline
\ \ \ \ \ \ \ \ \ \ \isacommand{define}\isamarkupfalse%
\ {\isasymPsi}\ \isakeyword{where}\ {\isachardoublequoteopen}{\isasymPsi}\ {\isasymequiv}\ {\isacharparenleft}{\kern0pt}{\isasymlambda}i{\isachardot}{\kern0pt}\ {\isacharparenleft}{\kern0pt}if\ i\ {\isacharequal}{\kern0pt}\ i{\isacharunderscore}{\kern0pt}{\isasymphi}\ then\ {\isasympsi}\ \isanewline
\ \ \ \ \ \ \ \ \ \ \ \ \ \ \ \ \ \ \ \ \ \ \ \ \ \ \ \ \ \ \ \ \ \ \ \ else\ {\isacharparenleft}{\kern0pt}if\ i\ {\isasymin}\ J\ then\ {\isacharparenleft}{\kern0pt}hml{\isacharunderscore}{\kern0pt}pos\ {\isacharparenleft}{\kern0pt}SOME\ {\isasymalpha}{\isachardot}{\kern0pt}\ {\isacharparenleft}{\kern0pt}{\isasymforall}s{\isachardot}{\kern0pt}\ {\isacharparenleft}{\kern0pt}s\ {\isasymTurnstile}\ {\isasymPhi}\ i{\isacharparenright}{\kern0pt}\ {\isasymlongleftrightarrow}\ {\isacharparenleft}{\kern0pt}s\ {\isasymTurnstile}\ {\isacharparenleft}{\kern0pt}hml{\isacharunderscore}{\kern0pt}pos\ {\isasymalpha}\ TT{\isacharparenright}{\kern0pt}{\isacharparenright}{\kern0pt}{\isacharparenright}{\kern0pt}{\isacharparenright}{\kern0pt}\ TT{\isacharparenright}{\kern0pt}\ \ \isanewline
\ \ \ \ \ \ \ \ \ \ \ \ \ \ \ \ \ \ \ \ \ \ \ \ \ \ \ \ \ \ \ \ \ \ \ \ \ \ \ \ \ \ \ \ \ \ \ \ else\ undefined{\isacharparenright}{\kern0pt}{\isacharparenright}{\kern0pt}{\isacharparenright}{\kern0pt}{\isachardoublequoteclose}\isanewline
\ \ \ \ \ \ \ \ \ \ \isacommand{have}\isamarkupfalse%
\ {\isachardoublequoteopen}{\isasymphi}\ {\isasymnotin}\ {\isasymPhi}\ {\isacharbackquote}{\kern0pt}\ J{\isachardoublequoteclose}\isanewline
\ \ \ \ \ \ \ \ \ \ \ \ \isacommand{using}\isamarkupfalse%
\ True\ {\isacartoucheopen}{\isasymphi}\ {\isasymin}\ {\isasymPhi}\ {\isacharbackquote}{\kern0pt}\ I{\isacartoucheclose}\ \isanewline
\ \ \ \ \ \ \ \ \ \ \ \ \isacommand{by}\isamarkupfalse%
\ blast\isanewline
\ \ \ \ \ \ \ \ \ \ \isacommand{hence}\isamarkupfalse%
\ {\isachardoublequoteopen}{\isasymforall}i\ {\isasymin}\ J{\isachardot}{\kern0pt}\ {\isasymPsi}\ i\ {\isacharequal}{\kern0pt}\ {\isacharparenleft}{\kern0pt}hml{\isacharunderscore}{\kern0pt}pos\ {\isacharparenleft}{\kern0pt}SOME\ {\isasymalpha}{\isachardot}{\kern0pt}\ {\isacharparenleft}{\kern0pt}{\isasymforall}s{\isachardot}{\kern0pt}\ {\isacharparenleft}{\kern0pt}s\ {\isasymTurnstile}\ {\isasymPhi}\ i{\isacharparenright}{\kern0pt}\ {\isasymlongleftrightarrow}\ {\isacharparenleft}{\kern0pt}s\ {\isasymTurnstile}\ {\isacharparenleft}{\kern0pt}hml{\isacharunderscore}{\kern0pt}pos\ {\isasymalpha}\ TT{\isacharparenright}{\kern0pt}{\isacharparenright}{\kern0pt}{\isacharparenright}{\kern0pt}{\isacharparenright}{\kern0pt}\ TT{\isacharparenright}{\kern0pt}{\isachardoublequoteclose}\isanewline
\ \ \ \ \ \ \ \ \ \ \ \ \isacommand{using}\isamarkupfalse%
\ True\ {\isachardoublequoteopen}{\isadigit{1}}{\isachardoublequoteclose}\ {\isasymPsi}{\isacharunderscore}{\kern0pt}def\ \ \isanewline
\ \ \ \ \ \ \ \ \ \ \ \ \isacommand{using}\isamarkupfalse%
\ {\isacartoucheopen}{\isasymPhi}\ i{\isacharunderscore}{\kern0pt}{\isasymphi}\ {\isacharequal}{\kern0pt}\ {\isasymphi}{\isacartoucheclose}\ \isacommand{by}\isamarkupfalse%
\ auto\isanewline
\ \ \ \ \ \ \ \ \ \ \isacommand{have}\isamarkupfalse%
\ {\isachardoublequoteopen}{\isasymforall}j\ {\isasymin}\ J{\isachardot}{\kern0pt}\ {\isasymexists}{\isasymalpha}{\isachardot}{\kern0pt}\ {\isacharparenleft}{\kern0pt}{\isasymforall}s{\isachardot}{\kern0pt}\ {\isacharparenleft}{\kern0pt}s\ {\isasymTurnstile}\ {\isasymPhi}\ j{\isacharparenright}{\kern0pt}\ {\isasymlongleftrightarrow}\ {\isacharparenleft}{\kern0pt}s\ {\isasymTurnstile}\ {\isacharparenleft}{\kern0pt}hml{\isacharunderscore}{\kern0pt}pos\ {\isasymalpha}\ TT{\isacharparenright}{\kern0pt}{\isacharparenright}{\kern0pt}{\isacharparenright}{\kern0pt}{\isachardoublequoteclose}\isanewline
\ \ \ \ \ \ \ \ \ \ \ \ \isacommand{using}\isamarkupfalse%
\ neg{\isacharunderscore}{\kern0pt}case\ stacked{\isacharunderscore}{\kern0pt}pos{\isacharunderscore}{\kern0pt}rewriting\ {\isachardoublequoteopen}{\isadigit{1}}{\isachardoublequoteclose}\ \isacommand{by}\isamarkupfalse%
\ blast\isanewline
\ \ \ \ \ \ \ \ \ \ \isacommand{hence}\isamarkupfalse%
\ {\isachardoublequoteopen}{\isasymforall}j\ {\isasymin}\ J{\isachardot}{\kern0pt}\ {\isacharparenleft}{\kern0pt}{\isasymforall}s{\isachardot}{\kern0pt}\ {\isacharparenleft}{\kern0pt}s\ {\isasymTurnstile}\ {\isasymPhi}\ j{\isacharparenright}{\kern0pt}\ {\isasymlongleftrightarrow}\ {\isacharparenleft}{\kern0pt}s\ {\isasymTurnstile}\ {\isasymPsi}\ j{\isacharparenright}{\kern0pt}{\isacharparenright}{\kern0pt}{\isachardoublequoteclose}\isanewline
\ \ \ \ \ \ \ \ \ \ \ \ \isacommand{using}\isamarkupfalse%
\ True\ {\isacartoucheopen}{\isasymforall}i\ {\isasymin}\ J{\isachardot}{\kern0pt}\ {\isasymPsi}\ i\ {\isacharequal}{\kern0pt}\ {\isacharparenleft}{\kern0pt}hml{\isacharunderscore}{\kern0pt}pos\ {\isacharparenleft}{\kern0pt}SOME\ {\isasymalpha}{\isachardot}{\kern0pt}\ {\isacharparenleft}{\kern0pt}{\isasymforall}s{\isachardot}{\kern0pt}\ {\isacharparenleft}{\kern0pt}s\ {\isasymTurnstile}\ {\isasymPhi}\ i{\isacharparenright}{\kern0pt}\ {\isasymlongleftrightarrow}\ {\isacharparenleft}{\kern0pt}s\ {\isasymTurnstile}\ {\isacharparenleft}{\kern0pt}hml{\isacharunderscore}{\kern0pt}pos\ {\isasymalpha}\ TT{\isacharparenright}{\kern0pt}{\isacharparenright}{\kern0pt}{\isacharparenright}{\kern0pt}{\isacharparenright}{\kern0pt}\ TT{\isacharparenright}{\kern0pt}{\isacartoucheclose}\isanewline
\ \ \ \ \ \ \ \ \ \ \ \ \isacommand{by}\isamarkupfalse%
\ {\isacharparenleft}{\kern0pt}smt\ {\isacharparenleft}{\kern0pt}verit{\isacharcomma}{\kern0pt}\ ccfv{\isacharunderscore}{\kern0pt}threshold{\isacharparenright}{\kern0pt}\ tfl{\isacharunderscore}{\kern0pt}some{\isacharparenright}{\kern0pt}\isanewline
\ \ \ \ \ \ \ \ \ \ \isacommand{have}\isamarkupfalse%
\ {\isachardoublequoteopen}{\isasymforall}i\ {\isasymin}\ I{\isachardot}{\kern0pt}\ {\isasymPhi}\ i\ {\isasymnoteq}\ {\isasymphi}\ {\isasymlongrightarrow}\ {\isacharparenleft}{\kern0pt}{\isasymforall}s{\isachardot}{\kern0pt}\ s\ {\isasymTurnstile}\ {\isasymPhi}\ i{\isacharparenright}{\kern0pt}{\isachardoublequoteclose}\isanewline
\ \ \ \ \ \ \ \ \ \ \ \ \isacommand{by}\isamarkupfalse%
\ {\isacharparenleft}{\kern0pt}simp\ add{\isacharcolon}{\kern0pt}\ {\isacartoucheopen}{\isasymforall}y{\isasymin}{\isasymPhi}\ {\isacharbackquote}{\kern0pt}\ I{\isachardot}{\kern0pt}\ {\isasymphi}\ {\isasymnoteq}\ y\ {\isasymlongrightarrow}\ {\isacharparenleft}{\kern0pt}{\isasymforall}s{\isachardot}{\kern0pt}\ s\ {\isasymTurnstile}\ y{\isacharparenright}{\kern0pt}{\isacartoucheclose}{\isacharparenright}{\kern0pt}\ \isanewline
\ \ \ \ \ \ \ \ \ \ \isacommand{have}\isamarkupfalse%
\ {\isachardoublequoteopen}{\isasymPsi}\ {\isacharbackquote}{\kern0pt}\ {\isacharbraceleft}{\kern0pt}i{\isacharunderscore}{\kern0pt}{\isasymphi}{\isacharbraceright}{\kern0pt}\ {\isacharequal}{\kern0pt}\ {\isacharbraceleft}{\kern0pt}{\isasympsi}{\isacharbraceright}{\kern0pt}{\isachardoublequoteclose}\ \isanewline
\ \ \ \ \ \ \ \ \ \ \ \ \isacommand{using}\isamarkupfalse%
\ {\isasymPsi}{\isacharunderscore}{\kern0pt}def\ {\isacartoucheopen}{\isasymphi}{\isasymin}{\isasymPhi}\ {\isacharbackquote}{\kern0pt}\ I{\isacartoucheclose}\ {\isacartoucheopen}{\isasymphi}\ {\isasymnotin}\ {\isasymPhi}\ {\isacharbackquote}{\kern0pt}\ J{\isacartoucheclose}\ {\isacartoucheopen}I\ {\isasyminter}\ J\ {\isacharequal}{\kern0pt}\ {\isacharbraceleft}{\kern0pt}{\isacharbraceright}{\kern0pt}{\isacartoucheclose}\isanewline
\ \ \ \ \ \ \ \ \ \ \ \ \isacommand{by}\isamarkupfalse%
\ simp\isanewline
\ \ \ \ \ \ \ \ \ \ \isacommand{hence}\isamarkupfalse%
\ {\isachardoublequoteopen}{\isasymforall}s{\isachardot}{\kern0pt}\ {\isacharparenleft}{\kern0pt}{\isasymforall}i\ {\isasymin}\ {\isacharbraceleft}{\kern0pt}i{\isacharunderscore}{\kern0pt}{\isasymphi}{\isacharbraceright}{\kern0pt}{\isachardot}{\kern0pt}\ s\ {\isasymTurnstile}\ {\isacharparenleft}{\kern0pt}{\isasymPsi}\ i{\isacharparenright}{\kern0pt}{\isacharparenright}{\kern0pt}\ {\isasymlongleftrightarrow}\ s\ {\isasymTurnstile}\ {\isasympsi}{\isachardoublequoteclose}\ \isanewline
\ \ \ \ \ \ \ \ \ \ \ \ \isacommand{using}\isamarkupfalse%
\ {\isacartoucheopen}{\isasymphi}\ {\isasymin}\ {\isasymPhi}\ {\isacharbackquote}{\kern0pt}\ I{\isacartoucheclose}\ {\isasymPsi}{\isacharunderscore}{\kern0pt}def\ {\isacartoucheopen}{\isasymPhi}\ i{\isacharunderscore}{\kern0pt}{\isasymphi}\ {\isacharequal}{\kern0pt}\ {\isasymphi}{\isacartoucheclose}\ \isacommand{by}\isamarkupfalse%
\ auto\isanewline
\ \ \ \ \ \ \ \ \ \ \isacommand{hence}\isamarkupfalse%
\ {\isachardoublequoteopen}{\isasymforall}s{\isachardot}{\kern0pt}\ s\ {\isasymTurnstile}\ {\isacharparenleft}{\kern0pt}hml{\isacharunderscore}{\kern0pt}conj\ I\ J\ {\isasymPhi}{\isacharparenright}{\kern0pt}\ {\isasymlongleftrightarrow}\ s\ {\isasymTurnstile}\ {\isacharparenleft}{\kern0pt}hml{\isacharunderscore}{\kern0pt}conj\ {\isacharbraceleft}{\kern0pt}i{\isacharunderscore}{\kern0pt}{\isasymphi}{\isacharbraceright}{\kern0pt}\ J\ {\isasymPsi}{\isacharparenright}{\kern0pt}{\isachardoublequoteclose}\isanewline
\ \ \ \ \ \ \ \ \ \ \ \ \isacommand{using}\isamarkupfalse%
\ {\isacartoucheopen}{\isasymforall}j\ {\isasymin}\ J{\isachardot}{\kern0pt}\ {\isacharparenleft}{\kern0pt}{\isasymforall}s{\isachardot}{\kern0pt}\ {\isacharparenleft}{\kern0pt}s\ {\isasymTurnstile}\ {\isasymPhi}\ j{\isacharparenright}{\kern0pt}\ {\isasymlongleftrightarrow}\ {\isacharparenleft}{\kern0pt}s\ {\isasymTurnstile}\ {\isasymPsi}\ j{\isacharparenright}{\kern0pt}{\isacharparenright}{\kern0pt}{\isacartoucheclose}\isanewline
\ \ \ \ \ \ \ \ \ \ \ \ \isacommand{by}\isamarkupfalse%
\ {\isacharparenleft}{\kern0pt}simp\ add{\isacharcolon}{\kern0pt}\ {\isacartoucheopen}{\isasymforall}s{\isachardot}{\kern0pt}\ {\isacharparenleft}{\kern0pt}{\isasymforall}i{\isasymin}I{\isachardot}{\kern0pt}\ s\ {\isasymTurnstile}\ {\isasymPhi}\ i{\isacharparenright}{\kern0pt}\ {\isacharequal}{\kern0pt}\ {\isacharparenleft}{\kern0pt}s\ {\isasymTurnstile}\ {\isasymphi}{\isacharparenright}{\kern0pt}{\isacartoucheclose}\ {\isacartoucheopen}{\isasymforall}s{\isachardot}{\kern0pt}\ {\isacharparenleft}{\kern0pt}s\ {\isasymTurnstile}\ {\isasymphi}{\isacharparenright}{\kern0pt}\ {\isacharequal}{\kern0pt}\ {\isacharparenleft}{\kern0pt}s\ {\isasymTurnstile}\ {\isasympsi}{\isacharparenright}{\kern0pt}{\isacartoucheclose}{\isacharparenright}{\kern0pt}\isanewline
\ \ \ \ \ \ \ \ \ \ \isacommand{have}\isamarkupfalse%
\ {\isachardoublequoteopen}{\isasymforall}j\ {\isasymin}\ {\isasymPsi}\ {\isacharbackquote}{\kern0pt}\ J{\isachardot}{\kern0pt}\ {\isasymexists}{\isasymalpha}{\isachardot}{\kern0pt}\ j\ {\isacharequal}{\kern0pt}\ {\isacharparenleft}{\kern0pt}hml{\isacharunderscore}{\kern0pt}pos\ {\isasymalpha}\ TT{\isacharparenright}{\kern0pt}{\isachardoublequoteclose}\ \isanewline
\ \ \ \ \ \ \ \ \ \ \ \ \isacommand{using}\isamarkupfalse%
\ {\isacartoucheopen}{\isasymforall}i{\isasymin}J{\isachardot}{\kern0pt}\ {\isasymPsi}\ i\ {\isacharequal}{\kern0pt}\ hml{\isacharunderscore}{\kern0pt}pos\ {\isacharparenleft}{\kern0pt}SOME\ {\isasymalpha}{\isachardot}{\kern0pt}\ {\isasymforall}s{\isachardot}{\kern0pt}\ {\isacharparenleft}{\kern0pt}s\ {\isasymTurnstile}\ {\isasymPhi}\ i{\isacharparenright}{\kern0pt}\ {\isacharequal}{\kern0pt}\ {\isacharparenleft}{\kern0pt}s\ {\isasymTurnstile}\ hml{\isacharunderscore}{\kern0pt}pos\ {\isasymalpha}\ TT{\isacharparenright}{\kern0pt}{\isacharparenright}{\kern0pt}\ TT{\isacartoucheclose}\ \isacommand{by}\isamarkupfalse%
\ blast\isanewline
\ \ \ \ \ \ \ \ \ \ \isacommand{have}\isamarkupfalse%
\ {\isachardoublequoteopen}{\isacharparenleft}{\kern0pt}{\isasymexists}i\ {\isasymin}\ {\isasymPsi}\ {\isacharbackquote}{\kern0pt}\ {\isacharbraceleft}{\kern0pt}i{\isacharunderscore}{\kern0pt}{\isasymphi}{\isacharbraceright}{\kern0pt}{\isachardot}{\kern0pt}\ {\isasymPsi}\ {\isacharbackquote}{\kern0pt}\ {\isacharbraceleft}{\kern0pt}i{\isacharunderscore}{\kern0pt}{\isasymphi}{\isacharbraceright}{\kern0pt}\ {\isacharequal}{\kern0pt}\ {\isacharbraceleft}{\kern0pt}i{\isacharbraceright}{\kern0pt}\ {\isasymand}\ hml{\isacharunderscore}{\kern0pt}failure{\isacharunderscore}{\kern0pt}trace\ i{\isacharparenright}{\kern0pt}{\isachardoublequoteclose}\isanewline
\ \ \ \ \ \ \ \ \ \ \ \ \isacommand{using}\isamarkupfalse%
\ {\isasymPsi}{\isacharunderscore}{\kern0pt}def\isanewline
\ \ \ \ \ \ \ \ \ \ \ \ \isacommand{using}\isamarkupfalse%
\ {\isacartoucheopen}{\isasymPsi}\ {\isacharbackquote}{\kern0pt}\ {\isacharbraceleft}{\kern0pt}i{\isacharunderscore}{\kern0pt}{\isasymphi}{\isacharbraceright}{\kern0pt}\ {\isacharequal}{\kern0pt}\ {\isacharbraceleft}{\kern0pt}{\isasympsi}{\isacharbraceright}{\kern0pt}{\isacartoucheclose}\ {\isacartoucheopen}hml{\isacharunderscore}{\kern0pt}failure{\isacharunderscore}{\kern0pt}trace\ {\isasympsi}{\isacartoucheclose}\ \isacommand{by}\isamarkupfalse%
\ auto\isanewline
\ \ \ \ \ \ \ \ \ \ \isacommand{have}\isamarkupfalse%
\ {\isachardoublequoteopen}hml{\isacharunderscore}{\kern0pt}failure{\isacharunderscore}{\kern0pt}trace\ {\isacharparenleft}{\kern0pt}hml{\isacharunderscore}{\kern0pt}conj\ {\isacharbraceleft}{\kern0pt}i{\isacharunderscore}{\kern0pt}{\isasymphi}{\isacharbraceright}{\kern0pt}\ J\ {\isasymPsi}{\isacharparenright}{\kern0pt}{\isachardoublequoteclose}\isanewline
\ \ \ \ \ \ \ \ \ \ \ \ \isacommand{by}\isamarkupfalse%
\ {\isacharparenleft}{\kern0pt}meson\ {\isacartoucheopen}{\isasymexists}i{\isasymin}{\isasymPsi}\ {\isacharbackquote}{\kern0pt}\ {\isacharbraceleft}{\kern0pt}i{\isacharunderscore}{\kern0pt}{\isasymphi}{\isacharbraceright}{\kern0pt}{\isachardot}{\kern0pt}\ {\isasymPsi}\ {\isacharbackquote}{\kern0pt}\ {\isacharbraceleft}{\kern0pt}i{\isacharunderscore}{\kern0pt}{\isasymphi}{\isacharbraceright}{\kern0pt}\ {\isacharequal}{\kern0pt}\ {\isacharbraceleft}{\kern0pt}i{\isacharbraceright}{\kern0pt}\ {\isasymand}\ hml{\isacharunderscore}{\kern0pt}failure{\isacharunderscore}{\kern0pt}trace\ i{\isacartoucheclose}\ {\isacartoucheopen}{\isasymforall}j{\isasymin}{\isasymPsi}\ {\isacharbackquote}{\kern0pt}\ J{\isachardot}{\kern0pt}\ {\isasymexists}{\isasymalpha}{\isachardot}{\kern0pt}\ j\ {\isacharequal}{\kern0pt}\ hml{\isacharunderscore}{\kern0pt}pos\ {\isasymalpha}\ TT{\isacartoucheclose}\ hml{\isacharunderscore}{\kern0pt}failure{\isacharunderscore}{\kern0pt}trace{\isachardot}{\kern0pt}intros{\isacharparenleft}{\kern0pt}{\isadigit{3}}{\isacharparenright}{\kern0pt}{\isacharparenright}{\kern0pt}\isanewline
\ \ \ \ \ \ \ \ \ \ \isacommand{then}\isamarkupfalse%
\ \isacommand{show}\isamarkupfalse%
\ {\isacharquery}{\kern0pt}thesis\ \isacommand{using}\isamarkupfalse%
\ {\isacartoucheopen}{\isasymforall}s{\isachardot}{\kern0pt}\ s\ {\isasymTurnstile}\ {\isacharparenleft}{\kern0pt}hml{\isacharunderscore}{\kern0pt}conj\ I\ J\ {\isasymPhi}{\isacharparenright}{\kern0pt}\ {\isasymlongleftrightarrow}\ s\ {\isasymTurnstile}\ {\isacharparenleft}{\kern0pt}hml{\isacharunderscore}{\kern0pt}conj\ {\isacharbraceleft}{\kern0pt}i{\isacharunderscore}{\kern0pt}{\isasymphi}{\isacharbraceright}{\kern0pt}\ J\ {\isasymPsi}{\isacharparenright}{\kern0pt}{\isacartoucheclose}\isanewline
\ \ \ \ \ \ \ \ \ \ \ \ \isacommand{by}\isamarkupfalse%
\ blast\isanewline
\ \ \ \ \ \ \ \ \isacommand{next}\isamarkupfalse%
\isanewline
\ \ \ \ \ \ \ \ \ \ \isacommand{case}\isamarkupfalse%
\ False\isanewline
\ \ \ \ \ \ \ \ \ \ \isacommand{hence}\isamarkupfalse%
\ {\isachardoublequoteopen}{\isasymforall}s{\isachardot}{\kern0pt}\ {\isasymnot}{\isacharparenleft}{\kern0pt}s\ {\isasymTurnstile}\ hml{\isacharunderscore}{\kern0pt}conj\ I\ J\ {\isasymPhi}{\isacharparenright}{\kern0pt}{\isachardoublequoteclose}\ \isanewline
\ \ \ \ \ \ \ \ \ \ \ \ \isacommand{by}\isamarkupfalse%
\ fastforce\isanewline
\ \ \ \ \ \ \ \ \ \ \isacommand{then}\isamarkupfalse%
\ \isacommand{obtain}\isamarkupfalse%
\ {\isasymphi}\ i{\isacharunderscore}{\kern0pt}{\isasymphi}\ \isakeyword{where}\ {\isachardoublequoteopen}{\isasymphi}\ {\isasymin}\ {\isasymPhi}\ {\isacharbackquote}{\kern0pt}\ I\ {\isasyminter}\ {\isasymPhi}\ {\isacharbackquote}{\kern0pt}\ J{\isachardoublequoteclose}\ {\isachardoublequoteopen}{\isasymPhi}\ i{\isacharunderscore}{\kern0pt}{\isasymphi}\ {\isacharequal}{\kern0pt}\ {\isasymphi}{\isachardoublequoteclose}\ \isanewline
\ \ \ \ \ \ \ \ \ \ \ \ \isacommand{using}\isamarkupfalse%
\ False\ \isacommand{by}\isamarkupfalse%
\ blast\isanewline
\ \ \ \ \ \ \ \ \ \ \isacommand{define}\isamarkupfalse%
\ {\isasymPsi}\ \isakeyword{where}\ {\isachardoublequoteopen}{\isasymPsi}\ {\isasymequiv}\ {\isacharparenleft}{\kern0pt}{\isasymlambda}i{\isachardot}{\kern0pt}\ {\isacharparenleft}{\kern0pt}if\ i\ {\isacharequal}{\kern0pt}\ i{\isacharunderscore}{\kern0pt}{\isasymphi}\ then\ TT{\isacharcolon}{\kern0pt}{\isacharcolon}{\kern0pt}{\isacharparenleft}{\kern0pt}{\isacharprime}{\kern0pt}a{\isacharcomma}{\kern0pt}\ {\isacharprime}{\kern0pt}s{\isacharparenright}{\kern0pt}hml\ else\ undefined{\isacharparenright}{\kern0pt}{\isacharparenright}{\kern0pt}{\isachardoublequoteclose}\isanewline
\ \ \ \ \ \ \ \ \ \ \isacommand{hence}\isamarkupfalse%
\ {\isachardoublequoteopen}{\isasymforall}s{\isachardot}{\kern0pt}\ {\isasymnot}{\isacharparenleft}{\kern0pt}s\ {\isasymTurnstile}\ hml{\isacharunderscore}{\kern0pt}conj\ {\isacharbraceleft}{\kern0pt}{\isacharbraceright}{\kern0pt}\ {\isacharbraceleft}{\kern0pt}i{\isacharunderscore}{\kern0pt}{\isasymphi}{\isacharbraceright}{\kern0pt}\ {\isasymPsi}{\isacharparenright}{\kern0pt}{\isachardoublequoteclose}\ \isanewline
\ \ \ \ \ \ \ \ \ \ \ \ \isacommand{by}\isamarkupfalse%
\ simp\isanewline
\ \ \ \ \ \ \ \ \ \ \isacommand{have}\isamarkupfalse%
\ {\isachardoublequoteopen}hml{\isacharunderscore}{\kern0pt}failure{\isacharunderscore}{\kern0pt}trace\ {\isacharparenleft}{\kern0pt}hml{\isacharunderscore}{\kern0pt}conj\ {\isacharbraceleft}{\kern0pt}{\isacharbraceright}{\kern0pt}\ {\isacharbraceleft}{\kern0pt}i{\isacharunderscore}{\kern0pt}{\isasymphi}{\isacharbraceright}{\kern0pt}\ {\isasymPsi}{\isacharparenright}{\kern0pt}{\isachardoublequoteclose}\ \isanewline
\ \ \ \ \ \ \ \ \ \ \ \ \isacommand{by}\isamarkupfalse%
\ {\isacharparenleft}{\kern0pt}simp\ add{\isacharcolon}{\kern0pt}\ {\isasymPsi}{\isacharunderscore}{\kern0pt}def\ hml{\isacharunderscore}{\kern0pt}failure{\isacharunderscore}{\kern0pt}trace{\isachardot}{\kern0pt}intros{\isacharparenleft}{\kern0pt}{\isadigit{3}}{\isacharparenright}{\kern0pt}{\isacharparenright}{\kern0pt}\isanewline
\ \ \ \ \ \ \ \ \ \ \isacommand{then}\isamarkupfalse%
\ \isacommand{show}\isamarkupfalse%
\ {\isacharquery}{\kern0pt}thesis\ \isacommand{using}\isamarkupfalse%
\ {\isacartoucheopen}{\isasymforall}s{\isachardot}{\kern0pt}\ {\isasymnot}{\isacharparenleft}{\kern0pt}s\ {\isasymTurnstile}\ hml{\isacharunderscore}{\kern0pt}conj\ {\isacharbraceleft}{\kern0pt}{\isacharbraceright}{\kern0pt}\ {\isacharbraceleft}{\kern0pt}i{\isacharunderscore}{\kern0pt}{\isasymphi}{\isacharbraceright}{\kern0pt}\ {\isasymPsi}{\isacharparenright}{\kern0pt}{\isacartoucheclose}\ {\isacartoucheopen}{\isasymforall}s{\isachardot}{\kern0pt}\ {\isasymnot}{\isacharparenleft}{\kern0pt}s\ {\isasymTurnstile}\ hml{\isacharunderscore}{\kern0pt}conj\ I\ J\ {\isasymPhi}{\isacharparenright}{\kern0pt}{\isacartoucheclose}\ \isanewline
\ \ \ \ \ \ \ \ \ \ \ \ \isacommand{by}\isamarkupfalse%
\ blast\isanewline
\ \ \ \ \ \ \ \ \isacommand{qed}\isamarkupfalse%
\isanewline
\ \ \ \ \ \ \isacommand{next}\isamarkupfalse%
\isanewline
\ \ \ \ \ \ \ \ \isacommand{case}\isamarkupfalse%
\ {\isadigit{2}}\isanewline
\ \ \ \ \ \ \ \ \isacommand{hence}\isamarkupfalse%
\ {\isachardoublequoteopen}{\isasymforall}s{\isachardot}{\kern0pt}\ {\isasymnot}s\ {\isasymTurnstile}\ {\isacharparenleft}{\kern0pt}hml{\isacharunderscore}{\kern0pt}conj\ I\ J\ {\isasymPhi}{\isacharparenright}{\kern0pt}{\isachardoublequoteclose}\ \isanewline
\ \ \ \ \ \ \ \ \ \ \isacommand{unfolding}\isamarkupfalse%
\ HML{\isacharunderscore}{\kern0pt}true{\isacharunderscore}{\kern0pt}def\ \isanewline
\ \ \ \ \ \ \ \ \ \ \isacommand{by}\isamarkupfalse%
\ fastforce\isanewline
\ \ \ \ \ \ \ \ \isacommand{obtain}\isamarkupfalse%
\ y\ \isakeyword{where}\ {\isachardoublequoteopen}y\ {\isasymin}\ {\isasymPhi}{\isacharbackquote}{\kern0pt}J{\isachardoublequoteclose}\ {\isachardoublequoteopen}{\isacharparenleft}{\kern0pt}{\isasymforall}s{\isachardot}{\kern0pt}\ s\ {\isasymTurnstile}\ y{\isacharparenright}{\kern0pt}{\isachardoublequoteclose}\ \isanewline
\ \ \ \ \ \ \ \ \ \ \isacommand{using}\isamarkupfalse%
\ {\isachardoublequoteopen}{\isadigit{2}}{\isachardoublequoteclose}\isanewline
\ \ \ \ \ \ \ \ \ \ \isacommand{unfolding}\isamarkupfalse%
\ HML{\isacharunderscore}{\kern0pt}true{\isacharunderscore}{\kern0pt}def\ \isanewline
\ \ \ \ \ \ \ \ \ \ \isacommand{by}\isamarkupfalse%
\ blast\isanewline
\ \ \ \ \ \ \isacommand{define}\isamarkupfalse%
\ {\isasymPsi}\ \isakeyword{where}\ {\isasymPsi}{\isacharunderscore}{\kern0pt}def{\isacharcolon}{\kern0pt}\ {\isachardoublequoteopen}{\isasymPsi}\ {\isacharequal}{\kern0pt}\ {\isacharparenleft}{\kern0pt}{\isasymlambda}i{\isachardot}{\kern0pt}\ {\isacharparenleft}{\kern0pt}if\ i\ {\isasymin}\ J\ then\ {\isacharparenleft}{\kern0pt}TT{\isacharcolon}{\kern0pt}{\isacharcolon}{\kern0pt}{\isacharparenleft}{\kern0pt}{\isacharprime}{\kern0pt}a{\isacharcomma}{\kern0pt}\ {\isacharprime}{\kern0pt}s{\isacharparenright}{\kern0pt}hml{\isacharparenright}{\kern0pt}\ else\ undefined{\isacharparenright}{\kern0pt}{\isacharparenright}{\kern0pt}{\isachardoublequoteclose}\isanewline
\ \ \ \ \ \ \isacommand{hence}\isamarkupfalse%
\ {\isachardoublequoteopen}{\isasymforall}s{\isachardot}{\kern0pt}\ {\isasymnot}s\ {\isasymTurnstile}\ {\isacharparenleft}{\kern0pt}hml{\isacharunderscore}{\kern0pt}conj\ {\isacharbraceleft}{\kern0pt}{\isacharbraceright}{\kern0pt}\ J\ {\isasymPsi}{\isacharparenright}{\kern0pt}{\isachardoublequoteclose}\ \isanewline
\ \ \ \ \ \ \ \ \isacommand{using}\isamarkupfalse%
\ {\isacartoucheopen}y\ {\isasymin}\ {\isasymPhi}\ {\isacharbackquote}{\kern0pt}\ J{\isacartoucheclose}\ \isacommand{by}\isamarkupfalse%
\ auto\isanewline
\ \ \ \ \ \ \isacommand{have}\isamarkupfalse%
\ {\isachardoublequoteopen}{\isasymPsi}\ {\isacharbackquote}{\kern0pt}\ {\isacharbraceleft}{\kern0pt}{\isacharbraceright}{\kern0pt}\ {\isacharequal}{\kern0pt}\ {\isacharbraceleft}{\kern0pt}{\isacharbraceright}{\kern0pt}{\isachardoublequoteclose}\ {\isachardoublequoteopen}{\isasymforall}j\ {\isasymin}\ {\isasymPsi}\ {\isacharbackquote}{\kern0pt}\ J{\isachardot}{\kern0pt}\ j\ {\isacharequal}{\kern0pt}\ TT{\isachardoublequoteclose}\ \isanewline
\ \ \ \ \ \ \ \ \ \isacommand{apply}\isamarkupfalse%
\ blast\isanewline
\ \ \ \ \ \ \ \ \isacommand{unfolding}\isamarkupfalse%
\ {\isasymPsi}{\isacharunderscore}{\kern0pt}def\ \isanewline
\ \ \ \ \ \ \ \ \isacommand{using}\isamarkupfalse%
\ {\isacartoucheopen}y\ {\isasymin}\ {\isasymPhi}{\isacharbackquote}{\kern0pt}J{\isacartoucheclose}\ \isanewline
\ \ \ \ \ \ \ \ \isacommand{by}\isamarkupfalse%
\ simp\isanewline
\ \ \ \ \ \ \isacommand{hence}\isamarkupfalse%
\ {\isachardoublequoteopen}hml{\isacharunderscore}{\kern0pt}failure{\isacharunderscore}{\kern0pt}trace\ {\isacharparenleft}{\kern0pt}hml{\isacharunderscore}{\kern0pt}conj\ {\isacharbraceleft}{\kern0pt}{\isacharbraceright}{\kern0pt}\ J\ {\isasymPsi}{\isacharparenright}{\kern0pt}{\isachardoublequoteclose}\ \isanewline
\ \ \ \ \ \ \ \ \isacommand{by}\isamarkupfalse%
\ {\isacharparenleft}{\kern0pt}meson\ hml{\isacharunderscore}{\kern0pt}failure{\isacharunderscore}{\kern0pt}trace{\isachardot}{\kern0pt}intros{\isacharparenleft}{\kern0pt}{\isadigit{3}}{\isacharparenright}{\kern0pt}{\isacharparenright}{\kern0pt}\isanewline
\ \ \ \ \ \ \isacommand{then}\isamarkupfalse%
\ \isacommand{show}\isamarkupfalse%
\ {\isacharquery}{\kern0pt}thesis\ \isacommand{using}\isamarkupfalse%
\ {\isacartoucheopen}{\isasymforall}s{\isachardot}{\kern0pt}\ {\isasymnot}s\ {\isasymTurnstile}\ {\isacharparenleft}{\kern0pt}hml{\isacharunderscore}{\kern0pt}conj\ I\ J\ {\isasymPhi}{\isacharparenright}{\kern0pt}{\isacartoucheclose}\ \isanewline
\ \ \ \ \ \ \ \ \isacommand{using}\isamarkupfalse%
\ {\isacartoucheopen}{\isasymforall}s{\isachardot}{\kern0pt}\ {\isasymnot}\ s\ {\isasymTurnstile}\ hml{\isacharunderscore}{\kern0pt}conj\ {\isacharbraceleft}{\kern0pt}{\isacharbraceright}{\kern0pt}\ J\ {\isasymPsi}{\isacartoucheclose}\ \isacommand{by}\isamarkupfalse%
\ blast\isanewline
\ \ \ \ \isacommand{qed}\isamarkupfalse%
\isanewline
\ \ \isacommand{qed}\isamarkupfalse%
\isanewline
\isacommand{next}\isamarkupfalse%
\isanewline
\ \ \ \ \isacommand{assume}\isamarkupfalse%
\ {\isachardoublequoteopen}{\isasymforall}y{\isasymin}{\isasymPhi}\ {\isacharbackquote}{\kern0pt}\ I{\isachardot}{\kern0pt}\ nested{\isacharunderscore}{\kern0pt}empty{\isacharunderscore}{\kern0pt}conj\ y{\isachardoublequoteclose}\isanewline
\ \ \ \ \isacommand{then}\isamarkupfalse%
\ \isacommand{show}\isamarkupfalse%
\ {\isacharquery}{\kern0pt}thesis\ \isacommand{proof}\isamarkupfalse%
{\isacharparenleft}{\kern0pt}cases\ {\isachardoublequoteopen}{\isasymexists}i{\isasymin}I{\isachardot}{\kern0pt}\ {\isacharparenleft}{\kern0pt}{\isasymforall}s{\isachardot}{\kern0pt}\ {\isacharparenleft}{\kern0pt}{\isasymnot}{\isacharparenleft}{\kern0pt}s\ {\isasymTurnstile}\ {\isacharparenleft}{\kern0pt}{\isasymPhi}\ i{\isacharparenright}{\kern0pt}{\isacharparenright}{\kern0pt}{\isacharparenright}{\kern0pt}{\isacharparenright}{\kern0pt}{\isachardoublequoteclose}{\isacharparenright}{\kern0pt}\isanewline
\ \ \ \ \ \ \isacommand{case}\isamarkupfalse%
\ True\isanewline
\ \ \ \ \ \ \isacommand{hence}\isamarkupfalse%
\ {\isachardoublequoteopen}{\isasymforall}s{\isachardot}{\kern0pt}\ {\isacharparenleft}{\kern0pt}{\isasymnot}{\isacharparenleft}{\kern0pt}s\ {\isasymTurnstile}\ hml{\isacharunderscore}{\kern0pt}conj\ I\ J\ {\isasymPhi}{\isacharparenright}{\kern0pt}{\isacharparenright}{\kern0pt}{\isachardoublequoteclose}\ \isanewline
\ \ \ \ \ \ \ \ \isacommand{using}\isamarkupfalse%
\ hml{\isacharunderscore}{\kern0pt}sem{\isacharunderscore}{\kern0pt}conj\ \isacommand{by}\isamarkupfalse%
\ blast\isanewline
\ \ \ \ \ \ \isacommand{define}\isamarkupfalse%
\ {\isasymPsi}\ \isakeyword{where}\ {\isachardoublequoteopen}{\isasymPsi}\ {\isasymequiv}\ {\isacharparenleft}{\kern0pt}{\isasymlambda}i{\isachardot}{\kern0pt}\ {\isacharparenleft}{\kern0pt}if\ i\ {\isasymin}\ I\ then\ TT{\isacharcolon}{\kern0pt}{\isacharcolon}{\kern0pt}\ {\isacharparenleft}{\kern0pt}{\isacharprime}{\kern0pt}a{\isacharcomma}{\kern0pt}\ {\isacharprime}{\kern0pt}s{\isacharparenright}{\kern0pt}\ hml\ else\ undefined{\isacharparenright}{\kern0pt}{\isacharparenright}{\kern0pt}{\isachardoublequoteclose}\isanewline
\ \ \ \ \ \ \isacommand{have}\isamarkupfalse%
\ {\isachardoublequoteopen}{\isasymforall}j\ {\isasymin}\ {\isasymPsi}\ {\isacharbackquote}{\kern0pt}\ I{\isachardot}{\kern0pt}\ j\ {\isacharequal}{\kern0pt}\ TT{\isachardoublequoteclose}\ \isanewline
\ \ \ \ \ \ \ \ \isacommand{using}\isamarkupfalse%
\ {\isasymPsi}{\isacharunderscore}{\kern0pt}def\ \isacommand{by}\isamarkupfalse%
\ force\isanewline
\ \ \ \ \ \ \isacommand{hence}\isamarkupfalse%
\ {\isachardoublequoteopen}hml{\isacharunderscore}{\kern0pt}failure{\isacharunderscore}{\kern0pt}trace\ {\isacharparenleft}{\kern0pt}hml{\isacharunderscore}{\kern0pt}conj\ {\isacharbraceleft}{\kern0pt}{\isacharbraceright}{\kern0pt}\ I\ {\isasymPsi}{\isacharparenright}{\kern0pt}{\isachardoublequoteclose}\ \isacommand{using}\isamarkupfalse%
\ hml{\isacharunderscore}{\kern0pt}failure{\isacharunderscore}{\kern0pt}trace{\isachardot}{\kern0pt}intros{\isacharparenleft}{\kern0pt}{\isadigit{3}}{\isacharparenright}{\kern0pt}\isanewline
\ \ \ \ \ \ \ \ \isacommand{by}\isamarkupfalse%
\ {\isacharparenleft}{\kern0pt}metis\ image{\isacharunderscore}{\kern0pt}empty{\isacharparenright}{\kern0pt}\isanewline
\ \ \ \ \ \ \isacommand{have}\isamarkupfalse%
\ {\isachardoublequoteopen}{\isasymforall}s{\isachardot}{\kern0pt}\ {\isacharparenleft}{\kern0pt}{\isasymnot}{\isacharparenleft}{\kern0pt}s\ {\isasymTurnstile}\ hml{\isacharunderscore}{\kern0pt}conj\ {\isacharbraceleft}{\kern0pt}{\isacharbraceright}{\kern0pt}\ I\ {\isasymPsi}{\isacharparenright}{\kern0pt}{\isacharparenright}{\kern0pt}{\isachardoublequoteclose}\ \isanewline
\ \ \ \ \ \ \ \ \isacommand{using}\isamarkupfalse%
\ True\ {\isasymPsi}{\isacharunderscore}{\kern0pt}def\ \isacommand{by}\isamarkupfalse%
\ force\isanewline
\ \ \ \ \ \ \isacommand{then}\isamarkupfalse%
\ \isacommand{show}\isamarkupfalse%
\ {\isacharquery}{\kern0pt}thesis\ \isacommand{using}\isamarkupfalse%
\ {\isacartoucheopen}hml{\isacharunderscore}{\kern0pt}failure{\isacharunderscore}{\kern0pt}trace\ {\isacharparenleft}{\kern0pt}hml{\isacharunderscore}{\kern0pt}conj\ {\isacharbraceleft}{\kern0pt}{\isacharbraceright}{\kern0pt}\ I\ {\isasymPsi}{\isacharparenright}{\kern0pt}{\isacartoucheclose}\ {\isacartoucheopen}{\isasymforall}s{\isachardot}{\kern0pt}\ {\isacharparenleft}{\kern0pt}{\isasymnot}{\isacharparenleft}{\kern0pt}s\ {\isasymTurnstile}\ hml{\isacharunderscore}{\kern0pt}conj\ I\ J\ {\isasymPhi}{\isacharparenright}{\kern0pt}{\isacharparenright}{\kern0pt}{\isacartoucheclose}\isanewline
\ \ \ \ \ \ \ \ \isacommand{by}\isamarkupfalse%
\ blast\isanewline
\ \ \ \ \isacommand{next}\isamarkupfalse%
\isanewline
\ \ \ \ \ \ \isacommand{case}\isamarkupfalse%
\ False\isanewline
\ \ \ \ \ \ \isacommand{consider}\isamarkupfalse%
\ {\isachardoublequoteopen}{\isasymforall}y\ {\isasymin}\ {\isasymPhi}{\isacharbackquote}{\kern0pt}J{\isachardot}{\kern0pt}\ {\isasymexists}{\isasymalpha}{\isachardot}{\kern0pt}\ {\isacharparenleft}{\kern0pt}{\isasymforall}s{\isachardot}{\kern0pt}\ {\isacharparenleft}{\kern0pt}s\ {\isasymTurnstile}\ y{\isacharparenright}{\kern0pt}\ {\isasymlongleftrightarrow}\ {\isacharparenleft}{\kern0pt}s\ {\isasymTurnstile}\ {\isacharparenleft}{\kern0pt}hml{\isacharunderscore}{\kern0pt}pos\ {\isasymalpha}\ TT{\isacharparenright}{\kern0pt}{\isacharparenright}{\kern0pt}{\isacharparenright}{\kern0pt}{\isachardoublequoteclose}\ {\isacharbar}{\kern0pt}\ {\isachardoublequoteopen}{\isacharparenleft}{\kern0pt}{\isasymexists}y{\isasymin}{\isasymPhi}\ {\isacharbackquote}{\kern0pt}\ J{\isachardot}{\kern0pt}\ HML{\isacharunderscore}{\kern0pt}true\ y{\isacharparenright}{\kern0pt}{\isachardoublequoteclose}\isanewline
\ \ \ \ \ \ \ \ \isacommand{using}\isamarkupfalse%
\ neg{\isacharunderscore}{\kern0pt}case\ stacked{\isacharunderscore}{\kern0pt}pos{\isacharunderscore}{\kern0pt}rewriting\ \isacommand{by}\isamarkupfalse%
\ blast\isanewline
\ \ \ \ \ \ \isacommand{then}\isamarkupfalse%
\ \isacommand{show}\isamarkupfalse%
\ {\isacharquery}{\kern0pt}thesis\ \isacommand{proof}\isamarkupfalse%
{\isacharparenleft}{\kern0pt}cases{\isacharparenright}{\kern0pt}\isanewline
\ \ \ \ \ \ \ \ \isacommand{case}\isamarkupfalse%
\ {\isadigit{1}}\isanewline
\ \ \ \ \ \ \ \ \isacommand{from}\isamarkupfalse%
\ False\ \isacommand{have}\isamarkupfalse%
\ {\isachardoublequoteopen}{\isasymforall}i\ {\isasymin}\ I{\isachardot}{\kern0pt}\ {\isacharparenleft}{\kern0pt}{\isasymforall}s{\isachardot}{\kern0pt}\ {\isacharparenleft}{\kern0pt}s\ {\isasymTurnstile}\ {\isacharparenleft}{\kern0pt}{\isasymPhi}\ i{\isacharparenright}{\kern0pt}{\isacharparenright}{\kern0pt}{\isacharparenright}{\kern0pt}{\isachardoublequoteclose}\isanewline
\ \ \ \ \ \ \ \ \isacommand{using}\isamarkupfalse%
\ nested{\isacharunderscore}{\kern0pt}empty{\isacharunderscore}{\kern0pt}conj{\isacharunderscore}{\kern0pt}TT{\isacharunderscore}{\kern0pt}or{\isacharunderscore}{\kern0pt}FF\ {\isacartoucheopen}{\isasymforall}y{\isasymin}{\isasymPhi}\ {\isacharbackquote}{\kern0pt}\ I{\isachardot}{\kern0pt}\ nested{\isacharunderscore}{\kern0pt}empty{\isacharunderscore}{\kern0pt}conj\ y{\isacartoucheclose}\ \isacommand{by}\isamarkupfalse%
\ blast\ \isanewline
\ \ \ \ \ \ \ \ \isacommand{have}\isamarkupfalse%
\ {\isachardoublequoteopen}{\isasymforall}i\ {\isasymin}\ {\isacharbraceleft}{\kern0pt}{\isacharbraceright}{\kern0pt}{\isachardot}{\kern0pt}\ {\isacharparenleft}{\kern0pt}{\isasymforall}s{\isachardot}{\kern0pt}\ {\isacharparenleft}{\kern0pt}s\ {\isasymTurnstile}\ {\isacharparenleft}{\kern0pt}{\isasymPhi}\ i{\isacharparenright}{\kern0pt}{\isacharparenright}{\kern0pt}{\isacharparenright}{\kern0pt}{\isachardoublequoteclose}\ \isacommand{by}\isamarkupfalse%
\ blast\isanewline
\ \ \ \ \ \ \ \ \isacommand{define}\isamarkupfalse%
\ {\isasymPsi}\ \isakeyword{where}\ {\isachardoublequoteopen}{\isasymPsi}\ {\isasymequiv}\ {\isacharparenleft}{\kern0pt}{\isasymlambda}i{\isachardot}{\kern0pt}\ {\isacharparenleft}{\kern0pt}if\ i\ {\isasymin}\ J\ \isanewline
\ \ \ \ \ \ \ \ \ \ \ \ \ \ then\ {\isacharparenleft}{\kern0pt}hml{\isacharunderscore}{\kern0pt}pos\ {\isacharparenleft}{\kern0pt}SOME\ {\isasymalpha}{\isachardot}{\kern0pt}\ {\isacharparenleft}{\kern0pt}{\isasymforall}s{\isachardot}{\kern0pt}\ {\isacharparenleft}{\kern0pt}s\ {\isasymTurnstile}\ {\isacharparenleft}{\kern0pt}{\isasymPhi}\ i{\isacharparenright}{\kern0pt}{\isacharparenright}{\kern0pt}\ {\isasymlongleftrightarrow}\ {\isacharparenleft}{\kern0pt}s\ {\isasymTurnstile}\ {\isacharparenleft}{\kern0pt}hml{\isacharunderscore}{\kern0pt}pos\ {\isasymalpha}\ TT{\isacharparenright}{\kern0pt}{\isacharparenright}{\kern0pt}{\isacharparenright}{\kern0pt}{\isacharparenright}{\kern0pt}\ TT{\isacharcolon}{\kern0pt}{\isacharcolon}{\kern0pt}\ {\isacharparenleft}{\kern0pt}{\isacharprime}{\kern0pt}a{\isacharcomma}{\kern0pt}\ {\isacharprime}{\kern0pt}s{\isacharparenright}{\kern0pt}\ hml{\isacharparenright}{\kern0pt}\ \isanewline
\ \ \ \ \ \ \ \ \ \ \ \ \ \ else\ undefined{\isacharparenright}{\kern0pt}{\isacharparenright}{\kern0pt}{\isachardoublequoteclose}\isanewline
\ \ \ \ \ \ \isacommand{have}\isamarkupfalse%
\ {\isachardoublequoteopen}{\isasymforall}j{\isasymin}{\isasymPhi}\ {\isacharbackquote}{\kern0pt}\ J{\isachardot}{\kern0pt}\ {\isacharparenleft}{\kern0pt}{\isasymexists}{\isasymalpha}{\isachardot}{\kern0pt}\ {\isacharparenleft}{\kern0pt}{\isasymforall}s{\isachardot}{\kern0pt}\ {\isacharparenleft}{\kern0pt}s\ {\isasymTurnstile}\ j{\isacharparenright}{\kern0pt}\ {\isasymlongleftrightarrow}\ {\isacharparenleft}{\kern0pt}s\ {\isasymTurnstile}\ {\isacharparenleft}{\kern0pt}hml{\isacharunderscore}{\kern0pt}pos\ {\isasymalpha}\ TT{\isacharparenright}{\kern0pt}{\isacharparenright}{\kern0pt}{\isacharparenright}{\kern0pt}{\isacharparenright}{\kern0pt}{\isachardoublequoteclose}\ \isanewline
\ \ \ \ \ \ \ \ \isacommand{using}\isamarkupfalse%
\ stacked{\isacharunderscore}{\kern0pt}pos{\isacharunderscore}{\kern0pt}rewriting\ neg{\isacharunderscore}{\kern0pt}case\ {\isadigit{1}}\ \isacommand{by}\isamarkupfalse%
\ blast\isanewline
\ \ \ \ \ \ \isacommand{hence}\isamarkupfalse%
\ {\isachardoublequoteopen}{\isasymforall}j\ {\isasymin}\ J{\isachardot}{\kern0pt}\ {\isacharparenleft}{\kern0pt}{\isasymexists}{\isasymalpha}{\isachardot}{\kern0pt}\ {\isacharparenleft}{\kern0pt}{\isasymforall}s{\isachardot}{\kern0pt}\ {\isacharparenleft}{\kern0pt}s\ {\isasymTurnstile}\ {\isasymPhi}\ j{\isacharparenright}{\kern0pt}\ {\isasymlongleftrightarrow}\ {\isacharparenleft}{\kern0pt}s\ {\isasymTurnstile}\ {\isacharparenleft}{\kern0pt}hml{\isacharunderscore}{\kern0pt}pos\ {\isasymalpha}\ TT{\isacharparenright}{\kern0pt}{\isacharparenright}{\kern0pt}{\isacharparenright}{\kern0pt}{\isacharparenright}{\kern0pt}{\isachardoublequoteclose}\ \isanewline
\ \ \ \ \ \ \ \ \isacommand{by}\isamarkupfalse%
\ blast\isanewline
\ \ \ \ \ \ \isacommand{hence}\isamarkupfalse%
\ {\isachardoublequoteopen}{\isasymforall}j\ {\isasymin}\ J{\isachardot}{\kern0pt}\ {\isasymexists}{\isasymalpha}\ {\isachardot}{\kern0pt}{\isasymPsi}\ j\ {\isacharequal}{\kern0pt}\ {\isacharparenleft}{\kern0pt}hml{\isacharunderscore}{\kern0pt}pos\ {\isasymalpha}\ TT{\isacharparenright}{\kern0pt}\ {\isasymand}\ {\isacharparenleft}{\kern0pt}{\isasymforall}s{\isachardot}{\kern0pt}\ {\isacharparenleft}{\kern0pt}s\ {\isasymTurnstile}\ {\isacharparenleft}{\kern0pt}{\isasymPhi}\ j{\isacharparenright}{\kern0pt}{\isacharparenright}{\kern0pt}\ {\isasymlongleftrightarrow}\ {\isacharparenleft}{\kern0pt}s\ {\isasymTurnstile}\ {\isacharparenleft}{\kern0pt}hml{\isacharunderscore}{\kern0pt}pos\ {\isasymalpha}\ TT{\isacharparenright}{\kern0pt}{\isacharparenright}{\kern0pt}{\isacharparenright}{\kern0pt}{\isachardoublequoteclose}\isanewline
\ \ \ \ \ \ \isacommand{proof}\isamarkupfalse%
{\isacharparenleft}{\kern0pt}safe{\isacharparenright}{\kern0pt}\isanewline
\ \ \ \ \ \ \ \ \isacommand{fix}\isamarkupfalse%
\ j\isanewline
\ \ \ \ \ \ \ \ \isacommand{assume}\isamarkupfalse%
\ {\isachardoublequoteopen}{\isasymforall}j{\isasymin}J{\isachardot}{\kern0pt}\ {\isasymexists}{\isasymalpha}{\isachardot}{\kern0pt}\ {\isasymforall}s{\isachardot}{\kern0pt}\ {\isacharparenleft}{\kern0pt}s\ {\isasymTurnstile}\ {\isasymPhi}\ j{\isacharparenright}{\kern0pt}\ {\isacharequal}{\kern0pt}\ {\isacharparenleft}{\kern0pt}s\ {\isasymTurnstile}\ hml{\isacharunderscore}{\kern0pt}pos\ {\isasymalpha}\ TT{\isacharparenright}{\kern0pt}{\isachardoublequoteclose}\ {\isachardoublequoteopen}j\ {\isasymin}\ J{\isachardoublequoteclose}\isanewline
\ \ \ \ \ \ \ \ \isacommand{then}\isamarkupfalse%
\ \isacommand{obtain}\isamarkupfalse%
\ {\isasymalpha}\ \isakeyword{where}\ {\isachardoublequoteopen}{\isasymPsi}\ j\ {\isacharequal}{\kern0pt}\ hml{\isacharunderscore}{\kern0pt}pos\ {\isasymalpha}\ TT{\isachardoublequoteclose}\ \isanewline
\ \ \ \ \ \ \ \ \ \ \isacommand{using}\isamarkupfalse%
\ {\isasymPsi}{\isacharunderscore}{\kern0pt}def\ \isacommand{by}\isamarkupfalse%
\ fastforce\isanewline
\ \ \ \ \ \ \ \ \isacommand{hence}\isamarkupfalse%
\ {\isachardoublequoteopen}{\isacharparenleft}{\kern0pt}{\isasymforall}s{\isachardot}{\kern0pt}\ {\isacharparenleft}{\kern0pt}s\ {\isasymTurnstile}\ {\isacharparenleft}{\kern0pt}{\isasymPhi}\ j{\isacharparenright}{\kern0pt}{\isacharparenright}{\kern0pt}\ {\isasymlongleftrightarrow}\ {\isacharparenleft}{\kern0pt}s\ {\isasymTurnstile}\ {\isacharparenleft}{\kern0pt}hml{\isacharunderscore}{\kern0pt}pos\ {\isasymalpha}\ TT{\isacharparenright}{\kern0pt}{\isacharparenright}{\kern0pt}{\isacharparenright}{\kern0pt}{\isachardoublequoteclose}\ \isacommand{unfolding}\isamarkupfalse%
\ {\isasymPsi}{\isacharunderscore}{\kern0pt}def\ \isacommand{using}\isamarkupfalse%
\ {\isacartoucheopen}j\ {\isasymin}\ J{\isacartoucheclose}\ \isanewline
\ \ \ \ \ \ \ \ \ \ \isacommand{by}\isamarkupfalse%
\ {\isacharparenleft}{\kern0pt}smt\ {\isacharparenleft}{\kern0pt}verit{\isacharcomma}{\kern0pt}\ best{\isacharparenright}{\kern0pt}\ {\isacartoucheopen}{\isasymforall}j{\isasymin}J{\isachardot}{\kern0pt}\ {\isasymexists}{\isasymalpha}{\isachardot}{\kern0pt}\ {\isasymforall}s{\isachardot}{\kern0pt}\ {\isacharparenleft}{\kern0pt}s\ {\isasymTurnstile}\ {\isasymPhi}\ j{\isacharparenright}{\kern0pt}\ {\isacharequal}{\kern0pt}\ {\isacharparenleft}{\kern0pt}s\ {\isasymTurnstile}\ hml{\isacharunderscore}{\kern0pt}pos\ {\isasymalpha}\ TT{\isacharparenright}{\kern0pt}{\isacartoucheclose}\ tfl{\isacharunderscore}{\kern0pt}some{\isacharparenright}{\kern0pt}\isanewline
\ \ \ \ \ \ \ \ \isacommand{then}\isamarkupfalse%
\ \isacommand{show}\isamarkupfalse%
\ {\isachardoublequoteopen}{\isasymexists}{\isasymalpha}{\isachardot}{\kern0pt}\ {\isasymPsi}\ j\ {\isacharequal}{\kern0pt}\ hml{\isacharunderscore}{\kern0pt}pos\ {\isasymalpha}\ TT\ {\isasymand}\ {\isacharparenleft}{\kern0pt}{\isasymforall}s{\isachardot}{\kern0pt}\ {\isacharparenleft}{\kern0pt}s\ {\isasymTurnstile}\ {\isasymPhi}\ j{\isacharparenright}{\kern0pt}\ {\isacharequal}{\kern0pt}\ {\isacharparenleft}{\kern0pt}s\ {\isasymTurnstile}\ hml{\isacharunderscore}{\kern0pt}pos\ {\isasymalpha}\ TT{\isacharparenright}{\kern0pt}{\isacharparenright}{\kern0pt}{\isachardoublequoteclose}\isanewline
\ \ \ \ \ \ \ \ \ \ \isacommand{using}\isamarkupfalse%
\ {\isacartoucheopen}{\isasymPsi}\ j\ {\isacharequal}{\kern0pt}\ hml{\isacharunderscore}{\kern0pt}pos\ {\isasymalpha}\ TT{\isacartoucheclose}\ \isacommand{by}\isamarkupfalse%
\ blast\isanewline
\ \ \ \ \ \ \isacommand{qed}\isamarkupfalse%
\isanewline
\ \ \ \ \ \ \isacommand{hence}\isamarkupfalse%
\ {\isachardoublequoteopen}{\isasymforall}j\ {\isasymin}\ J{\isachardot}{\kern0pt}\ {\isasymforall}s{\isachardot}{\kern0pt}\ s\ {\isasymTurnstile}\ {\isacharparenleft}{\kern0pt}{\isasymPsi}\ j{\isacharparenright}{\kern0pt}\ {\isasymlongleftrightarrow}\ s\ {\isasymTurnstile}\ {\isacharparenleft}{\kern0pt}{\isasymPhi}\ j{\isacharparenright}{\kern0pt}{\isachardoublequoteclose}\ \isacommand{using}\isamarkupfalse%
\ {\isasymPsi}{\isacharunderscore}{\kern0pt}def\ \isanewline
\ \ \ \ \ \ \ \ \isacommand{by}\isamarkupfalse%
\ metis\isanewline
\ \ \ \ \ \ \isacommand{hence}\isamarkupfalse%
\ {\isachardoublequoteopen}{\isasymforall}j\ {\isasymin}\ J{\isachardot}{\kern0pt}\ {\isasymforall}s{\isachardot}{\kern0pt}\ {\isasymnot}s\ {\isasymTurnstile}\ {\isacharparenleft}{\kern0pt}{\isasymPsi}\ j{\isacharparenright}{\kern0pt}\ {\isasymlongleftrightarrow}\ {\isasymnot}s\ {\isasymTurnstile}\ {\isacharparenleft}{\kern0pt}{\isasymPhi}\ j{\isacharparenright}{\kern0pt}{\isachardoublequoteclose}\ \isacommand{by}\isamarkupfalse%
\ blast\isanewline
\ \ \ \ \ \ \isacommand{hence}\isamarkupfalse%
\ {\isachardoublequoteopen}{\isacharparenleft}{\kern0pt}{\isasymforall}s{\isachardot}{\kern0pt}\ {\isacharparenleft}{\kern0pt}s\ {\isasymTurnstile}\ hml{\isacharunderscore}{\kern0pt}conj\ I\ J\ {\isasymPhi}{\isacharparenright}{\kern0pt}\ {\isacharequal}{\kern0pt}\ {\isacharparenleft}{\kern0pt}s\ {\isasymTurnstile}\ hml{\isacharunderscore}{\kern0pt}conj\ {\isacharbraceleft}{\kern0pt}{\isacharbraceright}{\kern0pt}\ J\ {\isasymPsi}{\isacharparenright}{\kern0pt}{\isacharparenright}{\kern0pt}{\isachardoublequoteclose}\isanewline
\ \ \ \ \ \ \ \ \isacommand{using}\isamarkupfalse%
\ {\isacartoucheopen}{\isasymforall}i\ {\isasymin}\ I{\isachardot}{\kern0pt}\ {\isacharparenleft}{\kern0pt}{\isasymforall}s{\isachardot}{\kern0pt}\ {\isacharparenleft}{\kern0pt}s\ {\isasymTurnstile}\ {\isacharparenleft}{\kern0pt}{\isasymPhi}\ i{\isacharparenright}{\kern0pt}{\isacharparenright}{\kern0pt}{\isacharparenright}{\kern0pt}{\isacartoucheclose}\ {\isacartoucheopen}{\isasymforall}i\ {\isasymin}\ {\isacharbraceleft}{\kern0pt}{\isacharbraceright}{\kern0pt}{\isachardot}{\kern0pt}\ {\isacharparenleft}{\kern0pt}{\isasymforall}s{\isachardot}{\kern0pt}\ {\isacharparenleft}{\kern0pt}s\ {\isasymTurnstile}\ {\isacharparenleft}{\kern0pt}{\isasymPhi}\ i{\isacharparenright}{\kern0pt}{\isacharparenright}{\kern0pt}{\isacharparenright}{\kern0pt}{\isacartoucheclose}\ \isanewline
\ \ \ \ \ \ \ \ \isacommand{by}\isamarkupfalse%
\ simp\isanewline
\ \ \ \ \ \ \isacommand{have}\isamarkupfalse%
\ {\isachardoublequoteopen}{\isasymforall}j\ {\isasymin}\ {\isasymPsi}\ {\isacharbackquote}{\kern0pt}\ J{\isachardot}{\kern0pt}\ {\isasymexists}{\isasymalpha}{\isachardot}{\kern0pt}\ j\ {\isacharequal}{\kern0pt}\ {\isacharparenleft}{\kern0pt}hml{\isacharunderscore}{\kern0pt}pos\ {\isasymalpha}\ TT{\isacharparenright}{\kern0pt}{\isachardoublequoteclose}\ \isanewline
\ \ \ \ \ \ \ \ \isacommand{by}\isamarkupfalse%
\ {\isacharparenleft}{\kern0pt}simp\ add{\isacharcolon}{\kern0pt}\ {\isasymPsi}{\isacharunderscore}{\kern0pt}def{\isacharparenright}{\kern0pt}\isanewline
\ \ \ \ \ \ \isacommand{hence}\isamarkupfalse%
\ {\isachardoublequoteopen}hml{\isacharunderscore}{\kern0pt}failure{\isacharunderscore}{\kern0pt}trace\ {\isacharparenleft}{\kern0pt}hml{\isacharunderscore}{\kern0pt}conj\ {\isacharbraceleft}{\kern0pt}{\isacharbraceright}{\kern0pt}\ J\ {\isasymPsi}{\isacharparenright}{\kern0pt}{\isachardoublequoteclose}\ \isanewline
\ \ \ \ \ \ \ \ \isacommand{by}\isamarkupfalse%
\ {\isacharparenleft}{\kern0pt}simp\ add{\isacharcolon}{\kern0pt}\ hml{\isacharunderscore}{\kern0pt}failure{\isacharunderscore}{\kern0pt}trace{\isachardot}{\kern0pt}intros{\isacharparenleft}{\kern0pt}{\isadigit{3}}{\isacharparenright}{\kern0pt}{\isacharparenright}{\kern0pt}\isanewline
\ \ \ \ \ \ \isacommand{then}\isamarkupfalse%
\ \isacommand{show}\isamarkupfalse%
\ {\isacharquery}{\kern0pt}thesis\isanewline
\ \ \ \ \ \ \ \ \isacommand{using}\isamarkupfalse%
\ {\isacartoucheopen}{\isasymforall}s{\isachardot}{\kern0pt}\ {\isacharparenleft}{\kern0pt}s\ {\isasymTurnstile}\ hml{\isacharunderscore}{\kern0pt}conj\ I\ J\ {\isasymPhi}{\isacharparenright}{\kern0pt}\ {\isacharequal}{\kern0pt}\ {\isacharparenleft}{\kern0pt}s\ {\isasymTurnstile}\ hml{\isacharunderscore}{\kern0pt}conj\ {\isacharbraceleft}{\kern0pt}{\isacharbraceright}{\kern0pt}\ J\ {\isasymPsi}{\isacharparenright}{\kern0pt}{\isacartoucheclose}\ \isacommand{by}\isamarkupfalse%
\ blast\isanewline
\ \ \ \ \ \ \isacommand{next}\isamarkupfalse%
\isanewline
\ \ \ \ \ \ \ \ \isacommand{case}\isamarkupfalse%
\ {\isadigit{2}}\isanewline
\ \ \ \ \ \ \ \ \isacommand{hence}\isamarkupfalse%
\ {\isachardoublequoteopen}{\isasymforall}s{\isachardot}{\kern0pt}\ {\isasymnot}s\ {\isasymTurnstile}\ {\isacharparenleft}{\kern0pt}hml{\isacharunderscore}{\kern0pt}conj\ I\ J\ {\isasymPhi}{\isacharparenright}{\kern0pt}{\isachardoublequoteclose}\ \isanewline
\ \ \ \ \ \ \ \ \ \ \isacommand{unfolding}\isamarkupfalse%
\ HML{\isacharunderscore}{\kern0pt}true{\isacharunderscore}{\kern0pt}def\ \isanewline
\ \ \ \ \ \ \ \ \ \ \isacommand{by}\isamarkupfalse%
\ fastforce\isanewline
\ \ \ \ \ \ \ \ \isacommand{obtain}\isamarkupfalse%
\ y\ \isakeyword{where}\ {\isachardoublequoteopen}y\ {\isasymin}\ {\isasymPhi}{\isacharbackquote}{\kern0pt}J{\isachardoublequoteclose}\ {\isachardoublequoteopen}{\isacharparenleft}{\kern0pt}{\isasymforall}s{\isachardot}{\kern0pt}\ s\ {\isasymTurnstile}\ y{\isacharparenright}{\kern0pt}{\isachardoublequoteclose}\ \isanewline
\ \ \ \ \ \ \ \ \ \ \isacommand{using}\isamarkupfalse%
\ {\isachardoublequoteopen}{\isadigit{2}}{\isachardoublequoteclose}\isanewline
\ \ \ \ \ \ \ \ \ \ \isacommand{unfolding}\isamarkupfalse%
\ HML{\isacharunderscore}{\kern0pt}true{\isacharunderscore}{\kern0pt}def\ \isanewline
\ \ \ \ \ \ \ \ \ \ \isacommand{by}\isamarkupfalse%
\ blast\isanewline
\ \ \ \ \ \ \ \ \isacommand{define}\isamarkupfalse%
\ {\isasymPsi}\ \isakeyword{where}\ {\isasymPsi}{\isacharunderscore}{\kern0pt}def{\isacharcolon}{\kern0pt}\ {\isachardoublequoteopen}{\isasymPsi}\ {\isacharequal}{\kern0pt}\ {\isacharparenleft}{\kern0pt}{\isasymlambda}i{\isachardot}{\kern0pt}\ {\isacharparenleft}{\kern0pt}if\ i\ {\isasymin}\ J\ then\ {\isacharparenleft}{\kern0pt}TT{\isacharcolon}{\kern0pt}{\isacharcolon}{\kern0pt}{\isacharparenleft}{\kern0pt}{\isacharprime}{\kern0pt}a{\isacharcomma}{\kern0pt}\ {\isacharprime}{\kern0pt}s{\isacharparenright}{\kern0pt}hml{\isacharparenright}{\kern0pt}\ else\ undefined{\isacharparenright}{\kern0pt}{\isacharparenright}{\kern0pt}{\isachardoublequoteclose}\isanewline
\ \ \ \ \ \ \ \ \isacommand{hence}\isamarkupfalse%
\ {\isachardoublequoteopen}{\isasymforall}s{\isachardot}{\kern0pt}\ {\isasymnot}s\ {\isasymTurnstile}\ {\isacharparenleft}{\kern0pt}hml{\isacharunderscore}{\kern0pt}conj\ {\isacharbraceleft}{\kern0pt}{\isacharbraceright}{\kern0pt}\ J\ {\isasymPsi}{\isacharparenright}{\kern0pt}{\isachardoublequoteclose}\ \isanewline
\ \ \ \ \ \ \ \ \ \ \isacommand{using}\isamarkupfalse%
\ {\isacartoucheopen}y\ {\isasymin}\ {\isasymPhi}\ {\isacharbackquote}{\kern0pt}\ J{\isacartoucheclose}\ \isacommand{by}\isamarkupfalse%
\ auto\isanewline
\ \ \ \ \ \ \ \ \isacommand{have}\isamarkupfalse%
\ {\isachardoublequoteopen}{\isasymPsi}\ {\isacharbackquote}{\kern0pt}\ {\isacharbraceleft}{\kern0pt}{\isacharbraceright}{\kern0pt}\ {\isacharequal}{\kern0pt}\ {\isacharbraceleft}{\kern0pt}{\isacharbraceright}{\kern0pt}{\isachardoublequoteclose}\ {\isachardoublequoteopen}{\isasymforall}j\ {\isasymin}\ {\isasymPsi}\ {\isacharbackquote}{\kern0pt}\ J{\isachardot}{\kern0pt}\ j\ {\isacharequal}{\kern0pt}\ TT{\isachardoublequoteclose}\ \isanewline
\ \ \ \ \ \ \ \ \ \ \ \isacommand{apply}\isamarkupfalse%
\ blast\isanewline
\ \ \ \ \ \ \ \ \ \ \isacommand{unfolding}\isamarkupfalse%
\ {\isasymPsi}{\isacharunderscore}{\kern0pt}def\ \isanewline
\ \ \ \ \ \ \ \ \ \ \isacommand{using}\isamarkupfalse%
\ {\isacartoucheopen}y\ {\isasymin}\ {\isasymPhi}{\isacharbackquote}{\kern0pt}J{\isacartoucheclose}\ \isanewline
\ \ \ \ \ \ \ \ \ \ \isacommand{by}\isamarkupfalse%
\ simp\isanewline
\ \ \ \ \ \ \ \ \isacommand{hence}\isamarkupfalse%
\ {\isachardoublequoteopen}hml{\isacharunderscore}{\kern0pt}failure{\isacharunderscore}{\kern0pt}trace\ {\isacharparenleft}{\kern0pt}hml{\isacharunderscore}{\kern0pt}conj\ {\isacharbraceleft}{\kern0pt}{\isacharbraceright}{\kern0pt}\ J\ {\isasymPsi}{\isacharparenright}{\kern0pt}{\isachardoublequoteclose}\ \isanewline
\ \ \ \ \ \ \ \ \ \ \isacommand{by}\isamarkupfalse%
\ {\isacharparenleft}{\kern0pt}meson\ hml{\isacharunderscore}{\kern0pt}failure{\isacharunderscore}{\kern0pt}trace{\isachardot}{\kern0pt}intros{\isacharparenleft}{\kern0pt}{\isadigit{3}}{\isacharparenright}{\kern0pt}{\isacharparenright}{\kern0pt}\isanewline
\ \ \ \ \ \ \ \ \isacommand{then}\isamarkupfalse%
\ \isacommand{show}\isamarkupfalse%
\ {\isacharquery}{\kern0pt}thesis\ \isacommand{using}\isamarkupfalse%
\ {\isacartoucheopen}{\isasymforall}s{\isachardot}{\kern0pt}\ {\isasymnot}s\ {\isasymTurnstile}\ {\isacharparenleft}{\kern0pt}hml{\isacharunderscore}{\kern0pt}conj\ I\ J\ {\isasymPhi}{\isacharparenright}{\kern0pt}{\isacartoucheclose}\ \isanewline
\ \ \ \ \ \ \ \ \ \ \isacommand{using}\isamarkupfalse%
\ {\isacartoucheopen}{\isasymforall}s{\isachardot}{\kern0pt}\ {\isasymnot}\ s\ {\isasymTurnstile}\ hml{\isacharunderscore}{\kern0pt}conj\ {\isacharbraceleft}{\kern0pt}{\isacharbraceright}{\kern0pt}\ J\ {\isasymPsi}{\isacartoucheclose}\ \isacommand{by}\isamarkupfalse%
\ blast\isanewline
\ \ \ \ \ \ \isacommand{qed}\isamarkupfalse%
\isanewline
\ \ \ \ \isacommand{qed}\isamarkupfalse%
\isanewline
\ \ \isacommand{qed}\isamarkupfalse%
\ \isanewline
\isacommand{qed}\isamarkupfalse%
%
\endisatagproof
{\isafoldproof}%
%
\isadelimproof
\isanewline
%
\endisadelimproof
\isanewline
\isacommand{end}\isamarkupfalse%
\isanewline
%
\isadelimtheory
%
\endisadelimtheory
%
\isatagtheory
\isacommand{end}\isamarkupfalse%
%
\endisatagtheory
{\isafoldtheory}%
%
\isadelimtheory
%
\endisadelimtheory
%
\end{isabellebody}%
\endinput
%:%file=~/Documents/Isabelle_HOL/HML_definitions.thy%:%
%:%10=1%:%
%:%11=1%:%
%:%12=2%:%
%:%13=3%:%
%:%18=3%:%
%:%21=4%:%
%:%22=5%:%
%:%23=5%:%
%:%24=6%:%
%:%25=7%:%
%:%26=8%:%
%:%27=9%:%
%:%28=9%:%
%:%29=10%:%
%:%30=11%:%
%:%31=12%:%
%:%32=13%:%
%:%33=14%:%
%:%34=15%:%
%:%35=16%:%
%:%36=16%:%
%:%37=17%:%
%:%38=18%:%
%:%39=19%:%
%:%40=20%:%
%:%41=21%:%
%:%42=22%:%
%:%43=23%:%
%:%44=23%:%
%:%45=24%:%
%:%46=25%:%
%:%47=26%:%
%:%48=27%:%
%:%49=28%:%
%:%50=29%:%
%:%51=30%:%
%:%52=30%:%
%:%53=31%:%
%:%54=32%:%
%:%55=33%:%
%:%56=34%:%
%:%57=35%:%
%:%58=36%:%
%:%59=37%:%
%:%60=37%:%
%:%61=38%:%
%:%62=39%:%
%:%63=40%:%
%:%64=40%:%
%:%65=41%:%
%:%66=42%:%
%:%67=43%:%
%:%68=44%:%
%:%69=45%:%
%:%70=46%:%
%:%71=47%:%
%:%72=47%:%
%:%73=48%:%
%:%74=49%:%
%:%75=50%:%
%:%76=51%:%
%:%77=52%:%
%:%78=53%:%
%:%79=54%:%
%:%80=55%:%
%:%81=56%:%
%:%82=56%:%
%:%83=57%:%
%:%84=58%:%
%:%85=59%:%
%:%86=60%:%
%:%87=61%:%
%:%88=62%:%
%:%89=63%:%
%:%90=63%:%
%:%91=64%:%
%:%92=65%:%
%:%93=66%:%
%:%94=67%:%
%:%95=68%:%
%:%96=69%:%
%:%97=70%:%
%:%98=70%:%
%:%99=71%:%
%:%100=72%:%
%:%101=72%:%
%:%102=73%:%
%:%103=74%:%
%:%104=75%:%
%:%107=76%:%
%:%111=76%:%
%:%112=76%:%
%:%113=77%:%
%:%114=77%:%
%:%115=78%:%
%:%116=78%:%
%:%117=79%:%
%:%118=79%:%
%:%119=79%:%
%:%120=79%:%
%:%121=79%:%
%:%122=80%:%
%:%123=80%:%
%:%124=81%:%
%:%125=81%:%
%:%126=82%:%
%:%127=82%:%
%:%128=82%:%
%:%129=82%:%
%:%130=83%:%
%:%131=83%:%
%:%132=84%:%
%:%133=84%:%
%:%134=85%:%
%:%135=85%:%
%:%136=85%:%
%:%137=86%:%
%:%138=86%:%
%:%139=87%:%
%:%140=87%:%
%:%141=87%:%
%:%142=88%:%
%:%143=88%:%
%:%144=88%:%
%:%145=89%:%
%:%151=89%:%
%:%154=90%:%
%:%155=91%:%
%:%156=91%:%
%:%157=92%:%
%:%158=93%:%
%:%159=94%:%
%:%162=95%:%
%:%166=95%:%
%:%167=95%:%
%:%168=95%:%
%:%169=96%:%
%:%170=96%:%
%:%171=97%:%
%:%172=97%:%
%:%173=97%:%
%:%174=98%:%
%:%175=98%:%
%:%176=98%:%
%:%177=99%:%
%:%178=99%:%
%:%179=100%:%
%:%180=100%:%
%:%181=101%:%
%:%182=101%:%
%:%183=102%:%
%:%184=102%:%
%:%185=102%:%
%:%186=103%:%
%:%187=103%:%
%:%188=104%:%
%:%189=104%:%
%:%190=105%:%
%:%191=105%:%
%:%192=105%:%
%:%193=105%:%
%:%194=105%:%
%:%195=106%:%
%:%196=106%:%
%:%197=107%:%
%:%198=107%:%
%:%199=108%:%
%:%200=108%:%
%:%201=108%:%
%:%202=108%:%
%:%203=109%:%
%:%204=109%:%
%:%205=109%:%
%:%206=109%:%
%:%207=110%:%
%:%208=110%:%
%:%209=110%:%
%:%210=111%:%
%:%211=111%:%
%:%212=111%:%
%:%213=112%:%
%:%214=112%:%
%:%215=112%:%
%:%216=112%:%
%:%217=113%:%
%:%218=113%:%
%:%219=114%:%
%:%225=114%:%
%:%228=115%:%
%:%229=116%:%
%:%230=116%:%
%:%231=117%:%
%:%232=118%:%
%:%235=119%:%
%:%239=119%:%
%:%240=119%:%
%:%241=119%:%
%:%246=119%:%
%:%249=120%:%
%:%250=121%:%
%:%251=121%:%
%:%252=122%:%
%:%253=123%:%
%:%254=124%:%
%:%257=125%:%
%:%261=125%:%
%:%262=125%:%
%:%263=125%:%
%:%264=126%:%
%:%265=126%:%
%:%266=127%:%
%:%267=127%:%
%:%268=127%:%
%:%269=128%:%
%:%270=128%:%
%:%271=128%:%
%:%272=129%:%
%:%273=129%:%
%:%274=130%:%
%:%275=130%:%
%:%276=131%:%
%:%277=131%:%
%:%278=131%:%
%:%279=132%:%
%:%280=132%:%
%:%281=132%:%
%:%282=133%:%
%:%283=133%:%
%:%284=134%:%
%:%285=134%:%
%:%286=135%:%
%:%287=135%:%
%:%288=136%:%
%:%289=136%:%
%:%290=137%:%
%:%291=137%:%
%:%292=137%:%
%:%293=138%:%
%:%294=138%:%
%:%295=139%:%
%:%301=139%:%
%:%304=140%:%
%:%305=141%:%
%:%306=141%:%
%:%307=142%:%
%:%308=143%:%
%:%309=144%:%
%:%312=145%:%
%:%316=145%:%
%:%317=145%:%
%:%318=145%:%
%:%319=146%:%
%:%320=146%:%
%:%321=147%:%
%:%322=147%:%
%:%323=147%:%
%:%324=148%:%
%:%325=148%:%
%:%326=148%:%
%:%327=149%:%
%:%328=149%:%
%:%329=150%:%
%:%330=150%:%
%:%331=151%:%
%:%332=151%:%
%:%333=151%:%
%:%334=151%:%
%:%335=152%:%
%:%336=152%:%
%:%337=153%:%
%:%338=153%:%
%:%339=154%:%
%:%340=154%:%
%:%341=154%:%
%:%342=154%:%
%:%343=155%:%
%:%344=155%:%
%:%345=156%:%
%:%346=156%:%
%:%347=157%:%
%:%348=157%:%
%:%349=157%:%
%:%350=157%:%
%:%351=158%:%
%:%352=158%:%
%:%353=159%:%
%:%354=159%:%
%:%355=160%:%
%:%356=160%:%
%:%357=160%:%
%:%358=161%:%
%:%359=161%:%
%:%361=163%:%
%:%362=164%:%
%:%363=164%:%
%:%364=165%:%
%:%365=165%:%
%:%366=166%:%
%:%367=166%:%
%:%368=167%:%
%:%369=167%:%
%:%370=167%:%
%:%371=168%:%
%:%372=168%:%
%:%373=169%:%
%:%374=169%:%
%:%375=169%:%
%:%376=170%:%
%:%377=170%:%
%:%378=171%:%
%:%379=171%:%
%:%380=171%:%
%:%381=172%:%
%:%382=172%:%
%:%383=173%:%
%:%384=173%:%
%:%385=174%:%
%:%386=174%:%
%:%387=175%:%
%:%388=175%:%
%:%389=176%:%
%:%390=176%:%
%:%391=177%:%
%:%392=177%:%
%:%393=178%:%
%:%394=178%:%
%:%395=178%:%
%:%396=179%:%
%:%397=179%:%
%:%398=179%:%
%:%399=180%:%
%:%400=180%:%
%:%401=181%:%
%:%402=181%:%
%:%403=182%:%
%:%404=182%:%
%:%405=183%:%
%:%406=183%:%
%:%407=183%:%
%:%408=184%:%
%:%409=184%:%
%:%410=184%:%
%:%411=185%:%
%:%412=185%:%
%:%413=186%:%
%:%414=186%:%
%:%415=187%:%
%:%416=187%:%
%:%417=188%:%
%:%418=188%:%
%:%419=189%:%
%:%420=189%:%
%:%421=190%:%
%:%422=190%:%
%:%423=191%:%
%:%424=191%:%
%:%425=191%:%
%:%426=192%:%
%:%427=192%:%
%:%428=192%:%
%:%429=193%:%
%:%430=193%:%
%:%431=194%:%
%:%437=194%:%
%:%440=195%:%
%:%441=196%:%
%:%442=196%:%
%:%443=197%:%
%:%444=198%:%
%:%447=199%:%
%:%451=199%:%
%:%452=199%:%
%:%453=199%:%
%:%458=199%:%
%:%461=200%:%
%:%462=201%:%
%:%463=201%:%
%:%464=202%:%
%:%465=203%:%
%:%466=204%:%
%:%469=205%:%
%:%473=205%:%
%:%474=205%:%
%:%475=205%:%
%:%476=206%:%
%:%477=206%:%
%:%478=207%:%
%:%479=207%:%
%:%480=207%:%
%:%481=208%:%
%:%482=208%:%
%:%483=208%:%
%:%484=209%:%
%:%485=209%:%
%:%486=210%:%
%:%487=210%:%
%:%488=211%:%
%:%489=211%:%
%:%490=211%:%
%:%491=212%:%
%:%492=212%:%
%:%493=212%:%
%:%494=213%:%
%:%495=213%:%
%:%496=214%:%
%:%497=214%:%
%:%498=215%:%
%:%499=215%:%
%:%500=216%:%
%:%501=216%:%
%:%502=217%:%
%:%503=217%:%
%:%504=217%:%
%:%505=217%:%
%:%506=218%:%
%:%512=218%:%
%:%515=219%:%
%:%516=220%:%
%:%517=220%:%
%:%518=221%:%
%:%519=222%:%
%:%520=223%:%
%:%523=224%:%
%:%527=224%:%
%:%528=224%:%
%:%529=224%:%
%:%530=225%:%
%:%531=225%:%
%:%532=226%:%
%:%533=226%:%
%:%534=226%:%
%:%535=227%:%
%:%536=227%:%
%:%537=227%:%
%:%538=228%:%
%:%539=228%:%
%:%540=229%:%
%:%541=229%:%
%:%542=230%:%
%:%543=230%:%
%:%544=230%:%
%:%545=230%:%
%:%546=231%:%
%:%547=231%:%
%:%548=232%:%
%:%549=232%:%
%:%550=233%:%
%:%551=233%:%
%:%552=233%:%
%:%553=233%:%
%:%554=234%:%
%:%555=234%:%
%:%556=235%:%
%:%557=235%:%
%:%558=236%:%
%:%559=236%:%
%:%560=236%:%
%:%561=237%:%
%:%562=238%:%
%:%563=238%:%
%:%564=239%:%
%:%565=239%:%
%:%566=239%:%
%:%567=239%:%
%:%568=240%:%
%:%569=240%:%
%:%570=241%:%
%:%571=241%:%
%:%572=242%:%
%:%573=242%:%
%:%574=243%:%
%:%575=243%:%
%:%577=245%:%
%:%578=246%:%
%:%579=246%:%
%:%580=247%:%
%:%581=247%:%
%:%582=248%:%
%:%583=248%:%
%:%584=249%:%
%:%585=249%:%
%:%586=250%:%
%:%587=250%:%
%:%588=251%:%
%:%589=251%:%
%:%590=252%:%
%:%591=252%:%
%:%592=253%:%
%:%593=253%:%
%:%594=254%:%
%:%595=254%:%
%:%596=255%:%
%:%597=255%:%
%:%598=256%:%
%:%599=256%:%
%:%600=257%:%
%:%601=257%:%
%:%602=258%:%
%:%603=258%:%
%:%604=259%:%
%:%605=259%:%
%:%606=260%:%
%:%607=260%:%
%:%608=261%:%
%:%609=261%:%
%:%610=262%:%
%:%611=262%:%
%:%612=263%:%
%:%613=263%:%
%:%614=264%:%
%:%615=264%:%
%:%616=265%:%
%:%617=265%:%
%:%618=266%:%
%:%619=266%:%
%:%620=267%:%
%:%621=267%:%
%:%622=268%:%
%:%623=268%:%
%:%624=269%:%
%:%625=269%:%
%:%626=270%:%
%:%627=270%:%
%:%628=271%:%
%:%629=271%:%
%:%630=272%:%
%:%631=272%:%
%:%632=273%:%
%:%633=273%:%
%:%634=274%:%
%:%635=274%:%
%:%636=274%:%
%:%637=275%:%
%:%638=275%:%
%:%639=275%:%
%:%640=276%:%
%:%641=276%:%
%:%642=277%:%
%:%643=277%:%
%:%644=278%:%
%:%645=278%:%
%:%646=279%:%
%:%647=279%:%
%:%648=279%:%
%:%649=280%:%
%:%650=280%:%
%:%651=281%:%
%:%652=281%:%
%:%653=281%:%
%:%654=282%:%
%:%655=282%:%
%:%656=283%:%
%:%657=283%:%
%:%658=284%:%
%:%659=284%:%
%:%660=285%:%
%:%661=285%:%
%:%662=286%:%
%:%663=286%:%
%:%664=287%:%
%:%665=287%:%
%:%666=287%:%
%:%667=288%:%
%:%668=288%:%
%:%669=288%:%
%:%670=289%:%
%:%671=289%:%
%:%672=290%:%
%:%678=290%:%
%:%681=291%:%
%:%682=292%:%
%:%683=292%:%
%:%684=293%:%
%:%685=294%:%
%:%688=295%:%
%:%692=295%:%
%:%693=295%:%
%:%694=295%:%
%:%699=295%:%
%:%702=296%:%
%:%703=297%:%
%:%704=297%:%
%:%705=298%:%
%:%706=299%:%
%:%707=300%:%
%:%710=301%:%
%:%714=301%:%
%:%715=301%:%
%:%716=301%:%
%:%717=302%:%
%:%718=302%:%
%:%719=303%:%
%:%720=303%:%
%:%721=303%:%
%:%722=304%:%
%:%723=304%:%
%:%724=304%:%
%:%725=305%:%
%:%726=305%:%
%:%727=306%:%
%:%728=306%:%
%:%729=307%:%
%:%730=307%:%
%:%731=307%:%
%:%732=308%:%
%:%733=308%:%
%:%734=308%:%
%:%735=309%:%
%:%736=309%:%
%:%737=310%:%
%:%738=310%:%
%:%739=311%:%
%:%740=311%:%
%:%741=311%:%
%:%742=312%:%
%:%743=313%:%
%:%744=314%:%
%:%745=314%:%
%:%746=315%:%
%:%747=315%:%
%:%748=315%:%
%:%749=315%:%
%:%750=316%:%
%:%751=316%:%
%:%752=317%:%
%:%753=317%:%
%:%754=318%:%
%:%755=318%:%
%:%756=318%:%
%:%757=319%:%
%:%758=319%:%
%:%759=320%:%
%:%760=320%:%
%:%761=320%:%
%:%762=321%:%
%:%763=321%:%
%:%764=322%:%
%:%765=322%:%
%:%766=323%:%
%:%767=323%:%
%:%768=323%:%
%:%769=324%:%
%:%770=324%:%
%:%771=325%:%
%:%772=325%:%
%:%773=326%:%
%:%774=326%:%
%:%775=326%:%
%:%776=327%:%
%:%777=327%:%
%:%778=327%:%
%:%779=328%:%
%:%780=328%:%
%:%781=329%:%
%:%782=329%:%
%:%783=330%:%
%:%784=330%:%
%:%785=331%:%
%:%786=331%:%
%:%787=332%:%
%:%788=332%:%
%:%789=333%:%
%:%790=333%:%
%:%791=334%:%
%:%792=334%:%
%:%793=334%:%
%:%794=335%:%
%:%795=335%:%
%:%796=336%:%
%:%797=336%:%
%:%798=337%:%
%:%799=337%:%
%:%800=338%:%
%:%801=338%:%
%:%802=339%:%
%:%803=339%:%
%:%806=342%:%
%:%807=343%:%
%:%808=343%:%
%:%809=344%:%
%:%810=344%:%
%:%811=345%:%
%:%812=345%:%
%:%813=346%:%
%:%814=346%:%
%:%815=347%:%
%:%816=347%:%
%:%817=347%:%
%:%818=348%:%
%:%819=348%:%
%:%820=349%:%
%:%821=349%:%
%:%822=349%:%
%:%823=350%:%
%:%824=350%:%
%:%825=351%:%
%:%826=351%:%
%:%827=352%:%
%:%828=352%:%
%:%829=352%:%
%:%830=353%:%
%:%831=353%:%
%:%832=354%:%
%:%833=354%:%
%:%834=355%:%
%:%835=355%:%
%:%836=356%:%
%:%837=356%:%
%:%838=356%:%
%:%839=357%:%
%:%840=357%:%
%:%841=358%:%
%:%842=358%:%
%:%843=358%:%
%:%844=359%:%
%:%845=360%:%
%:%846=360%:%
%:%847=361%:%
%:%848=361%:%
%:%849=362%:%
%:%850=362%:%
%:%851=363%:%
%:%852=363%:%
%:%853=363%:%
%:%854=364%:%
%:%855=364%:%
%:%856=364%:%
%:%857=365%:%
%:%858=365%:%
%:%859=366%:%
%:%860=366%:%
%:%863=369%:%
%:%864=370%:%
%:%865=370%:%
%:%866=371%:%
%:%867=371%:%
%:%868=371%:%
%:%869=372%:%
%:%870=373%:%
%:%871=373%:%
%:%872=373%:%
%:%873=374%:%
%:%874=374%:%
%:%875=374%:%
%:%876=375%:%
%:%877=376%:%
%:%878=377%:%
%:%879=378%:%
%:%880=379%:%
%:%881=379%:%
%:%882=380%:%
%:%883=380%:%
%:%884=381%:%
%:%885=381%:%
%:%886=382%:%
%:%887=382%:%
%:%888=382%:%
%:%889=383%:%
%:%890=383%:%
%:%891=383%:%
%:%892=384%:%
%:%893=384%:%
%:%894=385%:%
%:%895=385%:%
%:%896=386%:%
%:%897=386%:%
%:%898=387%:%
%:%899=387%:%
%:%900=388%:%
%:%901=388%:%
%:%902=389%:%
%:%903=389%:%
%:%904=390%:%
%:%905=390%:%
%:%906=391%:%
%:%907=391%:%
%:%908=392%:%
%:%909=392%:%
%:%910=392%:%
%:%911=393%:%
%:%912=393%:%
%:%913=394%:%
%:%914=394%:%
%:%915=395%:%
%:%916=395%:%
%:%917=396%:%
%:%918=396%:%
%:%919=396%:%
%:%920=397%:%
%:%921=397%:%
%:%922=398%:%
%:%923=398%:%
%:%924=399%:%
%:%925=399%:%
%:%926=399%:%
%:%927=400%:%
%:%928=400%:%
%:%929=400%:%
%:%930=401%:%
%:%931=401%:%
%:%932=402%:%
%:%933=402%:%
%:%934=403%:%
%:%935=403%:%
%:%936=404%:%
%:%937=404%:%
%:%938=405%:%
%:%939=405%:%
%:%940=405%:%
%:%941=405%:%
%:%942=406%:%
%:%943=406%:%
%:%944=407%:%
%:%950=407%:%
%:%953=408%:%
%:%954=409%:%
%:%955=409%:%
%:%956=410%:%
%:%957=411%:%
%:%958=412%:%
%:%961=413%:%
%:%965=413%:%
%:%966=413%:%
%:%967=413%:%
%:%968=414%:%
%:%969=414%:%
%:%970=415%:%
%:%971=415%:%
%:%972=415%:%
%:%973=416%:%
%:%974=416%:%
%:%975=416%:%
%:%976=417%:%
%:%977=417%:%
%:%978=418%:%
%:%979=418%:%
%:%980=419%:%
%:%981=419%:%
%:%982=419%:%
%:%983=420%:%
%:%984=420%:%
%:%985=420%:%
%:%986=421%:%
%:%987=421%:%
%:%988=422%:%
%:%989=422%:%
%:%990=423%:%
%:%991=423%:%
%:%992=424%:%
%:%993=424%:%
%:%994=424%:%
%:%995=425%:%
%:%996=425%:%
%:%997=426%:%
%:%998=426%:%
%:%999=427%:%
%:%1000=427%:%
%:%1001=427%:%
%:%1002=428%:%
%:%1003=428%:%
%:%1004=429%:%
%:%1005=430%:%
%:%1006=430%:%
%:%1007=431%:%
%:%1008=431%:%
%:%1009=431%:%
%:%1010=432%:%
%:%1011=432%:%
%:%1012=433%:%
%:%1013=433%:%
%:%1014=434%:%
%:%1015=434%:%
%:%1016=435%:%
%:%1017=435%:%
%:%1018=436%:%
%:%1019=436%:%
%:%1020=436%:%
%:%1021=437%:%
%:%1022=437%:%
%:%1023=438%:%
%:%1024=438%:%
%:%1025=438%:%
%:%1026=439%:%
%:%1027=439%:%
%:%1028=439%:%
%:%1029=440%:%
%:%1030=440%:%
%:%1031=440%:%
%:%1032=441%:%
%:%1038=441%:%
%:%1041=442%:%
%:%1042=443%:%
%:%1043=443%:%
%:%1044=444%:%
%:%1045=445%:%
%:%1046=446%:%
%:%1049=447%:%
%:%1053=447%:%
%:%1054=447%:%
%:%1055=447%:%
%:%1056=448%:%
%:%1057=448%:%
%:%1058=449%:%
%:%1059=449%:%
%:%1060=449%:%
%:%1061=450%:%
%:%1062=450%:%
%:%1063=450%:%
%:%1064=451%:%
%:%1065=451%:%
%:%1066=452%:%
%:%1067=452%:%
%:%1068=453%:%
%:%1069=453%:%
%:%1070=453%:%
%:%1071=453%:%
%:%1072=454%:%
%:%1073=454%:%
%:%1074=455%:%
%:%1075=455%:%
%:%1076=456%:%
%:%1077=456%:%
%:%1078=457%:%
%:%1079=457%:%
%:%1080=458%:%
%:%1081=458%:%
%:%1082=458%:%
%:%1083=459%:%
%:%1084=459%:%
%:%1085=459%:%
%:%1086=460%:%
%:%1087=460%:%
%:%1088=461%:%
%:%1089=461%:%
%:%1090=462%:%
%:%1091=462%:%
%:%1092=463%:%
%:%1093=463%:%
%:%1094=463%:%
%:%1095=464%:%
%:%1096=464%:%
%:%1097=465%:%
%:%1099=467%:%
%:%1100=468%:%
%:%1102=470%:%
%:%1103=471%:%
%:%1104=471%:%
%:%1105=471%:%
%:%1106=472%:%
%:%1107=472%:%
%:%1108=472%:%
%:%1109=472%:%
%:%1110=473%:%
%:%1111=473%:%
%:%1112=474%:%
%:%1113=474%:%
%:%1114=475%:%
%:%1115=475%:%
%:%1116=476%:%
%:%1117=476%:%
%:%1118=477%:%
%:%1119=477%:%
%:%1120=477%:%
%:%1121=477%:%
%:%1122=478%:%
%:%1123=478%:%
%:%1124=479%:%
%:%1125=479%:%
%:%1126=480%:%
%:%1127=480%:%
%:%1128=480%:%
%:%1129=481%:%
%:%1130=481%:%
%:%1131=482%:%
%:%1132=482%:%
%:%1133=483%:%
%:%1134=483%:%
%:%1135=483%:%
%:%1136=483%:%
%:%1137=483%:%
%:%1138=484%:%
%:%1139=484%:%
%:%1140=485%:%
%:%1141=485%:%
%:%1142=486%:%
%:%1143=486%:%
%:%1144=487%:%
%:%1145=487%:%
%:%1146=488%:%
%:%1147=488%:%
%:%1148=489%:%
%:%1149=489%:%
%:%1150=490%:%
%:%1151=490%:%
%:%1152=490%:%
%:%1153=491%:%
%:%1154=491%:%
%:%1155=492%:%
%:%1156=492%:%
%:%1157=492%:%
%:%1158=492%:%
%:%1159=493%:%
%:%1160=493%:%
%:%1161=494%:%
%:%1162=494%:%
%:%1163=494%:%
%:%1164=494%:%
%:%1165=495%:%
%:%1166=495%:%
%:%1167=496%:%
%:%1168=496%:%
%:%1169=496%:%
%:%1170=497%:%
%:%1171=497%:%
%:%1172=497%:%
%:%1173=498%:%
%:%1174=498%:%
%:%1175=498%:%
%:%1176=498%:%
%:%1177=498%:%
%:%1178=499%:%
%:%1179=499%:%
%:%1180=500%:%
%:%1181=500%:%
%:%1182=501%:%
%:%1183=501%:%
%:%1184=501%:%
%:%1185=502%:%
%:%1186=502%:%
%:%1187=503%:%
%:%1188=503%:%
%:%1189=503%:%
%:%1190=504%:%
%:%1191=504%:%
%:%1192=504%:%
%:%1193=505%:%
%:%1194=505%:%
%:%1195=505%:%
%:%1196=506%:%
%:%1197=506%:%
%:%1198=507%:%
%:%1199=507%:%
%:%1200=508%:%
%:%1201=508%:%
%:%1202=509%:%
%:%1203=509%:%
%:%1204=509%:%
%:%1205=510%:%
%:%1206=510%:%
%:%1207=511%:%
%:%1208=511%:%
%:%1209=512%:%
%:%1210=512%:%
%:%1211=512%:%
%:%1212=513%:%
%:%1213=513%:%
%:%1214=513%:%
%:%1215=514%:%
%:%1216=514%:%
%:%1217=514%:%
%:%1218=515%:%
%:%1219=515%:%
%:%1220=516%:%
%:%1221=516%:%
%:%1222=517%:%
%:%1223=517%:%
%:%1224=517%:%
%:%1225=518%:%
%:%1226=518%:%
%:%1228=520%:%
%:%1229=521%:%
%:%1230=521%:%
%:%1231=522%:%
%:%1232=522%:%
%:%1233=523%:%
%:%1234=523%:%
%:%1235=523%:%
%:%1236=523%:%
%:%1237=524%:%
%:%1238=524%:%
%:%1239=525%:%
%:%1240=525%:%
%:%1241=526%:%
%:%1242=526%:%
%:%1243=526%:%
%:%1244=527%:%
%:%1245=527%:%
%:%1246=527%:%
%:%1247=527%:%
%:%1248=528%:%
%:%1249=528%:%
%:%1250=528%:%
%:%1251=529%:%
%:%1252=529%:%
%:%1253=530%:%
%:%1259=530%:%
%:%1262=531%:%
%:%1263=532%:%
%:%1264=532%:%
%:%1265=533%:%
%:%1266=534%:%
%:%1269=535%:%
%:%1273=535%:%
%:%1274=535%:%
%:%1275=535%:%
%:%1276=536%:%
%:%1277=536%:%
%:%1278=537%:%
%:%1279=537%:%
%:%1280=537%:%
%:%1281=538%:%
%:%1282=538%:%
%:%1283=539%:%
%:%1284=539%:%
%:%1285=539%:%
%:%1286=540%:%
%:%1287=540%:%
%:%1288=541%:%
%:%1289=541%:%
%:%1290=542%:%
%:%1291=542%:%
%:%1292=542%:%
%:%1293=543%:%
%:%1294=543%:%
%:%1295=543%:%
%:%1296=544%:%
%:%1297=544%:%
%:%1298=545%:%
%:%1299=545%:%
%:%1300=546%:%
%:%1301=546%:%
%:%1302=547%:%
%:%1303=547%:%
%:%1304=547%:%
%:%1305=548%:%
%:%1306=548%:%
%:%1307=549%:%
%:%1308=549%:%
%:%1309=550%:%
%:%1310=550%:%
%:%1311=550%:%
%:%1312=551%:%
%:%1313=552%:%
%:%1314=553%:%
%:%1315=553%:%
%:%1316=554%:%
%:%1317=554%:%
%:%1318=554%:%
%:%1319=554%:%
%:%1320=555%:%
%:%1321=555%:%
%:%1322=556%:%
%:%1323=556%:%
%:%1324=556%:%
%:%1325=557%:%
%:%1326=557%:%
%:%1327=557%:%
%:%1328=558%:%
%:%1329=558%:%
%:%1330=559%:%
%:%1331=559%:%
%:%1332=559%:%
%:%1333=560%:%
%:%1334=560%:%
%:%1335=560%:%
%:%1336=560%:%
%:%1337=561%:%
%:%1343=561%:%
%:%1346=562%:%
%:%1347=563%:%
%:%1348=563%:%
%:%1349=564%:%
%:%1350=565%:%
%:%1353=566%:%
%:%1357=566%:%
%:%1358=566%:%
%:%1359=567%:%
%:%1360=567%:%
%:%1361=568%:%
%:%1362=568%:%
%:%1367=568%:%
%:%1370=569%:%
%:%1371=570%:%
%:%1372=570%:%
%:%1373=571%:%
%:%1374=572%:%
%:%1375=573%:%
%:%1378=574%:%
%:%1382=574%:%
%:%1383=574%:%
%:%1384=574%:%
%:%1385=575%:%
%:%1386=575%:%
%:%1387=576%:%
%:%1388=576%:%
%:%1389=576%:%
%:%1390=577%:%
%:%1391=577%:%
%:%1392=577%:%
%:%1393=578%:%
%:%1394=578%:%
%:%1395=579%:%
%:%1396=579%:%
%:%1397=580%:%
%:%1398=580%:%
%:%1399=580%:%
%:%1400=580%:%
%:%1401=581%:%
%:%1402=581%:%
%:%1403=582%:%
%:%1404=582%:%
%:%1405=582%:%
%:%1406=583%:%
%:%1407=583%:%
%:%1408=584%:%
%:%1409=584%:%
%:%1410=585%:%
%:%1411=585%:%
%:%1412=585%:%
%:%1413=586%:%
%:%1414=586%:%
%:%1415=586%:%
%:%1416=587%:%
%:%1417=587%:%
%:%1418=588%:%
%:%1419=588%:%
%:%1420=589%:%
%:%1421=589%:%
%:%1422=590%:%
%:%1423=590%:%
%:%1424=590%:%
%:%1425=591%:%
%:%1426=591%:%
%:%1427=591%:%
%:%1430=594%:%
%:%1431=595%:%
%:%1432=595%:%
%:%1433=596%:%
%:%1434=596%:%
%:%1435=596%:%
%:%1436=596%:%
%:%1437=597%:%
%:%1438=597%:%
%:%1440=599%:%
%:%1441=600%:%
%:%1442=600%:%
%:%1443=600%:%
%:%1444=601%:%
%:%1445=602%:%
%:%1446=602%:%
%:%1447=603%:%
%:%1448=603%:%
%:%1449=603%:%
%:%1450=604%:%
%:%1451=604%:%
%:%1452=605%:%
%:%1453=605%:%
%:%1454=606%:%
%:%1455=606%:%
%:%1456=607%:%
%:%1457=607%:%
%:%1458=608%:%
%:%1459=608%:%
%:%1460=608%:%
%:%1461=609%:%
%:%1462=609%:%
%:%1463=609%:%
%:%1464=610%:%
%:%1465=610%:%
%:%1466=611%:%
%:%1467=611%:%
%:%1468=612%:%
%:%1469=612%:%
%:%1470=613%:%
%:%1471=613%:%
%:%1472=613%:%
%:%1473=614%:%
%:%1474=614%:%
%:%1475=615%:%
%:%1476=615%:%
%:%1477=615%:%
%:%1478=616%:%
%:%1479=616%:%
%:%1480=617%:%
%:%1481=617%:%
%:%1482=618%:%
%:%1483=618%:%
%:%1484=618%:%
%:%1485=619%:%
%:%1486=619%:%
%:%1487=620%:%
%:%1488=620%:%
%:%1489=621%:%
%:%1490=621%:%
%:%1491=622%:%
%:%1492=622%:%
%:%1493=623%:%
%:%1494=623%:%
%:%1495=624%:%
%:%1496=624%:%
%:%1497=625%:%
%:%1498=625%:%
%:%1499=626%:%
%:%1500=626%:%
%:%1501=626%:%
%:%1502=626%:%
%:%1503=627%:%
%:%1504=627%:%
%:%1505=627%:%
%:%1506=628%:%
%:%1507=628%:%
%:%1508=629%:%
%:%1509=629%:%
%:%1510=630%:%
%:%1511=630%:%
%:%1512=631%:%
%:%1513=631%:%
%:%1514=632%:%
%:%1515=632%:%
%:%1516=633%:%
%:%1517=633%:%
%:%1518=633%:%
%:%1519=634%:%
%:%1520=634%:%
%:%1521=634%:%
%:%1522=634%:%
%:%1523=635%:%
%:%1524=635%:%
%:%1525=636%:%
%:%1526=636%:%
%:%1527=636%:%
%:%1528=637%:%
%:%1529=637%:%
%:%1530=638%:%
%:%1531=638%:%
%:%1532=639%:%
%:%1533=639%:%
%:%1534=640%:%
%:%1535=640%:%
%:%1536=640%:%
%:%1537=641%:%
%:%1538=641%:%
%:%1539=642%:%
%:%1540=642%:%
%:%1541=643%:%
%:%1542=643%:%
%:%1543=644%:%
%:%1544=644%:%
%:%1545=644%:%
%:%1546=645%:%
%:%1547=645%:%
%:%1549=647%:%
%:%1550=648%:%
%:%1551=648%:%
%:%1552=649%:%
%:%1553=649%:%
%:%1554=650%:%
%:%1555=650%:%
%:%1556=651%:%
%:%1557=651%:%
%:%1558=652%:%
%:%1559=652%:%
%:%1560=653%:%
%:%1561=653%:%
%:%1562=653%:%
%:%1563=654%:%
%:%1564=654%:%
%:%1565=655%:%
%:%1566=655%:%
%:%1567=655%:%
%:%1568=656%:%
%:%1569=656%:%
%:%1570=657%:%
%:%1571=657%:%
%:%1572=658%:%
%:%1573=658%:%
%:%1574=659%:%
%:%1575=659%:%
%:%1576=660%:%
%:%1577=660%:%
%:%1578=661%:%
%:%1579=661%:%
%:%1580=662%:%
%:%1581=662%:%
%:%1582=663%:%
%:%1583=663%:%
%:%1584=664%:%
%:%1585=664%:%
%:%1586=665%:%
%:%1587=665%:%
%:%1588=665%:%
%:%1589=666%:%
%:%1590=666%:%
%:%1591=667%:%
%:%1592=667%:%
%:%1593=668%:%
%:%1594=668%:%
%:%1595=669%:%
%:%1596=669%:%
%:%1597=670%:%
%:%1598=670%:%
%:%1599=670%:%
%:%1600=671%:%
%:%1601=671%:%
%:%1602=672%:%
%:%1603=672%:%
%:%1604=673%:%
%:%1605=673%:%
%:%1606=673%:%
%:%1607=674%:%
%:%1608=674%:%
%:%1609=675%:%
%:%1610=675%:%
%:%1611=676%:%
%:%1612=676%:%
%:%1613=676%:%
%:%1614=676%:%
%:%1615=677%:%
%:%1616=677%:%
%:%1617=678%:%
%:%1618=678%:%
%:%1619=679%:%
%:%1620=679%:%
%:%1621=680%:%
%:%1622=680%:%
%:%1623=681%:%
%:%1624=681%:%
%:%1625=682%:%
%:%1626=682%:%
%:%1627=682%:%
%:%1628=683%:%
%:%1629=683%:%
%:%1630=683%:%
%:%1631=684%:%
%:%1632=684%:%
%:%1633=685%:%
%:%1634=685%:%
%:%1635=686%:%
%:%1636=686%:%
%:%1637=687%:%
%:%1638=687%:%
%:%1639=688%:%
%:%1640=688%:%
%:%1641=689%:%
%:%1642=689%:%
%:%1643=689%:%
%:%1644=689%:%
%:%1645=690%:%
%:%1646=690%:%
%:%1647=691%:%
%:%1648=691%:%
%:%1649=692%:%
%:%1650=692%:%
%:%1651=693%:%
%:%1652=693%:%
%:%1653=694%:%
%:%1654=694%:%
%:%1655=695%:%
%:%1656=695%:%
%:%1657=696%:%
%:%1658=696%:%
%:%1659=697%:%
%:%1660=697%:%
%:%1661=698%:%
%:%1662=698%:%
%:%1663=699%:%
%:%1664=699%:%
%:%1665=700%:%
%:%1666=700%:%
%:%1667=701%:%
%:%1668=701%:%
%:%1669=702%:%
%:%1670=702%:%
%:%1671=703%:%
%:%1672=703%:%
%:%1673=703%:%
%:%1674=704%:%
%:%1675=704%:%
%:%1676=705%:%
%:%1677=705%:%
%:%1678=706%:%
%:%1679=706%:%
%:%1680=707%:%
%:%1681=707%:%
%:%1682=708%:%
%:%1683=708%:%
%:%1684=709%:%
%:%1685=709%:%
%:%1686=710%:%
%:%1687=710%:%
%:%1688=711%:%
%:%1689=711%:%
%:%1690=711%:%
%:%1691=711%:%
%:%1692=712%:%
%:%1693=712%:%
%:%1694=712%:%
%:%1695=713%:%
%:%1696=713%:%
%:%1697=714%:%
%:%1698=714%:%
%:%1699=715%:%
%:%1700=715%:%
%:%1701=716%:%
%:%1702=716%:%
%:%1703=717%:%
%:%1704=717%:%
%:%1705=717%:%
%:%1706=717%:%
%:%1707=718%:%
%:%1708=718%:%
%:%1709=719%:%
%:%1710=719%:%
%:%1711=720%:%
%:%1712=720%:%
%:%1713=720%:%
%:%1714=721%:%
%:%1715=721%:%
%:%1716=722%:%
%:%1717=722%:%
%:%1718=723%:%
%:%1719=723%:%
%:%1720=723%:%
%:%1721=724%:%
%:%1722=724%:%
%:%1723=724%:%
%:%1724=725%:%
%:%1725=725%:%
%:%1726=726%:%
%:%1727=726%:%
%:%1728=727%:%
%:%1729=727%:%
%:%1730=727%:%
%:%1731=728%:%
%:%1732=728%:%
%:%1733=728%:%
%:%1734=728%:%
%:%1735=729%:%
%:%1736=729%:%
%:%1737=730%:%
%:%1738=730%:%
%:%1739=731%:%
%:%1740=731%:%
%:%1741=732%:%
%:%1742=732%:%
%:%1743=733%:%
%:%1744=733%:%
%:%1745=733%:%
%:%1746=734%:%
%:%1747=734%:%
%:%1748=734%:%
%:%1749=734%:%
%:%1750=735%:%
%:%1751=735%:%
%:%1752=736%:%
%:%1753=736%:%
%:%1754=736%:%
%:%1755=737%:%
%:%1756=737%:%
%:%1757=737%:%
%:%1758=738%:%
%:%1759=738%:%
%:%1760=738%:%
%:%1761=739%:%
%:%1762=739%:%
%:%1764=741%:%
%:%1765=742%:%
%:%1766=742%:%
%:%1767=743%:%
%:%1768=743%:%
%:%1769=743%:%
%:%1770=744%:%
%:%1771=744%:%
%:%1772=745%:%
%:%1773=745%:%
%:%1774=746%:%
%:%1775=746%:%
%:%1776=747%:%
%:%1777=747%:%
%:%1778=748%:%
%:%1779=748%:%
%:%1780=749%:%
%:%1781=749%:%
%:%1782=750%:%
%:%1783=750%:%
%:%1784=750%:%
%:%1785=751%:%
%:%1786=751%:%
%:%1787=751%:%
%:%1788=752%:%
%:%1789=752%:%
%:%1790=752%:%
%:%1791=752%:%
%:%1792=753%:%
%:%1793=753%:%
%:%1794=754%:%
%:%1795=754%:%
%:%1796=754%:%
%:%1797=755%:%
%:%1798=755%:%
%:%1799=755%:%
%:%1800=756%:%
%:%1801=756%:%
%:%1802=757%:%
%:%1803=757%:%
%:%1804=757%:%
%:%1805=758%:%
%:%1806=758%:%
%:%1807=759%:%
%:%1808=759%:%
%:%1809=759%:%
%:%1810=760%:%
%:%1811=760%:%
%:%1812=761%:%
%:%1813=761%:%
%:%1814=762%:%
%:%1815=762%:%
%:%1816=763%:%
%:%1817=763%:%
%:%1818=764%:%
%:%1819=764%:%
%:%1820=765%:%
%:%1821=765%:%
%:%1822=766%:%
%:%1823=766%:%
%:%1824=767%:%
%:%1825=767%:%
%:%1826=767%:%
%:%1827=768%:%
%:%1828=768%:%
%:%1829=768%:%
%:%1830=769%:%
%:%1831=769%:%
%:%1832=770%:%
%:%1833=770%:%
%:%1834=771%:%
%:%1835=771%:%
%:%1836=772%:%
%:%1837=772%:%
%:%1838=773%:%
%:%1839=773%:%
%:%1840=774%:%
%:%1841=774%:%
%:%1842=775%:%
%:%1843=775%:%
%:%1844=776%:%
%:%1845=776%:%
%:%1846=777%:%
%:%1847=777%:%
%:%1848=778%:%
%:%1849=778%:%
%:%1850=779%:%
%:%1851=779%:%
%:%1852=780%:%
%:%1853=780%:%
%:%1854=780%:%
%:%1855=781%:%
%:%1856=781%:%
%:%1857=782%:%
%:%1858=782%:%
%:%1859=783%:%
%:%1860=783%:%
%:%1861=784%:%
%:%1862=784%:%
%:%1863=785%:%
%:%1864=785%:%
%:%1865=786%:%
%:%1866=786%:%
%:%1867=787%:%
%:%1868=787%:%
%:%1869=788%:%
%:%1870=788%:%
%:%1871=788%:%
%:%1872=788%:%
%:%1873=789%:%
%:%1874=789%:%
%:%1875=789%:%
%:%1876=790%:%
%:%1877=790%:%
%:%1878=791%:%
%:%1879=791%:%
%:%1880=792%:%
%:%1881=792%:%
%:%1882=793%:%
%:%1888=793%:%
%:%1891=794%:%
%:%1892=795%:%
%:%1893=795%:%
%:%1900=796%:%

%
\begin{isabellebody}%
\setisabellecontext{HML{\isacharunderscore}{\kern0pt}equivalences}%
%
\isadelimtheory
%
\endisadelimtheory
%
\isatagtheory
\isacommand{theory}\isamarkupfalse%
\ HML{\isacharunderscore}{\kern0pt}equivalences\isanewline
\isakeyword{imports}\ Main\isanewline
HML{\isacharunderscore}{\kern0pt}list\isanewline
\isakeyword{begin}%
\endisatagtheory
{\isafoldtheory}%
%
\isadelimtheory
\isanewline
%
\endisadelimtheory
\isanewline
\isacommand{context}\isamarkupfalse%
\ lts\ \isakeyword{begin}\isanewline
\isanewline
\isacommand{definition}\isamarkupfalse%
\ HML{\isacharunderscore}{\kern0pt}trace{\isacharunderscore}{\kern0pt}equivalent\ \isakeyword{where}\isanewline
{\isachardoublequoteopen}HML{\isacharunderscore}{\kern0pt}trace{\isacharunderscore}{\kern0pt}equivalent\ p\ q\ {\isasymequiv}\ {\isacharparenleft}{\kern0pt}{\isasymforall}\ {\isasymphi}{\isachardot}{\kern0pt}\ {\isasymphi}\ {\isasymin}\ HML{\isacharunderscore}{\kern0pt}trace{\isacharunderscore}{\kern0pt}formulas\ {\isasymlongrightarrow}\ {\isacharparenleft}{\kern0pt}p\ {\isasymTurnstile}\ {\isasymphi}{\isacharparenright}{\kern0pt}\ {\isasymlongleftrightarrow}\ {\isacharparenleft}{\kern0pt}q\ {\isasymTurnstile}\ {\isasymphi}{\isacharparenright}{\kern0pt}{\isacharparenright}{\kern0pt}{\isachardoublequoteclose}\isanewline
\isanewline
\isacommand{definition}\isamarkupfalse%
\ HML{\isacharunderscore}{\kern0pt}simulation{\isacharunderscore}{\kern0pt}equivalent\ {\isacharcolon}{\kern0pt}{\isacharcolon}{\kern0pt}\ {\isacartoucheopen}{\isacharprime}{\kern0pt}s\ {\isasymRightarrow}\ {\isacharprime}{\kern0pt}s\ {\isasymRightarrow}\ bool{\isacartoucheclose}\ \isakeyword{where}\isanewline
\ \ {\isachardoublequoteopen}HML{\isacharunderscore}{\kern0pt}simulation{\isacharunderscore}{\kern0pt}equivalent\ p\ q\ {\isasymequiv}\ \isanewline
{\isacharparenleft}{\kern0pt}{\isasymforall}{\isasymphi}{\isachardot}{\kern0pt}\ {\isasymphi}\ {\isasymin}\ HML{\isacharunderscore}{\kern0pt}simulation{\isacharunderscore}{\kern0pt}formulas\ {\isasymlongrightarrow}\ {\isacharparenleft}{\kern0pt}p\ {\isasymTurnstile}\ {\isasymphi}\ {\isasymlongleftrightarrow}\ q\ {\isasymTurnstile}\ {\isasymphi}{\isacharparenright}{\kern0pt}{\isacharparenright}{\kern0pt}{\isachardoublequoteclose}\isanewline
\isanewline
\isacommand{definition}\isamarkupfalse%
\ HML{\isacharunderscore}{\kern0pt}possible{\isacharunderscore}{\kern0pt}futures{\isacharunderscore}{\kern0pt}equivalent\ \isakeyword{where}\isanewline
{\isachardoublequoteopen}HML{\isacharunderscore}{\kern0pt}possible{\isacharunderscore}{\kern0pt}futures{\isacharunderscore}{\kern0pt}equivalent\ p\ q\ {\isasymequiv}\ {\isacharparenleft}{\kern0pt}{\isasymforall}\ {\isasymphi}{\isachardot}{\kern0pt}\ {\isasymphi}\ {\isasymin}\ HML{\isacharunderscore}{\kern0pt}possible{\isacharunderscore}{\kern0pt}futures{\isacharunderscore}{\kern0pt}formulas\ {\isasymlongrightarrow}\ {\isacharparenleft}{\kern0pt}p\ {\isasymTurnstile}\ {\isasymphi}{\isacharparenright}{\kern0pt}\ {\isasymlongleftrightarrow}\ {\isacharparenleft}{\kern0pt}q\ {\isasymTurnstile}\ {\isasymphi}{\isacharparenright}{\kern0pt}{\isacharparenright}{\kern0pt}{\isachardoublequoteclose}\isanewline
\isanewline
\isacommand{end}\isamarkupfalse%
\isanewline
%
\isadelimtheory
%
\endisadelimtheory
%
\isatagtheory
\isacommand{end}\isamarkupfalse%
%
\endisatagtheory
{\isafoldtheory}%
%
\isadelimtheory
%
\endisadelimtheory
%
\end{isabellebody}%
\endinput
%:%file=~/Documents/Isabelle_HOL/HML_equivalences.thy%:%
%:%10=1%:%
%:%11=1%:%
%:%12=2%:%
%:%13=3%:%
%:%14=4%:%
%:%19=4%:%
%:%22=5%:%
%:%23=6%:%
%:%24=6%:%
%:%25=7%:%
%:%26=8%:%
%:%27=8%:%
%:%28=9%:%
%:%29=10%:%
%:%30=11%:%
%:%31=11%:%
%:%32=12%:%
%:%33=13%:%
%:%34=14%:%
%:%35=15%:%
%:%36=15%:%
%:%37=16%:%
%:%38=17%:%
%:%39=18%:%
%:%40=18%:%
%:%47=19%:%

