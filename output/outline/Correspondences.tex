%
\begin{isabellebody}%
\setisabellecontext{Correspondences}%
%
\isadelimtheory
%
\endisadelimtheory
%
\isatagtheory
%
\endisatagtheory
{\isafoldtheory}%
%
\isadelimtheory
%
\endisadelimtheory
%
\isadelimdocument
%
\endisadelimdocument
%
\isatagdocument
%
\isamarkupchapter{Characterizing Equivalences%
}
\isamarkuptrue%
%
\endisatagdocument
{\isafolddocument}%
%
\isadelimdocument
%
\endisadelimdocument
%
\begin{isamarkuptext}%
In this chapter we introduce the modal-logical characterizations $\mathcal{O}_X$ of the various equivalences and link them to the HML sublanguages $\mathcal{O}_{e_X}$ determined 
certain by price bounds. The proofs follow the same structure: We first derive the modal characterization of $\mathcal{O}_{e_X}$ and then show that this characterization is equivalent to the corresponding $\mathcal{O}_X$.
We derive these modal-logical characterizations from (Glaabbeeck). In the appendix we prove for trace equivalence $\mathcal{O}_T$ and bisimilarity $\mathcal{O}_B$ that the modal-logical characterization really captures
the colloquial definitions via trace sets/the relational definition of bisimilarity.%
\end{isamarkuptext}\isamarkuptrue%
%
\isadelimtheory
%
\endisadelimtheory
%
\isatagtheory
%
\endisatagtheory
{\isafoldtheory}%
%
\isadelimtheory
%
\endisadelimtheory
%
\end{isabellebody}%
\endinput
%:%file=~/Documents/Isabelle_HOL/Correspondences.thy%:%
%:%24=8%:%
%:%36=10%:%
%:%37=11%:%
%:%38=12%:%
%:%39=13%:%
