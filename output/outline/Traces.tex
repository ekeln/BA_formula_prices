%
\begin{isabellebody}%
\setisabellecontext{Traces}%
%
\isadelimtheory
%
\endisadelimtheory
%
\isatagtheory
%
\endisatagtheory
{\isafoldtheory}%
%
\isadelimtheory
%
\endisadelimtheory
%
\isadelimdocument
%
\endisadelimdocument
%
\isatagdocument
%
\isamarkupsection{Trace semantics%
}
\isamarkuptrue%
%
\endisatagdocument
{\isafolddocument}%
%
\isadelimdocument
%
\endisadelimdocument
%
\begin{isamarkuptext}%
As discussed, trace semantics identifies two processes as equivalent if they allow for the same set of observations, or sequences of actions.%
\end{isamarkuptext}\isamarkuptrue%
%
\isadelimdocument
%
\endisadelimdocument
%
\isatagdocument
%
\isamarkupsubsubsection{Definition 3.1.1%
}
\isamarkuptrue%
%
\endisatagdocument
{\isafolddocument}%
%
\isadelimdocument
%
\endisadelimdocument
%
\begin{isamarkuptext}%
\textit{The \textnormal{modal-characterization of trace semantics} is given by the set $\mathcal{O}_T$ of trace formulas over Act, recursively defined by:}
\begin{align*}
&\langle a \rangle\varphi\in\mathcal{O}_T \text{ if } \varphi\in \mathcal{O}_T \text{ and }a\in\Act\\
&\bigwedge\varnothing\in \mathcal{O}_T
\end{align*}%
\end{isamarkuptext}\isamarkuptrue%
\isacommand{inductive}\isamarkupfalse%
\ hml{\isacharunderscore}{\kern0pt}trace\ {\isacharcolon}{\kern0pt}{\isacharcolon}{\kern0pt}\ {\isachardoublequoteopen}{\isacharparenleft}{\kern0pt}{\isacharprime}{\kern0pt}a{\isacharcomma}{\kern0pt}\ {\isacharprime}{\kern0pt}s{\isacharparenright}{\kern0pt}hml\ {\isasymRightarrow}\ bool{\isachardoublequoteclose}\ \isakeyword{where}\isanewline
trace{\isacharunderscore}{\kern0pt}tt{\isacharcolon}{\kern0pt}\ {\isachardoublequoteopen}hml{\isacharunderscore}{\kern0pt}trace\ TT{\isachardoublequoteclose}\ {\isacharbar}{\kern0pt}\isanewline
trace{\isacharunderscore}{\kern0pt}conj{\isacharcolon}{\kern0pt}\ {\isachardoublequoteopen}hml{\isacharunderscore}{\kern0pt}trace\ {\isacharparenleft}{\kern0pt}hml{\isacharunderscore}{\kern0pt}conj\ {\isacharbraceleft}{\kern0pt}{\isacharbraceright}{\kern0pt}\ {\isacharbraceleft}{\kern0pt}{\isacharbraceright}{\kern0pt}\ {\isasympsi}s{\isacharparenright}{\kern0pt}{\isachardoublequoteclose}{\isacharbar}{\kern0pt}\isanewline
trace{\isacharunderscore}{\kern0pt}pos{\isacharcolon}{\kern0pt}\ {\isachardoublequoteopen}hml{\isacharunderscore}{\kern0pt}trace\ {\isacharparenleft}{\kern0pt}hml{\isacharunderscore}{\kern0pt}pos\ {\isasymalpha}\ {\isasymphi}{\isacharparenright}{\kern0pt}{\isachardoublequoteclose}\ \isakeyword{if}\ {\isachardoublequoteopen}hml{\isacharunderscore}{\kern0pt}trace\ {\isasymphi}{\isachardoublequoteclose}\isanewline
\isanewline
\isacommand{definition}\isamarkupfalse%
\ hml{\isacharunderscore}{\kern0pt}trace{\isacharunderscore}{\kern0pt}formulas\isanewline
\ \ \isakeyword{where}\isanewline
{\isachardoublequoteopen}hml{\isacharunderscore}{\kern0pt}trace{\isacharunderscore}{\kern0pt}formulas\ {\isasymequiv}\ {\isacharbraceleft}{\kern0pt}{\isasymphi}{\isachardot}{\kern0pt}\ hml{\isacharunderscore}{\kern0pt}trace\ {\isasymphi}{\isacharbraceright}{\kern0pt}{\isachardoublequoteclose}%
\begin{isamarkuptext}%
This definition allows for the construction of traces such as $\langle a_1 \rangle \langle a_2 \rangle \ldots \langle a_n \rangle \textsf{T}$, which represents action sequences or traces.
Two processes $p$ and $q$ are considered trace-equivalent if they satisfy the same formulas in $\mathcal{O}_T$, namely 
$$p \sim_T q \longleftrightarrow \forall\varphi\in\mathcal{O}_T. p\models\varphi \longleftrightarrow q\models\varphi$$%
\end{isamarkuptext}\isamarkuptrue%
\isacommand{context}\isamarkupfalse%
\ lts\isanewline
\isakeyword{begin}\isanewline
\isanewline
\isacommand{definition}\isamarkupfalse%
\ hml{\isacharunderscore}{\kern0pt}trace{\isacharunderscore}{\kern0pt}equivalent\ \isakeyword{where}\isanewline
{\isachardoublequoteopen}hml{\isacharunderscore}{\kern0pt}trace{\isacharunderscore}{\kern0pt}equivalent\ p\ q\ {\isasymequiv}\ HML{\isacharunderscore}{\kern0pt}subset{\isacharunderscore}{\kern0pt}equivalent\ hml{\isacharunderscore}{\kern0pt}trace{\isacharunderscore}{\kern0pt}formulas\ p\ q{\isachardoublequoteclose}%
\begin{isamarkuptext}%
The subset $\mathcal{O}_X$ only allows for finite sequences of actions, without the use of conjunctions or negations. 
Therefore, the complexity of a trace formula is limited by its modal depth (and one conjunction for \textsf{T}). 
As a result, the language derivied from the price coordinate $(\infty, 1, 0, 0, 0, 0)$ encompasses all trace formulas. 
We refer to this HML-sublanguage as $\mathcal{O}{e_T}$.%
\end{isamarkuptext}\isamarkuptrue%
\isacommand{definition}\isamarkupfalse%
\ expr{\isacharunderscore}{\kern0pt}traces\isanewline
\ \ \isakeyword{where}\isanewline
{\isachardoublequoteopen}expr{\isacharunderscore}{\kern0pt}traces\ {\isacharequal}{\kern0pt}\ {\isacharbraceleft}{\kern0pt}{\isasymphi}{\isachardot}{\kern0pt}\ {\isacharparenleft}{\kern0pt}less{\isacharunderscore}{\kern0pt}eq{\isacharunderscore}{\kern0pt}t\ {\isacharparenleft}{\kern0pt}expr\ {\isasymphi}{\isacharparenright}{\kern0pt}\ {\isacharparenleft}{\kern0pt}{\isasyminfinity}{\isacharcomma}{\kern0pt}\ {\isadigit{1}}{\isacharcomma}{\kern0pt}\ {\isadigit{0}}{\isacharcomma}{\kern0pt}\ {\isadigit{0}}{\isacharcomma}{\kern0pt}\ {\isadigit{0}}{\isacharcomma}{\kern0pt}\ {\isadigit{0}}{\isacharparenright}{\kern0pt}{\isacharparenright}{\kern0pt}{\isacharbraceright}{\kern0pt}{\isachardoublequoteclose}\isanewline
\isanewline
\isacommand{definition}\isamarkupfalse%
\ expr{\isacharunderscore}{\kern0pt}trace{\isacharunderscore}{\kern0pt}equivalent\ \isanewline
\ \ \isakeyword{where}\isanewline
{\isachardoublequoteopen}expr{\isacharunderscore}{\kern0pt}trace{\isacharunderscore}{\kern0pt}equivalent\ p\ q\ {\isasymequiv}\ HML{\isacharunderscore}{\kern0pt}subset{\isacharunderscore}{\kern0pt}equivalent\ expr{\isacharunderscore}{\kern0pt}traces\ p\ q{\isachardoublequoteclose}\isanewline
\isacommand{end}\isamarkupfalse%
%
\isadelimdocument
%
\endisadelimdocument
%
\isatagdocument
%
\isamarkupsubsubsection{Proposition 3.1.2%
}
\isamarkuptrue%
%
\endisatagdocument
{\isafolddocument}%
%
\isadelimdocument
%
\endisadelimdocument
%
\begin{isamarkuptext}%
The language of formulas with prices below $(\infty, 1, 0, 0, 0, 0)$ characterizes trace equivalence. That is, for two processes $p$ and $q$, $p \sim_T q \longleftrightarrow p \sim_{e_T} q$. Explicitly: \\

\[
\forall \varphi \in \mathcal{O}_T. p \models \varphi \longleftrightarrow q \models \varphi \longleftrightarrow \forall \varphi \in \mathcal{O}_{e_T}. p \models \varphi \longleftrightarrow q \models \varphi
\]%
\end{isamarkuptext}\isamarkuptrue%
%
\begin{isamarkuptext}%
\textit{Proof.} We show that $\mathcal{O}_T$ and $\mathcal{O}_{e_T}$ capture the same set of formulas. We do this for both sides by induction over the structure of \textsf{HML}[$\Sigma$].

First, we show that if $\varphi \in \mathcal{O}_\text{T}$, then \textsf{expr}$(\varphi) \leq (\infty, 1, 0, 0, 0, 0)$:
\begin{align*}
&\textit{(Base) Case $\bigwedge\varnothing$:} &&\parbox[t]{0.8\textwidth}{We can easily derive that $\bigwedge\varnothing = (0, 1, 0, 0, 0, 0)$ and thus $\bigwedge\varnothing \leq (\infty, 1, 0, 0, 0, 0)$.}\\
&\textit{Case $\langle a \rangle \varphi$:} &&\parbox[t]{0.8\textwidth}{Since $\langle a \rangle$ only adds to $\textsf{expr}_1$, we can easily show that if $\textsf{expr}(\varphi) \leq (\infty, 1, 0, 0, 0, 0)$, then $\langle a \rangle\varphi \leq (\infty, 1, 0, 0, 0, 0)$.}
\end{align*}%
\end{isamarkuptext}\isamarkuptrue%
%
\begin{isamarkuptext}%
Next, we show that if $\textsf{expr}(\varphi) \leq (\infty, 1, 0, 0, 0, 0)$, then $\varphi \in \mathcal{O}_\text{X}$:
\begin{align*}
&\textit{Case $\bigwedge_{i\in I}(\psi_i)$:} &&\parbox[t]{0.8\textwidth}{Since every formula ends with \textsf{T}, and $\textsf{expr}_2$ denotes the depth of a conjunction, $\textsf{expr}_2(\bigwedge_{i\in I}(\psi_i)) \geq 2$ if $I\neq\varnothing$. Therefore, $I$ must be empty.} \\
&\textit{Case $\langle a \rangle\varphi$:} &&\parbox[t]{0.8\textwidth}{From the induction hypothesis and the monotonicity attribute, we have that $\varphi\in\mathcal{O}_T$. With the definition of $\mathcal{O}_T$, we have that $\langle a \rangle\varphi\in\mathcal{O}_T$.}
\end{align*}%
\end{isamarkuptext}\isamarkuptrue%
%
\isadelimvisible
%
\endisadelimvisible
%
\isatagvisible
%
\endisatagvisible
{\isafoldvisible}%
%
\isadelimvisible
\isanewline
%
\endisadelimvisible
\isacommand{lemma}\isamarkupfalse%
\ trace{\isacharunderscore}{\kern0pt}right{\isacharcolon}{\kern0pt}\ \isanewline
\ \ \isakeyword{assumes}\ {\isachardoublequoteopen}hml{\isacharunderscore}{\kern0pt}trace\ {\isasymphi}{\isachardoublequoteclose}\isanewline
\ \ \isakeyword{shows}\ {\isachardoublequoteopen}{\isacharparenleft}{\kern0pt}less{\isacharunderscore}{\kern0pt}eq{\isacharunderscore}{\kern0pt}t\ {\isacharparenleft}{\kern0pt}expr\ {\isasymphi}{\isacharparenright}{\kern0pt}\ {\isacharparenleft}{\kern0pt}{\isasyminfinity}{\isacharcomma}{\kern0pt}\ {\isadigit{1}}{\isacharcomma}{\kern0pt}\ {\isadigit{0}}{\isacharcomma}{\kern0pt}\ {\isadigit{0}}{\isacharcomma}{\kern0pt}\ {\isadigit{0}}{\isacharcomma}{\kern0pt}\ {\isadigit{0}}{\isacharparenright}{\kern0pt}{\isacharparenright}{\kern0pt}{\isachardoublequoteclose}\isanewline
%
\isadelimproof
\ \ %
\endisadelimproof
%
\isatagproof
\isacommand{using}\isamarkupfalse%
\ assms\isanewline
\isacommand{proof}\isamarkupfalse%
{\isacharparenleft}{\kern0pt}induct\ {\isasymphi}\ rule{\isacharcolon}{\kern0pt}hml{\isacharunderscore}{\kern0pt}trace{\isachardot}{\kern0pt}induct{\isacharparenright}{\kern0pt}\isanewline
\ \ \isacommand{case}\isamarkupfalse%
\ trace{\isacharunderscore}{\kern0pt}tt\isanewline
\ \ \isacommand{then}\isamarkupfalse%
\ \isacommand{show}\isamarkupfalse%
\ {\isacharquery}{\kern0pt}case\ \isacommand{by}\isamarkupfalse%
\ simp\isanewline
\isacommand{next}\isamarkupfalse%
\isanewline
\ \ \isacommand{case}\isamarkupfalse%
\ {\isacharparenleft}{\kern0pt}trace{\isacharunderscore}{\kern0pt}conj\ {\isasympsi}s{\isacharparenright}{\kern0pt}\isanewline
\ \ \isacommand{have}\isamarkupfalse%
\ {\isachardoublequoteopen}{\isacharparenleft}{\kern0pt}expr{\isacharunderscore}{\kern0pt}{\isadigit{4}}\ {\isacharparenleft}{\kern0pt}hml{\isacharunderscore}{\kern0pt}conj\ {\isacharbraceleft}{\kern0pt}{\isacharbraceright}{\kern0pt}\ {\isacharbraceleft}{\kern0pt}{\isacharbraceright}{\kern0pt}\ {\isasympsi}s{\isacharparenright}{\kern0pt}{\isacharparenright}{\kern0pt}\ {\isacharequal}{\kern0pt}\ {\isadigit{0}}{\isachardoublequoteclose}\isanewline
\ \ \ \ \isacommand{using}\isamarkupfalse%
\ expr{\isacharunderscore}{\kern0pt}{\isadigit{4}}{\isachardot}{\kern0pt}simps\ Sup{\isacharunderscore}{\kern0pt}enat{\isacharunderscore}{\kern0pt}def\ \isacommand{by}\isamarkupfalse%
\ auto\isanewline
\ \ \isacommand{then}\isamarkupfalse%
\ \isacommand{show}\isamarkupfalse%
\ {\isacharquery}{\kern0pt}case\ \isacommand{by}\isamarkupfalse%
\ auto\isanewline
\isacommand{next}\isamarkupfalse%
\isanewline
\ \ \isacommand{case}\isamarkupfalse%
\ {\isacharparenleft}{\kern0pt}trace{\isacharunderscore}{\kern0pt}pos\ {\isasymphi}\ {\isasymalpha}{\isacharparenright}{\kern0pt}\isanewline
\ \ \isacommand{then}\isamarkupfalse%
\ \isacommand{show}\isamarkupfalse%
\ {\isacharquery}{\kern0pt}case\ \isacommand{by}\isamarkupfalse%
\ simp\isanewline
\isacommand{qed}\isamarkupfalse%
%
\endisatagproof
{\isafoldproof}%
%
\isadelimproof
\isanewline
%
\endisadelimproof
\isanewline
\isacommand{lemma}\isamarkupfalse%
\ HML{\isacharunderscore}{\kern0pt}trace{\isacharunderscore}{\kern0pt}conj{\isacharunderscore}{\kern0pt}empty{\isacharcolon}{\kern0pt}\isanewline
\ \ \isakeyword{assumes}\ A{\isadigit{1}}{\isacharcolon}{\kern0pt}\ {\isachardoublequoteopen}less{\isacharunderscore}{\kern0pt}eq{\isacharunderscore}{\kern0pt}t\ {\isacharparenleft}{\kern0pt}expr\ {\isacharparenleft}{\kern0pt}hml{\isacharunderscore}{\kern0pt}conj\ I\ J\ {\isasymPhi}{\isacharparenright}{\kern0pt}{\isacharparenright}{\kern0pt}\ {\isacharparenleft}{\kern0pt}{\isasyminfinity}{\isacharcomma}{\kern0pt}\ {\isadigit{1}}{\isacharcomma}{\kern0pt}\ {\isadigit{0}}{\isacharcomma}{\kern0pt}\ {\isadigit{0}}{\isacharcomma}{\kern0pt}\ {\isadigit{0}}{\isacharcomma}{\kern0pt}\ {\isadigit{0}}{\isacharparenright}{\kern0pt}{\isachardoublequoteclose}\ \isanewline
\ \ \isakeyword{shows}\ {\isachardoublequoteopen}I\ {\isacharequal}{\kern0pt}\ {\isacharbraceleft}{\kern0pt}{\isacharbraceright}{\kern0pt}\ {\isasymand}\ J\ {\isacharequal}{\kern0pt}\ {\isacharbraceleft}{\kern0pt}{\isacharbraceright}{\kern0pt}{\isachardoublequoteclose}\isanewline
%
\isadelimproof
%
\endisadelimproof
%
\isatagproof
\isacommand{proof}\isamarkupfalse%
{\isacharminus}{\kern0pt}\isanewline
\ \ \isacommand{have}\isamarkupfalse%
\ {\isachardoublequoteopen}expr{\isacharunderscore}{\kern0pt}{\isadigit{2}}\ {\isacharparenleft}{\kern0pt}hml{\isacharunderscore}{\kern0pt}conj\ I\ J\ {\isasymPhi}{\isacharparenright}{\kern0pt}\ {\isacharequal}{\kern0pt}\ {\isadigit{1}}\ {\isacharplus}{\kern0pt}\ Sup\ {\isacharparenleft}{\kern0pt}{\isacharparenleft}{\kern0pt}expr{\isacharunderscore}{\kern0pt}{\isadigit{2}}\ {\isasymcirc}\ {\isasymPhi}{\isacharparenright}{\kern0pt}\ {\isacharbackquote}{\kern0pt}\ I\ {\isasymunion}\ {\isacharparenleft}{\kern0pt}expr{\isacharunderscore}{\kern0pt}{\isadigit{2}}\ {\isasymcirc}\ {\isasymPhi}{\isacharparenright}{\kern0pt}\ {\isacharbackquote}{\kern0pt}\ J{\isacharparenright}{\kern0pt}{\isachardoublequoteclose}\isanewline
\ \ \ \ \isacommand{using}\isamarkupfalse%
\ formula{\isacharunderscore}{\kern0pt}prices{\isacharunderscore}{\kern0pt}list{\isachardot}{\kern0pt}expr{\isacharunderscore}{\kern0pt}{\isadigit{2}}{\isacharunderscore}{\kern0pt}conj\ \isacommand{by}\isamarkupfalse%
\ blast\isanewline
\ \ \isacommand{with}\isamarkupfalse%
\ assms\ \isacommand{have}\isamarkupfalse%
\ {\isachardoublequoteopen}{\isachardot}{\kern0pt}{\isachardot}{\kern0pt}{\isachardot}{\kern0pt}\ {\isasymle}\ {\isadigit{1}}{\isachardoublequoteclose}\isanewline
\ \ \ \ \isacommand{using}\isamarkupfalse%
\ expr{\isachardot}{\kern0pt}simps\ less{\isacharunderscore}{\kern0pt}eq{\isacharunderscore}{\kern0pt}t{\isachardot}{\kern0pt}simps\isanewline
\ \ \ \ \isacommand{by}\isamarkupfalse%
\ simp\isanewline
\ \ \isacommand{hence}\isamarkupfalse%
\ le{\isacharunderscore}{\kern0pt}{\isadigit{0}}{\isacharcolon}{\kern0pt}\ {\isachardoublequoteopen}Sup\ {\isacharparenleft}{\kern0pt}{\isacharparenleft}{\kern0pt}expr{\isacharunderscore}{\kern0pt}{\isadigit{2}}\ {\isasymcirc}\ {\isasymPhi}{\isacharparenright}{\kern0pt}\ {\isacharbackquote}{\kern0pt}\ I\ {\isasymunion}\ {\isacharparenleft}{\kern0pt}expr{\isacharunderscore}{\kern0pt}{\isadigit{2}}\ {\isasymcirc}\ {\isasymPhi}{\isacharparenright}{\kern0pt}\ {\isacharbackquote}{\kern0pt}\ J{\isacharparenright}{\kern0pt}\ {\isasymle}\ {\isadigit{0}}{\isachardoublequoteclose}\isanewline
\ \ \ \ \isacommand{by}\isamarkupfalse%
\ simp\isanewline
\ \ \isacommand{hence}\isamarkupfalse%
\ le{\isacharunderscore}{\kern0pt}{\isadigit{0}}{\isacharcolon}{\kern0pt}\ {\isachardoublequoteopen}{\isasymforall}x\ {\isasymin}\ {\isacharparenleft}{\kern0pt}{\isacharparenleft}{\kern0pt}expr{\isacharunderscore}{\kern0pt}{\isadigit{2}}\ {\isasymcirc}\ {\isasymPhi}{\isacharparenright}{\kern0pt}\ {\isacharbackquote}{\kern0pt}\ I{\isacharparenright}{\kern0pt}{\isachardot}{\kern0pt}\ x\ {\isasymle}\ {\isadigit{0}}{\isachardoublequoteclose}\ {\isachardoublequoteopen}{\isasymforall}x\ {\isasymin}\ {\isacharparenleft}{\kern0pt}{\isacharparenleft}{\kern0pt}expr{\isacharunderscore}{\kern0pt}{\isadigit{2}}\ {\isasymcirc}\ {\isasymPhi}{\isacharparenright}{\kern0pt}\ {\isacharbackquote}{\kern0pt}\ J{\isacharparenright}{\kern0pt}{\isachardot}{\kern0pt}\ x\ {\isasymle}\ {\isadigit{0}}{\isachardoublequoteclose}\isanewline
\ \ \ \ \isacommand{using}\isamarkupfalse%
\ Sup{\isacharunderscore}{\kern0pt}le{\isacharunderscore}{\kern0pt}iff\ UnCI\isanewline
\ \ \ \ \isacommand{by}\isamarkupfalse%
\ metis{\isacharplus}{\kern0pt}\isanewline
\ \ \isacommand{have}\isamarkupfalse%
\ {\isachardoublequoteopen}{\isasymforall}x\ {\isasymin}\ {\isacharparenleft}{\kern0pt}{\isacharparenleft}{\kern0pt}expr{\isacharunderscore}{\kern0pt}{\isadigit{2}}\ {\isasymcirc}\ {\isasymPhi}{\isacharparenright}{\kern0pt}\ {\isacharbackquote}{\kern0pt}\ I{\isacharparenright}{\kern0pt}{\isachardot}{\kern0pt}\ x\ {\isasymge}\ {\isadigit{1}}{\isachardoublequoteclose}\ \isanewline
\ \ \ \ {\isachardoublequoteopen}{\isasymforall}x\ {\isasymin}\ {\isacharparenleft}{\kern0pt}{\isacharparenleft}{\kern0pt}expr{\isacharunderscore}{\kern0pt}{\isadigit{2}}\ {\isasymcirc}\ {\isasymPhi}{\isacharparenright}{\kern0pt}\ {\isacharbackquote}{\kern0pt}\ J{\isacharparenright}{\kern0pt}{\isachardot}{\kern0pt}\ x\ {\isasymge}\ {\isadigit{1}}{\isachardoublequoteclose}\ \isacommand{using}\isamarkupfalse%
\ expr{\isacharunderscore}{\kern0pt}{\isadigit{2}}{\isacharunderscore}{\kern0pt}lb\isanewline
\ \ \ \ \isacommand{by}\isamarkupfalse%
\ fastforce{\isacharplus}{\kern0pt}\isanewline
\ \ \isacommand{with}\isamarkupfalse%
\ le{\isacharunderscore}{\kern0pt}{\isadigit{0}}\ \isacommand{show}\isamarkupfalse%
\ {\isacharquery}{\kern0pt}thesis\ \isacommand{using}\isamarkupfalse%
\ imageI\ \isanewline
\ \ \ \ \isacommand{by}\isamarkupfalse%
\ simp\isanewline
\isacommand{qed}\isamarkupfalse%
%
\endisatagproof
{\isafoldproof}%
%
\isadelimproof
\isanewline
%
\endisadelimproof
\isanewline
\isacommand{lemma}\isamarkupfalse%
\ trace{\isacharunderscore}{\kern0pt}left{\isacharcolon}{\kern0pt}\isanewline
\ \ \isakeyword{assumes}\ {\isachardoublequoteopen}{\isacharparenleft}{\kern0pt}less{\isacharunderscore}{\kern0pt}eq{\isacharunderscore}{\kern0pt}t\ {\isacharparenleft}{\kern0pt}expr\ {\isasymphi}{\isacharparenright}{\kern0pt}\ {\isacharparenleft}{\kern0pt}{\isasyminfinity}{\isacharcomma}{\kern0pt}\ {\isadigit{1}}{\isacharcomma}{\kern0pt}\ {\isadigit{0}}{\isacharcomma}{\kern0pt}\ {\isadigit{0}}{\isacharcomma}{\kern0pt}\ {\isadigit{0}}{\isacharcomma}{\kern0pt}\ {\isadigit{0}}{\isacharparenright}{\kern0pt}{\isacharparenright}{\kern0pt}{\isachardoublequoteclose}\isanewline
\ \ \isakeyword{shows}\ {\isachardoublequoteopen}{\isacharparenleft}{\kern0pt}hml{\isacharunderscore}{\kern0pt}trace\ {\isasymphi}{\isacharparenright}{\kern0pt}{\isachardoublequoteclose}\isanewline
%
\isadelimproof
\ \ %
\endisadelimproof
%
\isatagproof
\isacommand{using}\isamarkupfalse%
\ assms\isanewline
\isacommand{proof}\isamarkupfalse%
{\isacharparenleft}{\kern0pt}induction\ {\isasymphi}{\isacharparenright}{\kern0pt}\isanewline
\ \ \isacommand{case}\isamarkupfalse%
\ TT\isanewline
\ \ \isacommand{then}\isamarkupfalse%
\ \isacommand{show}\isamarkupfalse%
\ {\isacharquery}{\kern0pt}case\isanewline
\ \ \ \ \isacommand{using}\isamarkupfalse%
\ trace{\isacharunderscore}{\kern0pt}tt\ \isacommand{by}\isamarkupfalse%
\ blast\isanewline
\isacommand{next}\isamarkupfalse%
\isanewline
\ \ \isacommand{case}\isamarkupfalse%
\ {\isacharparenleft}{\kern0pt}hml{\isacharunderscore}{\kern0pt}pos\ {\isasymalpha}\ {\isasymphi}{\isacharparenright}{\kern0pt}\isanewline
\ \ \isacommand{then}\isamarkupfalse%
\ \isacommand{show}\isamarkupfalse%
\ {\isacharquery}{\kern0pt}case\ \isanewline
\ \ \ \ \isacommand{using}\isamarkupfalse%
\ trace{\isacharunderscore}{\kern0pt}pos\ \isacommand{by}\isamarkupfalse%
\ simp\isanewline
\isacommand{next}\isamarkupfalse%
\isanewline
\ \ \isacommand{case}\isamarkupfalse%
\ {\isacharparenleft}{\kern0pt}hml{\isacharunderscore}{\kern0pt}conj\ I\ J\ {\isasymPhi}{\isacharparenright}{\kern0pt}\isanewline
\ \ \isacommand{then}\isamarkupfalse%
\ \isacommand{show}\isamarkupfalse%
\ {\isacharquery}{\kern0pt}case\ \isacommand{using}\isamarkupfalse%
\ HML{\isacharunderscore}{\kern0pt}trace{\isacharunderscore}{\kern0pt}conj{\isacharunderscore}{\kern0pt}empty\ trace{\isacharunderscore}{\kern0pt}conj\isanewline
\ \ \ \ \isacommand{by}\isamarkupfalse%
\ metis\isanewline
\isacommand{qed}\isamarkupfalse%
\isanewline
%
\endisatagproof
{\isafoldproof}%
%
\isadelimproof
%
\endisadelimproof
%
\isadelimvisible
%
\endisadelimvisible
%
\isatagvisible
%
\endisatagvisible
{\isafoldvisible}%
%
\isadelimvisible
\isanewline
%
\endisadelimvisible
\isacommand{context}\isamarkupfalse%
\ lts\ \isakeyword{begin}\isanewline
\isanewline
\isacommand{lemma}\isamarkupfalse%
\ {\isachardoublequoteopen}hml{\isacharunderscore}{\kern0pt}trace{\isacharunderscore}{\kern0pt}equivalent\ p\ q\ {\isasymlongleftrightarrow}\ expr{\isacharunderscore}{\kern0pt}trace{\isacharunderscore}{\kern0pt}equivalent\ p\ q{\isachardoublequoteclose}\isanewline
%
\isadelimproof
\ \ %
\endisadelimproof
%
\isatagproof
\isacommand{unfolding}\isamarkupfalse%
\ hml{\isacharunderscore}{\kern0pt}trace{\isacharunderscore}{\kern0pt}equivalent{\isacharunderscore}{\kern0pt}def\ HML{\isacharunderscore}{\kern0pt}subset{\isacharunderscore}{\kern0pt}equivalent{\isacharunderscore}{\kern0pt}def\ expr{\isacharunderscore}{\kern0pt}trace{\isacharunderscore}{\kern0pt}equivalent{\isacharunderscore}{\kern0pt}def\ expr{\isacharunderscore}{\kern0pt}traces{\isacharunderscore}{\kern0pt}def\ hml{\isacharunderscore}{\kern0pt}trace{\isacharunderscore}{\kern0pt}formulas{\isacharunderscore}{\kern0pt}def\isanewline
\ \ \isacommand{using}\isamarkupfalse%
\ HML{\isacharunderscore}{\kern0pt}trace{\isacharunderscore}{\kern0pt}lemma\ \isanewline
\ \ \isacommand{by}\isamarkupfalse%
\ blast%
\endisatagproof
{\isafoldproof}%
%
\isadelimproof
\isanewline
%
\endisadelimproof
\isanewline
\isacommand{end}\isamarkupfalse%
%
\begin{isamarkuptext}%
On Infinity...%
\end{isamarkuptext}\isamarkuptrue%
%
\isadelimproof
%
\endisadelimproof
%
\isatagproof
%
\endisatagproof
{\isafoldproof}%
%
\isadelimproof
%
\endisadelimproof
%
\isadelimproof
%
\endisadelimproof
%
\isatagproof
%
\endisatagproof
{\isafoldproof}%
%
\isadelimproof
%
\endisadelimproof
%
\isadelimproof
%
\endisadelimproof
%
\isatagproof
%
\endisatagproof
{\isafoldproof}%
%
\isadelimproof
\isanewline
%
\endisadelimproof
\isacommand{end}\isamarkupfalse%
\isanewline
%
\isadelimtheory
%
\endisadelimtheory
%
\isatagtheory
\isacommand{end}\isamarkupfalse%
%
\endisatagtheory
{\isafoldtheory}%
%
\isadelimtheory
%
\endisadelimtheory
%
\end{isabellebody}%
\endinput
%:%file=~/Documents/Isabelle_HOL/Traces.thy%:%
%:%24=7%:%
%:%36=9%:%
%:%45=11%:%
%:%57=13%:%
%:%58=14%:%
%:%59=15%:%
%:%60=16%:%
%:%61=17%:%
%:%63=18%:%
%:%64=18%:%
%:%65=19%:%
%:%66=20%:%
%:%67=21%:%
%:%68=22%:%
%:%69=23%:%
%:%70=23%:%
%:%71=24%:%
%:%72=25%:%
%:%74=26%:%
%:%75=27%:%
%:%76=28%:%
%:%78=30%:%
%:%79=30%:%
%:%80=31%:%
%:%81=32%:%
%:%82=33%:%
%:%83=33%:%
%:%84=34%:%
%:%86=36%:%
%:%87=37%:%
%:%88=38%:%
%:%89=39%:%
%:%91=41%:%
%:%92=41%:%
%:%93=42%:%
%:%94=43%:%
%:%95=44%:%
%:%96=45%:%
%:%97=45%:%
%:%98=46%:%
%:%99=47%:%
%:%100=48%:%
%:%108=49%:%
%:%120=50%:%
%:%121=51%:%
%:%122=52%:%
%:%123=53%:%
%:%124=54%:%
%:%128=56%:%
%:%129=57%:%
%:%130=58%:%
%:%131=59%:%
%:%132=60%:%
%:%133=61%:%
%:%134=62%:%
%:%138=64%:%
%:%139=65%:%
%:%140=66%:%
%:%141=67%:%
%:%142=68%:%
%:%155=75%:%
%:%158=76%:%
%:%159=76%:%
%:%160=77%:%
%:%161=78%:%
%:%164=79%:%
%:%168=79%:%
%:%169=79%:%
%:%170=80%:%
%:%171=80%:%
%:%172=81%:%
%:%173=81%:%
%:%174=82%:%
%:%175=82%:%
%:%176=82%:%
%:%177=82%:%
%:%178=83%:%
%:%179=83%:%
%:%180=84%:%
%:%181=84%:%
%:%182=85%:%
%:%183=85%:%
%:%184=86%:%
%:%185=86%:%
%:%186=86%:%
%:%187=87%:%
%:%188=87%:%
%:%189=87%:%
%:%190=87%:%
%:%191=88%:%
%:%192=88%:%
%:%193=89%:%
%:%194=89%:%
%:%195=90%:%
%:%196=90%:%
%:%197=90%:%
%:%198=90%:%
%:%199=91%:%
%:%205=91%:%
%:%208=92%:%
%:%209=93%:%
%:%210=93%:%
%:%211=94%:%
%:%212=95%:%
%:%219=96%:%
%:%220=96%:%
%:%221=97%:%
%:%222=97%:%
%:%223=98%:%
%:%224=98%:%
%:%225=98%:%
%:%226=99%:%
%:%227=99%:%
%:%228=99%:%
%:%229=100%:%
%:%230=100%:%
%:%231=101%:%
%:%232=101%:%
%:%233=102%:%
%:%234=102%:%
%:%235=103%:%
%:%236=103%:%
%:%237=104%:%
%:%238=104%:%
%:%239=105%:%
%:%240=105%:%
%:%241=106%:%
%:%242=106%:%
%:%243=107%:%
%:%244=107%:%
%:%245=108%:%
%:%246=108%:%
%:%247=109%:%
%:%248=109%:%
%:%249=110%:%
%:%250=110%:%
%:%251=110%:%
%:%252=110%:%
%:%253=111%:%
%:%254=111%:%
%:%255=112%:%
%:%261=112%:%
%:%264=113%:%
%:%265=114%:%
%:%266=114%:%
%:%267=115%:%
%:%268=116%:%
%:%271=117%:%
%:%275=117%:%
%:%276=117%:%
%:%277=118%:%
%:%278=118%:%
%:%279=119%:%
%:%280=119%:%
%:%281=120%:%
%:%282=120%:%
%:%283=120%:%
%:%284=121%:%
%:%285=121%:%
%:%286=121%:%
%:%287=122%:%
%:%288=122%:%
%:%289=123%:%
%:%290=123%:%
%:%291=124%:%
%:%292=124%:%
%:%293=124%:%
%:%294=125%:%
%:%295=125%:%
%:%296=125%:%
%:%297=126%:%
%:%298=126%:%
%:%299=127%:%
%:%300=127%:%
%:%301=128%:%
%:%302=128%:%
%:%303=128%:%
%:%304=128%:%
%:%305=129%:%
%:%306=129%:%
%:%307=130%:%
%:%308=130%:%
%:%327=137%:%
%:%330=138%:%
%:%331=138%:%
%:%332=139%:%
%:%333=140%:%
%:%334=140%:%
%:%337=141%:%
%:%341=141%:%
%:%342=141%:%
%:%343=142%:%
%:%344=142%:%
%:%345=143%:%
%:%346=143%:%
%:%351=143%:%
%:%354=144%:%
%:%355=145%:%
%:%358=147%:%
%:%397=192%:%
%:%400=193%:%
%:%401=193%:%
%:%408=194%:%
